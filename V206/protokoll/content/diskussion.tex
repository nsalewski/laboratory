\section{Diskussion}
\label{sec:Diskussion}

Zunächst ist anzumerken, dass die Abdeckung der Reservoire den Wärmeaustausch mit der Umgebung nicht vollständig verhindern können, da die zwischen den Reservoiren und ihren Abdeckungen trotz Korrekturversuchen nicht ohne Spalt aufeinander liegen.
Daher nimmt das kältere Reservoir weiterhin etwas Umgebungswärme auf und das wärmere Reservoir verliert Wärme an die Umgebung.

Zudem waren die Skalen auf den Manometer recht grob gegliedert, sodass eine genaues Ablesen erschwert ist. Gleiches gilt für das Ablesen der Leistungsaufnahme des Kompressors.

Die stark von den Theoriewerten abweichende Güteziffer lässt sich somit auf Wärmeverluste durch schlechte Isolierung, besonders an den beiden Reservoiren, zurückführen.
Die Kupferschlange im Reservoir hatte somit nahezu unisoliert Kontakt zur Umgebung außerhalb des Reservoirs. Da hier der Wärmeaustausch stattfindet, haben schlechte Insolierungen besonders an den Reservoiren signifikante Auswikungen auf die Güteziffer.
Auffällig war außerdem, dass trotz der Isolierung Wärme mit der Umgebung ausgetauscht wurde. Bei der Berührung der Eimer war eine Wärmeabstrahlung deutlich zu spüren

\newpage
