\section{Theoretische Grundlagen}
\label{sec:Theorie}

Der zweite Hauptsatz der Thermodynamik besagt, dass Wärme immer vom wärmeren zum kälteren Reservoir fließt. Dieser Prozess lässt sich allerdings auch umkehren.
Es ist allerdings auch möglich, diesen Prozess umzukehren. Dies kann man realisieren mittels einer Wärmepumpe, welche dem System mechanische Arbeit zuführt. Dadurch kann dann
Wärme vom kälteren ins wärmere Reservoir transportiert wird.
.

\subsection {Güteziffer}
\label{sec:güteziffer}
Die Güteziffer $\upnu$ gibt das Verhältnis zwischen transportierter Wärmemenge und der dafür benötigten Arbeit an. Nach dem 1. Hauptsatz der Thermodynamik muss die dem Reservoir 1 hinzugefügte Wärmemenge $Q_1$
der Summe der aus Reservoir 2 entzogenen Wärmemenge $Q_2$ und der mechanischen Arbeit $A$, im vorliegenden Versuch durch die Kompressionsarbeit realisiert, sein.
Daraus ergibt sich dann für
\begin{equation}
  \upnu=\frac{Q_1}{A}
\end{equation}

\subsection {Massendurchsatz}
\label{sec:massendurchsatz}
\subsection {Berechnung der Kompressionsleistung}
\label{sec:kompressorleistung}








\cite{Anleitung}
