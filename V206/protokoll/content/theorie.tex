\section{Theoretische Grundlagen}
\label{sec:Theorie}

\subsection{Das Prinzip der Wärmepumpe}
Betrachtet man zwei Flüssigkeitsreservoire mit den Temperaturen $T_1$ und $T_2$, wobei $T_1 > T_2$ gilt, dann wird solange Wärmeenergie vom Reservoir $1$ zum Reservoir $2$ übertragen, bis die Temperaturdifferenz $\Delta T := T_1 - T_2$ gleich Null ist, also die Temperaturen gleich sind.
Dieser Wärmetransport lässt sich allerdings mit Hilfe einer Wärmepumpe umkehren. Unter Aufwendung von mechanischer Arbeit kann einem kälteren Reservoir Wärmeenergie entzogen werden und dem wärmeren Reservoir hinzugefügt werden.
Eine Kenngröße für die Effizienz einer Wärmepumpe ist die Güteziffer $\upnu$ (auch effektive Leistungszahl \cite{geschke}). Im Folgenden soll nun ein Ausdruck für die effektive Leistungszahl hergeleitet werden.

Die Wärmepumpe wird idealisierend als abgeschlossenes System aufgefasst.
Demnach gilt nach dem 1. Hauptsatz der Thermodynamik, dass die von einem Transportmedium an das Reservoir $1$ übertragene Wärmemenge $Q_1$ gleich der Summe der vom Reservoir $2$ entzogenen Wärmemenge $Q_2$ und der verrichteten Arbeit $A$ ist. Also gilt:
\begin{equation}
	\label{eqn:dQ}
	Q_1 = Q_2 + A
\end{equation}
Offensichtlich ist eine Wärmepumpe umso effizienter, wenn eine möglichst kleine mechanische Arbeit für eine möglichst große übertragene Wärmemenge $Q_1$ benötigt wird. Daher wird die Güteziffer $\upnu$ wie folgt definiert:
\begin{equation}
	\label{eqn:guete}
	\upnu := \frac{Q_1}{A}
\end{equation}
Aus der Annahme, dass die Wärmepumpe als abgeschlossenes isoliertes System angenommen wird, ergibt sich für die Entropieänderung $\symup{d}S$ des Systems
\begin{equation}
	\label{eqn:entropie}
	\symup{d}S = \frac{Q_1}{T_1} - \frac{Q_2}{T_2}
\end{equation}
Die Entropieänderung eines isolierten Systems ist gleich Null, wenn die Wärmeübertragung reversibel verläuft. 
Ein reversibler Umwandlungsprozess entspricht der Annahme, dass während der Wärmeübertragung die Temperaturen $T_1$ und $T_2$, sowie die zugehörigen Drücke $p_b$ und $p_a$ konstant bleiben. Der Prozess müsste also unendlich langsam verlaufen. 
Dies muss erfüllt sein, damit die vom Transportmedium aufgenommene Energie jederzeit in einem umgekehrtem Prozess wieder zurückgewonnen werden kann \cite{Anleitung}.

Aus den Gleichungen \eqref{eqn:dQ}, \eqref{eqn:guete} und \eqref{eqn:entropie} erhält man nun für die Güteziffer den Ausdruck:
\begin{equation}
	\label{eqn:gueteneu}
	\upnu = \frac{Q_1}{A} = \frac{T_1}{T_1 - T_2}
\end{equation}
Allerdings gilt Gleichung \eqref{eqn:gueteneu} nicht für den realen irreversiblen Fall. 
Dann ist nämlich $\symup{d}S > 0$ und es gilt für die reale Güteziffer:
\begin{equation}
	\upnu_{real} < \frac{T_1}{T_1 - T_2}
\end{equation}
Die Wärmepumpe arbeitet also umso effektiver, je kleiner die Temperaturdifferenz $\Delta T$ ist.

Verwendet wird die Wärmepumpe, um preiswert zu heizen.
Wird mechanische Arbeit unmittelbar in Wärme umgewandelt, ist die übertragene Wärmemenge höchstens so groß wie die aufgewendete Arbeit $A$.
Mit Hilfe einer Wärmepumpe kann Wärmeenergie allerdings viel effektiver übertragen werden. Es gilt:
\begin{equation}
	Q_{1_{real}} < A \frac{T_1}{T_1 - T_2}
\end{equation}


\subsection{Die Arbeitsweise einer Wärmepumpe}

% Julian


\subsection{Bestimmung von Kenngrößen einer realen Wärmepumpe}

\subsubsection {Reale Güteziffer}
\label{sec:gueteziffer}
\subsubsection {Massendurchsatz}
\label{sec:massendurchsatz}
\subsubsection {Mechanischen Kompressionsleistung} 
\label{sec:kompressorleistung}
