\section{Anhang}




  \begin{longtable}{ccc}
  \caption{Aufgenommene Messwerte des Silberzerfalls abzüglich der Nullmessung.}\label{tab:anhang}\\
  \hline
$t$/$\si{\second}$ & Anzahl der Zerfälle $N-N_0$ & Logarithmus $\ln{N-N_0}$ \\
\hline
8 & 136 & 4.913 \\
16 & 113 & 4.728 \\
24 & 91 & 4.511 \\
32 & 84 & 4.431 \\
40 & 62 & 4.128 \\
48 & 75 & 4.318 \\
56 & 51 & 3.933 \\
64 & 45 & 3.808 \\
72 & 37 & 3.612 \\
80 & 27 & 3.297 \\
88 & 33 & 3.498 \\
96 & 23 & 3.137 \\
104 & 21 & 3.047 \\
112 & 23 & 3.137 \\
120 & 11 & 2.402 \\
128 & 21 & 3.047 \\
136 & 9 & 2.202 \\
144 & 16 & 2.775 \\
152 & 17 & 2.836 \\
160 & 11 & 2.402 \\
168 & 5 & 1.618 \\
176 & 8 & 2.085 \\
184 & 13 & 2.568 \\
192 & 6 & 1.799 \\
200 & 12 & 2.489 \\
208 & 5 & 1.618 \\
216 & 8 & 2.085 \\
224 & 10 & 2.307 \\
232 & 5 & 1.618 \\
240 & 8 & 2.085 \\
248 & 4 & 1.397 \\
256 & 6 & 1.799 \\
264 & 2 & 0.715 \\
272 & 10 & 2.307 \\
280 & 8 & 2.085 \\
288 & 6 & 1.799 \\
296 & 4 & 1.397 \\
304 & 7 & 1.952 \\
312 & 6 & 1.799 \\
320 & 1 & 0.043 \\
328 & 2 & 0.715 \\
336 & 6 & 1.799 \\
344 & 2 & 0.715 \\
352 & 6 & 1.799 \\
360 & 3 & 1.113 \\
368 & 2 & 0.715 \\
376 & 6 & 1.799 \\
384 & 1 & 0.043 \\
392 & 3 & 1.113 \\
400 & 1 & 0.043 \\
408 & 2 & 0.715 \\
416 & 3 & 1.113 \\
424 & 0 & -3.114 \\
432 & 5 & 1.618 \\
440 & 3 & 1.113 \\
448 & -1 & -- \\
456 & 0 & -3.114 \\
464 & 0 & -3.114 \\
472 & 1 & 0.043 \\
480 & 3 & 1.113 \\
\hline
\end{longtable}
