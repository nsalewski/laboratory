\section{Theorie}
\label{sec:Theorie}
Zur Untersuchung der Halbwertszeit $T$ von instabilen Nukliden ist es zunächst nötig
Nuklide einer im vorliegenden Rahmen beobachtbaren Halbwertszeit von wenigen Sekunden bis Stunden. Die Halbwertszeit meint hierbei die Zeit, bei der genau die Hälfte einer großen Anzahl an instabilen Nukliden zerfallen ist.
Nuklide sind stabil, wenn das Verhältnis ihrer Neutronen-und Protonenzahl innerhalb bestimmer Grenzen liegt.
Je nach Nuklid muss die Neutronenzahl hierzu etwa $\SI{20}{\percent}$ bis $\SI{50}{\percent}$ betragen. Instabile Kerne zerfallen mit bestimmten Wahrscheinlichkeiten in stabile oder instabile Kerne, welche wiederum weiter zerfallen.
Instabile Nuklide werden im vorliegenden Experiment über den Beschuss stabiler Kerne mit Neutronen.
\subsection{Wechselwirkungen zwischen Neutronen mit Nukliden}
Wechselwirkungen zwischen Teilchen und Kernen werden als Kernreaktionen bezeichnet.
Die ablaufende Reaktion bei Beschuss eines Kerns $A$ mit einem Neutron lässt sich wie folgt schreiben:
\begin{equation*}
  ^{\mathrm{m}}_\mathrm{z}A+
\end{equation*}
Wird ein Kern $A$ mit einem Neutron beschossen, entsteht zunächst ein sogenannter \textbf{Zwischenkern} $A^{*}$, auch \textbf{Coupondkern} genannt, erzeugt. Die zusätzlich eingebrachte Energie gegenüber dem Kern $A$ durch die kinetische Energie und die Bindungsenergie des Neutrons wird auf eine große Anzahl an Nukleonen im Kern verteilt.
Nach einer kurzen Zeitspanne von $\SI{10e-16}{\second}$ geht der Zwischenkern unter Emission eines $\gamma$-Quants in seinen Grundzustand zurück.
\subsection{Erzeugung niederenergetischer Neutronen}



\subsection{Untersuchung des Zerfalls instabiler Isotope}
