\section{Diskussion}
\label{sec:Diskussion}
Bei der Messung des Zerfalls der Silberisotope zeigten sich besonders für große Zeiten $t$ (ab etwa $t=\SI{320}{\second}$) nur noch sehr kleine Anzahlen an registrierten Impulsen. Abzüglich der Nullmessung wurde so häufig nur ein Impuls registriert, für Zeiten größer als $t=\SI{416}{\second}$ wird zudem häufig abzüglich der Nullmessung gar kein Impuls mehr gemessen.
Zu beachten ist auch, dass der Zerfall der Isotope bereits beginnt, wenn die Probe aus der Quelle der thermischen Neutronen entnommen wird.
Theoretisch müssten also noch einige Sekunden auf die Messzeit addiert werden und es würden sich eventuell leicht verschiedene Messergebnisse ergeben.
Ein Vergleich der bestimmten Halbwertszeiten mit dem Theoriewert nach \cite{silber} und
\cite{indium} samt der Abweichung findet sich in Tabelle \ref{tab:theor}. Die recht große Abweichung für die Halbwertszeit des langlebigen Zerfalls des Silber-Isotops $\ce{^{108}_47Ag}$ lässt sich hierbei mit den bereits genannten Unsicherheiten in der Messung besonders für große Zeiten erklären.
Es ist jedoch anzumerken, dass sich für den langlebigen Zerfall ein recht großes Fehlerintervall ergibt, sodass der Literaturwert nur unweit des Fehlerintervalls liegt.
\begin{table}
  \centering
  \caption{Vergleich der Messergebnisse mit den Theoriewerten für die Halbwertszeiten des Zerfalls der gemessenen Isotope.}
  \label{tab:theor}
  \begin{tabular}{cccc}
  \toprule
  Isotop&experimentelles Ergebnis $T_\frac{1}{2}$/$\si{\second}$&Theoriewert $T_\frac{1}{2}$/$\si{\second}$ &Abweichung\\
  \midrule
	  $\ce{^{116}In}$&$\num{3.08(11)E03}$ & \num{3.2574E03} & $\SI{5}{\percent}$ \\
  $\ce{^{108}_47Ag}$&\num{113(24)}&\num{142.9(7)}&$\SI{21}{\percent}$\\
  $\ce{^{110}_47Ag}$&\num{25(2)}&\num{24,56(11)}&$\SI{2}{\percent}$\\
  \bottomrule
  \end{tabular}
\end{table}
