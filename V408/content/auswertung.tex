\section{Auswertung}
\label{sec:Auswertung}

\subsection{Bestimmung der Brennweite durch Messung der Gegenstandsweite und Bildweite und Überprüfung der Linsengleichung und des Abbildungsgesetzes}

\subsubsection{Bestimmung der Brennweite einer bekannten Sammellinse}
\label{sec:zuvor}
Zur Bestimmung der Brennweite über die Bildweite und die Gegenstandsweite werden die gemessenen Gegenstandsweiten $g_{\mathrm{i}}$ auf der x-Achse und die gemessenen Bildweiten $b_{\mathrm{i}}$ auf der y-Achse aufgetragen und die zueinandergehörigen Datentupel miteinander verbunden. \\
Die verwendeten Datentupel sind hierbei in Tabelle \ref{tab:bundg} angegeben.
\\Der sich ergebende Schnittpunkt aller Verbindungsgraden hat schließlich als Koordinaten für beide Raumrichtungen die Brennweite $f$ der verwendeten Linse. \\Um die Brennweite aus dem Graphen möglichst genau abzulesen, wird der Bereich des Schnittpunkts vergrößert in Abbildung \ref{fig:plota} dargestellt.
Die Brennweite wird abgelesen zu
\begin{equation*}
  f_{\mathrm{abgelesen}}=\SI{9.7(1)}{\centi\meter}\text{.}
\end{equation*}
\subsubsection{Experimentelle Überprüfung der Linsengleichung und des Abbildungsgesetz.}
Über die Linsengleichung \eqref{eqn:linsi} ergibt sich, über die Mittelung der Brennweiten $f_{\mathrm{i}}$, berechnet aus den Datentupeln aus Bild-und Gegenstandsweite, die Brennweite $f_{\mathrm{Linsengleichung}}$.
Ebenso wie hier, wird auch in den folgenden Mittelwertberechnungen, selbiger mittels python/numpy \cite{numpy} ermittelt.\\
Hierbei wird als Fehler, wie in den weiteren Messungen ebenfalls, lediglich der Fehler des Mittelwerts angegeben.\\
Es ergibt sich
\begin{equation*}
  f_{\mathrm{Linsengleichung}}=\SI{9.70(9)}{\centi\meter}\text{.}
\end{equation*}

\begin{figure}
  \centering
  \includegraphics[width=0.8\textwidth]{Bilder/plot_a.pdf}
  \caption{Schnittpunkt der Verbindungslinien zwischen den auf der x-Achse aufgetragenen Gegenstandsweiten $g_\mathrm{i}$ und den zugehörigen Bildweiten $b_\mathrm{i}$ auf der y-Achse zur Bestimmung der Brennweite.}
  \label{fig:plota}
\end{figure}
Ein Vergleich mit dem aus dem Graphen abgelesenen $f_{\mathrm{abgelesen}}$ zeigt, dass die Linsengleichung für unsere Messung bestätigt werden kann.


\begin{table}
  \caption{Abgelesene Datenpaare aus Bildweite $b$ und Gegenstandsweite $g$ zur Untersuchung der Linsengleichung.}
  \label{tab:bundg}
  \centering
  \begin{tabular}{ccc}
    \toprule
  $b/\si{\centi\meter}$ & $g/\si{\centi\meter}$ & $f/\si{\centi\meter}$ \\
\midrule
  19.8 \pm 0.1 & 19.2 \pm 0.1 & 9.75 \pm 0.04 \\
  33.7 \pm 0.1 & 13.3 \pm 0.1 & 9.54 \pm 0.05 \\
  21.8 \pm 0.1 & 17.4 \pm 0.1 & 9.68 \pm 0.04 \\
  15.6 \pm 0.1 & 25.8 \pm 0.1 & 9.72 \pm 0.04 \\
  14.0 \pm 0.1 & 31.0 \pm 0.1 & 9.64 \pm 0.05 \\
  100.5 \pm 0.1 & 11.0 \pm 0.1 & 9.91 \pm 0.08 \\
  16.1 \pm 0.1 & 24.4 \pm 0.1 & 9.70 \pm 0.04 \\
  13.8 \pm 0.1 & 33.3 \pm 0.1 & 9.76 \pm 0.05 \\
  13.0 \pm 0.1 & 37.9 \pm 0.1 & 9.68 \pm 0.06 \\
  13.7 \pm 0.1 & 32.8 \pm 0.1 & 9.66 \pm 0.05 \\
\bottomrule
\end{tabular}
\end{table}



\FloatBarrier


Zur Überprüfung des Abbildungsgesetzes werden zudem die Bildgrößen $B$ und Gegenstandsgrößen $G$ benötigt. \\Die gemessenen Datentupel finden sich in Tabelle \ref{tab:groesse}. \\
Ein Vergleich des aus  Bildweite und Gegenstandsweite berechneten Abbildungsmaßstabs $V_\mathrm{1}$ mit dem Abbildungsmaßstab $V_\mathrm{2}$ berechnet aus Bildgröße und Gegenstandsgröße, zeigt, dass sich nur geringe Unterschiede ergeben.\\ So liegt $\Delta V$ zumeist im Fehlerintervall beider Abbildungsmaßstäbe $V_\mathrm{i}$.\\
Die Abweichungen liegen also zum größten Teil unter der Messauflösung, oder zumindest sehr nah an den Fehlerintervallen beider Abbildungsmaßstäbe $V_\mathrm{i}$.
\\Die Gültigkeit des Abbildungsgesetzes wird also durch die vorliegende Messung bestätigt.
\begin{table}
  \caption{Aufgenommene Messdaten zur Untersuchung des Abbildungsgesetzes.}
  \label{tab:groesse}
  \centering
\begin{tabular}{ccccccc}
  \toprule
  $b/\si{\centi\meter} $& $g/\si{\centi\meter} $& $B/\si{\centi\meter} $& $G/\si{\centi\meter}$ &$ V_{\mathrm{1}}=\frac{b}{g}$ & $V_\mathrm{2}=\frac{B}{G}$ & $\Delta V$ \\
\midrule
  19.8 \pm 0.1 & 19.2 \pm 0.1 & 3.0 \pm 0.1 & 3.0 \pm 0.1 & 1.031 \pm 0.007 & 1.00 \pm 0.05 & 0.03 \pm 0.05 \\
  33.7 \pm 0.1 & 13.3 \pm 0.1 & 7.4 \pm 0.1 & 3.0 \pm 0.1 & 2.534 \pm 0.020 & 2.47 \pm 0.09 & 0.07 \pm 0.09 \\
  21.8 \pm 0.1 & 17.4 \pm 0.1 & 3.7 \pm 0.1 & 3.0 \pm 0.1 & 1.253 \pm 0.009 & 1.23 \pm 0.05 & 0.02 \pm 0.05 \\
  15.6 \pm 0.1 & 25.8 \pm 0.1 & 1.9 \pm 0.1 & 3.0 \pm 0.1 & 0.605 \pm 0.005 & 0.63 \pm 0.04 & 0.03 \pm 0.04 \\
  14.0 \pm 0.1 & 31.0 \pm 0.1 & 1.5 \pm 0.1 & 3.0 \pm 0.1 & 0.452 \pm 0.004 & 0.50 \pm 0.04 & 0.05 \pm 0.04 \\
\bottomrule
\end{tabular}
\end{table}

%%%%%%%%%%%%%%%%%%%%%%%%%%%%%%%%%%%%%%%%%%%%%%%%%%%%%%%%%%%%%%%%%%%%%%%%%%%%%%%%%%%%%

\FloatBarrier
\subsubsection{Bestimmung der Brennweite einer mit Wasser gefüllten Linse}


\begin{table}
	\caption{Abgelesene Datenpaare aus Bildweite $b$ und Gegenstandsweite $g$ zur Bestimmung der Brennweite einer mit Wasser gefüllten Linse.}
	\label{tab:hquer}
	\centering
	\begin{tabular}{cc}
		\toprule
		$b/\si{\centi\meter}$ & $g/\si{\centi\meter}$ \\
	\midrule
		18.3 & 47.9 \\
		31.6 & 45.3 \\
		23.3 & 46.2 \\
		13.5 & 52.4 \\
		27.8 & 45.5 \\
		24.7 & 46.1 \\
		33.2 & 45.5 \\
		15.2 & 50.5 \\
		17.3 & 48.7 \\
		33.0 & 45.6 \\
	\bottomrule
	\end{tabular}
\end{table}

Die Messwerte zur Bestimmung einer mit Wasser gefüllten Linse sind in Tabelle \ref{tab:hquer}
aufgetragen.

Für die Brennweite der mit Wasser gefüllten Linse ergibt sich mit Gleichung \eqref{eqn:linsi}
und den Werten aus Tabelle \ref{tab:hquer} die Brennweite des Linsensystems zu
\begin{equation*}
	f_{\mathrm{Linsengleichung}} = \SI{7.5(1)}{\centi\meter} \mathrm{.}
\end{equation*}

\begin{figure}
  \centering
  \includegraphics[width=0.8\textwidth]{Bilder/schnitti.pdf}
  \caption{Schnittpunkt der Verbindungslinien zwischen den auf der x-Achse aufgetragenen Gegenstandsweiten $g_\mathrm{i}$ und den zugehörigen Bildweiten $b_\mathrm{i}$ auf der y-Achse zur Bestimmung der Brennweite der mit Wasser gefüllten Linse.}
  \label{fig:kopiert}
\end{figure}


Analog zur Bestimmung der Brennweite der bekannten Linse (vgl. \ref{sec:zuvor}) lassen sich
die Wertetupel $(g_{\mathrm{i}}, b_{\mathrm{i}})$ auf die Achsen eines Koordinatensystems
auftragen, und aus dem Schnittpunkt der Verbindungslinien die Brennweite ablesen.
Diese Darstellung  ist in Abbildung \ref{fig:kopiert} zu finden.
Die Brennweite wird zu
\begin{equation*}
	f_{\mathrm{abgelesen}} = \SI{7.9(1)}{\centi\meter}
\end{equation*}
abgelesen.




\FloatBarrier
\subsection{Bestimmung der Brennweite mit der Methode von Bessel}
\subsubsection{Berechnung der Brennweite}
Zur Bestimmung der Brennweite einer Linse nach der Methode von Bessel wird jede der beiden Linsenpositionen, bei denen ein scharfes Bild auf dem Schirm entsteht, einzeln ausgewertet. \\
Theoretisch sollte die Linsenanordnung völlig symmetrisch sein, weswegen auch ein annähernd gleiches Ergebnis für die Brennweite erwartet wird. Allerdings treten sicherlich Messungenauigkeiten und kleine Ablesefehler auf, sodass die Linsenanordnung sich in der Realität als nicht völlig symmetrisch darstellt.
\\In den Tabellen \ref{tab:besseli} und \ref{tab:bessii} sind die jeweilige Gegenstandsweite $g_\mathrm{i}$ und die Bildweite $b_\mathrm{i}$, sowie der Abstand $e_\mathrm{i}=g_\mathrm{i} + b_\mathrm{i}$ zwischen Objekt und Bild und dem Abstand $d_\mathrm{i}=g_\mathrm{i} - b_\mathrm{i}$ zwischen den beiden Linsenpositionen eingetragen.
\\Zudem wurde jeweils nach Formel \ref{eqn:nochmalbessel} die Brennweite $f_\mathrm{i}$ berechnet.
\begin{table}
  \caption{Gemessene Bild-und Gegenstandsweiten sowie zugehörige berechnete $e$, $d$ und der Brennweite bezüglich der ersten Linsenposition bei ungefiltertem Licht. }
  \label{tab:besseli}
  \centering
\begin{tabular}{ccccc}
  \toprule
$b_\mathrm{1}/\si{\centi\meter}$ & $g_\mathrm{1}/\si{\centi\meter}$ & $e_\mathrm{1}/\si{\centi\meter}$ & $d_\mathrm{1}/\si{\centi\meter}$ & $f_\mathrm{1}/\si{\centi\meter}$ \\
\midrule
37.1 \pm 0.1 & 13.4 \pm 0.1 & 50.50 \pm 0.14 & -23.70 \pm 0.14 & 9.84 \pm 0.05 \\
46.2 \pm 0.1 & 12.4 \pm 0.1 & 58.60 \pm 0.14 & -33.80 \pm 0.14 & 9.78 \pm 0.06 \\
51.6 \pm 0.1 & 12.0 \pm 0.1 & 63.60 \pm 0.14 & -39.60 \pm 0.14 & 9.74 \pm 0.07 \\
30.1 \pm 0.1 & 14.5 \pm 0.1 & 44.60 \pm 0.14 & -15.60 \pm 0.14 & 9.79 \pm 0.05 \\
41.9 \pm 0.1 & 12.7 \pm 0.1 & 54.60 \pm 0.14 & -29.20 \pm 0.14 & 9.75 \pm 0.06 \\
22.0 \pm 0.1 & 17.6 \pm 0.1 & 39.60 \pm 0.14 & -4.40 \pm 0.14 & 9.78 \pm 0.04 \\
52.6 \pm 0.1 & 12.0 \pm 0.1 & 64.60 \pm 0.14 & -40.60 \pm 0.14 & 9.77 \pm 0.07 \\
39.7 \pm 0.1 & 12.9 \pm 0.1 & 52.60 \pm 0.14 & -26.80 \pm 0.14 & 9.74 \pm 0.06 \\
35.0 \pm 0.1 & 13.6 \pm 0.1 & 48.60 \pm 0.14 & -21.40 \pm 0.14 & 9.79 \pm 0.05 \\
27.5 \pm 0.1 & 15.1 \pm 0.1 & 42.60 \pm 0.14 & -12.40 \pm 0.14 & 9.75 \pm 0.04 \\
\bottomrule
\end{tabular}
\end{table}

\begin{table}
  \caption{Messdaten zur Bestimmung der Brennweite mittels Bessel-Methode für ungefiltertes Licht bezüglich Linsenposition 2.}
  \label{tab:bessii}
  \centering
\begin{tabular}{ccccc}
  \toprule
$b_\mathrm{2}/\si{\centi\meter}$ & $g_\mathrm{2}/\si{\centi\meter}$ & $e_\mathrm{2}/\si{\centi\meter}$ & $d_\mathrm{2}/\si{\centi\meter}$ & $f_\mathrm{2}/\si{\centi\meter}$ \\
\midrule
13.3 \pm 0.1 & 37.2 \pm 0.1 & 50.50 \pm 0.14 & 23.90 \pm 0.14 & 9.80 \pm 0.05 \\
12.0 \pm 0.1 & 46.6 \pm 0.1 & 58.60 \pm 0.14 & 34.60 \pm 0.14 & 9.54 \pm 0.06 \\
12.0 \pm 0.1 & 51.6 \pm 0.1 & 63.60 \pm 0.14 & 39.60 \pm 0.14 & 9.74 \pm 0.07 \\
14.4 \pm 0.1 & 30.2 \pm 0.1 & 44.60 \pm 0.14 & 15.80 \pm 0.14 & 9.75 \pm 0.05 \\
12.7 \pm 0.1 & 41.9 \pm 0.1 & 54.60 \pm 0.14 & 29.20 \pm 0.14 & 9.75 \pm 0.06 \\
17.3 \pm 0.1 & 22.3 \pm 0.1 & 39.60 \pm 0.14 & 5.00 \pm 0.14 & 9.74 \pm 0.04 \\
11.8 \pm 0.1 & 52.8 \pm 0.1 & 64.60 \pm 0.14 & 41.00 \pm 0.14 & 9.64 \pm 0.07 \\
12.9 \pm 0.1 & 39.7 \pm 0.1 & 52.60 \pm 0.14 & 26.80 \pm 0.14 & 9.74 \pm 0.06 \\
13.4 \pm 0.1 & 35.2 \pm 0.1 & 48.60 \pm 0.14 & 21.80 \pm 0.14 & 9.71 \pm 0.05 \\
15.1 \pm 0.1 & 27.5 \pm 0.1 & 42.60 \pm 0.14 & 12.40 \pm 0.14 & 9.75 \pm 0.04 \\
\bottomrule
\end{tabular}
\end{table}

Der Mittelwert der Brennweiten $f_\mathrm{i}$ ergibt sich erneut mittels python/numpy \cite{numpy} zu
\begin{gather*}
  f_\mathrm{weiß, 1}= \SI{9.77(3)}{\centi\meter}\text{,}\\
  f_\mathrm{weiß, 2}= \SI{9.71(7)}{\centi\meter}\text{.}
\end{gather*}
\FloatBarrier
\subsubsection{Untersuchung der chromatischen Abberation}
Analog zur Bestimmung der Brennweite der Linse für das Licht der Halogenlampe wird zur Untersuchung der chromatischen Abberration vorgegangen.
\\Die verwendeten Messdaten für blaues Licht befinden sich in Tabelle \ref{tab:blaueins} und \ref{tab:blauzwo}.
\begin{table}
  \caption{Gemessene und berechnete Werte zur Bestimmung der Brennweite für blaues Licht zur Untersuchung der chromatischen Abberration der Linsenposition 1.}
  \label{tab:blaueins}
  \centering
\begin{tabular}{ccccc}
  \toprule
$b_\mathrm{1}/\si{\centi\meter}$ & $g_\mathrm{1}/\si{\centi\meter}$ & $e_\mathrm{1}/\si{\centi\meter}$ & $d_\mathrm{1}/\si{\centi\meter}$ & $f_\mathrm{1}/\si{\centi\meter}$ \\
\midrule
37.2 \pm 0.1 & 13.3 \pm 0.1 & 50.50 \pm 0.14 & -23.90 \pm 0.14 & 9.80 \pm 0.05 \\
46.3 \pm 0.1 & 12.3 \pm 0.1 & 58.60 \pm 0.14 & -34.00 \pm 0.14 & 9.72 \pm 0.06 \\
51.5 \pm 0.1 & 12.1 \pm 0.1 & 63.60 \pm 0.14 & -39.40 \pm 0.14 & 9.80 \pm 0.07 \\
30.2 \pm 0.1 & 14.4 \pm 0.1 & 44.60 \pm 0.14 & -15.80 \pm 0.14 & 9.75 \pm 0.05 \\
41.8 \pm 0.1 & 12.8 \pm 0.1 & 54.60 \pm 0.14 & -29.00 \pm 0.14 & 9.80 \pm 0.06 \\
\bottomrule
\end{tabular}
\end{table}
\begin{table}
\caption{Messdaten zur Bestimmung der Brennweite für blaues Licht bezüglich der Linsenposition 2.}
  \label{tab:blauzwo}
  \centering
\begin{tabular}{ccccc}
  \toprule
$b_\mathrm{2}/\si{\centi\meter}$ & $g_\mathrm{2}/\si{\centi\meter}$ & $e_\mathrm{2}/\si{\centi\meter}$ & $d_\mathrm{2}/\si{\centi\meter}$ & $f_\mathrm{2}/\si{\centi\meter}$ \\
\midrule
13.2 \pm 0.1 & 37.3 \pm 0.1 & 50.50 \pm 0.14 & 24.10 \pm 0.14 & 9.75 \pm 0.05 \\
12.1 \pm 0.1 & 46.5 \pm 0.1 & 58.60 \pm 0.14 & 34.40 \pm 0.14 & 9.60 \pm 0.06 \\
12.1 \pm 0.1 & 51.5 \pm 0.1 & 63.60 \pm 0.14 & 39.40 \pm 0.14 & 9.80 \pm 0.07 \\
14.3 \pm 0.1 & 30.3 \pm 0.1 & 44.60 \pm 0.14 & 16.00 \pm 0.14 & 9.72 \pm 0.05 \\
12.7 \pm 0.1 & 41.9 \pm 0.1 & 54.60 \pm 0.14 & 29.20 \pm 0.14 & 9.75 \pm 0.06 \\
\bottomrule
\end{tabular}
\end{table}
\\Die Mittelwerte der Brennweiten ergeben sich mit python/numpy zu:
\begin{gather*}
  f_\mathrm{blau, 1}= \SI{9.77(3)}{\centi\meter} \text{,}\\
  f_\mathrm{blau, 2}= \SI{9.72(7)}{\centi\meter}\text{.}
\end{gather*}


Für die Untersuchung der Brennweite des roten Lichts finden sich die gemessenen und berechneten Daten in Tabelle \ref{tab:red} und \ref{tab:notblue}.
\begin{table}
  \caption{Messdaten zur Untersuchung der chromatischen Abberration über die Bestimmung der Brennweite nach der Bessel-Methode bei rotem Licht bezüglich Linsenposition 1.}
  \label{tab:red}
  \centering
\begin{tabular}{ccccc}
  \toprule
$b_\mathrm{1}/\si{\centi\meter}$ & $g_\mathrm{1}/\si{\centi\meter}$ & $e_\mathrm{1}/\si{\centi\meter}$ & $d_\mathrm{1}/\si{\centi\meter}$ & $f_\mathrm{1}/\si{\centi\meter}$ \\
\midrule
37.0 \pm 0.1 & 13.5 \pm 0.1 & 50.50 \pm 0.14 & -23.50 \pm 0.14 & 9.89 \pm 0.05 \\
46.1 \pm 0.1 & 12.5 \pm 0.1 & 58.60 \pm 0.14 & -33.60 \pm 0.14 & 9.83 \pm 0.06 \\
51.5 \pm 0.1 & 12.1 \pm 0.1 & 63.60 \pm 0.14 & -39.40 \pm 0.14 & 9.80 \pm 0.07 \\
29.9 \pm 0.1 & 14.7 \pm 0.1 & 44.60 \pm 0.14 & -15.20 \pm 0.14 & 9.85 \pm 0.05 \\
41.7 \pm 0.1 & 12.9 \pm 0.1 & 54.60 \pm 0.14 & -28.80 \pm 0.14 & 9.85 \pm 0.06 \\
\bottomrule
\end{tabular}
\end{table}
\begin{table}
\caption{Messdaten zur Untersuchung der chromatischen Abberration über die Bestimmung der Brennweite nach der Bessel-Methode bei rotem Licht bezüglich Linsenposition 2.}
  \label{tab:notblue}
  \centering
\begin{tabular}{ccccc}
  \toprule
$b_\mathrm{2}/\si{\centi\meter}$ & $g_\mathrm{2}/\si{\centi\meter}$ & $e_\mathrm{2}/\si{\centi\meter}$ & $d_\mathrm{2}/\si{\centi\meter}$ & $f_\mathrm{2}/\si{\centi\meter}$ \\
\midrule
13.4 \pm 0.1 & 37.1 \pm 0.1 & 50.50 \pm 0.14 & 23.70 \pm 0.14 & 9.84 \pm 0.05 \\
12.2 \pm 0.1 & 46.4 \pm 0.1 & 58.60 \pm 0.14 & 34.20 \pm 0.14 & 9.66 \pm 0.06 \\
11.9 \pm 0.1 & 51.7 \pm 0.1 & 63.60 \pm 0.14 & 39.80 \pm 0.14 & 9.67 \pm 0.07 \\
14.5 \pm 0.1 & 30.1 \pm 0.1 & 44.60 \pm 0.14 & 15.60 \pm 0.14 & 9.79 \pm 0.05 \\
12.6 \pm 0.1 & 42.0 \pm 0.1 & 54.60 \pm 0.14 & 29.40 \pm 0.14 & 9.69 \pm 0.06 \\
\bottomrule
\end{tabular}
\end{table}
\\Der Mittelwert der Brennweiten samt dessen Fehler ergibt sich zu:
\begin{gather*}
  f_\mathrm{rot, 1}= \SI{9.85(3)}{\centi\meter}\text{,}\\
  f_\mathrm{rot, 2}= \SI{9.73(7)}{\centi\meter}\text{.}
\end{gather*}



\FloatBarrier
\subsection{Bestimmung der Brennweite eines Linsensystems mit der Methode von Abbe}

Die Messwerte zur Bestimmung der Lage der Hauptebenen des Linsensystems und der Brennweite sind
in Tabelle \ref{tab:abbemess} aufgetragen.

\begin{table}
	\caption{Messwerte zur Bestimmung der Brennweite $f$ und der Lage der Hauptebenen}
	\label{tab:abbemess}
	\centering
	\begin{tabular}{ccc}
		\toprule
		$g'$ / $\si{\centi\meter}$ & $b'$ / $\si{\centi\meter}$ & $V$ \\
		\midrule
		22.30 \pm 0.10 & 86.90 \pm 0.10 & 1.267 \pm 0.033 \\
		31.00 \pm 0.10 & 78.20 \pm 0.10 & 0.933 \pm 0.033 \\
		35.80 \pm 0.10 & 76.20 \pm 0.10 & 0.667 \pm 0.033 \\
		19.90 \pm 0.10 & 92.10 \pm 0.10 & 1.600 \pm 0.033 \\
		17.00 \pm 0.10 & 103.00 \pm 0.10 & 2.233 \pm 0.033 \\
		47.10 \pm 0.10 & 72.90 \pm 0.10 & 0.467 \pm 0.033 \\
		18.40 \pm 0.10 & 96.60 \pm 0.10 & 1.867 \pm 0.033 \\
		40.50 \pm 0.10 & 74.50 \pm 0.10 & 0.567 \pm 0.033 \\
		15.10 \pm 0.10 & 114.90 \pm 0.10 & 2.900 \pm 0.033 \\
		58.70 \pm 0.10 & 71.30 \pm 0.10 & 0.367 \pm 0.033 \\
		\bottomrule
	\end{tabular}
\end{table}

\begin{figure}
  \centering
  \includegraphics[width=0.8\textwidth]{Messdaten/123.pdf}
  \caption{Lineare Ausgleichsrechnung zur Bestimmung der Brennweite des Linsensystems und Lage der ersten Hauptebene.}
  \label{fig:ausgleichd}
\end{figure}

Es werden nach der ersten Formel von \eqref{eqn:abbe} die Gegenstandsweiten $g'$ des Bezugspunkts A
gegen $(1+\frac{1}{V})$ aufgetragen. Diese Zusammenhang ist in Abbildung \ref{fig:ausgleichd}
dargestellt, wobei die Ausgleichsgerade durch eine lineare Ausgleichsrechnung der Form
\begin{equation}
	g' = m \cdot x + n
\end{equation}
bestimmt wird. Die Steigung $m$ der Gleichung \ref{eqn:abbe} entspricht der Brennweite des
Linsensystems und der Achsenabschnitt $n$ dem Abstand $h$ vom Bezugspunkt A zur ersten
Hauptebene.

Die lineare Ausgleichsrechnung mit python ergibt für die Parameter:
\begin{gather*}
	f = \SI{0.181(4)}{\meter} \text{,} \\
	h = -\SI{0.092(4)}{\meter} \text{.}
\end{gather*}
Die erste Hauptebene befindet sich also $h=\SI{9.2(4)}{\centi\meter}$ vor dem Bezugspunkt A.

\begin{figure}
  \centering
  \includegraphics[width=0.8\textwidth]{Messdaten/1234.pdf}
  \caption{Lineare Ausgleichsrechnung zur Bestimmung der Brennweite des Linsensystems und Lage der zweiten Hauptebene.}
  \label{fig:ausgleichd2}
\end{figure}



Die Lage der zweiten Hauptebene wird nach der zweiten Formel von \eqref{eqn:abbe} analog wie
zuvor mit einer linearen Ausgleichsrechnung bestimmt.
Die Ausgleichsgerade samt aufgennommener Messwerte befinden sich in Abbildung \ref{fig:ausgleichd}. Hier wird die Bildweite $b'$ des Bezugspunkts A gegen $(1+V)$ aufgetragen.
Es ergeben sich die Parameter:
\begin{gather*}
	f = \SI{0.173(4)}{\meter} \text{,}  \\
	h = \SI{0.120(4)}{\meter} \text{.}
\end{gather*}
Die zweite Hauptebene liegt also $h=\SI{12.0(4)}{\centi\meter}$ hinter dem Bezugspunkt A.
