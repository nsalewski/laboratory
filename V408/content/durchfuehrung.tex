\section{Durchführung}
\label{sec:Durchführung}

\subsection{Versuchsaufbau}
\label{sec:Versuchsaufbau}
Der Versuchsaufbau besteht aus einem Ultraschallechoskop, an dessen Ausgänge zwei Ultraschallsonden mit \SI{2}{\mega\Hz} gekoppelt sind, und einem Rechner zur Datenaufnahme und -analyse.\\
An das Ultraschallechoskop sind zwei Ultraschallsonden angeschlossen, mithilfe derer sich sowohl eine Impuls-Echo-Messung, als auch eine Durchschallmessung realisieren lässt.
Am Rechner werden die gemessenen Daten mittels des Programms \textquote{Echoview} ausgewertet.\\
Hierbei ist \textquote{Echoview} in der Lage, vier verschiedene Diagramme darzustellen.
Im linken oberen Graphen wird der A-Scan dargestellt, also die Amplitude gegen die Zeit aufgetragen.
Der linke untere Graph stellt die gewählte Verstärkung dar. Die Verstärkung lässt sich am Ultraschallechoskop über die Drehknöpfe zur laufzeit-bzw. tiefenabhängigen Verstärkung (TGC; Time Gain Control) und ebenso über die Verstärkung des Outputs und der Empfindlichkeit der Sonden regulieren.\\
Zu Beachten ist, dass eine Verstärkung nur gewählt werden darf, wenn die auszuwertende Messreihe nicht zur Untersuchung der Amplitudenhöhe dient.
Die beiden rechten Graphiken sind das berechnete Spektrum der Messdaten (FFT), bzw. ihr
Cepstrum.
Erzeugte Graphiken und Messdaten können aus dem Programm heraus exportiert werden.
Als zu untersuchende Versuchsobjekte stehen Acrylzylinder verschiedener Länge, Acrylplatten unterschiedlicher Dicke sowie das Modell eines menschlichen Auges im Maßstab 3:1 zur Verfügung.


\subsection{Versuchsbeschreibung}
\label{sec:Versuchsbeschreibung}


\subsubsection{Messung zur Verifizierung der Gleichungen}

Für die Verifizierung der Gleichungen \eqref{eqn:abbi} und \eqref{eqn:linsi} wird der
Aufbau wie in Abschnitt \ref{sec:gundb} mit einer Sammellinse der Brennweite
\SI{100}{\milli\meter} verwendet.\\
Zu Beginn wird die Gegenstandgröße $G$ - die Größe des "Perl L"s - mit dem Lineal gemessen.
Daraufhin werden für zehn Gegenstandsweiten $g$ die Bildweiten $b$ bestimmt, bei denen auf dem
Schirm ein scharfes reelles Bild zu erkennen ist. Hierfür soll der Schirm verschoben werden.
Des Weiteren werden für die ersten fünf Messungen außerdem die Bildgrößen auf dem Schirm mit
einem Lineal bestimmt. Es werden also zehn Wertepaare $(g_{\mathrm{i}},b_{\mathrm{i}})$ und
fünf Bildgrößen notiert.

\subsubsection{Messung für die unbekannte Sammellinse}

Die Messung der Brennweite der mit Wasser gefüllten Linse erfolgt analog zur vorherigen
Messung. Es werden allerdings nur zehn Wertepaare $(g_{\mathrm{i}},b_{\mathrm{i}})$
benötigt.\\
Hierbei ist zu beachten, dass ein konstanter Druck auf die Spritze ausgeübt wird,
damit das Wasservolumen in der Linse konstant bleibt.

\subsubsection{Messung für die Methode nach Bessel}
\begin{figure}
  \centering
  \includegraphics[width=0.6\textwidth]{Bilder/Bessel.png}
  \caption{Schematische Darstellung der beiden Linsenpositionen an denen ein scharfes Bild auf dem Schirm entsteht zur Bestimmung der Brennweite nach der bessel-Methode.}
  \label{fig:besselmess}
\end{figure}
Bei der Messung für die Methode nach Bessel wird wieder die Linse mit der Brennweite
$f=\SI{100}{\milli\meter}$ angebracht.
Dann wird der Abstand zwischen Gegenstand und Schirm festgehalten und die Linse solange variiert
bis beide Punkte bestimmt sind (vgl. Abbildung \ref{fig:besselmess}) , an denen ein scharfes Bild am Schirm zu sehen ist.
Für diese beiden Punkte werden jeweils die Gegenstandsweite $g_{1,2}$ und die Bildweite
$b_{1,2}$ notiert. Diese Messung soll für zehn verschiedene Abstände von Gegenstand und Schirm
durchgeführt werden.\\
Da außerdem die chromatische Abberation untersucht werden soll, werden bei den ersten fünf
Messungen jeweils ein Rot- und ein Blaufilter angebracht und die Messung wiederholt.

\subsubsection{Messung für die Methode nach Abbe}

Für die Messung nach der Methode von Abbe wird vor der Sammellinse die Zerstreuungslinse mit
Brennweite $f_{\mathrm{Z}}=-\SI{100}{\milli\meter}$ - wie in Abschnitt \ref{sec:abbe}
beschrieben - angebracht. Der Punkt A wird als Berührungspunkt der beiden Reiter, auf denen
sich die Linsen befinden, festgelegt und die Hilfsgegenstandsweite $g'$  und Hilfsbildweite $b'$ wieder zehnmal
gemessen bis ein scharfes Bild auf dem Schirm zu erkennen ist.
