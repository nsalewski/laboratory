\FloatBarrier
\section{Diskussion}
\label{sec:Diskussion}
Beim Vergleich aller bestimmter Brennweiten (vgl. Tabelle \ref{tab:diskus}) fällt auf, dass diese dauerhaft um etwa $3\%$ nach unten abweichen.
\\Dies liegt vermutlich daran, dass sich meist nur relativ ungenau die exakte Position bestimmen ließ, bei der das Bild auf dem Schirm am schärfsten ist.
\\Die Übereinstimmung unter den Messungen und die allgemein doch recht kleine Abweichung zur Herstellerangabe zeigen, dass in allen Messungen wahrscheinlich ein annähernd gleich scharfes Bild als schärfstes Bild angenommen wurde, welches nur wenig neben dem schärfsten Bild mit der Brennweite nach Herstellerangaben liegt.

\begin{table}
  \centering
  \caption{Vergleich aller experimentell bestimmter Brennweiten mit der Herstellerangabe.}
  \label{tab:diskus}
  \begin{tabular}{cccc}
    \toprule
    &Experiment/$\si{\centi\meter}$&Herstellerangabe/$\si{\centi\meter}$&Abweichung\\
    \midrule
    $f_{\mathrm{abgelesen}}$&$\num{9.70(10)}$&$\num{10(0)}$&$3.0\%$\\
    $f_{\mathrm{Linsengleichung}}$&$\num{9.70(9)}$&$\num{10(0)}$&$3.0\%$\\
    $f_\mathrm{Bessel, 1}$&$\num{9.77(3)}$&$\num{10(0)}$&$2.3\%$\\
    $f_\mathrm{Bessel, 2}$&$\num{9.71(7)}$&$\num{10(0)}$&$2.9\%$\\
\bottomrule
\end{tabular}
\end{table}

\begin{table}
  \centering
  \caption{Ergebnisse für die Brennweite der mit Wasser gefüllten Linse.}
  \label{tab:diskus}
  \begin{tabular}{cc}
    \toprule
	  & Brennweite / \si{\centi\meter} \\
    \midrule
    $f_{\mathrm{abgelesen}}$&$\num{7.9(1)}$ \\
	  $f_{\mathrm{Linsengleichung}}$ & \num{7.5(1)} \\
\bottomrule
\end{tabular}
\end{table}


Bei der Untersuchung der chromatischen Abberration (vgl. Tabelle \ref{tab:chroma}) war erwartet worden, dass die Brennweite für blaues Licht etwas kürzer ist als die für weißes Licht und die Brennweite für rotes Licht etwas länger als die des weißen ist.\\
Allerdings zeigt sich lediglich für das rote Licht, besonders wenn die berechnete Brennweite $f_\mathrm{1}$ betrachtet wird, einigermaßen deutlich das erwartete Ergebnis.
\\Die Brennweite für das blaue und das weiße Licht wurde hingegen als nahezu identisch gemessen.
\\Eventuell sind im Licht der Halogenlampe auch recht große Anteile kurzwelligen Lichts enthalten, sodass die Brennweiten für blaues Licht und für das ungefilterte Licht der Halogenlampe nahezu gleich sind.
\\ Zudem lässt sich über alle drei Messungen beobachten, dass die ermittelten Brennweiten $f_\mathrm{2}$ etwas kürzer ausfallen. \\Möglicherweise hängt dies mit Ablesefehlern zusammen, da bei großer Gegenstandsweite im Verhältnis zu kleiner Bildweite der Abbildungsmaßstab kleiner als eins ist, wird das Bild immer kleiner und es lässt sich schwieriger ein eindeutig scharfes Bild bestimmen.
\\Zudem war es, wie in den anderen Messreihen auch, recht schwierig, eine eindeutige Position für die Linse zu finden, sodass das Bild scharf wird.

\begin{table}
  \centering
  \caption{Vergleich der bestimmten Brennweiten für verschiedenfarbiges Licht zur Untersuchung der chromatischen Abberration.}
  \label{tab:chroma}
  \begin{tabular}{ccc}
    \toprule
    &Brennweite bzgl. Linsenpos. 1 $f_\mathrm{1}$ /$\si{\centi\meter}$&Brennweite bzgl. Linsenpos. 2 $f_\mathrm{2}$/$\si{\centi\meter}$\\
    \midrule
    $f_{\mathrm{blau}}$&$\num{9.77(3)}$&$\num{9.72(7)}$\\
    $f_{\mathrm{weiß}}$&$\num{9.77(3)}$&$\num{9.71(7)}$\\
    $f_{\mathrm{rot}}$&$\num{9.85(3)}$&$\num{9.73(7)}$\\
\bottomrule
\end{tabular}
\end{table}

\begin{table}
  \centering
  \caption{Ergebnisse für die Methode nach Abbe.}
  \label{tab:diskus}
  \begin{tabular}{cc}
    \toprule
	  & Brennweite bzw. Position der Hauptebenen / \si{\centi\meter} \\
    \midrule
	$f_{1}$ & $\num{18.1(4)}$ \\
	$f_{2}$ & $\num{17.3(4)}$ \\
	$h_1$ & $-\num{9.2(4)}$ \\
	$h_2$ & $\num{12.0(4)}$ \\
\bottomrule
\end{tabular}
\end{table}



