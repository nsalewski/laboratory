\section{Diskussion}
\label{sec:Diskussion}
\begin{equation}
f_{\mathrm{abgelesen}}=\SI{9.7(1)}{\centi\meter}\text{.}
\end{equation}

\begin{equation}
f_{\mathrm{Linsengleichung}}=\SI{9.70(9)}{\centi\meter}\text{.}
\end{equation}
\begin{table}
  \caption{title}
  \label{tab:diskus}
  \begin{tabular}{cccc}
    \toprule
    &Experiment&Theorie&Abweichung\\
    $f_{\mathrm{abgelesen}}$&$\SI{9.7(1)}{\centi\meter}$&&\\
    $f_{\mathrm{Linsengleichung}}$&$\SI{9.70(9)}{\centi\meter}$&&\\
    $f_\mathrm{Bessel, 1}$&$\SI{9.77(3)}{\centi\meter}$&&\\
    $f_\mathrm{Bessel, 2}$&$\SI{9.71(7)}{\centi\meter}$&&\\
\end{tabular}
\end{table}


Bei der Untersuchung der chromatischen Abberration war erwartet worden, dass die Brennweite für blaues Licht etwas kürzer ist als die für weißes Licht und die Brennweite für rotes Licht etwas länger als die des weißen ist.\\
Allerdings zeigt sich lediglich für das rote Licht, besonders wenn lediglich die berechnete Brennweite $F_\mathrm{1}$ betrachtet wird, einigermaßen deutlich das erwartete Ergebnis.
\\Die Brennweite für das blaue und das weiße Licht wurde hingegen als nahezu identisch gemessen.
\\Eventuell ist sind im Licht der Halogenlampe auch recht große Anteile kurzwelligen Lichts enthalten, sodass die Brennweiten für blaues Licht und für das Licht der Halogenlampe nahezu gleich sind.
\\ Zudem lässt sich über alle drei Messungen beobachtet, dass die ermittelten Brennweiten $f_\mathrm{2}$ etwas kürzer ausfallen. \\Möglicherweise hängt dies mit Ablesefehlern zusammen, da bei großer Gegenstandsweite im Verhältnis zu kleiner Bildweite der Abbildungsmaßstab kleiner als 1 ist, wird das Bild immer kleiner und es lässt sich schwieriger ein eindeutig scharfes Bild bestimmen.
\\Zudem war es, wie in den anderen Messreihen auch, recht schwierig, einen eindeutige Position für die Linse zu finden, sodass das Bild scharf wird.

\begin{gather*}
f_\mathrm{weiß, 1}= \SI{9.77(3)}{\centi\meter}\text{,}\\
f_\mathrm{weiß, 2}= \SI{9.71(7)}{\centi\meter}\text{.}
\end{gather*}

\begin{gather*}
f_\mathrm{blau, 1}= \SI{9.77(3)}{\centi\meter} \text{,}\\
f_\mathrm{blau, 2}= \SI{9.72(7)}{\centi\meter}\text{.}
\end{gather*}

\begin{gather*}
f_\mathrm{rot, 1}= \SI{9.85(3)}{\centi\meter}\text{,}\\
f_\mathrm{rot, 2}= \SI{9.73(7)}{\centi\meter}\text{.}
\end{gather*}
