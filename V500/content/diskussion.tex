\section{Diskussion}
\label{sec:Diskussion}
Bei der Bestimmung der Grenzspannung zeigt sich für alle vermessenen Spektrallinien eine hohe Übereinstimmung mit dem $\sqrt{I}$-Gesetz. Nur wenige Messpunkte weichen minimal von den jeweiligen Regressionsgraden ab.
Ein Vergleich des experimentell bestimmten Verhältnis $\frac{h}{e_0}$ mit dem Theoriewert nach \cite{h} und \cite{e} zeigt zudem nur eine relativ geringe Abweichung von knapp $4\%$.
\begin{gather*}
  {\frac{h}{e_0}}_\mathrm{Experiment}=\SI{4.3(3)e-15}{\electronvolt}\\
  {\frac{h}{e_0}}_\mathrm{Literaturwert}=\SI{4.13566766(6)e-15}{\electronvolt}
\end{gather*}
Während der Messung viel auf, dass bei jedem Wechsel der Skala am Pikoamperemeter große Schwankungen im gemessenen Photostrom festgestellt wurden. Selbst wenn die Skala direkt anschließend wieder zurückgestellt wurde, wurden Werte gemessen, die zum Teil um $50\%$ von den zuvor gemessenen Werten abwichen. Wenn etwas gewartet wurde, stellte sich meist wieder der zuvor gemessene Wert ein.\\
Eine weitere mögliche Fehlerquelle sind besonders in der Messung der gelben Spektrallinie unterschiedlich große Lichtintensitäten an der Photozelle. Bereits Unruhe am Nachbartisch oder eine kleine Änderung der Beleuchtungssituation des Raumes reichten aus, um den Zeiger des Picoamperemeters schwanken zu lassen.\\
Daher war es zum Teil schwierig, den Photostrom exakt abzulesen.
