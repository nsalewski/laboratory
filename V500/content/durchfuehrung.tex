\section{Durchführung}
\label{sec:Durchführung}

\subsection{Versuchsaufbau}
\label{sec:Versuchsaufbau}
Der Versuchsaufbau besteht aus einem Ultraschallechoskop, an dessen Ausgänge zwei Ultraschallsonden mit \SI{2}{\mega\Hz} gekoppelt sind, und einem Rechner zur Datenaufnahme und -analyse.\\
An das Ultraschallechoskop sind zwei Ultraschallsonden angeschlossen, mithilfe derer sich sowohl eine Impuls-Echo-Messung, als auch eine Durchschallmessung realisieren lässt.
Am Rechner werden die gemessenen Daten mittels des Programms \textquote{Echoview} ausgewertet.\\
Hierbei ist \textquote{Echoview} in der Lage, vier verschiedene Diagramme darzustellen.
Im linken oberen Graphen wird der A-Scan dargestellt, also die Amplitude gegen die Zeit aufgetragen.
Der linke untere Graph stellt die gewählte Verstärkung dar. Die Verstärkung lässt sich am Ultraschallechoskop über die Drehknöpfe zur laufzeit-bzw. tiefenabhängigen Verstärkung (TGC; Time Gain Control) und ebenso über die Verstärkung des Outputs und der Empfindlichkeit der Sonden regulieren.\\
Zu Beachten ist, dass eine Verstärkung nur gewählt werden darf, wenn die auszuwertende Messreihe nicht zur Untersuchung der Amplitudenhöhe dient.
Die beiden rechten Graphiken sind das berechnete Spektrum der Messdaten (FFT), bzw. ihr
Cepstrum.
Erzeugte Graphiken und Messdaten können aus dem Programm heraus exportiert werden.
Als zu untersuchende Versuchsobjekte stehen Acrylzylinder verschiedener Länge, Acrylplatten unterschiedlicher Dicke sowie das Modell eines menschlichen Auges im Maßstab 3:1 zur Verfügung.


\subsection{Versuchsbeschreibung}
\label{sec:Versuchsbeschreibung}
Vor Beginn der Messung muss ein möglichst scharfes Bild am Eintrittsspalt der Photozelle erzeugt werden.\\
Dazu wird direkt vor dem Eintrittsspalt eine Mattscheibe angebracht und die Positionen der optischen Elemente zueinander so variiert, dass ein scharfes Bild möglichst großer Intensität entsteht.\\
Mithilfe des Schwenkarms wird die Photozelle so ausgerichtet, dass nur monochromatisches Licht, also einfarbiges Licht mit nur einer Wellenlänge, auf den Eintrittsspalt der Photozelle fällt.\\
Die anliegende variable Gegenspannung wird schrittweise erhöht, bis kein Photostrom mehr am Pikometer abgelesen wird. Es werden etwa 10-15 Datenpaare aus angelegter Gegenspannung und Photostrom notiert.\\
Ebenso wird mit allen Spektrallinien verfahren, welche zu Beginn auf der Mattscheibe sichtbar waren.\\
Für niedrige Photoenergien, also recht langwelliges Licht, kann es nötig sein, ebenso beschleunigte Spannungen anzulegen, um eine genügend große Anzahl an Datenpaaren aus Spannung und Photostrom aufnehmen zu können.\\
Anschließend wird für die Spektrallinie $\lambda=\SI{578}{\nano\meter}$, also gelbes Licht, sowohl mit beschleunigendem als auch bremsenden Potential der Photostrom gemessen.
Die angelegte Spannung wird hierfür von $U=\SI{+20}{\volt}$ als beschleunigendes Potential bis zum bremsenden Potential, bei welchem der Photostrom verschwindet, gemessen.
Es werden etwa 40 Datenpaare aus anliegendem Potential und dem Photostrom notiert.
