\section{Auswertung}
\label{sec:Auswertung}

\subsection{Bestimunng Ug}
violett

\begin{table}
  \centering
  \caption{Messdaten zur Bestimmung der Grenzspannung $U_\mathrm{g}$.}
  \label{tab:ug}
\begin{tabular}{cccc}
  \toprule
    &$a\cdot 10^{-5}$/$\frac{\sqrt{\si{\ampere}}}{\si{\volt}}$& $b\cdot 10^{-5}$/$\sqrt{\si{\ampere}}$&$U_\mathrm{g}$/$\si{\volt}$\\
\midrule
Violett&(-4.76  \pm 0.07) &(6.31  \pm 0.06) & 1.324  \pm 0.023\\
Violett 2&(-5.11  \pm 0.05) &(8.61  \pm 0.05) & 1.686  \pm 0.018\\
Violett 3& (-6.91  \pm 0.09) &(7.82  \pm 0.06) & 1.133  \pm 0.017\\
Blaugrün&(-1.64  \pm 0.06) &(1.484  \pm 0.029) &0.90  \pm 0.04\\
Grün&(-8.50  \pm 0.18) &(4.22  \pm 0.05) & 0.496  \pm 0.012\\
Gelb& (-6.30  \pm 0.22) &(2.23  \pm 0.04) & 0.354  \pm 0.014\\
\bottomrule
\end{tabular}
\end{table}

\begin{table}
  \centering
  \caption{Messdaten zur Bestimmung der Grenzspannung $U_\mathrm{g}$ bei der ersten, ultravioletten Spektrallinie.}
  \label{tab:uguv}
\begin{tabular}{ccc}
\toprule
$U$/$\si{\volt}$ & $I\cdot 10^{-9}$/$\si{\ampere}$ & $sqrt(I)\cdot 10^(-5)$/$\sqrt{\si{\ampere}}$ \\
\midrule
1.33 & 0.000  \pm 0.010 & 0.0  \pm nan \\
1.2 & 0.060  \pm 0.010 & 0.77  \pm 0.06 \\
1.1 & 0.120  \pm 0.010 & 1.10  \pm 0.05 \\
1.0 & 0.240  \pm 0.010 & 1.549  \pm 0.032 \\
0.9 & 0.380  \pm 0.010 & 1.949  \pm 0.026 \\
0.8 & 0.540  \pm 0.010 & 2.324  \pm 0.022 \\
0.7 & 0.760  \pm 0.010 & 2.757  \pm 0.018 \\
0.6 & 1.20  \pm 0.10 & 3.46  \pm 0.14 \\
0.5 & 1.60  \pm 0.10 & 4.00  \pm 0.12 \\
0.4 & 2.00  \pm 0.10 & 4.47  \pm 0.11 \\
0.3 & 2.40  \pm 0.10 & 4.90  \pm 0.10 \\
0.2 & 2.80  \pm 0.10 & 5.29  \pm 0.09 \\
0.1 & 3.40  \pm 0.10 & 5.83  \pm 0.09 \\
0.02 & 4.00  \pm 0.10 & 6.32  \pm 0.08 \\
\bottomrule
\end{tabular}
\end{table}

\begin{figure}
  \centering
  \caption{title}
  \label{fig:jakdfn}
  \includegraphics{Bilder/violett.pdf}
\end{figure}





violett2

\begin{table}
  \centering
  \caption{Messdaten zur Bestimmung der Grenzspannung $U_\mathrm{g}$ bei der zweiten violetten Spektrallinie.}
  \label{tab:ugv2}
\begin{tabular}{ccc}
\toprule
$U$/$\si{\volt}$ & $I\cdot 10^{-9}$/$\si{\ampere}$ & $sqrt(I)\cdot 10^(-5)$/$\sqrt{\si{\ampere}}$ \\
\midrule
1.68 & 0.000  \pm 0.010 & 0.0  \pm nan \\
1.5 & 0.120  \pm 0.010 & 1.10  \pm 0.05 \\
1.4 & 0.220  \pm 0.010 & 1.483  \pm 0.034 \\
1.3 & 0.380  \pm 0.010 & 1.949  \pm 0.026 \\
1.2 & 0.600  \pm 0.010 & 2.449  \pm 0.020 \\
1.1 & 0.880  \pm 0.010 & 2.966  \pm 0.017 \\
1.0 & 1.10  \pm 0.10 & 3.32  \pm 0.15 \\
0.9 & 1.60  \pm 0.10 & 4.00  \pm 0.12 \\
0.8 & 2.10  \pm 0.10 & 4.58  \pm 0.11 \\
0.7 & 2.60  \pm 0.10 & 5.10  \pm 0.10 \\
0.5 & 3.70  \pm 0.10 & 6.08  \pm 0.08 \\
0.3 & 5.00  \pm 0.10 & 7.07  \pm 0.07 \\
0.15 & 6.00  \pm 0.10 & 7.75  \pm 0.06 \\
0.02 & 7.40  \pm 0.10 & 8.60  \pm 0.06 \\
\bottomrule
\end{tabular}
\end{table}



\begin{figure}
  \centering
  \caption{title}
  \label{fig:jakdeffn}
  \includegraphics{Bilder/ultraviolett.pdf}
\end{figure}



violett3

\begin{table}
  \centering
  \caption{Messdaten zur Bestimmung der Grenzspannung $U_\mathrm{g}$ bei der dritten violetten Spektrallinie.}
  \label{tab:ugv2}
\begin{tabular}{ccc}
\toprule
$U$/$\si{\volt}$ & $I\cdot 10^{-9}$/$\si{\ampere}$ & $sqrt(I)\cdot 10^(-5)$/$\sqrt{\si{\ampere}}$ \\
\midrule
1.12 & 0.000  \pm 0.010 & 0.0  \pm nan \\
1.1 & 0.020  \pm 0.010 & 0.45  \pm 0.11 \\
1.0 & 0.100  \pm 0.010 & 1.00  \pm 0.05 \\
0.9 & 0.240  \pm 0.010 & 1.549  \pm 0.032 \\
0.8 & 0.460  \pm 0.010 & 2.145  \pm 0.023 \\
0.7 & 0.860  \pm 0.010 & 2.933  \pm 0.017 \\
0.6 & 1.20  \pm 0.10 & 3.46  \pm 0.14 \\
0.5 & 2.00  \pm 0.10 & 4.47  \pm 0.11 \\
0.4 & 2.60  \pm 0.10 & 5.10  \pm 0.10 \\
0.35 & 3.00  \pm 0.10 & 5.48  \pm 0.09 \\
0.3 & 3.40  \pm 0.10 & 5.83  \pm 0.09 \\
0.2 & 4.20  \pm 0.10 & 6.48  \pm 0.08 \\
0.1 & 5.00  \pm 0.10 & 7.07  \pm 0.07 \\
0.02 & 5.90  \pm 0.10 & 7.68  \pm 0.07 \\
\bottomrule
\end{tabular}
\end{table}


\begin{figure}
  \centering
  \caption{title}
  \label{fig:jakdwewwfn}
  \includegraphics{Bilder/violett_drittkleinste.pdf}
\end{figure}


blaugrün

\begin{table}
\centering
\caption{Messdaten zur Bestimmung der Grenzspannung $U_\mathrm{g}$ bei der vierten blaugrünen Spektrallinie.}
\label{tab:ugbg}
\begin{tabular}{ccc}
\toprule
$U$/$\si{\volt}$ & $I\cdot 10^{-9}$/$\si{\ampere}$ & $sqrt(I)\cdot 10^(-5)$/$\sqrt{\si{\ampere}}$ \\
\midrule
0.9 & 0.000  \pm 0.010 & 0.0  \pm nan \\
0.8 & 0.010  \pm 0.010 & 0.32  \pm 0.16 \\
0.7 & 0.010  \pm 0.010 & 0.32  \pm 0.16 \\
0.6 & 0.020  \pm 0.010 & 0.45  \pm 0.11 \\
0.55 & 0.030  \pm 0.010 & 0.55  \pm 0.09 \\
0.5 & 0.040  \pm 0.010 & 0.63  \pm 0.08 \\
0.45 & 0.050  \pm 0.010 & 0.71  \pm 0.07 \\
0.4 & 0.060  \pm 0.010 & 0.77  \pm 0.06 \\
0.35 & 0.080  \pm 0.010 & 0.89  \pm 0.06 \\
0.3 & 0.100  \pm 0.010 & 1.00  \pm 0.05 \\
0.25 & 0.120  \pm 0.010 & 1.10  \pm 0.05 \\
0.2 & 0.140  \pm 0.010 & 1.18  \pm 0.04 \\
0.1 & 0.180  \pm 0.010 & 1.34  \pm 0.04 \\
0.02 & 0.220  \pm 0.010 & 1.483  \pm 0.034 \\
\bottomrule
\end{tabular}
\end{table}



\begin{figure}
  \centering
  \caption{title}
  \label{fig:jakdfewwn}
  \includegraphics{Bilder/blaugrün.pdf}
\end{figure}

grün
\begin{table}
\centering
\caption{Messdaten zur Bestimmung der Grenzspannung $U_\mathrm{g}$ bei der fünften, grünen Spektrallinie.}
\label{tab:uggruen}
\begin{tabular}{ccc}
\toprule
$U$/$\si{\volt}$ & $I\cdot 10^{-9}$/$\si{\ampere}$ & $sqrt(I)\cdot 10^(-5)$/$\sqrt{\si{\ampere}}$ \\
\midrule
-0.6 & 6.60  \pm 0.10 & 8.12  \pm 0.06 \\
-0.5 & 5.80  \pm 0.10 & 7.62  \pm 0.07 \\
-0.4 & 5.50  \pm 0.10 & 7.42  \pm 0.07 \\
-0.3 & 4.60  \pm 0.10 & 6.78  \pm 0.07 \\
-0.2 & 3.60  \pm 0.10 & 6.00  \pm 0.08 \\
-0.1 & 2.80  \pm 0.10 & 5.29  \pm 0.09 \\
0.02 & 1.50  \pm 0.10 & 3.87  \pm 0.13 \\
0.1 & 1.000  \pm 0.010 & 3.162  \pm 0.016 \\
0.2 & 0.620  \pm 0.010 & 2.490  \pm 0.020 \\
0.3 & 0.240  \pm 0.010 & 1.549  \pm 0.032 \\
0.35 & 0.140  \pm 0.010 & 1.18  \pm 0.04 \\
0.4 & 0.070  \pm 0.010 & 0.84  \pm 0.06 \\
0.45 & 0.040  \pm 0.010 & 0.63  \pm 0.08 \\
0.5 & 0.000  \pm 0.010 & 0.0  \pm nan \\
\bottomrule
\end{tabular}
\end{table}


\begin{figure}
  \centering
  \caption{title}
  \label{fig:jakfedfewwn}
  \includegraphics{Bilder/grün.pdf}
\end{figure}

gelb
\begin{table}
\centering
\caption{Messdaten zur Bestimmung der Grenzspannung $U_\mathrm{g}$ bei der sechsten, gelben Spektrallinie.}
\label{tab:ugg}
\begin{tabular}{ccc}
\toprule
$U$/$\si{\volt}$ & $I\cdot 10^{-9}$/$\si{\ampere}$ & $sqrt(I)\cdot 10^(-5)$/$\sqrt{\si{\ampere}}$ \\
\midrule
-19.0 & 15.00  \pm 0.10 & 12.25  \pm 0.04 \\
-17.5 & 14.00  \pm 0.10 & 11.83  \pm 0.04 \\
-16.0 & 14.00  \pm 0.10 & 11.83  \pm 0.04 \\
-14.5 & 14.00  \pm 0.10 & 11.83  \pm 0.04 \\
-13.0 & 14.00  \pm 0.10 & 11.83  \pm 0.04 \\
-11.5 & 13.00  \pm 0.10 & 11.40  \pm 0.04 \\
-10.0 & 12.00  \pm 0.10 & 10.95  \pm 0.05 \\
-9.0 & 12.00  \pm 0.10 & 10.95  \pm 0.05 \\
-8.0 & 12.00  \pm 0.10 & 10.95  \pm 0.05 \\
-7.0 & 11.00  \pm 0.10 & 10.49  \pm 0.05 \\
-6.0 & 10.00  \pm 0.10 & 10.00  \pm 0.05 \\
-5.0 & 9.00  \pm 0.10 & 9.49  \pm 0.05 \\
-4.0 & 8.40  \pm 0.10 & 9.17  \pm 0.05 \\
-3.0 & 6.90  \pm 0.10 & 8.31  \pm 0.06 \\
-2.5 & 6.00  \pm 0.10 & 7.75  \pm 0.06 \\
-2.0 & 5.10  \pm 0.10 & 7.14  \pm 0.07 \\
-1.8 & 5.00  \pm 0.10 & 7.07  \pm 0.07 \\
-1.6 & 4.80  \pm 0.10 & 6.93  \pm 0.07 \\
-1.4 & 4.20  \pm 0.10 & 6.48  \pm 0.08 \\
-1.2 & 3.80  \pm 0.10 & 6.16  \pm 0.08 \\
-1.1 & 3.50  \pm 0.10 & 5.92  \pm 0.08 \\
-1.0 & 3.20  \pm 0.10 & 5.66  \pm 0.09 \\
-0.9 & 3.10  \pm 0.10 & 5.57  \pm 0.09 \\
-0.8 & 2.90  \pm 0.10 & 5.39  \pm 0.09 \\
-0.7 & 2.70  \pm 0.10 & 5.20  \pm 0.10 \\
-0.6 & 2.40  \pm 0.10 & 4.90  \pm 0.10 \\
-0.5 & 2.00  \pm 0.10 & 4.47  \pm 0.11 \\
-0.4 & 1.80  \pm 0.10 & 4.24  \pm 0.12 \\
-0.3 & 1.40  \pm 0.10 & 3.74  \pm 0.13 \\
-0.2 & 1.10  \pm 0.10 & 3.32  \pm 0.15 \\
-0.1 & 0.900  \pm 0.010 & 3.000  \pm 0.017 \\
-0.01 & 0.600  \pm 0.010 & 2.449  \pm 0.020 \\
0.01 & 0.500  \pm 0.010 & 2.236  \pm 0.022 \\
0.1 & 0.240  \pm 0.010 & 1.549  \pm 0.032 \\
0.15 & 0.140  \pm 0.010 & 1.18  \pm 0.04 \\
0.2 & 0.080  \pm 0.010 & 0.89  \pm 0.06 \\
0.25 & 0.040  \pm 0.010 & 0.63  \pm 0.08 \\
0.3 & 0.020  \pm 0.010 & 0.45  \pm 0.11 \\
0.35 & 0.000  \pm 0.010 & 0.0  \pm nan \\
\bottomrule
\end{tabular}
\end{table}




\begin{figure}
  \centering
  \caption{title}
  \label{fig:jakfedfewwn}
  \includegraphics{Bilder/gelb.pdf}
\end{figure}
