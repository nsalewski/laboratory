\section{Auswertung}
\label{sec:Auswertung}

\subsection{Bestimmung der Grenzspannung $U_\mathrm{G}$}
Zunächst wird die Grenzspannung $U_\mathrm{G}$ für die jeweiligen Spektrallinien bestimmt. \\Dazu wird die Wurzel des gemessenen Stroms $I$ gegen die zugehörige Bremsspannung $U_\mathrm{B}$ aufgetragen.\\
Es wird ein linearer Zusammenhang angenommen, daher wird mittels scipy/python \cite{scipy} eine lineare Ausgleichsrechnung durchgeführt.
Die Ausgleichgerade ist gegeben durch
\begin{equation}
  \label{eqn:ausgleich}
y=a \cdot x +b \text{.}
\end{equation}
Die Grenzspannung $U_\mathrm{G}$ ergibt sich aus dem Schnittpunkt der Ausgleichsgerade mit der $U_\mathrm{B}$-Achse, also aus dem Verhältnis $U_\mathrm{G}=\frac{-b}{a}$.
\\In den Tabellen \ref{tab:uguv} bis \ref{tab:ugg} sind die gemessenen Bremsspannungen $U_\mathrm{B}$ sowie die Photoströme $I$ samt der zugehörigen Wurzeln der Photoströme $\sqrt{I}$ aufgetragen.\\
In den Abbildungen \ref{fig:uguv} bis \ref{fig:ugg} wurde jeweils die Wurzel des Photostroms $\sqrt{I}$ gegen die Bremsspannung $U_\mathrm{B}$ aufgetragen, sowie, wie zuvor beschrieben, eine Regressionsgrade \ref{eqn:ausgleich} eingezeichnet.\\
Die verschieden großen Unsicherheiten ergeben sich hierbei durch die jeweils verwendete Messskala zur Messung des Photostroms. \\Der Ablesefehler wurde jeweils auf den halben Abstand zwischen zwei Markierungen der Messskala geschätzt. \\Der Ablesefehler wird daher auch in den Abbildungen als verschieden hohe Fehlerbalken berücksichtigt.


Für große Wellenlängen (grüne und gelbe Spektrallinie) mussten zum Teil beschleunigende Potentiale angelegt werden, um eine ausreichende Anzahl an Messwerten aufnehmen zu können. Der lineare Zusammenhang zwischen der Wurzel des Photostroms und der Bremsspannung gilt nicht für große beschleunigende Potentiale.\\
Für die lineare Regression wurden daher bei der grünen Spektrallinie die ersten drei Datentupel ebenso wenig berücksichtigt, wie für die gelbe Spektrallinie lediglich die letzten zehn Datentupel verwendet wurden.\\


In Tabelle \ref{tab:ug} sind errechneten Geradenparameter der Regressionsgraden sowie die daraus bestimmten Grenzspannungen $U_\mathrm{G}$ zu den jeweiligen Spektrallinien eingetragen.\\
\begin{table}
  \centering
  \caption{Messdaten zur Bestimmung der Grenzspannung $U_\mathrm{G}$.}
  \label{tab:ug}
  \begin{tabular}{cccc}
    \toprule
    Farbe d. Spektrallinie&$a\cdot 10^{-5}$/$\frac{\sqrt{\si{\ampere}}}{\si{\volt}}$& $b\cdot 10^{-5}$/$\sqrt{\si{\ampere}}$&$U_\mathrm{G}$/$\si{\volt}$\\
    \midrule
    Violett 1&(-4.76  \pm 0.07) &(6.31  \pm 0.06) & 1.32  \pm 0.02\\
    Violett 2&(-5.11  \pm 0.05) &(8.61  \pm 0.05) & 1.69  \pm 0.02\\
    Violett 3& (-6.91  \pm 0.09) &(7.82  \pm 0.06) & 1.13  \pm 0.02\\
    Blaugrün&(-1.64  \pm 0.06) &(1.48  \pm 0.03) &0.90  \pm 0.04\\
    Grün&(-8.5  \pm 0.2) &(4.22  \pm 0.05) & 0.50  \pm 0.01\\
    Gelb& (-6.3  \pm 0.2) &(2.23  \pm 0.04) & 0.35  \pm 0.01\\
    \bottomrule
  \end{tabular}
\end{table}



\begin{table}
  \centering
  \caption{Messdaten zur Bestimmung der Grenzspannung $U_\mathrm{G}$ bei der ersten, violetten Spektrallinie.}
  \label{tab:uguv}
  \begin{tabular}{ccc}
    \toprule
    $U$/$\si{\volt}$ & $I\cdot 10^{-9}$/$\si{\ampere}$ & $\sqrt{I}\cdot 10^{-5}$/$\sqrt{\si{\ampere}}$ \\
    \midrule
    1.68 & 0.0&0.0 \\
    1.5 & 0.12 \pm 0.01 & 1.10  \pm 0.05 \\
    1.4 & 0.22 \pm 0.01 & 1.48  \pm 0.03 \\
    1.3 & 0.38 \pm 0.01 & 1.95  \pm 0.03 \\
    1.2 & 0.60 \pm 0.01 & 2.45  \pm 0.02 \\
    1.1 & 0.88 \pm 0.01 & 2.97  \pm 0.02 \\
    1.0 & 1.1 \pm 0.1& 3.3  \pm 0.2 \\
    0.9 & 1.6 \pm 0.1& 4.0  \pm 0.1 \\
    0.8 & 2.1 \pm 0.1& 4.6  \pm 0.1 \\
    0.7 & 2.6 \pm 0.1& 5.1  \pm 0.1 \\
    0.5 & 3.7 \pm 0.1& 6.08  \pm 0.08 \\
    0.3 & 5.0 \pm 0.1& 7.07  \pm 0.07 \\
    0.15 & 6.0 \pm 0.1 & 7.75  \pm 0.06\\
    0.02 & 7.4 \pm 0.1 & 8.60  \pm 0.06 \\
    \bottomrule
  \end{tabular}
\end{table}

\begin{figure}
  \centering
  \caption{Datenpaare aus  $\sqrt{I}$ und $U_\mathrm{B}$ samt linearer Regression zur Bestimmung Grenzspannung $U_\mathrm{G}$ der ersten violetten Spektrallinie.}
  \label{fig:uguv}
  \includegraphics{Bilder/violett.pdf}
\end{figure}



\FloatBarrier


\begin{table}
  \centering
  \caption{Messdaten zur Bestimmung der Grenzspannung $U_\mathrm{G}$ bei der zweiten violetten Spektrallinie.}
  \label{tab:ugv2}
  \begin{tabular}{ccc}
    \toprule
    $U$/$\si{\volt}$ & $I\cdot 10^{-9}$/$\si{\ampere}$ & $\sqrt{I}\cdot 10^{-5}$/$\sqrt{\si{\ampere}}$ \\
    \midrule
    1.33 & 0.0 & 0.0 \\
    1.2 & 0.06  \pm 0.01 & 0.77  \pm 0.06 \\
    1.1 & 0.12  \pm 0.01 & 1.10  \pm 0.05 \\
    1.0 & 0.24  \pm 0.01 & 1.55  \pm 0.03 \\
    0.9 & 0.38  \pm 0.01 & 1.95  \pm 0.03 \\
    0.8 & 0.54  \pm 0.01 & 2.32  \pm 0.02 \\
    0.7 & 0.76  \pm 0.01 & 2.76  \pm 0.02 \\
    0.6 & 1.2  \pm 0.1 & 3.5  \pm 0.1 \\
    0.5 & 1.6  \pm 0.1 & 4.0  \pm 0.1 \\
    0.4 & 2.0  \pm 0.1 & 4.5  \pm 0.1 \\
    0.3 & 2.4  \pm 0.1 & 4.9  \pm 0.1 \\
    0.2 & 2.8  \pm 0.1 & 5.29  \pm 0.09 \\
    0.1 & 3.4  \pm 0.1 & 5.83  \pm 0.09 \\
    0.02 & 4.0  \pm 0.1 & 6.32  \pm 0.08 \\
    \bottomrule
  \end{tabular}
\end{table}

\begin{figure}
  \centering
  \caption{Datenpaare aus  $\sqrt{I}$ und $U_\mathrm{B}$ samt linearer Regression zur Bestimmung Grenzspannung $U_\mathrm{G}$ der zweiten violetten Spektrallinie.}
  \label{fig:ugv2}
  \includegraphics{Bilder/violett.pdf}
\end{figure}



\FloatBarrier


\begin{table}
  \centering
  \caption{Messdaten zur Bestimmung der Grenzspannung $U_\mathrm{G}$ bei der dritten violetten Spektrallinie.}
  \label{tab:ugv3}
  \begin{tabular}{ccc}
    \toprule
    $U$/$\si{\volt}$ & $I\cdot 10^{-9}$/$\si{\ampere}$ & $\sqrt{I}\cdot 10^{-5}$/$\sqrt{\si{\ampere}}$ \\
    \midrule
    1.12 & 0.0& 0.0\\
    1.1 & 0.02  \pm 0.01 & 0.5  \pm 0.1 \\
    1.0 & 0.10  \pm 0.01 & 1.00  \pm 0.05 \\
    0.9 & 0.24  \pm 0.01 & 1.55  \pm 0.03 \\
    0.8 & 0.46  \pm 0.01 & 2.15  \pm 0.02 \\
    0.7 & 0.86  \pm 0.01 & 2.93  \pm 0.02 \\
    0.6 & 1.2  \pm 0.1 & 3.5  \pm 0.1 \\
    0.5 & 2.0  \pm 0.1 & 4.5  \pm 0.1 \\
    0.4 & 2.6  \pm 0.1 & 5.1  \pm 0.1 \\
    0.35 & 3.0  \pm 0.1 & 5.48  \pm 0.09 \\
    0.3 & 3.4  \pm 0.1 & 5.83  \pm 0.09 \\
    0.2 & 4.2  \pm 0.1 & 6.48  \pm 0.08 \\
    0.1 & 5.0  \pm 0.1 & 7.07  \pm 0.07 \\
    0.02 & 5.9  \pm 0.1 & 7.68  \pm 0.07 \\
    \bottomrule
  \end{tabular}
\end{table}

\begin{figure}
  \centering
  \caption{Datenpaare aus  $\sqrt{I}$ und $U_\mathrm{B}$ samt linearer Regression zur Bestimmung Grenzspannung $U_\mathrm{G}$ der dritten violetten Spektrallinie.}
  \label{fig:ugv3}
  \includegraphics{Bilder/violett_drittkleinste.pdf}
\end{figure}


\FloatBarrier


\begin{table}
  \centering
  \caption{Messdaten zur Bestimmung der Grenzspannung $U_\mathrm{G}$ bei der vierten blaugrünen Spektrallinie.}
  \label{tab:ugbg}
  \begin{tabular}{ccc}
    \toprule
    $U$/$\si{\volt}$ & $I\cdot 10^{-9}$/$\si{\ampere}$ & $\sqrt{I}\cdot 10^{-5}$/$\sqrt{\si{\ampere}}$ \\
    \midrule
    0.9 & 0.0 & 0.0\\
    0.8 & 0.01  \pm 0.01 & 0.3  \pm 0.2 \\
    0.7 & 0.01  \pm 0.01 & 0.3  \pm 0.2 \\
    0.6 & 0.02  \pm 0.01 & 0.5  \pm 0.1 \\
    0.55 & 0.03  \pm 0.01 & 0.55  \pm 0.09 \\
    0.5 & 0.04  \pm 0.01 & 0.63  \pm 0.08 \\
    0.45 & 0.05  \pm 0.01 & 0.71  \pm 0.07 \\
    0.4 & 0.06  \pm 0.01 & 0.77  \pm 0.06 \\
    0.35 & 0.08  \pm 0.01 & 0.89  \pm 0.06 \\
    0.3 & 0.10  \pm 0.01 & 1.00  \pm 0.05 \\
    0.25 & 0.12  \pm 0.01 & 1.10  \pm 0.05 \\
    0.2 & 0.14  \pm 0.01 & 1.18  \pm 0.04 \\
    0.1 & 0.18  \pm 0.01 & 1.34  \pm 0.04 \\
    0.02 & 0.22  \pm 0.01 & 1.48  \pm 0.03 \\
    \bottomrule
  \end{tabular}
\end{table}



\begin{figure}
  \centering
  \caption{Datenpaare aus  $\sqrt{I}$ und $U_\mathrm{B}$ samt linearer Regression zur Bestimmung Grenzspannung $U_\mathrm{G}$ der blaugrünen Spektrallinie.}
  \label{fig:ugbg}
  \includegraphics{Bilder/blaugrün.pdf}
\end{figure}

\FloatBarrier

\begin{table}
  \centering
  \caption{Messdaten zur Bestimmung der Grenzspannung $U_\mathrm{G}$ bei der fünften, grünen Spektrallinie.}
  \label{tab:uggruen}
  \begin{tabular}{ccc}
    \toprule
    $U$/$\si{\volt}$ & $I\cdot 10^{-9}$/$\si{\ampere}$ & $\sqrt{I}\cdot 10^{-5}$/$\sqrt{\si{\ampere}}$ \\
    \midrule
    -0.6 & 6.6  \pm 0.1 & 8.12  \pm 0.06 \\
    -0.5 & 5.8  \pm 0.1 & 7.62  \pm 0.07 \\
    -0.4 & 5.5  \pm 0.1 & 7.42  \pm 0.07 \\
    -0.3 & 4.6  \pm 0.1 & 6.78  \pm 0.07 \\
    -0.2 & 3.6  \pm 0.1 & 6.00  \pm 0.08 \\
    -0.1 & 2.8  \pm 0.1 & 5.29  \pm 0.09 \\
    0.02 & 1.5  \pm 0.1 & 3.9  \pm 0.1 \\
    0.1 & 1.00  \pm 0.01 & 3.16  \pm 0.02 \\
    0.2 & 0.62  \pm 0.01 & 2.49  \pm 0.02 \\
    0.3 & 0.24  \pm 0.01 & 1.55  \pm 0.03 \\
    0.35 & 0.14  \pm 0.01 & 1.18  \pm 0.04 \\
    0.4 & 0.07  \pm 0.01 & 0.84  \pm 0.06 \\
    0.45 & 0.04  \pm 0.01 & 0.63  \pm 0.08 \\
    0.5 & 0.0  & 0.0 \\
    \bottomrule
  \end{tabular}
\end{table}

\begin{figure}
  \centering
  \caption{Datenpaare aus  $\sqrt{I}$ und $U_\mathrm{B}$ samt linearer Regression zur Bestimmung Grenzspannung $U_\mathrm{G}$ der grünen Spektrallinie.}
  \label{fig:uggruen}
  \includegraphics{Bilder/grün.pdf}
\end{figure}

\FloatBarrier

\begin{table}
  \centering
  \caption{Messdaten zur Bestimmung der Grenzspannung $U_\mathrm{G}$ bei der sechsten, gelben Spektrallinie.}
  \label{tab:ugg}
  \begin{tabular}{ccc}
    \toprule
    $U$/$\si{\volt}$ & $I\cdot 10^{-9}$/$\si{\ampere}$ & $\sqrt{I}\cdot 10^{-5}$/$\sqrt{\si{\ampere}}$ \\
    \midrule
    -19.0 & 15.0  \pm 0.1 & 12.25  \pm 0.04 \\
    -17.5 & 14.0  \pm 0.1 & 11.83  \pm 0.04 \\
    -16.0 & 14.0  \pm 0.1 & 11.83  \pm 0.04 \\
    -14.5 & 14.0  \pm 0.1 & 11.83  \pm 0.04 \\
    -13.0 & 14.0  \pm 0.1 & 11.83  \pm 0.04 \\
    -11.5 & 13.0  \pm 0.1 & 11.40  \pm 0.04 \\
    -10.0 & 12.0  \pm 0.1 & 10.95  \pm 0.05 \\
    -9.0 & 12.0  \pm 0.1 & 10.95  \pm 0.05 \\
    -8.0 & 12.0  \pm 0.1 & 10.95  \pm 0.05 \\
    -7.0 & 11.0  \pm 0.1 & 10.49  \pm 0.05 \\
    -6.0 & 10.0  \pm 0.1 & 10.00  \pm 0.05 \\
    -5.0 & 9.0  \pm 0.1 & 9.49  \pm 0.05 \\
    -4.0 & 8.4  \pm 0.1 & 9.17  \pm 0.05 \\
    -3.0 & 6.9  \pm 0.1 & 8.31  \pm 0.06 \\
    -2.5 & 6.0  \pm 0.1 & 7.75  \pm 0.06 \\
    -2.0 & 5.1  \pm 0.1 & 7.14  \pm 0.07 \\
    -1.8 & 5.0  \pm 0.1 & 7.07  \pm 0.07 \\
    -1.6 & 4.8  \pm 0.1 & 6.93  \pm 0.07 \\
    -1.4 & 4.2  \pm 0.1 & 6.48  \pm 0.08 \\
    -1.2 & 3.8  \pm 0.1 & 6.16  \pm 0.08 \\
    -1.1 & 3.5  \pm 0.1 & 5.92  \pm 0.08 \\
    -1.0 & 3.2  \pm 0.1 & 5.66  \pm 0.09 \\
    -0.9 & 3.1  \pm 0.1 & 5.57  \pm 0.09 \\
    -0.8 & 2.9  \pm 0.1 & 5.39  \pm 0.09 \\
    -0.7 & 2.7  \pm 0.1 & 5.2  \pm 0.1 \\
    -0.6 & 2.4  \pm 0.1 & 4.9  \pm 0.1 \\
    -0.5 & 2.0  \pm 0.1 & 4.5  \pm 0.1 \\
    -0.4 & 1.8  \pm 0.1 & 4.2  \pm 0.1 \\
    -0.3 & 1.4  \pm 0.1 & 3.7  \pm 0.1 \\
    -0.2 & 1.1  \pm 0.1 & 3.3  \pm 0.2 \\
    -0.1 & 0.90  \pm 0.01 & 3.00  \pm 0.02 \\
    -0.01 & 0.60  \pm 0.01 & 2.45  \pm 0.02 \\
    0.01 & 0.50  \pm 0.01 & 2.24  \pm 0.02 \\
    0.1 & 0.24  \pm 0.01 & 1.55  \pm 0.03 \\
    0.15 & 0.14  \pm 0.01 & 1.18  \pm 0.04 \\
    0.2 & 0.08  \pm 0.01 & 0.89  \pm 0.06 \\
    0.25 & 0.04  \pm 0.01 & 0.63  \pm 0.08 \\
    0.3 & 0.02  \pm 0.01 & 0.4  \pm 0.1 \\
    0.35 & 0.0&0.0\\
    \bottomrule
  \end{tabular}
\end{table}




\begin{figure}
  \centering
  \caption{Datenpaare aus  $\sqrt{I}$ und $U_\mathrm{B}$ samt linearer Regression zur Bestimmung Grenzspannung $U_\mathrm{G}$ der gelben Spektrallinie.}
  \label{fig:ugg}
  \includegraphics{Bilder/gelb.pdf}
\end{figure}
\FloatBarrier
\subsection{Bestimmung des Verhältnis $\frac{\symup{h}}{\symup{e}_0}$ und der Austrittsarbeit $A_{K}$}
Nach Gleichung \eqref{eqn:gleichi} besteht ein linearer Zusammenhang zwischen der Grenzspannung und der Frequenz $\nu$ der zugehörigen Spektralfarbe.
Zwischen der Wellenlänge $\lambda$ und der Frequenz $\nu$ besteht der Zusammenhang
\begin{equation*}
  \nu=\frac{c}{\lambda} \text{.}
\end{equation*}
Hierbei ist $c$ die Lichtgeschwindigkeit im Vakuum nach \cite{c}.\\
In Tabelle \ref{tab:ak} sind hierzu die Wellenlängen $\lambda$ nach \cite{Anleitung} der Spektrallinien, die zugehörigen Frequenzen $\nu$ sowie die zuvor berechneten Grenzspannungen $U_\mathrm{G}$ eingetragen.\\
Aus Gleichung \eqref{eqn:gleichi} ergibt sich durch Umformung
\begin{equation*}
  U_\mathrm{G}=\frac{\symup{h}}{\symup{e}_0} \cdot{\nu}-A_\mathrm{K} \text{.}
\end{equation*}
Hierbei ist $A_\mathrm{K}$ die Austrittsarbeit in $\si{\electronvolt}$.
In Abbildung \ref{fig:ak} ist daher die Grenzspannung $U_\mathrm{G}$ gegen die Frequenz $\nu$ der Spektrallinien aufgetragen.\\
Aus den Parametern der mit scipy/python \cite{scipy} ermittelten Regressionsgraden ergibt sich:
\begin{gather*}
  a=\frac{\symup{h}}{\symup{e}_0}= \SI{4.3(3)e-15}{\electronvolt}\\
b=A_\mathrm{K}=\SI{1.8(2)}{\electronvolt}
\end{gather*}


\begin{table}
  \centering
\caption{Messdaten zur Bestimmung des Verhältnis $\frac{h}{\symup{e}_0}$ und der Austrittsarbeit $A_\mathrm{K}$.}
\label{tab:ak}
\begin{tabular}{ccc}
  \toprule
$\lambda \cdot 10^{-9}$/$\si{\meter}$& $\nu \cdot 10^{12}$/$\si{\hertz}$ & $U_\mathrm{G}$/$\si{\volt}$ \\
\midrule
365 & 821.3 & 1.69 \\
405 & 740.2 & 1.32 \\
435 & 689.2 & 1.13 \\
492 & 609.3 & 0.90 \\
546 & 549.1 & 0.50 \\
577 & 519.5 & 0.35 \\
\bottomrule
\end{tabular}
\end{table}
\begin{figure}
  \centering
  \caption{Grenzspannung $U_\mathrm{G}$ aufgetragen gegen die Frequenz $\nu$ der vermessenen Spektralfarben samt Regressionsgrade.}
  \label{fig:ak}
  \includegraphics{Bilder/Ug.pdf}
\end{figure}


%%%%%%%%%%%%%%%%%%%%%%%%%%%%%%%%%%%%%%%%%%%%%%%%%%%%%%%%%%%%%%%%%%%%%%%%%%%%%%%%%%%%%%%%%
\FloatBarrier
\subsection{Diskussion des Photostroms in Abhängigkeit von der anliegenden Spannung beim gelben Licht}

Die Messdaten für die Messung mit gelbem Licht sind in Tabelle \ref{tab:ugg} zu finden.
Diese sind in Abbildung \ref{fig:datayellow} dargestellt, wobei der Photostrom $I$ gegen die
anliegende Spannung $U$ aufgetragen ist. Erwähnenswert ist, dass die negativen Spannungen $U$
einer beschleunigenden Spannung entsprechen.
\begin{figure}
	\centering
	\caption{Photostrom $I$ in Abhängigkeit von der anliegenden Spannung $U$ beim gelben Licht}
	\label{fig:datayellow}
	\includegraphics{Bilder/gelbplot.pdf}
\end{figure}
Insgesamt lässt sich erkennen, dass der Photostrom bei steigender Spannung $U$ monton fällt.
Für hohe beschleunigende Spannungen -- im Bereich von $-\SI{20}{\volt}$ bis $-\SI{10}{\volt}$ --
lässt sich eine Annäherung an einen Sättigungswert beobachten. Auf den ersten Blick scheint
dieser Zusammenhang wie ein Widerspruch zum Ohm'schen Gesetz
\begin{equation*}
	I = \frac{U}{R} \mathrm{,}
\end{equation*}
nach dem ein wachsender Strom bei wachsender Spannung gegeben sein müsste.
Allerdings kommt diese obere Schranke zustande, da nach Formel \eqref{eqn:gleichi} ab einer
bestimmten Spannung $U$ die Energie aller ausgelösten Elektronen ausreicht, um die Anode zu
erreichen. Die obere Schranke des Photostroms ist also gegeben durch die Anzahl der
ausgelösten Elektronen, die durch die Intensität des Lichts festgelegt wird. Die Intensität
des Lichts wird allerdings bei diesem Versuch konstant gehalten und hängt nicht mit der
anliegenden Spannung zusammen. Dieses Ergebnis stellt also keinen Widerspruch zum
Ohm'schen Gesetz dar.

Dieser Sättigungswert des Photostroms wird nur asymptotisch erreicht, weil in der Realität
nicht jedes ausgelöste Elektron mit einer hinreichend großen kinetischen Energie die Anode
erreicht. Dafür müsste eine unendliche Spannung anliegen, die dafür sorgt, dass jedes
ausgelöste Elektron durch das entstehende Feld zur Anode gelangt. Des Weiteren ist die
Größe der Anode beschränkt.
Um den Sättigungswert bei endlicher Spannung zu erreichen, müsste die Anode eine größere
Oberfläche haben, sodass kein Elektron nicht zur Anode gelangt.

Dass der Photostrom vor der Grenzspannung beginnt zu sinken, liegt an der Energieverteilung
der Elektronen im Material. Diese durch die Fermi-Dirac-Statistik gegebene Verteilung sorgt
dafür, dass ausgelöste Elektronen eine unterschiedlich hohe kinetische Energie beim Austritt
aus der Oberfläche besitzen und somit auch eine statistisch verteilte Gegenspannung, bei der
das Elektron die Anode nicht mehr erreichen kann. Dadurch erreichen einzelne Elektronen die
Anode, während andere Elektronen keine hinreichend große Energie besitzen, um das Gegenfeld
zu passieren. Daher verschwindet der Photostrom nicht instantan bei der Grenzspannung.

Weiterhin sorgt die Verdampfungstemperetur des verwendeten Kathodenmaterials von
$\SI{20}{\celsius}$ dafür, dass Elektronen durch den Verdampfungsprozess aus dem Material
gelöst werden und sich an der Anode anlagern. Durch die Gegenspannung -- für die durch das
Licht ausgelösten Elektronen -- entsteht für die an der Anode liegenden Elektronen
ein beschleunigendes Feld, sodass diese Elektronen aus der Anode gelöst und  zur Kathode
hin beschleunigt werden.
Dies hat zur Folge, dass ein dem Photostrom entgegengerichteter Strom auftreten kann.
Dazu lässt sich anmerken, dass aufgrund der geringen Quantität der durch den Verdampfungsprozess
ausgelösten Elektronen und der Potentialdifferenz zwischen Anode und Kathode der Sättigungswert
bei diesem umgekehrtem Prozess erreicht werden kann.

Da bei relativ langwelligem Licht der negative Strom auftritt, lässt sich sich für
die Austrittsarbeit der Anode sagen, dass sie höchstens so groß sein kann wie bei der
Kathode.
