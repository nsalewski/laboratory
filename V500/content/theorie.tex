\section{Theorie}
\label{sec:Theorie}

Der Photoeffekt tritt auf, wenn eine Metalloberfläche mit Licht bestrahlt wird.
Ist die Energie $E=\symup{h}\nu$ der einzelnen Photonen groß genug, können Elektronen durch 
diese quantitisierte Energie aus der Oberfläche austreten.

Die experimentellen Beobachtungen dieses Phänomens liefern drei grundlegende Ergebnisse.
Zum Einen ist die Anzahl der pro Zeit gelösten Elektronen proportional 





\cite{Anleitung}
