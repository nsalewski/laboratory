\begin{table}
 \caption{Werte der $\Delta \theta$ zwischen undotiertem und dotiertem $\ce{GaAs}$ zur Bestimmung der effektiven Masse der Kristallelektronen}
 \label{tab:eff_mass}
 \centering
\sisetup{table-format=4.2} \begin{tabular}{SSSSSS}
 \toprule 
    {$\lambda$/$\si{\micro\meter}$}& {$\theta_{\mathrm{und}}$/$\si{\radian\per\meter}$}& {$\theta_{\mathrm{d1}}$/$\si{\radian\per\meter}$}& {$\theta_{\mathrm{d2}}$/$\si{\radian\per\meter}$}& {$\Delta \theta_{\mathrm{d1}}$/$\si{\radian\per\meter}$}& {$\Delta \theta_{\mathrm{d2}}$/$\si{\radian\per\meter}$} \\
     \midrule
           0.00 &     162.18 &      14.93 &     185.66 &    -147.25 &      23.47 \\
           0.00 &     -12.21 &      30.97 &     234.74 &      43.18 &     246.95 \\
           0.00 &     -13.83 &     150.05 &     217.74 &     163.88 &     231.57 \\
           0.00 &       1.05 &     -84.39 &     278.80 &     -85.45 &     277.75 \\
           0.00 &     -27.44 &     160.48 &     224.58 &     187.92 &     252.02 \\
           0.00 &     -41.30 &      97.64 &      78.71 &     138.94 &     120.01 \\
           0.00 &     -18.27 &      93.37 &     -87.80 &     111.64 &     -69.53 \\
           0.00 &     193.26 &     194.37 &     165.23 &       1.11 &     -28.03 \\
 \bottomrule
 \end{tabular}
\end{table}