\section{Diskussion}
\label{sec:Diskussion}
Die im Experiment bestimmten Werte für das Verhältnis zwischen der effektiven Masse $m^{*}$ des Kristallelektrons in $\ce{GaAs}$ und der Ruhemasse des Elektrons $m_{0}$ weichen deutlich vom Literaturwert
$\frac{m^{*}}{m_{0}}=0.067$ ab. In Tabelle \ref{tab:diskussion} sind die prozentualen Abweichungen zum Literaturwert für beide Messungen angegeben.
\begin{table}
  \centering
  \caption{Abweichungen der gemessenen Größen zum Literaturwert $\frac{m^{*}}{m_{0}}=0.067$.}
  \label{tab:diskussion}
  \begin{tabular}{ccc}
    \toprule
    Messung&$m^{*}$/$m_{0}$&Abweichung zum Literaturwert\\
    \midrule
    $\ce{GaAs}_{\mathrm{d1}}-\ce{GaAs}_{\mathrm{und}}$&$\num{0.055(7)}$&$18\%$\\
    $\ce{GaAs}_{\mathrm{d2}}-\ce{GaAs}_{\mathrm{und}}$&$\num{0.027(4)}$&$60\%$\\
    \bottomrule
  \end{tabular}
\end{table}
Ein Vergleich mit Abbildung \ref{fig:delta1} und \ref{fig:delta2} zeigt, dass viele Werte recht deutlich von der Ausgleichsgraden abweichen. Zudem wurden in beiden Messungen Messpunkte nicht zur Berechnung der Regressionsgraden berücksichtigt. Die Auswahl der nicht zu berücksichtigen Werte für die Berechnung der Ausgleichsgraden erfolgte jeweils anhand der Überlegung, dass ein negatives $\Delta\theta$ unphysikalisch wäre. Schließlich würde ein negatives $\Delta\theta$ bedeuten, dass hochreines $\ce{GaAs}$ besser leitet als dotiertes $\ce{GaAs}$.\\
Es wäre auch noch möglich, weitere Punkte von der Regressionsrechnung auszuschließen oder die ausgeschlossenen Messpunkte weiterhin zu berücksichtigen, was zu sehr großen Änderungen im Steigungsparameter $A$ führen würde.
Letztlich zeigt sich in den Rohdaten der Messung kein eindeutiger linearer Zusammenhang zwischen der Rotationswinkeldifferenz $\Delta\theta$ und dem Quadrat der Wellenlänge $\lambda^{2}$.\\
Eine größere Anzahl an betrachteten Wellenlängen könnte ebenso zu einer Verbesserung der Genauigkeit der Messung führen, wie eine Optimierung der Messapparatur.
Der Nullabgleich am Oszilloskop war beispielsweise recht subjektiv und ließ sich nicht für alle Filter gleich gut realisieren.
Einige der verwendeten Filter ließen nur eine sehr geringe Restintensität durch.
Zu vermuten ist auch, dass die Apparatur eventuell nicht für jeden Filter und jede Probe optimal einjustiert ist.\\
Da zudem manchmal mit identischem Filter ein sehr guter Abgleich auf Null möglich war und in einer späteren Messung (bei einer anderen Probe bzw. bei Umpolung des Magnetfelds) der Abgleich deutlich schlechter gelang, könnte auch die Frequenz des Lichtzerhackers instabil sein, sodass der Selektiv-Verstärker nicht immer gut auf diesen abgestimmt war.\\
Möglicherweise schwankte die Stärke des Magnetfelds und der maximale Wert des Magnetfelds bei unterschiedlicher Polung könnte sich voneinander unterscheidet. Zudem wurde festgestellt, dass die Probe nicht genau an der Stelle der größten Feldstärke montiert ist, sondern leicht daneben. Inhomogenitäten des Magnetfelds blieben ebenso unberücksichtigt.\\
Beachtet wurde außerdem nicht, dass der Brechungsindex $n$ im Allgemeinen nicht konstant ist, sondern eine Dispersionsabhängigkeit besitzt.\\

Trotz der vielen Ungenauigkeiten im Versuchsaufbau lässt sich aus den Ergebnissen vermuten, dass prinzipiell eine Untersuchung der Faraday-Rotation mit der vorliegenden Apparatur mit höher dotiertem $\ce{GaAs}$ genauer ist (vgl. Tabelle \ref{tab:diskussion}).
