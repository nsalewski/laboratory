\section{Auswertung}
\label{sec:Auswertung}
%include tables
%\begin{table}
 \caption{Im Experiment gemessene Ströme der Sweep-Spule und der Horizontalfeldspule für die Transparenzminima beider Isotope sowie die Frequenz des angelegten RF-Wechselfelds }
 \label{tab:current}
 \centering
 \begin{tabular}{ccccc}
 \toprule 
    RF-Wechselfeld/$\si{\kilo\hertz}$ & $I_{\mathrm{Horizontal,1}}/\si{\milli \ampere}$ & $I_{\mathrm{Horizontal,2}}/\si{\milli \ampere}$ & $I_{\mathrm{Sweep,1}}/\si{\milli \ampere}$ & $I_{\mathrm{Sweep,2}}/\si{\milli \ampere}$ \\
     \midrule
     100.3 & 0 & 0 & 597 & 666 \\
     201.0 & 35 & 35 & 245 & 484 \\
     301.0 & 56 & 56 & 187 & 545 \\
     400.0 & 76 & 76 & 139 & 617 \\
     499.0 & 91 & 91 & 162 & 753 \\
     603.0 & 116 & 116 & 47 & 763 \\
     699.0 & 88 & 159 & 688 & 473 \\
     801.0 & 101 & 182 & 736 & 508 \\
     899.0 & 149 & 207 & 265 & 493 \\
     1004.0 & 156 & 237 & 410 & 424 \\
 \bottomrule
 \end{tabular}
\end{table}
\subsection{Bestimmung des B-Felds am Ort der Probe}
Nach der Versuchsanleitung soll angenommen werden, dass sich die Probe am Ort des Maximums des Magnetfelds befindet \cite{Anleitung}.
Da das Magnetfeld durch zwei Spulen erzeugt wird und sich die Probe annähernd mittig zwischen diesen befindet, wird von einer symmetrischen Verteilung der magnetischen Feldstärke bezüglich ihres Maximums ausgegangen.\\ Zur Ermittlung der $z$-Position des stärksten Magnetfelds wird daher eine Gaußglocke an die aufgenommenen Daten angepasst:
\begin{equation}
	\phi(z)=a\cdot\exp\left({-(z-\mu)^2/b}\right) \, \mathrm{.}
\end{equation}
Die Parameter $a=\SI{404(9)}{\per\centi\meter}$  und $b=\SI{1.51(9)}{\square\centi\meter}$ sind dabei Fitparameter, welche für die Betrachtung des Orts der größten Feldstärke nicht relevant sind.\\
%(array([ 4.04175662e+02, -2.60722876e-01,  1.50983396e+00]), array([8.72498252, 0.02434217, 0.0911311 ]))
Der Erwartungswert der Gaußglocke ergibt sich zu $\mu=\SI{-0.26(2)}{\centi\meter}$. Da zu diesem $z$-Wert keine magnetische Feldstärke bestimmt wurde, wird die Feldstärke der beiden umgebenden $z$-Werte gemittelt.
Es ergibt sich somit ein Magnetfeld $B\approx \SI{377.5}{\milli\tesla}$ an der Position der Probe.\\
Dieser Wert wird für die weiteren Berechnungen als Magnetfeld am Ort der Probe angenommen.\\
In Tabelle \ref{tab:magnetfeld} sind die gemessenen Werte für das B-Feld in Abhängigkeit vom Ort angegeben. Hierbei wurde das Magnetfeld in einem Intervall von $\SI{6}{\centi\meter}$ um die Position der Probe vermessen, wobei bezüglich der ungefähren Probenposition bei $z=\SI{0}{\centi\meter}$ sowohl in positiver, als auch in negativer $z$-Richtung gemessen wurde. In Abbildung \ref{fig:magnetfeld} sind die gemessenen Werte gegen den Ort $z$ aufgetragen. Zudem wurde durch die gestrichelte rote Linie der Erwartungswert der Gaußglocke eingezeichnet.
\begin{figure}
  \centering
  \includegraphics[width=0.75\textwidth]{pictures/B_feld.pdf}
  \caption{Das gemessene B-Feld aufgetragen gegen den Ort $z$. Die Probe befand sich etwa bei $z=\SI{0}{\centi\meter}$. Zudem ist der Erwartungswert der Gaußverteilung eingezeichnet, an welchem die größte Magnetfeldstärke angenommen wird.}
  \label{fig:magnetfeld}
\end{figure}
\begin{table}
 \caption{Messung des Magnetfelds in Abhängigkeit zum Ort $z$ (Probe ist etwa bei $\SI{0}{\centi\meter}$ platziert)}
 \label{tab:magnetfeld}
 \centering
\sisetup{table-format=3.2} \begin{tabular}{SS|SS}
 \toprule
    {$z$/$\si{\centi\meter}$}& {$B$/$\si{\milli\tesla}$}&{$z$/$\si{\centi\meter}$}& {$B$/$\si{\milli\tesla}$} \\
     \midrule
		 -3.10 &          1 &0.10 &        367 \\
		 -2.60 &          4 &     0.20 &        359 \\
		 -2.10 &         19 &     0.40 &        331 \\
		 -1.60 &         96 &     0.65 &        263 \\
		 -1.35 &        193 &     0.90 &        170 \\
		 -1.10 &        287 &     1.15 &         85 \\
		 -0.85 &        341 &     1.40 &         41 \\
		 -0.60 &        368 &     1.90 &          9 \\
		 -0.35 &        378 &     2.40 &          2 \\
		 -0.10 &        377 &     2.90 &          0 \\
			0.00 &        373 & & \\
 \bottomrule
 \end{tabular}
\end{table}



\FloatBarrier
\subsection{Bestimmung der Faraday-Rotation}
Die Faraday-Rotation wurde im vorliegenden Versuch an zwei unterschiedlich dotierten $\ce{GaAs}$-Proben, sowie einer undotierten $\ce{GaAs}$-Probe durchgeführt.
Die Parameter der Proben finden sich in Tabelle \ref{tab:params}.
\begin{table}
  \centering
  \caption{Parameter der verwendeten Proben}
  \label{tab:params}
  \begin{tabular}{ccc}
    \toprule
    Probe&Dotierung $N$/$\si{\per\cubic\centi\metre}$&Dicke $L$/$\si{\milli\meter}$\\
    \midrule
    $\ce{GaAs}_{\mathrm{und}}$&-&$\num{5.110}$\\
    $\ce{GaAs}_{\mathrm{d1}}$&$\num{2.8e18}$&$\num{1.296}$\\
    $\ce{GaAs}_{\mathrm{d2}}$&$\num{1.2e18}$&$\num{1.360}$\\
    \bottomrule
  \end{tabular}
\end{table}

Für die Proben wurden jeweils die Winkel $\theta_1$ und $\theta_2$ für die beiden unterschiedlichen Polungen des Magnetfelds gemessen. Da zum Teil Winkel kleiner als $\theta=\ang{0}$ gemessen wurden, werden zunächst  alle Winkel auf einen gemeinsamen Bezugspunkt gedreht. Hierzu werden alle Winkel um $\theta=\ang{180}$ gedreht, was experimentell der genau gleichen Polarisation entspricht und die weitere Auswertung der Daten vereinfacht.
Da die Faraday-Rotation bezüglich beider Orientierungen des Magnetfelds und für verschiedene Probendicken aufgenommen wurde, wird die Differenz beider Rotationswinkel betrachtet und auf die Dicke $L$ der Proben normiert:
\begin{equation}
  \theta=\frac{1}{2L}\left(\theta_1-\theta_2\right)\mathrm{.}
\end{equation}
In den Tabellen \ref{tab:probe1}-\ref{tab:probe3} finden sich die um den Halbwinkel modifizierten Messwerte für die $\theta_i$, sowie die daraus bestimmten Faraday-Rotationen $\theta$.
\begin{table}
 \caption{Messwerte der Faraday-Rotation für die dotierte Probe $\ce{GaAs}_{d1}$}
 \label{tab:probe1}
 \centering
\sisetup{table-format=4.2} \begin{tabular}{SSSS}
 \toprule 
    {$\lambda$/$\si{\micro\meter}$}& {$\theta_1$/$\si{\degree}$}& {$\theta_2$/$\si{\degree}$}& {$\theta$/$\si{\radian\per\meter}$} \\
     \midrule
           1.06 &     187.22 &     189.43 &      14.93 \\
           1.29 &     182.30 &     186.90 &      30.97 \\
           1.45 &     180.67 &     202.95 &     150.05 \\
           1.96 &     189.30 &     176.77 &     -84.39 \\
           2.16 &     168.65 &     192.48 &     160.48 \\
           2.34 &     186.98 &     201.48 &      97.64 \\
           2.51 &     185.00 &     198.87 &      93.37 \\
           3.18 &     162.65 &     191.52 &     194.37 \\
 \bottomrule
 \end{tabular}
\end{table}
\begin{table}
 \caption{Messwerte der Faraday-Rotation für die dotierte Probe $\ce{GaAs}_{d2}$}
 \label{tab:probe2}
 \centering
\sisetup{table-format=4.2} \begin{tabular}{SSSS}
 \toprule 
    {$\lambda$/$\si{\micro\meter}$}& {$\theta_1$/$\si{\degree}$}& {$\theta_2$/$\si{\degree}$}& {$|\theta|$/$\si{\radian\per\meter}$} \\
     \midrule
           1.06 &     166.07 &     195.00 &     185.66 \\
           1.29 &     163.82 &     200.40 &     234.74 \\
           1.45 &     157.12 &     191.05 &     217.74 \\
           1.96 &     164.83 &     208.28 &     278.80 \\
           2.16 &     163.58 &     198.58 &     224.58 \\
           2.34 &     188.20 &     200.47 &      78.71 \\
           2.51 &     204.02 &     190.33 &     -87.80 \\
           3.18 &     167.33 &     193.08 &     165.23 \\
 \bottomrule
 \end{tabular}
\end{table}
\begin{table}
 \caption{Messwerte der Faraday-Rotation für die undotierte Probe $\ce{GaAs}_{und}$}
 \label{tab:probe3}
 \centering
\sisetup{table-format=4.2} \begin{tabular}{SSSS}
 \toprule 
    {$\lambda$/$\si{\micro\meter}$}& {$\theta_1$/$\si{\degree}$}& {$\theta_2$/$\si{\degree}$}& {$\theta$/$\si{\radian\per\meter}$} \\
     \midrule
           1.06 &     157.42 &     252.38 &     162.18 \\
           1.29 &     186.48 &     179.33 &     -12.21 \\
           1.45 &     196.15 &     188.05 &     -13.83 \\
           1.96 &     191.03 &     191.65 &       1.05 \\
           2.16 &     159.50 &     143.43 &     -27.44 \\
           2.34 &     204.18 &     180.00 &     -41.30 \\
           2.51 &     208.53 &     197.83 &     -18.27 \\
           3.18 &     162.32 &     275.48 &     193.26 \\
 \bottomrule
 \end{tabular}
\end{table}

In Abbildung \ref{fig:faraday} finden sich die berechneten Faraday-Rotationswinkel aufgetragen gegen die jeweils untersuchte Wellenlänge $\lambda$ des einfallenden Lichts.

\begin{figure}
  \centering
  \includegraphics[width=\textwidth]{pictures/winkel_gg_wellenlaenge.pdf}
  \caption{Die Faraday-Rotationen $\theta$ der verwendeten Proben aufgetragen gegen die Wellenlänge $\lambda$ des jeweils einfallenden Lichts.}
  \label{fig:faraday}
\end{figure}
\FloatBarrier
\subsection{Bestimmung der effektiven Masse der Kristallelektronen}
Zur Bestimmung der effektiven Masse der Elektronen im dotierten $\ce{GaAs}$ wird die Differenz der Faraday-Rotation zwischen dotierten und undotierten $\ce{GaAs}$ betrachtet:
\begin{equation}
\Delta \theta=\theta_{\mathrm{dotiert}}-\theta_{\mathrm{undotiert}}\mathrm{.}
\end{equation}
Durch die Bildung der Differenz zwischen der Faraday-Rotation in dotiertem und reinem $\ce{GaAs}$ ist der verbleibende Rotationswinkel $\Delta\theta$ nur noch verursacht durch die Dotieratome.
Somit ist in $\Delta\theta$ nur noch der durch die Leitungselektronen verursachte Anteil der Faraday-Rotation enthalten und ermöglicht die Berechnung der Masse der Leitungselektronen.\\
In Tabelle \ref{tab:eff_mass} sind die berechneten $\Delta\theta$ eingetragen. In Abbildung \ref{fig:delta1} findet sich die Faraday-Rotation für die Probe $\ce{GaAs}_{\mathrm{d1}}$ sowie in Abbildung \ref{fig:delta2}
für Probe $\ce{GaAs}_{\mathrm{d2}}$.
\begin{table}
 \caption{Werte der $\Delta \theta$ zwischen undotiertem und dotiertem $\ce{GaAs}$ zur Bestimmung der effektiven Masse der Kristallelektronen}
 \label{tab:eff_mass}
 \centering
\sisetup{table-format=4.2} \begin{tabular}{SSSSSS}
 \toprule 
    {$\lambda$/$\si{\micro\meter}$}& {$\theta_{\mathrm{und}}$/$\si{\radian\per\meter}$}& {$\theta_{\mathrm{d1}}$/$\si{\radian\per\meter}$}& {$\theta_{\mathrm{d2}}$/$\si{\radian\per\meter}$}& {$\Delta \theta_{\mathrm{d1}}$/$\si{\radian\per\meter}$}& {$\Delta \theta_{\mathrm{d2}}$/$\si{\radian\per\meter}$} \\
     \midrule
           0.00 &     162.18 &      14.93 &     185.66 &    -147.25 &      23.47 \\
           0.00 &     -12.21 &      30.97 &     234.74 &      43.18 &     246.95 \\
           0.00 &     -13.83 &     150.05 &     217.74 &     163.88 &     231.57 \\
           0.00 &       1.05 &     -84.39 &     278.80 &     -85.45 &     277.75 \\
           0.00 &     -27.44 &     160.48 &     224.58 &     187.92 &     252.02 \\
           0.00 &     -41.30 &      97.64 &      78.71 &     138.94 &     120.01 \\
           0.00 &     -18.27 &      93.37 &     -87.80 &     111.64 &     -69.53 \\
           0.00 &     193.26 &     194.37 &     165.23 &       1.11 &     -28.03 \\
 \bottomrule
 \end{tabular}
\end{table}
\begin{figure}
  \centering
  \includegraphics[width=\textwidth]{pictures/delta1.pdf}
  \caption{Die Faraday-Rotation $\Delta \theta$ aufgetragen gegen $\lambda^{2}$ mit Regressionsgerade für die Probe $\ce{GaAs}_{\mathrm{d1}}$}
  \label{fig:delta1}
\end{figure}
\begin{figure}
  \centering
  \includegraphics[width=\textwidth]{pictures/delta2.pdf}
  \caption{Die Faraday-Rotation $\Delta \theta$ aufgetragen gegen $\lambda^{2}$ mit Regressionsgerade für die Probe $\ce{GaAs}_{\mathrm{d2}}$}
  \label{fig:delta2}
\end{figure}

Nach Gleichung \eqref{eqn:Masse} besteht ein linearer Zusammenhang zwischen der Faraday-Rotation und dem Quadrat der Wellenlänge:
\begin{equation}
    \theta(\lambda)=A\cdot\lambda^2 \, \mathrm{.}
\end{equation}
Die effektive Masse der Kristallelektronen lässt sich aus dem Steigungsparameter bestimmen.
Es ist:
\begin{equation}
  A=\frac{e_{\mathrm{0}}^3 \cdot NB}{8\pi \epsilon_{\mathrm{0}} c^3 n {m^{*}}^2} \, \mathrm{,}
\end{equation}
und somit:
\begin{equation}
  {m^{*}}=\sqrt{\frac{e_{\mathrm{0}}^3 \cdot NB}{8\pi^2 \epsilon_{\mathrm{0}} c^3 n A}} \, \mathrm{.}
\end{equation}
Es ist $N$ die Dotierung der jeweils betrachteten Probe und $n=3,3$ der Brechungsindex von $\ce{GaAs}$ nach \cite{ioffe}.
Bereits bei einer Betrachtung von Abbildung \ref{fig:faraday} fällt auf, dass einige Messwerte Unstimmigkeiten aufweisen. Dies gilt im Besonderen für die größte eingestrahlte Wellenlänge. Daher werden die Werte für $\lambda=\SI{3.18}{\micro\meter}$ jeweils in der Berechnung der Ausgleichsgraden nicht berücksichtigt.
Zudem weichen jeweils weitere Werte deutlich ab. In der Berechnung für $\Delta \theta_{\mathrm{d1}}$ werden daher zusätzlich die Werte für $\lambda=\SI{1.06}{\micro\meter}$ und $\lambda=\SI{1.96}{\micro\meter}$ in der Ausgleichsrechnung ebensowenig berücksichtigt wie $\lambda=\SI{2.51}{\micro\meter}$ in der Berechnung für
 $\Delta \theta_{\mathrm{d2}}$.
In Tabelle \ref{tab:steigung} sind die so ermittelten Steigungen und die daraus bestimmten effektiven Massen $m^{*}$ eingetragen. Zudem wurden jeweils die Verhältnisse zur Ruhemasse $m_{0}$ des Elektrons berechnet.
\begin{table}
  \centering
  \caption{Ergebnisse der Ausgleichsrechnung und der daraus bestimmten Massen für beide Dotierungen.}
  \label{tab:steigung}
  \begin{tabular}{cccc}
    \toprule
    Messung&$A$/$10^{12}\si{\radian\per\cubic\metre}$&effektive Masse  $m^{*}$/$10^{-32}\si{\kilo\gram}$&$m^{*}$/$m_{0}$\\
    \midrule
    $\ce{GaAs}_{\mathrm{d1}}-\ce{GaAs}_{\mathrm{und}}$&$\num{27(7)}$&$\num{5.0(6)}$&$\num{0.055(7)}$\\
    $\ce{GaAs}_{\mathrm{d2}}-\ce{GaAs}_{\mathrm{und}}$&$\num{51(14)}$&$\num{2.4(3)}$&$\num{0.027(4)}$\\
    \bottomrule
  \end{tabular}
\end{table}
