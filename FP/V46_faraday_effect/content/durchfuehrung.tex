\section{Durchführung}
\label{sec:Durchführung}
Für den Versuch wird der Aufbau \ref{fig:Bild2} verwendet.
\begin{figure}
	\centering
	\includegraphics[width=0.8\textwidth]{pictures/aufbau.png}
	\caption{Schematische Darstellung der Messapparatur \cite{Anleitung}}
	\label{fig:Bild2}
\end{figure}
Als Lichtquelle dient eine Halogenlampe mit einem Emissionsspektrum im nahen Infrarotbereich.
Das Licht wird zunächst mittels einer Kondensorlinse auf das erste Glan-Thompson-Prisma fokussiert, welches das Licht linear polarisiert.\\
Im Mittelpunkt eines großen Elektromagneten befindet sich die zu überprüfende Probe.
Nach dem Austreten aus dem Medium trifft das Licht auf einen Interferenzfilter, welcher das Licht monochromatisiert und somit die Messung der Faraday-Rotation für fest definierte Lichtwellenlängen ermöglicht.\\
Anschließend passiert das Licht ein zweites Glan-Thompson-Prisma das
benötigt wird, um zwei senkrecht zueinander polarisierte Strahlenbündel zu erzeugen, mit denen
schließlich ein Zweistrahlverfahren durchgeführt werden kann. Dadurch kann der Drehwinkel $\theta$ mit einer höheren Präzision gemessen werden. Die beiden linear polarisierten Lichtstrahlen, welche aus dem
Glan-Thompson-Prisma austreten, werden jeweils auf einen Photowiderstand ausgerichtet, um die
Lichtintensität zu messen. An den mit Gleichstrom betriebenen Photowiderständen wird nun in
Abhängigkeit der Lichtintensität ein Spannungsabfall beobachtet. Dieser entsteht, da die auftreffenden Photonen im Halbleitermaterial der Photowiderstände Elektronen aus dem Valenzband in das Leitungsband anheben und damit die elektrische Leitfähigkeit des Materials erhöhen (photoelektrischer Effekt).\\
Da die Photowiederstände jedoch
einen hohen Innenwiederstand ($\SI{1}{\mega\ohm}$) besitzen, produzieren diese eine
Rauschspannung und es bietet sich so an, in diesem Versuch mit der sogenannten
Wechsellichtmethode zu arbeiten. Dementsprechend wird ein mechanischer Lichtzerhacker direkt hinter der
Kondensorlinse angebracht, der das Licht in Impulse fester Länge, bestimmt durch die Frequenz des Lichtzerhackers, zerhackt. Die an den Photowiederständen
abfallende Wechselspannung wird über einen Kondensator ausgekoppelt und die Differenz der
beiden Strahlen verstärkt durch einen Differenzverstärkers auf einem Oszilloskop
ausgegeben. Vor dem Oszilloskop befindet sich ein, auf die Frequenz des Lichtzerhackers abgestimmter,
Selektivverstärker. Dieser dient dazu, das Signal genau dann zu verstärken, wenn tatsächlich ein experimentell zu betrachtender Lichtimpuls auf die Photowiderstände trifft und unterdrückt somit den Einfluss des Rauschens verursacht durch das Umgebungslicht.
Gibt das Oszilloskop gerade eine Null-Linie aus, so stimmen beide Signale in Betrag und Phase überein.
\subsection{Justage der Apparatur}
Zuerst wird die Apparatur ohne Medium betrieben und es wird sichergestellt, dass der Lichtstrahl
auf die photosensitiven Flächen des Photowiederstandes trifft. Als nächstes wird die Frequenz
des Lichtzerhackers auf die des Selektivverstärkers eingestellt. Hierzu wird der Lichtzerhacker
eingeschaltet, ein Ausgang am Differenzverstärker auf "Ground" gestellt und das Oszilloskop an den
Ausgang "Resonance" angeschlossen. Dann wird die Frequenz am Selektivverstärker, angefangen
bei der Frequenz des Lichtzerhackers, so eingestellt, dass sich am Oszilloskop ein Maximalstrom
ergibt.
\subsection{Messvorgang}
\begin{itemize}
	\item Um den Drehwinkel des Lichts in einer Probe zu messen wird diese in die Probenhalterung in der Mitte des Elektromagneten gesetzt.
	Der Elektromagnet  wird zuerst auf maximalen Strom eingestellt und das Oszilloskop mittels des Drehwinkels $\theta_1$
		des ersten Glan-Thompson-Prismas auf eine Null-Linie geregelt. Danach wird das
		Feld umgepolt und wieder auf eine Null-Linie gerichtet. Hierbei wird wieder der
		Drehwinkel $\theta_2$ des Prismas am Goniometer abgelesen. Der resultierende
		Drehwinkel ergibt sich also zu:
		\begin{equation*}
			\theta = \frac{1}{2}(\theta_1-\theta_2) \, \mathrm{.}
		\end{equation*}
		Zudem werden acht verschiedene Filter verwendet, um die Faraday-Rotation anhand verschiedener Lichtwellenlängen zu untersuchen.
	\item Am Ende des Experiments wird noch die Kraftflussdichte $B(z)$ in Richtung des
		einfallenden Lichtes mittels einer Hallsonde bei maximalem Feldstrom gemessen.
\end{itemize}
