\section{Theorie}
\label{sec:Theorie}
Das Ziel dieses Versuches ist die präzise Messung der Hyperfeinstruktur- sowie
Zeemann-Aufspaltung mit Hilfe der Methode des optischen Pumpens. Weiterhin sollen die Land\'{e}-
Faktoren von zwei Rubidium-Isotopen ermittelt werden.

In der Elektronenhülle eines Atoms können die Elektronen verschiedene Energieniveaus einnehmen. 
Dabei wird immer der energetisch günstigste Zustand angestrebt, sodass die innersten Schalen
vollständig besetzt sind. Interessant für diesen Versuch sind die äußeren Niveaus, dessen 
Besetzung eine Temperaturabhängigkeit aufweist. Seien nun die beiden Energieniveaus 
$W_{\mathrm{i}}$ und $W_{\mathrm{i}+1}$ ($W_{\mathrm{i}+1} >  W_{\mathrm{i}}$) gegeben.
Dann erfüllen diese die Boltzmannsche Gleichung
\begin{equation}
	\frac{N_{\mathrm{i}+1}}{N_{\mathrm{i}}} = \frac{g_{\mathrm{i}+1}}{g_{\mathrm{i}}} \frac{\exp(-W_{\mathrm{i}+1} / \mathrm{k}_{\mathrm{B}}T)}{\exp(-W_{\mathrm{i}} / \mathrm{k}_{\mathrm{B}}T)} \, \mathrm{,}
\end{equation}
wobei $g_{\mathrm{i}}$ dem statistischen Gewicht und $N_{\mathrm{i}}$ der Besetzungszahl 
des jeweiligen Energieniveaus, 
$\mathrm{k}_{\mathrm{B}}$ der Boltzmann-Konstanten und $T$ der Temperatur im thermischen 
Gleichgewicht entspricht.
Im Normalfall sollte folglich $W_{\mathrm{i}}$ eine höhere Besetzungszahl aufweisen. 
Mit dem optischen Pumpen wird eine Inversion ($N_{\mathrm{i}+1} > N_{\mathrm{i}}$) möglich.
Wenn diese Konstellation hergestellt wid, können Übergänge zwischen den beiden Niveaus 
induziert werden, bei denen ein Photon mit der Energie 
\begin{equation}
	\mathrm{h} \nu = W_{\mathrm{i}+1} - W_{\mathrm{i}} \, \mathrm{,}
\end{equation}
emittiert wird. Die Frequenz wird hier $\nu$ genannt und h entspricht dem Planckschen 
Wirkungsquantum.

\subsection{Zusammenhang zwischen Drehimpuls, magnetischem Moment und Kernspin}

Die Elektronenhülle des Atoms besitzt einen Drehimpuls $\vec{J}$, der sich aus dem
Bahndrehimpuls $\vec{L}$ und dem Spin $\vec{S}$ zusammensetzt. An diesen koppelt das
zugehörige magnetische Moment
\begin{equation}
	\vec{\mu}_J = - g_J \mu_{\mathrm{B}} \vec{J} \, \mathrm{,}
\end{equation}
mit dem Bohrschen Magneton $\mu_{\mathrm{B}}$ und dem Land\'{e}-Faktor $g_J$.
Die Zusammensetzung des magnetischen Momentes der Elektronenhülle gemäß der
Russel-Saunders-Kopplung $\vec{\mu}_J = \vec{\mu}_L + \vec{\mu}_S$ ermöglicht unter
Verwendung des Kosinussatzes den Ausdruck
\begin{equation}
	g_J = \frac{3,0023 J(J+1) + 1,0023 (S(S+1)-L(L+1))}{2J(J+1)}
\end{equation}
für den Land\'{e}-Faktor.

Befindet sich das Atom in einem externen B-Feld, so tritt eine Aufspaltung der bisherigen 
Energieniveaus ein. Dieser Effekt wird Zeemann-Effekt genannt.
Das B-Feld wirkt auf das magnetische Moment des Atoms und dieses führt eine Präzessionsbewegung
in Feldrichtung aus, wodurch sich eine Richtungsquantelung des Drehimpulses ergibt. Damit 
ergibt sich die Wechselwirkungsenergie
\begin{equation}
	U_{\mathrm{mag}} = - \vec{\mu}_J \cdot \vec{B} = M_J g_J \mu_{\mathrm{B}} B \, \mathrm{,} \, M_J \in \{J \in \mathbb{Z} \, \vert -J \le M_J \le J\} \, \mathrm{.}
\end{equation}
Die Orientierungsquantenzahl $M_J$ folgt aus der Richtungsquantelung. 

Bei dieser Überlegung wurde allerdings noch nicht der Einfluss des Kernspins berücksichtigt.
Der Gesamtdrehimpuls des Atoms ergibt sich nämlich durch eine Drehimpulsaddition von dem 
Gesamtdrehimpuls der Elektronenhülle und dem Drehimpuls des Kerns: $\vec{F}=\vec{J}+\vec{I}$.
Durch den Kernspin, der wieder ein magnetisches Moment impliziert, welches im B-Feld der 
Elektronenhülle wieder eine Richtungsquantelung aufweist, findet eine Aufspaltung in die 
sogenannte Hyperfeinstrukturaufspaltung statt. Jeder Zustand in dieser Struktur wird durch
die Quantenzahl $F \in \{F \in \mathbb{N} \, \vert I+J \le F \le |I-J| \}$ klassifiziert.
