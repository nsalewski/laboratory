\section{Durchführung}
\label{sec:Durchführung}

\subsection{Versuchsaufbau}
\label{sec:Versuchsaufbau}
Der Versuchsaufbau besteht aus einem Ultraschallechoskop, an dessen Ausgänge zwei Ultraschallsonden mit \SI{2}{\mega\Hz} gekoppelt sind, und einem Rechner zur Datenaufnahme und -analyse.\\
An das Ultraschallechoskop sind zwei Ultraschallsonden angeschlossen, mithilfe derer sich sowohl eine Impuls-Echo-Messung, als auch eine Durchschallmessung realisieren lässt.
Am Rechner werden die gemessenen Daten mittels des Programms \textquote{Echoview} ausgewertet.\\
Hierbei ist \textquote{Echoview} in der Lage, vier verschiedene Diagramme darzustellen.
Im linken oberen Graphen wird der A-Scan dargestellt, also die Amplitude gegen die Zeit aufgetragen.
Der linke untere Graph stellt die gewählte Verstärkung dar. Die Verstärkung lässt sich am Ultraschallechoskop über die Drehknöpfe zur laufzeit-bzw. tiefenabhängigen Verstärkung (TGC; Time Gain Control) und ebenso über die Verstärkung des Outputs und der Empfindlichkeit der Sonden regulieren.\\
Zu Beachten ist, dass eine Verstärkung nur gewählt werden darf, wenn die auszuwertende Messreihe nicht zur Untersuchung der Amplitudenhöhe dient.
Die beiden rechten Graphiken sind das berechnete Spektrum der Messdaten (FFT), bzw. ihr
Cepstrum.
Erzeugte Graphiken und Messdaten können aus dem Programm heraus exportiert werden.
Als zu untersuchende Versuchsobjekte stehen Acrylzylinder verschiedener Länge, Acrylplatten unterschiedlicher Dicke sowie das Modell eines menschlichen Auges im Maßstab 3:1 zur Verfügung.


\subsection{Versuchsbeschreibung}
\label{sec:Versuchsbeschreibung}

Eine halbe Stunde vor der Messung wird der Ofen eingeschaltet, um den richtigen
Dampfdruck zu erzeugen.
Daraufhin werden die optischen Objekte justiert, sodass die Intensität maximal wird.
Dafür müssen an der Kontrollvorrichtung alle "GAIN"-Knöpfe auf "1" gestellt werden. 
Nach der Justierung wird eine schwarze Decke über den Aufbau gelegt, um externes Licht 
abzuschirmen. 

Als zweiter Schritt wird der gesamte Aufbau in Nord-Süd-Richtung gedreht, damit 
die Horizontalkomponente des Erdmagnetfeldes entweder parallel oder antiparallel 
zur Horizontalfeldspule ausgerichtet ist.
Dann wird die vertikale Komponente des Erdmagnetfeldes mit der Vertikalfeldspule 
kompensiert. Hierzu soll auf dem Oszilloskop ein möglichst scharfer Peak zu erkennen sein.
Dies geschieht mit abwechselnder Justierung der Position des Aufbaus in Nord-Süd-Richtung und 
dem Einstellen des Spulenstroms der Vertikalfeldspule.
Hierfür muss der untere Schalter für die Sweep-Spule auf "CONTINUOUS" und der Obere
auf "START" gestellt werden. Weiterhin muss für beide Kanäle am Oszilloskop DC-Kopplung
eingestellt sein. Der "GAIN" soll auf 20, der "GAIN MULTIPLIER" auf x10 und der "METER
MULTIPLIER" soll auf x2 geschaltet sein. Der Wert für den eingestellten Spulenstroms
wird notiert und nicht mehr verändert.

Nun wird das gesamte Horizontalfeld in Abhängigkeit von den Resonanzfrequenzen der beiden 
Rubidium-Isotope gemessen. Hierfür wird die Frequenz (Sinus-Spannung) im Bereich
von $\SI{100}{\kilo\hertz}-\SI{1}{\mega\hertz}$ in $\SI{100}{\kilo\hertz}$-Schritten
variiert.
Für höhere Frequenzen (ab $\SI{200}{\kilo\hertz}$) wird ein zusätzliches Horizontalfeld 
benötigt, damit die Resonanzen im Sweet-Feld-Bereich liegen.
Es werden jeweils für die Frequenzen die beiden Resonanzen notiert. Der Strom bei dem 
die Resonanzen auftreten kann am Potentiometer abgelesen werden, wobei eine Umdrehung
$\SI{0,1}{\ampere}$ entspricht.
