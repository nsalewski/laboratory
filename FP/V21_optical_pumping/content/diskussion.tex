\section{Diskussion}
\label{sec:Diskussion}
Das Experiment ist äußerst empfindlich für äußere Einflüsse. Leichte Bewegungen oder Positionsänderungen in der Nähe des Experiments verursachten große Schwankungen im Oszilloskopbild.\\
Es könnten daher prinzipiell unbemerkte systematische Fehler aufgetreten sein.

Da das Amplitudenverhältnis nur aus einem abfotografierten Oszilloskopbild bestimmt wurde und der Wert maximaler Transparenz nicht ganz eindeutig bestimmbar ist, sind besonders hier Ablesefehler nicht auszuschließen.\\
In der Dampfkammer lag ein Isotopenverhältnis von $\frac{A_1}{A_2}=\frac{\ce{^87Rb}}{\ce{^85Rb}}\approx 0.58\pm 0.02$ vor. Natürlich kommt ein Verhältnis von $\frac{\ce{^87Rb}}{\ce{^85Rb}}\approx 0.39$ vor \cite{muenster}.\\
Nach unserer Messung ist also die Konzentration von $\ce{^87Rb}$ deutlich höher, als seine Konzentration im natürlichen Vorkommen.\\
Die vorliegende Messung zeigt eine Abweichung von etwa $33\%$ im Verhältnis der Rubidium-Isotope im Vergleich zum natürlichen Vorkommen.\\
Wie bereits erwähnt, könnte das bestimmte Amplitudenverhältnis allerdings aufgrund von Ablesefehlern ungenau sein.

In Tabelle \ref{tab:discuss} findet sich ein Vergleich der experimentell bestimmten Werte mit den Theoriewerten.\\
Die größte Abweichung zeigt sich in der Bestimmung der Vertikalkomponente des Erdmagnetfelds. Der experimentelle Wert errechnete sich lediglich über einen Messwert, ein Fehler hierbei würde sich also direkt deutlich auswirken.\\
Alle anderen experimentellen Werte liegen allerdings sehr nah an den Theoriewerten, besonders in der Messung der Kernspins und der Landéschen Faktoren (Abweichungen $<1\%$) scheint also kein systematischer Fehler vorliegen zu können.\\
\begin{table}
  \caption{Vergleich der bestimmten Größen mit den Literaturwerten}
  \label{tab:discuss}
 \centering
 \begin{tabular}{cccc}
   \toprule
Messgröße&Experiment&Theorie&prozentuale Abweichung\\
\midrule
Erdmagnetfeld, vertikal&$\SI{35400}{\nano\tesla}$&$\SI{45012.9}{\nano\tesla}$&$21.3\%$\\
Erdmagnetfeld, horizontal&$\SI{18.3(5)e-6}{\tesla}$&$\SI{19304.4}{\nano\tesla}$&$5.2\%$\\
Kernspin $I$ $\ce{^87Rb}$&$1.491\pm0.021$&$\frac{3}{2}$&$0.5\%$\\
$g_{\mathrm{F}}$ von $\ce{^87Rb}$&$0.503\pm0.005$&$\frac{1}{2}$&$0.6\%$\\
Kernspin $I$ $\ce{^85Rb}$&$2.515\pm0.009$&$\frac{5}{2}$&$0.6\%$\\
$g_{\mathrm{F}}$ von $\ce{^85Rb}$&$0.332\pm0.001$&$\frac{1}{3}$&$0.4\%$\\
\bottomrule
\end{tabular}
\end{table}
Die Theoriedaten der Magnetfeldstärken des Erdmagnetfelds wurden hierbei mit dem Deklinationsrechner des GFZ Potsdam für Dortmund ermittelt \cite{dekli}.
Die Theoriewerte der Kernspins ergeben sich nach \cite{muenster}, die Theoriewerte der Landéschen Faktoren nach \cite{gf} für $\ce{^87Rb}$ und \cite{gf2} für $\ce{^85Rb}$.
