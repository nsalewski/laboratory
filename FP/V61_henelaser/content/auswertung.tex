\section{Auswertung}
\label{sec:Auswertung}
Im Folgenden werden die aufgenommenen Messdaten ausgewertet, um die Polarisation, die Stabilitätsmessung für unterschiedliche Resonatorspiegel, die Grundmode sowie die erste angeregte Mode und die Wellenlänge des Lasers
zu untersuchen.
\subsection{Bestimmung der Polarisation des Laserstrahls}
Die Messdaten sind in Tabelle~\ref{tab:polarisation} dargestellt. Die Intensität $I(\phi)$ wurde mittels einer Photodiode in Abhängigkeit des Drehwinkels $\phi$ des Polarisationsfilters gemessen. Für den gemessenen Photostrom wurde aufgrund der schwankenden Intensität ein Ablesefehler von $\delta I=\SI{0.1}{\micro\ampere}$ und für den Polarisationswinkel aufgrund des Ablesefehlers ein Fehler von $\delta \phi=\SI{1}{\degree}$ angenommen.
Die Messdaten werden gefittet an eine Funktion der Form:
\begin{equation}
    I(\phi)=I_0\cdot \sin{\phi-\phi_0}^2 \,\text{.}
\end{equation}
Hierbei stellt $I_0$ die Maximalintensität und $\phi_0$ den Anfangswinkel der Polarisation dar.
Die Parameter der mittels python/scipy~\cite{scipy} durchgeführten Ausgleichsrechnung ergeben sich zu:
\begin{align}
  I_0&=\SI{16.4(6)}{\micro\ampere}\,\text {;}\\ \phi_0&=\SI{-3.50(3)}{\radian}\,\text{.}
\end{align}
In Abbildung~\ref{fig:polarisation} sind die Messdaten samt zugehöriger Regressionsfunktion dargestellt.
\begin{figure}
  \centering
  \includegraphics[width=0.9\columnwidth]{daten/polarisation.pdf}
  \caption{Darstellung der Messdaten der Polarisationsmessung samt zugehöriger Regressionsfunktion.}
  \label{fig:polarisation}
\end{figure}
\begin{table}
 \caption{Messdaten der Polarisationsmessung}
 \label{tab:polarisation}
 \centering
\sisetup{table-format=3.3} \begin{tabular}{SS}
 \toprule 
    {Polarisationsmessung $\phi$/$\si{\degree}$}& {Intensität $I_{\mathrm{Pol}}$/$\si{\micro\ampere}$} \\
     \midrule
     $\num{0.0 \pm 1.0}$ & $\num{0.02\pm0.10}$ \\
     $\num{0.2 \pm 1.0}$ & $\num{0.38\pm0.10}$ \\
     $\num{0.3 \pm 1.0}$ & $\num{2.20\pm0.10}$ \\
     $\num{0.5 \pm 1.0}$ & $\num{4.20\pm0.10}$ \\
     $\num{0.7 \pm 1.0}$ & $\num{5.50\pm0.10}$ \\
     $\num{0.9 \pm 1.0}$ & $\num{6.30\pm0.10}$ \\
     $\num{1.0 \pm 1.0}$ & $\num{11.50\pm0.10}$ \\
     $\num{1.2 \pm 1.0}$ & $\num{16.20\pm0.10}$ \\
     $\num{1.4 \pm 1.0}$ & $\num{17.60\pm0.10}$ \\
     $\num{1.6 \pm 1.0}$ & $\num{18.00\pm0.10}$ \\
     $\num{1.7 \pm 1.0}$ & $\num{15.00\pm0.10}$ \\
     $\num{1.9 \pm 1.0}$ & $\num{16.90\pm0.10}$ \\
     $\num{2.1 \pm 1.0}$ & $\num{12.20\pm0.10}$ \\
     $\num{2.3 \pm 1.0}$ & $\num{7.50\pm0.10}$ \\
     $\num{2.4 \pm 1.0}$ & $\num{4.40\pm0.10}$ \\
     $\num{2.6 \pm 1.0}$ & $\num{2.90\pm0.10}$ \\
     $\num{2.8 \pm 1.0}$ & $\num{1.10\pm0.10}$ \\
     $\num{3.0 \pm 1.0}$ & $\num{0.20\pm0.10}$ \\
     $\num{3.1 \pm 1.0}$ & $\num{0.03\pm0.10}$ \\
 \bottomrule
 \end{tabular}
\end{table}
\subsection{Messung von Schwingungsmoden}
Die Messdaten der Messung der Grundmode $\text{TEM}_{(00)}$ finden sich in Tabelle~\ref{tab:t00} und die der ersten angeregten Mode $\text{TEM}_{(01)}$ in Tabelle~\ref{tab:t01}.
Für die Verschiebung der Photodiode senkrecht zur Strahlachse wird ein Ablesefehler von $\Delta L=\SI{0.5}{\milli\meter}$ angenommen.
Aufgrund der schwankenden Intensität wird für die Messung der $\text{TEM}_{(00)}$-Mode ein Fehler von $\Delta I=\SI{0.2}{\micro\ampere}$ und für die Messung der $\text{TEM}_{(01)}$-Mode ein Fehler von $\Delta I=\SI{0.05}{\micro\ampere}$ für die Intensität angenommen.
Die Grundmode entspricht dabei der Form einer Gaußglocke:
\begin{equation}
  I_{(00)}(L)=I_0 \exp{-2\left(\frac{L-d_0}{\omega})^2\right)}\,\text{.}
\end{equation}
Es ist $I_0$ erneut die Maximalintensität, $d_0$ ist eine Verschiebung der Gaußglocke entlang der $x$-Achse und $\omega$ der Radius der Fundamentalmode. $L$ ist die gemessene Verschiebung der Photodiode senkrecht zur Strahlachse des Lasers.
Die erste angeregte Mode ist an eine doppelte Gaußglocke mit verschiedener Maximalintensität $I_{(0,i)}$ gefittet:
\begin{equation}
    I_{(01)}(L)=I_{(0,1)} \exp{-2\left( \frac{L-d_{(0,1)}}{\omega_1})^2\right)}+I_{(0,2)} \exp{-2\left( \frac{L-d_{(0,2)}}{\omega_2})^2\right)}
\end{equation}
In Abbildung~\ref{fig:tem00} sind die Messdaten der $\text{TEM}_{(00)}$-Mode samt der mittels python/scipy ermittelten Regressionsfunktion dargestellt.
Die zugehörigen Regressionsparameter ergeben sich zu:
\begin{align}
  d_{00}&=\SI{1.069(5)}{\milli\meter}\,\text{;}\\ w_{00}&=\SI{0.60(1)}{\centi\meter}\,\text{;}\\ I_{00}&=\SI{11.5(2)}{\micro\ampere}\,\text{.}
\end{align}
\begin{figure}
  \centering
  \includegraphics[width=0.9\columnwidth]{daten/tem00.pdf}
  \caption{Messdaten samt zugehöriger Regressionsfunktion der Grundmode $\text{TEM}_{(00)}$.}
  \label{fig:tem00}
\end{figure}
\begin{table}
 \caption{Messdaten der Messung der Grundmode $T_{00}$}
 \label{tab:t00}
 \centering
\sisetup{table-format=3.3} \begin{tabular}{SS}
 \toprule 
    {Verschiebung $\Delta L$/$\si{\centi\meter}$}& {Intensität $I_{\mathrm{00}}$/$\si{\micro\ampere}$} \\
     \midrule
     $\num{0.00 \pm 0.05}$ & $\num{0.13\pm0.20}$ \\
     $\num{0.10 \pm 0.05}$ & $\num{0.28\pm0.20}$ \\
     $\num{0.20 \pm 0.05}$ & $\num{0.43\pm0.20}$ \\
     $\num{0.30 \pm 0.05}$ & $\num{0.81\pm0.20}$ \\
     $\num{0.40 \pm 0.05}$ & $\num{1.34\pm0.20}$ \\
     $\num{0.50 \pm 0.05}$ & $\num{2.20\pm0.20}$ \\
     $\num{0.60 \pm 0.05}$ & $\num{3.80\pm0.20}$ \\
     $\num{0.70 \pm 0.05}$ & $\num{5.50\pm0.20}$ \\
     $\num{0.80 \pm 0.05}$ & $\num{7.10\pm0.20}$ \\
     $\num{0.90 \pm 0.05}$ & $\num{9.40\pm0.20}$ \\
     $\num{1.00 \pm 0.05}$ & $\num{11.10\pm0.20}$ \\
     $\num{1.10 \pm 0.05}$ & $\num{12.20\pm0.20}$ \\
     $\num{1.20 \pm 0.05}$ & $\num{10.20\pm0.20}$ \\
     $\num{1.30 \pm 0.05}$ & $\num{8.90\pm0.20}$ \\
     $\num{1.40 \pm 0.05}$ & $\num{6.30\pm0.20}$ \\
     $\num{1.50 \pm 0.05}$ & $\num{3.80\pm0.20}$ \\
     $\num{1.60 \pm 0.05}$ & $\num{2.30\pm0.20}$ \\
     $\num{1.70 \pm 0.05}$ & $\num{0.90\pm0.20}$ \\
     $\num{1.80 \pm 0.05}$ & $\num{0.46\pm0.20}$ \\
     $\num{1.90 \pm 0.05}$ & $\num{0.26\pm0.20}$ \\
     $\num{2.00 \pm 0.05}$ & $\num{0.10\pm0.20}$ \\
     $\num{2.10 \pm 0.05}$ & $\num{0.01\pm0.20}$ \\
 \bottomrule
 \end{tabular}
\end{table}
\begin{table}
 \caption{Messdaten der Messung der $T_{01}$-Mode}
 \label{tab:t01}
 \centering
\sisetup{table-format=3.3} \begin{tabular}{SS}
 \toprule 
    {Verschiebung $\Delta L$/$\si{\centi\meter}$}& {Intensität $I_{\mathrm{01}}$/$\si{\micro\ampere}$} \\
     \midrule
     $\num{0.00 \pm 0.05}$ & $\num{0.13\pm0.05}$ \\
     $\num{0.10 \pm 0.05}$ & $\num{0.35\pm0.05}$ \\
     $\num{0.20 \pm 0.05}$ & $\num{0.64\pm0.05}$ \\
     $\num{0.30 \pm 0.05}$ & $\num{0.93\pm0.05}$ \\
     $\num{0.40 \pm 0.05}$ & $\num{1.30\pm0.05}$ \\
     $\num{0.50 \pm 0.05}$ & $\num{1.50\pm0.05}$ \\
     $\num{0.60 \pm 0.05}$ & $\num{1.40\pm0.05}$ \\
     $\num{0.70 \pm 0.05}$ & $\num{1.00\pm0.05}$ \\
     $\num{0.80 \pm 0.05}$ & $\num{0.54\pm0.05}$ \\
     $\num{0.90 \pm 0.05}$ & $\num{0.15\pm0.05}$ \\
     $\num{1.00 \pm 0.05}$ & $\num{0.01\pm0.05}$ \\
     $\num{1.10 \pm 0.05}$ & $\num{0.07\pm0.05}$ \\
     $\num{1.20 \pm 0.05}$ & $\num{0.37\pm0.05}$ \\
     $\num{1.30 \pm 0.05}$ & $\num{0.79\pm0.05}$ \\
     $\num{1.40 \pm 0.05}$ & $\num{1.00\pm0.05}$ \\
     $\num{1.50 \pm 0.05}$ & $\num{0.80\pm0.05}$ \\
     $\num{1.60 \pm 0.05}$ & $\num{0.53\pm0.05}$ \\
     $\num{1.70 \pm 0.05}$ & $\num{0.37\pm0.05}$ \\
     $\num{1.80 \pm 0.05}$ & $\num{0.22\pm0.05}$ \\
     $\num{1.90 \pm 0.05}$ & $\num{0.14\pm0.05}$ \\
     $\num{2.00 \pm 0.05}$ & $\num{0.04\pm0.05}$ \\
 \bottomrule
 \end{tabular}
\end{table}
Die Regressionsparameter der Ausgleichsrechnung der ersten angeregten Mode ergeben sich zu:
\begin{align}
  d_{01,1}&=\SI{0.495(9)}{\milli\meter}\,\text{;}\\
  w_{01,1}&=\SI{0.42(2)}{\centi\meter} \,\text{;}\\
  I_{01,1}&=\SI{1.51(6)}{\micro\ampere}\,\text{;}\\
  d_{01,2}&=\SI{1.43(1)}{\milli\meter} \,\text{;}\\
  w_{01,1}&=\SI{0.35(3)}{\centi\meter} \,\text{;}\\
  I_{01,2}&=\SI{0.9(6)}{\micro\ampere}\,\text{.}\\
\end{align}
In Abbildung~\ref{fig:tem01} finden sich die Messdaten samt ermittelter Regressionsrechnung.
\begin{figure}
  \centering
  \includegraphics[width=0.9\columnwidth]{daten/tem01.pdf}
  \caption{Messdaten samt zugehöriger Regressionsfunktion der ersten angeregten Mode $\text{TEM}_{(01)}$.}
  \label{fig:tem01}
\end{figure}
\subsection{Stabilitätsmessung für verschiedene Resonatorspiegel}
Die Messdaten der Stabilitätsmessung des Lasers für verschiedene Resonatorspiegel sind in Tabelle~\ref{tab:stabilitaetpk} beziehungsweise in Tabelle~\ref{tab:stabilitaetkk} dargestellt.
Da die Resonatorlänge mittels eines Maßbandes vermessen wurde, dessen Nullanschlag nicht stabil fixiert war und welches sich zudem leicht durchbog, wird eine Ungenauigkeit von $\Delta L=\SI{3}{\milli\meter}$ angenommen.
Für die Intensitätsmessung wird bei zwei konkaven Resonatoren ein Ablesefehler von $\Delta I=\SI{0.2}{\micro\ampere}$ und für die Resonatorkombination planar-konkav eine Unsicherheit von $\delta I=\SI{0.025}{\micro\ampere}$ angenommen.
Die Messdaten des Resonators mit der Spiegelkombination konkav-konkav wurden dabei an eine quadratische Funktion der Form
\begin{equation}
  I_{\text{kk}}=a\cdot L^2+b\cdot L+c
\end{equation}
gefittet.
Die zugehörigen Parameter ergeben sich zu:
\begin{align}
  a&=\SI{6(442)e-6}{\micro\ampere\per\square\centi\meter}\,\text{;}\\
  b&=-\SI{88(88)e-2}{\micro\ampere\per\centi\meter} \,\text{;}\\
  c&=\SI{11(4)}{\micro\ampere}\,\text{.}
\end{align}
Die Messdaten samt zugehöriger Regression sind in Abbildung~\ref{fig:kk} dargestellt.
\begin{figure}
  \centering
  \includegraphics[width=0.9\columnwidth]{daten/stabilitaetkp.pdf}
  \caption{Messdaten samt zugehöriger Regressionsfunktion zur Messung der Stabilität des Lasers bei der Resonatorspiegelkombination planar-konkav.}
  \label{fig:pk}
\end{figure}
\begin{table}
 \caption{Messdaten der Stabilitätsmessung für zwei konkave Resonatorspiegel}
 \label{tab:stabilitaetkk}
 \centering
\sisetup{table-format=3.3} \begin{tabular}{SS}
 \toprule 
    {Resonatorlänge $ L$/$\si{\centi\meter}$}& {Intensität $I_{\mathrm{konkav-konkav}}$/$\si{\micro\ampere}$} \\
     \midrule
     $\num{64.80 \pm 0.30}$ & $\num{6.90\pm0.20}$ \\
     $\num{74.20 \pm 0.30}$ & $\num{4.60\pm0.20}$ \\
     $\num{84.00 \pm 0.30}$ & $\num{4.30\pm0.20}$ \\
     $\num{94.10 \pm 0.30}$ & $\num{3.90\pm0.20}$ \\
     $\num{104.90 \pm 0.30}$ & $\num{3.20\pm0.20}$ \\
     $\num{113.90 \pm 0.30}$ & $\num{1.90\pm0.20}$ \\
     $\num{124.00 \pm 0.30}$ & $\num{1.40\pm0.20}$ \\
     $\num{134.10 \pm 0.30}$ & $\num{0.00\pm0.20}$ \\
 \bottomrule
 \end{tabular}
\end{table}
Für den Resonatorspiegel mit der Kombination planar-konkav werden die Messdaten an eine lineare Regression der Form:
\begin{equation}
  I_{\text{pk}}=a\cdot L+b
\end{equation}
gefittet.
Die Parameter ergeben sich zu:
\begin{align}
    a&=-\SI{13(5)e-3}{\micro\ampere\per\centi\meter}\,\text{;}\\
  b&=\SI{1.6(4)}{\micro\ampere}\,\text{.}
\end{align}
In Abbildung~\ref{fig:pk} sind die Messdaten samt Regressiongeraden geplottet.
\begin{table}
 \caption{Messdaten der Stabilitätsmessung für einen konkaven und einen planaren Resonatorspiegel}
 \label{tab:stabilitaetpk}
 \centering
\sisetup{table-format=3.3} \begin{tabular}{SS}
 \toprule 
    {Resonatorlänge $\Delta L$/$\si{\centi\meter}$}& {Intensität $I_{\mathrm{planar-konkav}}$/$\si{\micro\ampere}$} \\
     \midrule
     $\num{85.00 \pm 0.30}$ & $\num{0.900\pm0.025}$ \\
     $\num{75.00 \pm 0.30}$ & $\num{0.600\pm0.025}$ \\
     $\num{65.30 \pm 0.30}$ & $\num{0.700\pm0.025}$ \\
     $\num{55.10 \pm 0.30}$ & $\num{0.900\pm0.025}$ \\
     $\num{88.50 \pm 0.30}$ & $\num{0.300\pm0.025}$ \\
     $\num{90.50 \pm 0.30}$ & $\num{0.100\pm0.025}$ \\
     $\num{94.50 \pm 0.30}$ & $\num{0.100\pm0.025}$ \\
     $\num{110.50 \pm 0.30}$ & $\num{0.400\pm0.025}$ \\
     $\num{100.70 \pm 0.30}$ & $\num{0.100\pm0.025}$ \\
 \bottomrule
 \end{tabular}
\end{table}
\begin{figure}
  \centering
  \includegraphics[width=0.9\columnwidth]{daten/stabilitaetkk.pdf}
  \caption{Messdaten samt zugehöriger Regressionsfunktion zur Messung der Stabilität des Lasers bei der Resonatorspiegelkombination konkav-konkav.}
  \label{fig:kk}
\end{figure}

\subsection{Bestimmung der Wellenlänge}
\begin{table}
 \caption{Gemessene Abstände $r_i$ der Nebenmaxima zum Hauptmaximum und daraus berechnete Wellenlängen $\lambda_i$.}
 \label{tab:wellenlaenge}
 \centering
\sisetup{table-format=3.2} \begin{tabular}{rrrrr}
 \toprule 
    {Nr. des Nebenmaxima}& {Abstand $r_{\mathrm{rechts}}$/$\si{\centi\meter}$}& {$\lambda_{\mathrm{rechts}}$/$\si{\nano\meter}$}& {Abstand $r_{\mathrm{links}}$/$\si{\centi\meter}$}& {$\lambda_{\mathrm{links}}$/$\si{\nano\meter}$} \\
     \midrule
              1 & $\num{2.05 \pm 0.05}$ & $\num{664 \pm 16}$ & $\num{2.05 \pm 0.05}$ & $\num{664\pm16}$ \\
              2 & $\num{4.10 \pm 0.05}$ & $\num{660 \pm 8}$ & $\num{4.05 \pm 0.05}$ & $\num{652\pm8}$ \\
              3 & $\num{6.10 \pm 0.05}$ & $\num{648 \pm 5}$ & $\num{6.00 \pm 0.05}$ & $\num{637\pm5}$ \\
 \bottomrule
 \end{tabular}
\end{table}
Die Wellenlänge des Lasers lässt sich anhand des Abstands der Beugungsreflexe auf dem Schirm ermitteln.
Die Berechnung der Wellenlänge erfolgt über die Beziehung:
\begin{equation}
  \lambda=\frac{g}{n}\cdot\sin{\arctan{\frac{d_n}{L}}}\,\text{.}
\end{equation}
Es ist $g=\SI{0.01}{\milli\meter}$ die Gitterkonstante, $d_n$ der Abstand des Hauptmaximums zu den Nebenmaxima, $L=\SI{30.8(1)}{\centi\meter}$ der Abstand zwischen Beugungsgitter und Schirm, sowie $n$ die Ordnung des Beugungsmaximums, wobei $n$ ausgehend vom Hauptmaximum mit $n=0$ aufsteigend mit größer werdenden Abstand der Nebenmaxima gezählt wird.
Für die Messung der Abstände der Nebenmaxima zum Hauptmaximum wird ein Ablesefehler von $\Delta d=\SI{0.5}{\milli\meter}$ angenommen.
In Tabelle~\ref{tab:wellenlaenge} finden sich die ermittelten Messdaten samt der daraus berechneten Wellenlänge für jedes Nebenmaximum.
Eine Mittelung über alle Wellenlängen ergibt die mittlere Wellenlänge
$\lambda_{\mathrm{mean}}=\SI{654(4)}{\nano\meter}$
als beste Schätzung der tatsächlichen Laserwellenlänge.
