\section{Diskussion}
\label{sec:Diskussion}
Die Wellenlängenmessung des Lasers ergibt eine ermittelte Wellenlänge von $\lambda_{\mathrm{mean}}=\SI{654(4)}{\nano\meter}$.
Der Theoriewert der Laserwellenlänge beträgt $\lambda_{\mathrm{mean}}=\SI{632.82}{\nano\meter}$. Dies entspricht einer Abweichung von $3.3\%$. Ein möglicher Grund für die Ungenauigkeit sind Ablesefehler. Diese wurden allerdings bestmöglichst abgeschätzt und bereits berücksichtigt. Möglicherweise ist daher die angegebene Gitterkonstante des verwendeten Gitters nicht korrekt. Eine genauere Messung ließe sich für einen größeren Abstand zwischen Schirm und Gitter erreichen, da sich hierdurch mögliche Ungenauigkeiten beim Ablesen der Abstände der Beugungsmaxima weniger stark auf das Endergebnis auswirken würden. Dies war mit dem vorliegenden Aufbau nicht realisierbar, da der verwendete Schirm zu klein war um auch für große Abstände genügend Beugungsmaxima abzudecken.
Sämtliche Intensitätsmessungen unterliegen einer möglichen Ungenauigkeit durch eine nicht immer optimalen Ausrichtung des Eintrittsspalts der Photodiode.
Denkbar ist eine leichte Verkippung der Photodiode bezüglich des Laserstrahls. Zudem war die Halterung der Photodiode zu kurz, sodass der Laserstrahl zum Teil nur die obere Hälfte des Eintrittsspalts der Photodiode traf.

Die Stabilitätsmessung zeigt prinzipiell für beide Resonatorkombinationen den erwarteten Verlauf mit abnehmender Intensität für große Resonatorlängen. Der quadratische Verlauf für zwei konkave Resonatoren zeigt sich allerdings kaum. Ein möglicher Grund sind Intensitätsschwankungen. Der undurchlässige Resonatorspiegel musste ebenso wie der Laser zum Teil nachjustiert werden, sodass nicht garantiert werden kann, dass der Laser für jede Resonatorlänge optimal ausgerichtet war. Dies zeigt sich vor allem in den zum Teil deutlich abweichenden Messwerten der planar-konkaven Resonatorspiegelkombination.
Hier ließen sich der Laser nur schwer zum lasen bringen, weshalb von der äquidistanten Datenaufnahme abgewichen wurde und zusätzlich einige Zwischenwerte aufgezeichnet wurden.

Die Modenmessung zeigt prinzipiell den erwarteten Verlauf. Für die Grundmode ließ sich die erwartete Gaußverteilung messen. Für die erste angeregte Mode $\text{TEM}_{01}$ zeigt sich ebenfalls der erwartete Verlauf der Intensitätsverteilung mit zwei relativ zueinander verschobenen Gaußglocken. Die unterschiedlichen Maximalintensitäten der beiden Gaußglocken lassen sich durch eine nicht exakte Ausrichtung des Golddrahts erklären, welcher zur Ausblendung der Grundmode verwendet wurde.
Möglicherweise wirft dieser einen Schatten auf die rechte Gaußglocke und schwächt so deren Intensität ab.

Die Polarisationsmessung zeigt prinzipiell den erwarteten Verlauf. Eindeutigere Aussagen, ob die Intensität tatsächlich eine $\sin{\phi}^2$-Abhängigkeit aufweißt, ließe sich mit einer Messung über den vollen Winkelbereich, sodass beide Maxima des quadratischen Sinus betrachtet würden, überprüfen.
