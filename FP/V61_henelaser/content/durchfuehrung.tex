\section{Durchführung}
\label{sec:Durchführung}

\subsection{Versuchsaufbau}
\label{sec:Versuchsaufbau}
Der Versuchsaufbau besteht aus einem Ultraschallechoskop, an dessen Ausgänge zwei Ultraschallsonden mit \SI{2}{\mega\Hz} gekoppelt sind, und einem Rechner zur Datenaufnahme und -analyse.\\
An das Ultraschallechoskop sind zwei Ultraschallsonden angeschlossen, mithilfe derer sich sowohl eine Impuls-Echo-Messung, als auch eine Durchschallmessung realisieren lässt.
Am Rechner werden die gemessenen Daten mittels des Programms \textquote{Echoview} ausgewertet.\\
Hierbei ist \textquote{Echoview} in der Lage, vier verschiedene Diagramme darzustellen.
Im linken oberen Graphen wird der A-Scan dargestellt, also die Amplitude gegen die Zeit aufgetragen.
Der linke untere Graph stellt die gewählte Verstärkung dar. Die Verstärkung lässt sich am Ultraschallechoskop über die Drehknöpfe zur laufzeit-bzw. tiefenabhängigen Verstärkung (TGC; Time Gain Control) und ebenso über die Verstärkung des Outputs und der Empfindlichkeit der Sonden regulieren.\\
Zu Beachten ist, dass eine Verstärkung nur gewählt werden darf, wenn die auszuwertende Messreihe nicht zur Untersuchung der Amplitudenhöhe dient.
Die beiden rechten Graphiken sind das berechnete Spektrum der Messdaten (FFT), bzw. ihr
Cepstrum.
Erzeugte Graphiken und Messdaten können aus dem Programm heraus exportiert werden.
Als zu untersuchende Versuchsobjekte stehen Acrylzylinder verschiedener Länge, Acrylplatten unterschiedlicher Dicke sowie das Modell eines menschlichen Auges im Maßstab 3:1 zur Verfügung.


\subsection{Versuchsbeschreibung}
\label{sec:Versuchsbeschreibung}
Mittels der Interferenz an einem optischen Gitter soll die Wellenlänge des erzeugten Laserstrahls bestimmt werden. Hierzu werden das Gitter
und ein Schirm hinter dem Laser positioniert. Aus einer Messung des Abstandes zwischen Schirm und Gitter, sowie den auftretenden Intensitätsmaxima
auf dem Schirm kann die Wellenlänge berechnet werden:
\begin{equation}
	  \lambda = \frac{g \cdot \sin(\varphi)}{n}, \quad \varphi = \arctan\left(\frac{d_n}{L} \right) ,\quad n \in \mathbb{N}
	    \label{eq:interferenz}
\end{equation}
Dabei ist $g$ die Gitterkonstante, $d_n$ der Abstand des $n$-ten Interferenzmaximums zum Hauptmaximum und $L$ der Abstand zwichen Schirm und Gitter.

Zur Untersuchung der TEM-Moden wird eine defokussierende Linse hinter den Laser gestellt, die den Laserstrahl verbreitert.
Mittels einer Mikrometerschraube kann die Photodiode
senkrecht zur Strahlachse verschoben werden und somit die Abhängigkeit zwischen Intensität und Achsenabstand untersucht werden. Zunächst
wird die Grundmode untersucht, die ohne Einsatz von Blenden gegenüber anderen Moden dominiert. Anschließeden wird die $I_{01}$-Mode
vermessen, welche eine Nullstelle bei $r = 0$ besitzt. Innherhalb des Resonators wird ein Wolframdraht so positioniert, dass die Grundmode
unterdrückt wird. Anschließend kann die $I_{01}$-Mode analog mit der Photodiode ausgemessen werden.

An den Ausgängen des Laserrohrs sind Brewster-Fenster angebracht. Hierbei handelt es sich um gläserne Platten, die
im Brewsterwinkel zu optischen Achse eingestellt werden. Gemäß den Fresnelschen Gleichungen ist der Brewsterwinkel jener Winkel, unter dem
die zur Einfallsebene parallele Komponente des elektrischen Lichtfeldes nicht reflektiert wird. Die senkrechte Komponente wird somit
durch Refelxion stark unterdrückt und das Licht linear polarisiert. Diese Gegebenheit soll
experimentell untersucht werden. Hinter den Laser wird hierzu ein drehbarer Polarisationsfilter platziert. In Abhängigkeit zum eingestellten Winkel
wird der Photostrom gemessen. Nach Malus ist die Intensität der linear polarisierten Lichtwelle hinter dem Polarisationsfilter
durch folgende Funktion zu beschreiben:
\begin{equation}
	  I(\varphi) = I_{0} \sin^2\left(\varphi + \varphi_0\right)
	    \label{eq:malus}
\end{equation}
Es sind $I_{0}$ und $\varphi_0$ die konstanten Parameter und $\varphi$ der relativ zu einer beliebigen Startachse eingestellte Winkel.

Abschließend soll die Stabilitätsbedingung \eqref{eq:stabilität} für
einen gekrümmten Spiegel $r = \SI{1.4}{\meter}$ und einen ebenen Spiegel quantitativ untersucht werden. Unter variabler Resonatorlänge $L$ wird hierzu der
Photostrom aufgezeichnet. Hierbei ist es notwendig bei jedem Abstand den Laser neu zu justieren und den Photostrom zu maximieren.
