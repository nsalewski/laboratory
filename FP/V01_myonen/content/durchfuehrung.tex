\section{Durchführung}
\label{sec:Durchführung}

\subsection{Versuchsaufbau}
\label{sec:Versuchsaufbau}
Der Versuchsaufbau besteht aus einem Ultraschallechoskop, an dessen Ausgänge zwei Ultraschallsonden mit \SI{2}{\mega\Hz} gekoppelt sind, und einem Rechner zur Datenaufnahme und -analyse.\\
An das Ultraschallechoskop sind zwei Ultraschallsonden angeschlossen, mithilfe derer sich sowohl eine Impuls-Echo-Messung, als auch eine Durchschallmessung realisieren lässt.
Am Rechner werden die gemessenen Daten mittels des Programms \textquote{Echoview} ausgewertet.\\
Hierbei ist \textquote{Echoview} in der Lage, vier verschiedene Diagramme darzustellen.
Im linken oberen Graphen wird der A-Scan dargestellt, also die Amplitude gegen die Zeit aufgetragen.
Der linke untere Graph stellt die gewählte Verstärkung dar. Die Verstärkung lässt sich am Ultraschallechoskop über die Drehknöpfe zur laufzeit-bzw. tiefenabhängigen Verstärkung (TGC; Time Gain Control) und ebenso über die Verstärkung des Outputs und der Empfindlichkeit der Sonden regulieren.\\
Zu Beachten ist, dass eine Verstärkung nur gewählt werden darf, wenn die auszuwertende Messreihe nicht zur Untersuchung der Amplitudenhöhe dient.
Die beiden rechten Graphiken sind das berechnete Spektrum der Messdaten (FFT), bzw. ihr
Cepstrum.
Erzeugte Graphiken und Messdaten können aus dem Programm heraus exportiert werden.
Als zu untersuchende Versuchsobjekte stehen Acrylzylinder verschiedener Länge, Acrylplatten unterschiedlicher Dicke sowie das Modell eines menschlichen Auges im Maßstab 3:1 zur Verfügung.


\subsection{Versuchsbeschreibung}
\label{sec:Versuchsbeschreibung}
\subsubsection{Kalibration des Messaufbaus}
Zunächst wird der für die Rauschunterdrückung gedachte Teil des Versuchsaufbaus aufgebaut und mittels eines Oszillosgraphens geprüft.
Nachdem die Hochspannungs an den SEV eingeschaltet ist, werden die an den Ausgängen der SEV auftretenden Impulse unterschiedlicher Höhe zunächst vermessen.
Anschließend wird die Höhe der Diskriminatorimpulse gemessen. Zudem wird über ein Zählwerk die Zahl der pro Sekunde am Diskriminator auftretenden Impulse gemessen und die Diskriminatorschwellen an beiden Diskriminatoren so eingeregelt, dass etwa an beiden die gleiche Impulsrate mit etwa $20$ bis $40$ Impulsen pro Sekunde auftritt.
Beide Signale werden anschließend auf die Koinzidenzschaltung aufgegeben und am Ausgang der Koinzidenz wird die Zählrate  in Abhängigkeit der Verzögerung gemessen. Es bildet sich ein Maximumsplateau aus, innerhalb dessen die Verzögerungszeit zu wählen ist, welche bei der späteren Lebensdauermessung verwendet wird.
Aus der Messung lässt sich zudem die Koinzidenzzeit $\Delta t_{\mathrm{K}}$ der Koinzidenz bestimmen.
Geprüft wird außerdem, ob die Zählrate hinter der Koinzidenz kleiner ist als vor ihr. Ist dies nicht der Fall, wäre sie wirkungslos. Die Diskriminatorschwellen müssen dann entsprechend etwas abgesenkt werden.
Für die weitere Kalibrierung der Schaltung wird ein Doppelimpulsgenerator an die Koinzidenzschaltung gelegt.
Die Zeit zwischen zwei durch den Doppelimpulsgenerator erzeugten Impulse lässt sich dabei mit einer Genauigkeit von $\SI{0.1}{\micro\second}$ eingestellt werden.
Die monostabile Kippstufe wird nun wie in Abbildung \ref{fig:aufbau} gezeigt, in die Schaltung eingebaut und an ihrem Ausgängen geprüft, dass die Suchzeit $T_{\mathrm{S}}$ nur minimal über der maximalen Zeitauflösung des TAC liegt.
Nachdem nun ebenfalls die AND-Gatter eingebaut werden, ist an deren Ausgängen zu prüfen, dass sich diese um genau den Zeitabstand unterscheiden, welcher am Doppelimpulsgenerator eingestellt wurde.
Zudem wird geprüft, dass die Signalhöhe am Ausgang des TAC proportional zum eingestellten Impulsabstand ist.
Anschließend wird noch durch Variation der Impulsabstände geprüft, welcher Kanal am Mehrkanalanalysator welchem Zeitabstand entspricht.


\subsubsection{Ablauf der Messung}
Die Messung wird begonnen, indem sowohl das Zählwerk als auch der Vielkanalanalysator gestartet werden.
Nach einer Messung von blup Stunden, wird die Messung über das Stoppen des Zählwerks und des Vielkanalanalysators beendet.
Aufgezeichnet wird die Anzahl der tatsächlichen Myonenzerfälle, die Anzahl der Fehlmessungen, die Ergebnisse des Vielkanalanalysators sowie die Messzeit, welche sich am Vielkanalanalysator ablesen lässt.
