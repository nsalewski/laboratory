\section{Theorie}
\label{sec:Theorie}
Das Standardmodell der Teilchenphysik beschreibt den Aufbau der Materie sowie die Wechselwirkungen dieser auf atomarer Ebene.
Es enthält alle bisher bekannten Elementarteilchen. Diese Teilchen lassen sich grundlegend in zwei Gruppen aufteilen.
Dies sind die Bosonen mit ganzzahligen Spin, welche die Kräfte zwischen den Fermionen vermitteln. Ein Beispiel hierfür ist das Photon, welches Mittler der elektromagnetischen Kraft ist.
Die zweite Gruppe bilden die elementaren Fermionen, also Elementarteilchen mit halbzahligen Spin. Sie sind sozusagen die Bausteine der Materie und lassen sich wiederum in zwei Gruppen, die Quarks und die Leptonen einteilen.
Fermionen sind zudem in drei Generationen gegliedert, die hier betrachteten Myonen sind dabei Leptonen der zweiten Generation.
Im Wesentlichen unterscheiden sich Myonen von Elektronen (Lepton der 1. Generation) durch ihre etwa $200$-mal größeren Masse. Ansonsten wechselwirken beide Leptonen gleich.
Dem geladenen, massebehafteten Myon $\symup\mu^-$ ist ein ungeladenes Myonneutrino $\symup{\nu_{\mu}}$ mit verschwindender Masse zugeordnet. Das Myon trägt hierbei die Ladung eines Elektrons. Das zugehörige Antiteilchen $\symup\mu^+$ trägt die umgekehrte Ladung und das zugehörige Antimyonneutrino wird als $\symup{\bar{\nu}_{\mu}}$ bezeichnet.
\\
Durch die Wechselwirkung von hochenergetischen Protonen mit der hohen Erdatmosphäre entstehen kurzlebige Pionen, welche bevorzugt in Myonen zerfallen:

\begin{align}
  \symup{\pi^+}\to\symup{\mu^+}+\symup{\nu_\mu}\\
  \symup{\pi^-}\to\symup{\mu^-}+\symup{\bar{\nu}_\mu}
\end{align}
Die so entstandenen Myonen bewegen sich relativistisch, sodass sie an der Erdoberfläche mit Szintillationsdetektoren nachgewiesen werden können.
Beim Durchqueren des Szintillatormaterials geben die Myonen einen Teil ihrer Energie ab. Dies regt das Szintillatormaterial an. Bei der anschließenden Abregung des Szintillatormaterials wird ein Photon emittiert.
Die so entstehenden Lichtblitze können detektiert werden.
Das Myon kann auch seine ganze kinetische Energie im Szintillatormaterial deponieren, wenn dies geschieht, zerfällt das Myon anschließend auch im Szintillatormaterial.
Der Zerfall geschieht bevorzugt wie folgt:
\begin{align}
  \symup{\mu^+}\to\symup{e^+}+\symup{\nu_e}+\symup{\bar{\nu}_\mu} \\
  \symup{\mu^-}\to\symup{e^-}+\symup{\bar{\nu}_e}+\symup{\nu_\mu}
\end{align}
Die so entstandenen Elektronen beziehungsweise Positronen sind aufgrund ihrer viel kleineren Masse hochenergetisch und erzeugen ebenfalls einen Lichtblitz im Szintillator.
Allerdings zerfallen nicht alle einfallenden Myonen im Szintillator.
Es lassen sich drei unterschiedliche Fälle betrachten:
\begin{enumerate}
  \item Das Myon hat bereits viel Energie auf seinem Weg durch die Atmosphäre deponiert. Es zerfällt daher im Szintillator. Aus der Zeitdifferenz zwischen den beiden somit detektierbaren Lichtblitzen (ein Lichtblitz wird erzeugt bei Eintritt des Myons in das Szintillatormaterial, der zweite beim Zerfall des Myons durch Reaktionen der Zerfallsprodukte mit dem Szintillatormaterial) ist eine Bestimmung der Myonenlebensdauer möglich.
  \item Das Myon zerfällt nicht im Detektor, sondern durchquert diesen lediglich. Hier wird somit nur ein Lichtblitz erzeugt.
  \item Negativ geladene Myonen können zudem ähnlich wie beim Elektroneneinfang durch Szintillatoratome eingefangen werden. Genauso wie beim Durchqueren des Detektors tritt auch hier nur ein Lichtblitz beim Eindringen in den Detektor auf.
\end{enumerate}

Zu Beachten ist hierbei, dass die beiden letzten Fälle nicht zur Bestimmung der Lebensdauer der Myonen geeignet sind. Im experimentellen Aufbau müssen sie daher schaltungstechnisch geeignet herausgefiltert werden. Näheres dazu im Kapitel \ref{sec:Durchführung}.

\subsection{Bestimmung der Lebensdauer instabiler Teilchen}
Der Zerfall eines instabilen Teilchens ist ein statistischer Prozess und unabhängig von vorangegangenen Zerfallsprozessen.
Im vorliegenden Versuch werden zwar Individuallebensdauern einzelner Myonen gemessen, interessant ist allerdings die mittlere Lebensdauer von Myonen.
Zwischen der Wahrscheinlichkeit $\upd W$, dass ein Zerfall im Zeitraum $\upd t$ stattfindet, besteht eine Proportionalität:
\begin{equation}
  \label{eqn:wkeit}
  \upd W=\lambda\upd t\mathrm{.}
\end{equation}

Hierbei ist $\lambda$ eine für den Zerfallsprozess charakteristische Konstante.
Da die Zerfallsprozesse unabhängig sind, lässt sich mit Formel \ref{eqn:wkeit} die Teilchenzahländerung $\upd N$ herleiten:
\begin{equation*}
\upd N = -N \upd W = - \lambda N \upd t  \mathrm{.}
\end{equation*}
Für große Teilchenzahlen N folgt mittels Integration:
\begin{equation}
\frac{N(t)}{N_0} = \exp{(-\lambda t)}\mathrm{.}
\end{equation}
Hierbei ist $N_0$ die Gesamtzahl der betrachteten Teilchen.
Die Betrachtung der Teilchenzahländerung $\upd N(t)$ liefert schließlich eine Verteilungsfunktion für die Lebensdauer der Teilchen:
\begin{equation}
  \label{eqn:verteilung}
  \frac{\upd N(t)}{N_0} = \lambda \exp{(-\lambda t)}\upd t\mathrm{.}
\end{equation}
Zur Bestimmung der charakteristischen Lebensdauer $\tau$ wird der Erwartungswert der Verteilungfunktion, auch bezeichnet als ihr erstes Moment, berechnet:
\begin{equation}
  \tau=\Braket{t}=\int_0^\infty t\lambda\exp(-\lambda t)\upd t=\frac{1}{\lambda}\mathrm{.}
\end{equation}
Die zuvor eingeführte Konstante $\lambda$ wird daher auch als Zerfallskonstante bezeichnet.

\subsection{Statistische Probleme}
Da es in einem Experiment lediglich eine endliche Anzahl an Messwerten aufgenommen werden kann, also immer nur eine Stichprobe betrachtet wird, lassen sich exakte Werte für $\tau$ nicht angeben.
Allerdings ist es möglich, den wahrscheinlichsten Wert für $\tau$ zu bestimmen.
Das arithmetische Mittel $\bar{t}$ aller Einzelmessungen kommt dem gesuchten Wert für $\tau$
für eine genügend große Anzahl an Einzelmessungen $N$
beliebig nahe und stellt damit eine gute Schätzung für den tatsächlichen Wert dar:
\begin{equation}
  \bar{t}=\frac{1}{N}\sum_{j=1}^{N} t_j \mathrm{.}
\end{equation}
Die Berechnung einer Regression der Verteilungsfunktion \eqref{eqn:verteilung} an die Messwerte mittels einer nichtlinearen Ausgleichsrechnung (Methode der kleinsten Fehlerquadrate) bildet daher eine gute Approximation der tatsächlichen Verteilungsfunktion und ermöglicht eine gute Abschätzung des tatsächlichen Werts für $\tau$.
