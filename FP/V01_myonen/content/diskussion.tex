\section{Diskussion}
\label{sec:Diskussion}
Ein Vergleich der berechneten Untergrudrate $U \approx \num{1.90286(1)} $ sowie der aus dem Fit berechneten Untergrundrate $B=\num{1.9(2)}$ zeigt eine gute Übereinstimmung. Die Untergrundrate $U$ liegt innerhalb des Fehlerbereichs der Untergrundrate $B$.

Die Auflösungszeit $t_{\text{K}}=\SI{50.1}{\nano\second}$ der Koinzidenz stimmt gut überein mit der Länge der Diskriminatorimpulse von etwa $\Delta t=\SI{50}{\nano\second}$.

Der Literaturwert für die Lebensdauer des Myons beträgt nach \cite{myon}
\begin{equation*}
	\tau_\text{Literatur} = \SI{2.1969811(22)}{\micro\second}\, \mathrm{.}
\end{equation*}
Mit dem experimentell bestimmten Wert von $\tau=\SI{2.07(2)}{\micro\second}$ liegt eine Abweichung von ungefähr $6 \%$ gegenüber dem Theoriewert vor.

Anhand der großen Anzahl an aufgenommenen Datensätzen ist eine hohe Übereinstimmung zu erwarten. Ein besseres Ergebnis ließe sich bestimmt durch eine Linearisierung des Problems erhalten, bedauerlicherweise müsste hierzu nahezu die Hälfte der aufgenommenen Daten von der Regression ausgeschlossen werden, da bereits ab dem Kanal $220$ Nulleinträge vorhanden sind.
Die ausgeschlossenen Messwerte in Abbildung \ref{fig:blubbb} sind ermutlich zum einen zu kleine Zeitabstände gewesen, welche der Vielkanalanalysator nicht korrekt erfassen konnte, zum anderen verteilt der Vielkanalanalysator Ereignisse nahezu gleicher Zeitabstände zum Teil auf mehrere Kanäle. Dies kann in einigen Kanälen zu zu geringen Zählraten führen, während in anderen Kanälen zuviele Ereignisse detektiert werden.

Wird zudem die Summe aller im Vielkanalanalysator gemessenen Ereignisse gebildet, ergibt sich $n=\num{9481}$. Am Zählwerk wurden $n_{\text{Zähl}}=\num{7972}$ Stopsignale gezählt. Dies entspricht einer Abweichung von etwa $19\%$, wenn angenommen wird, dass der Wert am Zählwerk der tatsächlichen Ereignisanzahl entspricht. Dies deutet ebenfalls darauf hin, dass einige Ereignisse doppelt gezählt wurden.

Weitere Fehlerquellen sind eventuelle Wackelkontakte in der Apparatur, welche durch locker eingesteckte Kabel erzeugt werden, sowie das händische Einstellen der Diskriminatorschwellen, was zum Filtern tatsächlicher Ereignisse führen könnte. Außerdem unterliegt die Elektronik technischen Limitationen, sehr kurze Zeitabstände können daher nicht gemessen werden.
