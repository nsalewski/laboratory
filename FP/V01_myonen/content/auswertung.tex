\section{Auswertung}
\label{sec:Auswertung}

\subsection{Berechnung der Untergrundrate}
\label{sec:unter}
Es sind insgesamt $N_\text{start}=1998242$ Startsignale und
$N_\text{stop}=7972$ Stopsignale registriert worden.
Damit ergibt sich mit der Messdauer $t_\text{measure}=\SI{128829}{\second}$
die Eintrittsfrequenz der Myonen zu
\begin{equation*}
	f = \frac{N_\text{start}}{t_\text{measure}} \approx \SI{21,04}{\per\second} \, \text{.}
\end{equation*}
Weiterhin soll die Wahrscheinlichkeit dafür bestimmt werden, dass in der
Suchzeit $T_\text{S}$ genau $n$ weitere Myonen in den Szintillationsdetektor
eintreffen. Diese Myonen würden eine Fehlmessung auslösen. Die Wahrscheinlichkeit ergibt sich mit der Poissonverteilung zu
\begin{equation*}
	p(n)=\frac{(f T_\text{S})^n}{n!} \, \symup{e}^{f T_\text{S}} \, \text{.}
\end{equation*}
Für genau ein Myon ergibt sich somit die erwartete Anzahl an Fehlmessungen
$N_\text{false}=p(1) \cdot N_\text{start} \approx 420,52$. Damit ergibt sich
nach Division durch die Kanalzahl direkt die Untergrundrate
\begin{equation*}
	U \approx 0,95 \, \mathrm{,}
\end{equation*}
wobei sich die Kanalanzahl nach Nichtberücksichtigung der Einträge mit "$0$" zu $442$ ergibt.
\subsection{Zeiteichung der Apparatur}
\FloatBarrier
\label{sec:blaaaaaaa}
Die Messwerte für die Zeiteichung der Apparatur sind in Tabelle
\ref{tab:eichi} aufgetragen.
\begin{table}
 \caption{Kanalnummer in Abhängigkeit des Doppelimpulsabstandes $T$ für die Zeiteichung der Apparatur.}
 \label{tab:eichi}
 \centering
\sisetup{table-format=3.3} \begin{tabular}{SS}
 \toprule 
    {Kanalnummer}& {Doppelimpulsabstand T / $\si{\second}$} \\
     \midrule
          35,00 &          1 \\
          57,04 &          2 \\
          79,83 &          3 \\
         102,00 &          4 \\
         124,00 &          5 \\
         146,00 &          6 \\
         168,00 &          7 \\
         190,00 &          8 \\
         212,00 &          9 \\
 \bottomrule
 \end{tabular}
\end{table}
Hierbei soll die Abhängigkeit zwischen Doppelimpulsabstand $T$ und Kanalnummer $N$ linear sein.
Daher wird ein linearer Fit der Form
\begin{equation}
	T = m \, N + b
\end{equation}
mit Python \cite{numpy} durchgeführt, wobei $m$ und $b$ die Fitparameter in $\si{\micro\second}$
sind.
Es ergeben sich die beiden Fitparameter zu
\begin{equation*}
	\begin{split}
		m &= \SI{0.0452(8)}{\micro\second} \, \mathrm{,} \\
		b &= \SI{-0.5949(103)}{\micro\second} \, \mathrm{.}
	\end{split}
\end{equation*}
Im Folgenden wird allerdings nur die Steigung $m$ benötigt, sodass der Parameter $b$ nicht weiter
verwendet wird.
%include tables
%\begin{table}
 \caption{Im Experiment gemessene Ströme der Sweep-Spule und der Horizontalfeldspule für die Transparenzminima beider Isotope sowie die Frequenz des angelegten RF-Wechselfelds }
 \label{tab:current}
 \centering
 \begin{tabular}{ccccc}
 \toprule 
    RF-Wechselfeld/$\si{\kilo\hertz}$ & $I_{\mathrm{Horizontal,1}}/\si{\milli \ampere}$ & $I_{\mathrm{Horizontal,2}}/\si{\milli \ampere}$ & $I_{\mathrm{Sweep,1}}/\si{\milli \ampere}$ & $I_{\mathrm{Sweep,2}}/\si{\milli \ampere}$ \\
     \midrule
     100.3 & 0 & 0 & 597 & 666 \\
     201.0 & 35 & 35 & 245 & 484 \\
     301.0 & 56 & 56 & 187 & 545 \\
     400.0 & 76 & 76 & 139 & 617 \\
     499.0 & 91 & 91 & 162 & 753 \\
     603.0 & 116 & 116 & 47 & 763 \\
     699.0 & 88 & 159 & 688 & 473 \\
     801.0 & 101 & 182 & 736 & 508 \\
     899.0 & 149 & 207 & 265 & 493 \\
     1004.0 & 156 & 237 & 410 & 424 \\
 \bottomrule
 \end{tabular}
\end{table}
\FloatBarrier
\subsection{Ermittlung der Lebensdauer von Myonen}
\begin{figure}
  \centering
  \includegraphics[width=0.75\columnwidth]{Daten/regress.pdf}
	\caption{Aufgenommene Messtupel mit Ausgleichsrechnung gemäß Formel \eqref{eqn:nnn} für die registrierte Myonenzahl $N(t)$ in Abhängigkeit der Zeit $t$.}
  \label{fig:blubbb}
\end{figure}

Aus den Messdaten und dem Zusammenhang aus Abschnitt \ref{sec:blaaaaaaa} lassen sich die
Messtupel $\big(t, N(t)\big)$ ableiten, die den Zusammenhang
\begin{equation}
	\label{eqn:nnn}
	N(t) = N_0 \cdot \mathrm{e}^{-\lambda t} + U \, \mathrm{,}
\end{equation}
wobei $U$ der Untergrundrate aus Abschnitt \ref{sec:unter} entspricht.
Die Messergebnisse mit zugehöriger Theoriekurve gemäß \eqref{eqn:nnn} sind in Abbildung
\ref{fig:blubbb} dargestellt.
Mit Python \cite{numpy} ergeben sich die Fitparamter zu
\begin{align*}
	N_0 = (198 \pm 4 ) \, \mathrm{,} \\
	\lambda = \SI{0.47(1)}{\per\micro\second} \, \mathrm{.}
\end{align*}
Damit ist die experimentell bestimmte Lebensdauer der Myonen
\begin{equation*}
	\tau = \frac{1}{\lambda} = \SI{2.14(6)}{\micro\second} \, \mathrm{.}
\end{equation*}
