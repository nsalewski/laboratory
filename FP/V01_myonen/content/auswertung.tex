\section{Auswertung}
\label{sec:Auswertung}

\subsection{Berechnung der Untergrundrate}
Es sind insgesamt $N_\text{start}=1998242$ Startsignale und 
$N_\text{stop}=7972$ Stopsignale registriert worden.
Damit ergibt sich mit der Messdauer $t_\text{measure}=\SI{128829}{\second}$
die Eintrittsfrequenz der Myonen zu
\begin{equation*}
	f = \frac{N_\text{start}}{t_\text{measure}} \approx \SI{21,04}{\per\second} \, \text{.}
\end{equation*}
Weiterhin soll die Wahrscheinlichkeit dafür bestimmt werden, dass in der 
Suchzeit $T_\text{S}$ genau $n$ weitere Myonen in den Szintillationsdetektor
eintreffen. Diese Myonen würden eine Fehlmessung auslösen. Die Wahrscheinlichkeit ergibt sich mit der Poissonverteilung zu
\begin{equation*}
	p(n)=\frac{(f T_\text{S})^n}{n!} \, \symup{e}^{f T_\text{S}} \, \text{.}
\end{equation*}
Für genau ein Myon ergibt sich somit die erwartete Anzahl an Fehlmessungen 
$N_\text{false}=p(1) \cdot N_\text{start} \approx 420,52$. Damit ergibt sich
nach Division durch die Kanalzahl direkt die Untergrundrate 
\begin{equation*}
	U \approx 0,95 \, \mathrm{,}
\end{equation*}
wobei sich die Kanalanzahl nach Nichtberücksichtigung der Einträge mit "$0$" zu $442$ ergibt.
\subsection{Zeiteichung der Apparatur}
Die Messwerte für die Zeiteichung der Apparatur sind in Tabelle
\ref{tab:eichi} aufgetragen. 
\begin{table}
 \caption{Kanalnummer in Abhängigkeit des Doppelimpulsabstandes $T$ für die Zeiteichung der Apparatur.}
 \label{tab:tab:eichi}
 \centering
\sisetup{table-format=3.3} \begin{tabular}{SS}
 \toprule 
    {Kanalnummer}& {Doppelimpulsabstand T / $\si{\second}$} \\
     \midrule
           1.00 &         70 \\
           2.00 &        114 \\
           3.00 &        160 \\
           4.00 &        204 \\
           5.00 &        248 \\
           6.00 &        292 \\
           7.00 &        336 \\
           8.00 &        380 \\
           9.00 &        424 \\
 \bottomrule
 \end{tabular}
\end{table}

%include tables
%\begin{table}
 \caption{Im Experiment gemessene Ströme der Sweep-Spule und der Horizontalfeldspule für die Transparenzminima beider Isotope sowie die Frequenz des angelegten RF-Wechselfelds }
 \label{tab:tab:current}
 \centering
 \begin{tabular}{ccccc}
 \toprule 
    RF-Wechselfeld/$\si{\kilo\hertz}$ & $I_{\mathrm{Horizontal,1}}/\si{\milli \ampere}$ & $I_{\mathrm{Horizontal,2}}/\si{\milli \ampere}$ & $I_{\mathrm{Sweep,1}}/\si{\milli \ampere}$ & $I_{\mathrm{Sweep,2}}/\si{\milli \ampere}$ \\
     \midrule
     100.3 & 0 & 0 & 597 & 666 \\
     201.0 & 35 & 35 & 245 & 484 \\
     301.0 & 56 & 56 & 187 & 545 \\
     400.0 & 76 & 76 & 139 & 617 \\
     499.0 & 91 & 91 & 162 & 753 \\
     603.0 & 116 & 116 & 47 & 763 \\
     699.0 & 88 & 159 & 688 & 473 \\
     801.0 & 101 & 182 & 736 & 508 \\
     899.0 & 149 & 207 & 265 & 493 \\
     1004.0 & 156 & 237 & 410 & 424 \\
 \bottomrule
 \end{tabular}
\end{table}
