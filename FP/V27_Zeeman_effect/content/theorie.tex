\section{Theorie}
\label{sec:Theorie}
Der Zeeman-Effekt ist ein quantenmechanischer Effekt.
Unter Einfluss eines äußeren Magnetfelds bildet sich eine Feinstruktur der Energieniveaus im Atom aus, da die den Drehimpulsen zugeordneten magnetischen Momente der Elektronenhülle verschieden an das Magnetfeld koppeln und daher unterschiedliche Energieniveaus annehmen. Bei Übergängen zwischen verschiedenen Energieniveaus werden entsprechend ihrer Energiedifferenz Lichtquanten unterschiedlicher Wellenlänge emittiert.
Im vorliegenden Versuch wird anhand der roten und der blauen Linie des Emissionsspektrum einer Cadmiumdampflampe
optisch anhand von Wellenlängenänderungen, verursacht durch geänderte Energieniveaus beim Anlegen eines Magnetfelds, der Zeeman-Effekt untersucht.
\subsection{Magnetische Momente von Elektronen und Atomen}
Der Gesamtdrehimpuls eines Elektron der Elektronenhülle setzt sich zusammen aus seinem Bahndrehimpuls $\vec{l}$ und seinem Spin $\vec{s}$. Beiden ist aufgrund der Elektronenladung ein magnetisches Moment zugeordnet.
Eine Drehimpulseinheit definiert hierbei das sogenannte Bohrsche Magneton:
\begin{equation}
  \mu_{\mathrm{B}}=-\frac{1}{2}e_0\frac{\hbar}{m_0}=\mu_{\mathrm{l}}(m=1)
\end{equation}
Mit dem Bohrschen Magneton ergeben sich die magnetischen Momente des Bahndrehimpulses $\mu_{\mathrm{l}}$ und des Spins $\mu_{\mathrm{s}}$ zu:
\begin{gather}
  \mu_{\mathrm{l}}=-\mu_{\mathrm{B}}\sqrt{l\left(l+1\right)}\mathrm{;}\\
  \mu_{\mathrm{s}}=-\mu_{\mathrm{B}}g_{\mathrm{S}}\sqrt{s\left(s+1\right)}\mathrm{.}
\end{gather}
Es ist hierbei $\hbar$ das Planksche Wirkungsquantum und $g_{\mathrm{S}}$ der sogenannte Landé-Faktor. Dieser berücksichtigt, dass die gemessenen magnetischen Momente des Spins
etwa doppelt so groß sind wie die theoretisch berechneten Werte des magnetischen Moments zum Spin des Elektrons nach der klassischen Physik. Der Landé-Faktor des 
\subsection{Verhalten bei Anlegen eines Magnetfelds}
\subsection{Der normale Zeeman-Effekt}
\subsection{Der anomale Zeeman-Effekt}
