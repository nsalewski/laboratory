\section{Einleitung}
\label{sec:Einleitung}
Der Zeeman-Effekt ist ein quantenmechanischer Effekt.
Unter dem Einfluss eines äußeren Magnetfeldes bildet sich eine Feinstruktur der Energieniveaus im Atom aus, da die, den Drehimpulsen zugeordneten magnetischen Momente der Elektronenhülle, verschieden an das Magnetfeld koppeln und daher unterschiedliche Energieniveaus annehmen.\\
Bei Übergängen zwischen verschiedenen Energieniveaus werden entsprechend ihrer Energiedifferenz Lichtquanten unterschiedlicher Wellenlänge emittiert.\\
Im vorliegenden Versuch wird anhand der roten und der blauen Linie des Emissionsspektrums einer Cadmiumdampflampe
optisch anhand von Wellenlängenänderungen, verursacht durch die geänderten Energieniveaus beim Anlegen eines Magnetfelds, der Zeeman-Effekt untersucht.
