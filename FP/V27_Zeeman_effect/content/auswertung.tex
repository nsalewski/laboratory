\section{Auswertung}
\label{sec:Auswertung}
%include tables
%\begin{table}
 \caption{Im Experiment gemessene Ströme der Sweep-Spule und der Horizontalfeldspule für die Transparenzminima beider Isotope sowie die Frequenz des angelegten RF-Wechselfelds }
 \label{tab:current}
 \centering
 \begin{tabular}{ccccc}
 \toprule 
    RF-Wechselfeld/$\si{\kilo\hertz}$ & $I_{\mathrm{Horizontal,1}}/\si{\milli \ampere}$ & $I_{\mathrm{Horizontal,2}}/\si{\milli \ampere}$ & $I_{\mathrm{Sweep,1}}/\si{\milli \ampere}$ & $I_{\mathrm{Sweep,2}}/\si{\milli \ampere}$ \\
     \midrule
     100.3 & 0 & 0 & 597 & 666 \\
     201.0 & 35 & 35 & 245 & 484 \\
     301.0 & 56 & 56 & 187 & 545 \\
     400.0 & 76 & 76 & 139 & 617 \\
     499.0 & 91 & 91 & 162 & 753 \\
     603.0 & 116 & 116 & 47 & 763 \\
     699.0 & 88 & 159 & 688 & 473 \\
     801.0 & 101 & 182 & 736 & 508 \\
     899.0 & 149 & 207 & 265 & 493 \\
     1004.0 & 156 & 237 & 410 & 424 \\
 \bottomrule
 \end{tabular}
\end{table}
In der Auswertung wird zunächst das Eichen des Elektromagneten ausgewertet

\subsection{Eichen des Elektromagneten}%Tabelle mit den Messdaten für B-Feld Eichung?
Das gemessene Magnetfeld $B$ ist in Abbildung \ref{fig:eichi} gegen den Feldstrom $I$ auf dem
Intervall $\SI{1}{\ampere} \le I \le \SI{16}{\ampere}$ mit zugehörigem Ausgleichspolynom
aufgetragen. Dabei ist das Ausgleichspolynom dritter Ordnung.
Die Ausgleichsrechnung wird mit Python \cite{scipy} durchgeführt.
\begin{figure}
	\centering
	\includegraphics[width=\textwidth]{pictures/eichi.pdf}
	\caption{Gemessenes B-Feld aufgetragen gegen den Feldstrom $I$ mit zugehörigem Ausgleichspolynom.}
	\label{fig:eichi}
\end{figure}

\subsection{Vorbereitende Aufgaben}
	Bei der verwendeten Cadmium-Lampe wird die rote Linie für den normalen Zeemann-Effekt
	mit dem Übergang
	\begin{align*}
		5^1D_2 \rightarrow 5^1P_1
	\end{align*}
	erzeugt, wobei sich durch die Multiplizität $M=2S+1=1$ jeweils ein Gesamtspin von
	$S=0$ ergibt -- für den normalen Zeemann-Effekt notwendigerweise erfüllt --
	was zur Folge hat, dass der Landé-Faktor eins wird.
	Die Niveaus, die an der Erzeugung der blauen Linie beteiligt sind, tragen offensichtlich
	einen Spin, sodass der Landé-Faktor von $L$, $S$ und $J$ abhängt. Der angedeutete
	Übergang lautet
	\begin{align*}
		6^3S_1 (g_J=2) \rightarrow 5^3S_1 (g_J=\frac{3}{2}) \, \mathrm{.}
	\end{align*}

	Weiterhin wird das Dispersionsgebiet $\Delta \lambda_{\mathrm{D}}$ der
	Lummer-Gehrcke-Platte mit Formel
	\eqref{eqn:dispersionsgebiet} bestimmt. Dabei ist die Dicke der Platte $d=\SI{4}{\milli\meter}$
	und die Länge der Platte  $L = \SI{120}{\milli\meter}$. Zuletzt ergibt sich der
	Brechungsindex $n$ aus den Wellenlängen des verwendeten Lichtes zu
	$n(\lambda=\SI{644}{\nano\meter}) = 1,4567$ für das rote Licht und zu
	$n(\lambda=\SI{480}{\nano\meter}) = 1,4635$ für das blaue Licht. Damit ergibt sich
	das Dispersionsgebiet zu
	\begin{align*}
		\Delta \lambda_{\mathrm{D}}(\lambda=\SI{643.8}{\nano\meter}) &= \SI{48.913}{\pico\meter} \, \mathrm{,} \\
		\Delta \lambda_{\mathrm{D}}(\lambda=\SI{480.0}{\nano\meter}) &= \SI{26.952}{\pico\meter} \, \mathrm{.}
	\end{align*}
	Für das Auflösungsvermögen $A$ ergeben sich mit Gleichung \eqref{eqn:A} die Werte:
\begin{align*}
	A(\lambda=\SI{643.8}{\nano\meter}) &= 209063,644 \, \mathrm{,} \\
	A(\lambda=\SI{480}{\nano\meter}) &= 285458,063 \, \mathrm{.}
\end{align*}

\subsection{Bestimmung der Landé-Faktoren $g$}
	Die Landé-Faktoren werden mit Formel \eqref{eqn:lande} bestimmt. Hierfür werden die mit der
	Digital-Kamera aufgenommenen Aufspaltungen mit und ohne B-Feld übereinander aufgetragen.
