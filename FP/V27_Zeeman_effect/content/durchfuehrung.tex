\section{Durchführung}
\label{sec:Durchführung}

\subsection{Versuchsaufbau}
\label{sec:Versuchsaufbau}
Der Versuchsaufbau besteht aus einem Ultraschallechoskop, an dessen Ausgänge zwei Ultraschallsonden mit \SI{2}{\mega\Hz} gekoppelt sind, und einem Rechner zur Datenaufnahme und -analyse.\\
An das Ultraschallechoskop sind zwei Ultraschallsonden angeschlossen, mithilfe derer sich sowohl eine Impuls-Echo-Messung, als auch eine Durchschallmessung realisieren lässt.
Am Rechner werden die gemessenen Daten mittels des Programms \textquote{Echoview} ausgewertet.\\
Hierbei ist \textquote{Echoview} in der Lage, vier verschiedene Diagramme darzustellen.
Im linken oberen Graphen wird der A-Scan dargestellt, also die Amplitude gegen die Zeit aufgetragen.
Der linke untere Graph stellt die gewählte Verstärkung dar. Die Verstärkung lässt sich am Ultraschallechoskop über die Drehknöpfe zur laufzeit-bzw. tiefenabhängigen Verstärkung (TGC; Time Gain Control) und ebenso über die Verstärkung des Outputs und der Empfindlichkeit der Sonden regulieren.\\
Zu Beachten ist, dass eine Verstärkung nur gewählt werden darf, wenn die auszuwertende Messreihe nicht zur Untersuchung der Amplitudenhöhe dient.
Die beiden rechten Graphiken sind das berechnete Spektrum der Messdaten (FFT), bzw. ihr
Cepstrum.
Erzeugte Graphiken und Messdaten können aus dem Programm heraus exportiert werden.
Als zu untersuchende Versuchsobjekte stehen Acrylzylinder verschiedener Länge, Acrylplatten unterschiedlicher Dicke sowie das Modell eines menschlichen Auges im Maßstab 3:1 zur Verfügung.


\subsection{Versuchsbeschreibung}
\label{sec:Versuchsbeschreibung}
Vor Beginn der eigentlichen Untersuchung des Zeeman-Effekts muss der im Versuchsaufbau verwendete Elektromagnet zunächst geeicht werden.\\
Hierzu wird eine Hall-Sonde möglichst senkrecht in das B-Feld des Elektromagneten eingebracht und der Feldstrom in Schritten von $\SI{0.5}{\ampere}$ hochgeregelt und das zugehörige Magnetfeld in Abhängigkeit der Stromstärke notiert.
Anschließend wird der Versuchsaufbau wie in Abbildung \ref{fig:aufbau} gezeigt, aufgebaut. Lediglich der Polarisationsfilter wird noch nicht eingebaut, da dieser einen großen Teil der Intensität des Lichtstrahls herausfiltert und daher hinderlich beim Optimieren des Strahlenganges ist.\\
Um ein möglichst gutes Bild zu erhalten, wird zunächst die Position des Objektiv $O$ und der Linse $L1$ solange variiert, bis ein möglichst scharfes Lichtbündel auf den Spalt $S1$ abgebildet wird.\\
Die Linse $L2$ wiederum wird solange verschoben, bis ein möglichst paralleles Lichtbündel auf das Gradsichtprisma trifft.
Hierbei ist darauf zu achten, dass das Lichtbündel etwa so groß ist wie das Prisma ist, sodass möglichst wenig Intensität in der Optik verloren geht.\\
Die Linse $L3$ wird nun so eingestellt, dass auf den Spalt $S2$ ein scharfes Linienspektrum abgebildet wird. Mit dem Spalt $S2$ kann nun eine diskrete Linie des Spektrums ausgewählt werden und in die weitere Optik gelassen werden.\\
Begonnen wird mit der grünen Spektrallinie, diese entsteht aufgrund eines Quecksilberanteils in der Lampe und wird zur Kalibrierung des Messaufbaus genutzt, da das menschliche Auge für sie besonders empfindlich ist.\\
Mit der Linse $L4$ wird der Lichtstrahl schließlich auf das Eintrittsprisma der Lummer-Gehrcke-Platte fokussiert. Die Lummer-Gehrcke-Platte wird nun so ausgerichtet, dass ihre planparallelen Platten in einem möglichst spitzen Winkel zum Strahlengang stehen, damit eine hohe Auflösung erzielt werden kann.\\
Die Kamera wird zunächst auf den manuellen Modus geschaltet und jede Autofokus-Funktion deaktiviert.
Da die Kamera parallele Strahlen aufnehmen soll, muss ihr Zoom auf unendlich gestellt werden.\\
Es wird nun solange etwas an der Lummer-Gehrcke-Platte gedreht, bis ein gutes Bild auf dem Kamerabildschirm abgebildet wird.
Nun kann die eigentliche Messung des Zeeman-Effekts beginnen. Dafür wird der Polarisationsfilter an eine beliebige Stelle des Strahlengangs eingebracht (zum Beispiel direkt nach der Linse $L1$).\\
Durch ein leichtes Verdrehen der Linse $L3$ kann nun eine der zu untersuchenden Spektrallinien auf den Spalt $S2$ abgebildet. Dies hat gegenüber einer Verschiebung des Spalts den Vorteil, dass die dahinterliegende Optik nicht neu angepasst werden muss.
Für beide zu vermessenden Linien wird schließlich sowohl mindestens ein Bild ohne, als auch mindestens ein Bild mit eingeschalteten Magnetfeld für die jeweils gewünschte Polarisation gemacht.\\
Für jeden zu untersuchenden Übergang wird zunächst noch das aufgrund des Auflösungsvermögen der Lummer-Gehrcke-Platte minimal nötige Magnetfeld zur deutlichen Auflösung der Aufspaltung berechnet und im weiteren Verlauf jeweils passend eingestellt.
Da dies nicht für alle Übergänge mit dem vorliegenden Versuchsaufbau möglich ist, wird entsprechend das maximal mögliche Magnetfeld von etwa $\SI{1}{\tesla}$ eingestellt.\\
Für die rote Spektrallinie wird ein Bild ohne Magnetfeld und jeweils durch das Einstellen des Polarisators ein Bild nur mit zirkular und eins mit linear polarisierten Licht gespeichert.\\
Für die Untersuchung der blauen Spektrallinie wird analog vorgegangen.\\
Es empfiehlt sich für jede Messung nicht nur ein Bild zu speichern, sondern jeweils eine Bildserie mit verschiedenen Belichtungszeiten und ISO-Werten aufzunehmen um möglichst gute Bilder zu erhalten.
