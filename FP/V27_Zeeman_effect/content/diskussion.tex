\section{Diskussion}
\label{sec:Diskussion}
Die berechneten Landé-Faktoren mit den zugehörigen Theoriewerten gemäß Formel \eqref{eqn:lande}
sind in Tabelle \ref{tab:result} aufgetragen.
Hierbei ist auffällig, dass die Messergebnisse bei der roten und der blauen $\pi$-Linie
im Intervall der Standardabweichung liegen.
\begin{table}
	\centering
	\caption{Experimentell bestimmte Landé-Faktoren mit zugehörigen Theoriewerten gemäß Formel \eqref{eqn:lande} mit relativen Fehlern.} 
	 \label{tab:result}
	 \begin{tabular}{c | c | c | c}
	 	\toprule
		 Linie & $g_{\text{exp}}$ & $g_{\text{theo}}$ & $\Delta g$ \\
		\midrule
		rot & $1$ & $1,014\pm0,028$ & $1,4\, \si{\percent}$ \\
		blau ($\sigma$) & $2$ & $1,84\pm0,04$ & $8 \, \si{\percent}$ \\
		blau ($\pi$)    & $\frac{1}{2}$ & $0,488\pm0,023$ & $2,4\,\si{\percent}$ \\
		\bottomrule
	\end{tabular}
\end{table}
Die größte Fehlerquelle wird dem Ablesen aus den Bildern entsprechen. Hier ermöglicht
Gimp \cite{gimp} zwar eine Messung in Pixeln, allerdings ist die Bestimmung der intensivsten
Stellen sehr subjektiv und liefert eine große Sensibilität für Fehler.

Außerdem war die Kamera geneigt gegenüber der Aufspaltung, sodass die Werte für $\Delta s$ und
$\delta s$ nicht äquidistant waren. Dieser Effekt wurde mit der Neigungsfunktion von
Gimp \cite{gimp} ein wenig aufgehoben, allerdings sind doch noch deutliche Abweichungen
zwischen den einzelnen Werten gemessen worden.

Die größte Abweichung liegt bei der blauen $\sigma$-Linie auf. Dies lässt sich vielleicht durch
die inadäquate Auflösung zwischen den Niveaus $6^3S_1$ mit $g_J=2$ und $5^3P_1$ mit $g_J=\frac{3}{2}$ und eine dadurch bedingte Überlagerung erklären, was den Trend zu kleineren $g$ erklären würde.
