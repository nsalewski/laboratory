\section{Diskussion}
\label{sec:Diskussion}
Der berechnete Landé-Faktor mit den zugehörigen Theoriewert gemäß Formel \eqref{eqn:lande} für die rote Linie
ist in Tabelle \ref{tab:result} aufgetragen.
Für die blaue Linie wird hier der Faktor $\alpha$ angegeben, welcher die Landé-Faktoren der beteiligten Niveaus enthält.
Der Theoriewert des $\alpha$-Faktors ergibt sich mit:
\begin{equation*}
\alpha=m_1g_1-m_2g_2 \mathrm{.}
\end{equation*}
Hierbei bezeichnen die $m_i$ die Orientierungsquantenzahlen der beteiligten Niveaus und $g_i$ die Landé-Faktoren der Niveaus.
Hierbei ist auffällig, dass die Messergebnisse bei der roten und der blauen $\pi$-Linie
im Intervall der Standardabweichung liegen.
\begin{table}
	\centering
	\caption{Experimentell bestimmte Landé-Faktoren mit zugehörigen Theoriewerten gemäß Formel \eqref{eqn:lande} mit relativen Fehlern.}
	 \label{tab:result}
	 \begin{tabular}{c | c | c | c}
	 	\toprule
		 Linie & $\alpha_{\text{theo}}$ & $\alpha_{\text{exp}}$ & $\Delta g$ \\
		\midrule
		rot & $1$ & $1,014\pm0,028$ & $1,4\, \si{\percent}$ \\
		blau ($\sigma$) & $2$ & $1,84\pm0,04$ & $8 \, \si{\percent}$ \\
		blau ($\pi$)    & $\frac{1}{2}$ & $0,488\pm0,023$ & $2,4\,\si{\percent}$ \\
		\bottomrule
	\end{tabular}
\end{table}
Die größte Fehlerquelle wird dem Ablesen aus den Bildern entsprechen. Hier ermöglicht
Gimp \cite{gimp} zwar eine Messung in Pixeln, allerdings ist die Bestimmung der intensivsten
Stellen sehr subjektiv und liefert eine große Sensibilität für Fehler.

Außerdem war die Kamera geneigt gegenüber der Aufspaltung, sodass die Werte für $\Delta s$ und
$\delta s$ nicht äquidistant waren.\\
Zudem war die Kamera leicht gekippt bezüglich des restlichen Versuchsaufbau, sodass das Linienspektrum nicht ganz senkrecht auf die Kameralinse fiel.\\
Dieser Effekt wurde durch eine Rotation der Bilder um $1,3°$ mithilfe von Gimp \cite{gimp} ein wenig aufgehoben.

Die größte Abweichung tritt bei der blauen $\sigma$-Linie auf. Dies lässt sich vielleicht durch
die inadäquate Auflösung zwischen den Niveaus $6^3S_1$ mit $g_J=2$ und $5^3P_1$ mit $g_J=\frac{3}{2}$ und eine dadurch bedingte Überlagerung erklären, was den Trend zu kleineren $g$ erklären würde.
