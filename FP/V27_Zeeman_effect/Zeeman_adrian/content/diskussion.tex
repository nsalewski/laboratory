\section{Diskussion}
\label{sec:Diskussion}
In der Tabelle \ref{tab:vergleich} sind Theoriewerte und experimentelle
Werte zum Vergleich aufgelistet.
Eine Fehlerquelle in der Auswertung liegt vorallem darin, dass
im Interferenzbild die Intensitätsmaxima nur nach Augenmaß abgeschätzt werden.
Dies ist vorallem bei dem $\pi$-Übergang blau ein Problem (Abbildung \ref{fig:blaupicut}),
da dieses mehr Bildrauschen enthält, als die anderen Bilder.
Bei dieser Messung fällt außerdem noch auf, dass die abgeschätzten Maxima
nicht unter denen der $B=0\,\si{\milli\tesla}$ Messung lagen, bzw.
auch nicht unter denen, bei gleichem Feld, aber mit Polarisationsfilter.
Sie sind also verschoben.
Bei der blauen $\sigma$-Linie fällt die Abweichung nach unten auf, welche
vermutlich dadurch verursacht wird, dass sich die $g=\frac{3}{2}$ und die
$g=2$ Linie sich mit der verwendeten Apparatur nicht genau auflösen lassen und sich
deswegen überlagern. Was dafür spricht ist, dass der gemessene Wert
näher an dem Mittelwert $g=1.75$ liegt, als an dem angenommenen Wert $g=2$.

\begin{table}
  \centering
  \caption{Gemessener Wert und Theorie Wert sind zum Vergleich aufgelistet.
  Zudem werden die relativen Fehler berechnet.}
  \label{tab:vergleich}
  \begin{tabular}{c | c | c | c}
    \toprule
    Art & $g_\text{theo}$ & $g_\text{exp}$ & Abweichung\\
    \midrule
    rot $\pi$     & 1 & $0.995\pm0.018$ & $0.5 \, \si{\percent}$ \\
    blau $\pi$    & $\frac{1}{2}$ &$0.51 \pm 0.01$ & $2 \, \si{\percent}$ \\
    blau $\sigma$ & 2 & $1.76 \pm 0.06$ & $12 \, \si{\percent}$ \\
    \bottomrule
  \end{tabular}
\end{table}
