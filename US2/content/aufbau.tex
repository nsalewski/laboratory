\subsection{Versuchsaufbau}
\label{sec:Versuchsaufbau}
Der vorliegende Versuchsaufbau besteht aus einem Ultraschallechoskop, Ultraschallsonden verschiedener Frequenzen (farblich markiert: \SI{1}{\mega\Hz} (blau),  \SI{2}{\mega\Hz} (rot),  \SI{4}{\mega\Hz} (grün)), sowie einem Rechner zur Datenaufnahme und-analyse.
Am Echoskop kann über einen Kippschalter REFLEC./TRANS. eine Messung mittels einer Ultraschallsonde (Impuls-Echo-Verfahren), bzw. mittels zweier gleicher Ultraschallsonden (Durchschallungs-Verfahren) eingestellt werden.\\
Am Rechner können die erfassten Daten in dem Programm \textbf{A-Scan} angezeigt und ausgewertet werden.
Im oberen Bereich des Programmfensters kann die Scan-Art gewählt werden.
Erstellte Grafiken können ebenso wie aufgenommene Daten aus dem Programm heraus exportiert werden.
Wird der Scan-Modus \textbf{A-Scan} gewählt, werden zwei Graphen dargestellt. Im oberen Fenster befindet sich der A-Scan. Die Amplitude lässt sich über die Programmeinstellung hierbei sowohl gegen die Eindringtiefe in Microsekunden, als auch gegen die Eindringtiefe in Millimetern auftragen. Für letzteres muss die Schallgeschwindigkeit im Probenmaterial in das Programm eingetragen werden.\\
Im unteren Fenster befindet sich die laufzeit-und tiefenabhängige Verstärkung des Signals (TGC; Time Gain Control). Diese kann über das Echoskop mittels der Verstärkungsparameter Treshold, Wide,Slope und Start am Ultraschallechoskop eingestellt werden.\\
Bei der Wahl des \textbf{TM-Mode} wird auf der y-Achse die Laufzeit der Impulse aufgetragen und auf der x-Achse wird kontinuierlich fortschreitend die Messzeit aufgetragen.
Die Messung kann über die Knöpfe START und STOP gestartet, bzw. gestoppt werden.
Die Achsen lassen sich über die Zoom Area skalieren.
Zudem lässt sich im Programm noch über den \textbf{FFT}-Modus das Frequenzspektrum und das zugehörige Cepstrum anzeigen.\\
Zur Untersuchung steht ein Acrylblock mit elf Fehlstellen sowie ein Herzmodell, bestehend aus zwei Bechergläsern, einer Gummimembran und einem Handblasebalg zur Simulation des Herzschlags, zur Verfügung.
