\section{Diskussion}
\label{sec:Diskussion}
\begin{table}
  \centering
	\caption{Bestimmten Durchmesser der Fehlstellen im Acrylblock mit zugehörigen Fehlern.}
	\label{tab:werte_ascan}
	\begin{tabular}{cccccc}
		\toprule
		$n$ & $d_\mathrm{A}$/$\si{\milli\meter}$ & $d_{\mathrm{SL}}$/$\si{\milli\meter}$ & $\Delta d_\mathrm{A}$&$d_\mathrm{B}$&$\Delta d_\mathrm{B}$ \\
		\midrule
		1 & 2.0 & 1.3 & \SI{53.8}{\percent}&  --      & --     \\
		2 & 1.4 & 1.3 & \SI{7.7}{\percent} &   1.5     & \SI{15.8}{\percent}    \\
		3 & 6.0 & 5.8 & \SI{3.4}{\percent} &   5.3     & \SI{8.6}{\percent}    \\
		4 & 4.7 & 4.8 & \SI{2.1}{\percent} &   4.7     & \SI{2.1}{\percent}    \\
		5 & 3.6 & 4.0 & \SI{10.0}{\percent}&   3.8     & \SI{5.0}{\percent}     \\
		6 & 2.7 & 2.9 & \SI{6.9}{\percent} &   2.6     & \SI{10.3}{\percent}    \\
		7 & 2.3 & 2.9 & \SI{20.7}{\percent}&   2.3     & \SI{20.7}{\percent}     \\
		8 & 2.2 & 2.9 & \SI{24.1}{\percent}&   2.6     & \SI{10.3}{\percent}     \\
		9 & 2.6 & 2.9 & \SI{10.3}{\percent}&   2.5     & \SI{13.8}{\percent}     \\
		10 & 2.1 & 2.9 & \SI{27.6}{\percent}&  --      &--      \\
		11 & 9.3 & 9.5 & \SI{2.1}{\percent}&   9.5     & \SI{0.0}{\percent}     \\
		\bottomrule
	\end{tabular}
\end{table}
Ein Großteil der mittels Ultraschall-Scan ermittelten Werte liegt relativ nahe an den erwarteten Werten. Insgesamt lässt sich sagen, dass sehr kleine Fehlstellen mit deutlich größerer Ungenauigkeit, beziehungsweise sogar gar nicht über den Ultraschallscan vermessen werden konnten.
Je größer die Fehlstelle war, desto kleiner ist bei beiden Messverfahren tendenziell der Fehler zum mittels Schiebelehre ermittelten Durchmesser.
