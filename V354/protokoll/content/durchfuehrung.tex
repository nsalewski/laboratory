\section{Durchführung}
\label{sec:Durchführung}

\subsection{Versuchsaufbau}
\label{sec:Versuchsaufbau}
Der Versuchsaufbau besteht aus einem Ultraschallechoskop, an dessen Ausgänge zwei Ultraschallsonden mit \SI{2}{\mega\Hz} gekoppelt sind, und einem Rechner zur Datenaufnahme und -analyse.\\
An das Ultraschallechoskop sind zwei Ultraschallsonden angeschlossen, mithilfe derer sich sowohl eine Impuls-Echo-Messung, als auch eine Durchschallmessung realisieren lässt.
Am Rechner werden die gemessenen Daten mittels des Programms \textquote{Echoview} ausgewertet.\\
Hierbei ist \textquote{Echoview} in der Lage, vier verschiedene Diagramme darzustellen.
Im linken oberen Graphen wird der A-Scan dargestellt, also die Amplitude gegen die Zeit aufgetragen.
Der linke untere Graph stellt die gewählte Verstärkung dar. Die Verstärkung lässt sich am Ultraschallechoskop über die Drehknöpfe zur laufzeit-bzw. tiefenabhängigen Verstärkung (TGC; Time Gain Control) und ebenso über die Verstärkung des Outputs und der Empfindlichkeit der Sonden regulieren.\\
Zu Beachten ist, dass eine Verstärkung nur gewählt werden darf, wenn die auszuwertende Messreihe nicht zur Untersuchung der Amplitudenhöhe dient.
Die beiden rechten Graphiken sind das berechnete Spektrum der Messdaten (FFT), bzw. ihr
Cepstrum.
Erzeugte Graphiken und Messdaten können aus dem Programm heraus exportiert werden.
Als zu untersuchende Versuchsobjekte stehen Acrylzylinder verschiedener Länge, Acrylplatten unterschiedlicher Dicke sowie das Modell eines menschlichen Auges im Maßstab 3:1 zur Verfügung.


\subsection{Versuchsdurchführung}
\label{sec:Versuchsbeschreibung}

Vor Beginn der Messung werden die Kenngrößen der verwendeten Bauteile notiert.

Zur Bestimmung des effektiven Dämpfungswiderstandes $R_\text{eff}$ und der Abklingdauer $T_ex$ wird die Abnahme der Spannungsamplitude der Kondensatorspannung $U_\text{C}$ untersucht.
Es wird der kleinere der beiden fixen Widerstände des Schaltkreises, $R_\text{1}=(48.1 \pm 0.1)\,\si{\ohm}$, verwendet.
Dazu wird am Funktionengenerator ein Nadelimpuls eingestellt.
Die Kondensatorspannung wird hierbei auf den ersten Kanal des Oszilloskops gegeben.
Auf dem Bildschirm des Oszilloskops ist nun die Abnahme der Spannungsamplitude der Kondensators zu sehen.
Der Nadelimpuls soll nun so eingestellt werden, dass die Spannungsamplitude am Kondensator sich zwischen zwei Impulsen mindestens um den Faktor 3 bis 8 ändert.
Dazu werden die Frequenz und Amplitude des Nadelimpulses ebenso wie das Triggerlevel am Oszilloskop variiert, bis sich der gewünschte Spannungsverlauf auf dem Bildschirm des Oszilloskops zeigt.
Mit der Cursorfunktion werden nun die positiven Spannungsamplituden mit den zugehörigen Zeitdifferenzen zum auslösenden Nadelimpuls bestimmt und der Spannungsverlauf über die USB-Ausgabe des Oszilloskops gespeichert.

Für die zweite Messung zur Bestimmung des Widerstands $R_\text{ap}$ im aperiodischen Grenzfall wird der regelbare Widerstand $R_\text{3}$ verwendet.
Dieser wird zunächst auf seinen maximalen wert gestellt. Am Oszilloskop zeigt sich nun lediglich die erwartete stetig abfallende Spannung am Kondensator.
Der Widerstand $R_\text{3}$ wird nun so weit heruntergeregelt, dass am Spannungsverlauf Überschwinger sichtbar werden.
Der Widerstand $R_\text{3}$ wird nun langsam so weit wieder hoch geregelt, dass die Überschwinger soeben verwschwinden. Der Wert des Widerstandes $R_\text{3}$ entspricht nun dem Widerstand für den aperiodischen Grenzfall des Schaltkreises und wird notiert.

Zur Bestimmung der Frequenzabhängigkeit der Kondensatorspannung und der Phase zwischen Kondensator- und Erregerspannung wird am Funktionengenerator eine Sinusspannung eingestellt und die Generatorspannung $U(t)$ über den zweiten Kanal des Oszilloskops abgegriffen.
Am Oszilloskop werden beide Spannunsgverläufe übereinander gelegt.
Zunächst wird nun die Frequenz einmal über den Bereich $1 \,\si{\kilo\Hz}$ bis etwa $100 \,\si{\kilo\Hz}$ und dabei die Spannungsverläufe am Oszilloskop beobachtet, sodass etwa der Bereich der Resonatorfrequenz bestimmt werden kann.
Für Frequenzen nahe der Resonatorfrequenz wird im Weiteren in höherer Auflösung gemessen.
Es werden nun etwa 20 Datensätze aus Generatorspannung $U(t)$, Kondensatorspannung $U_\text{C}$, dem Abstand $a$ zwischen den Nulldurchgängen der beiden Spannungen und der jeweiligen Frequenz $f$ der Sinusspannung aufgenommen.
Die Spannungsmaxima beider Spannungen werden jeweils über die Measure-Funktion des Oszilloskops bestimmt.
Der Abstand zwischen den Nulldurchgängen der beiden Spannungsverläufe wird mit der Cursorfunktion bestimmt.
Die Periodenlänge $b$, welche zusätzlich zur Berechnung der Phasendifferenz $\phi$ benötigt wird, muss nicht gemessen werden, da sie über den Kehrwert der verwendeten Frequenz berechnet werden kann.
