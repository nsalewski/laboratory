\section{Auswertung}
\label{sec:Auswertung}
Die zu Beginn notierten Werte für die Komponenten des Schaltkreises lauten:
\begin{equation*}
  L=(10.11 \pm 0.03) \,\si{\milli\henry}
\end{equation*}

\begin{equation*}
C=(2.098 \pm 0.006) \cdot 10^{-9} \, \si{\farad}
\end{equation*}

\begin{equation*}
R_\text{1}= (48.1 \pm 0.1) \, \si{\ohm}
\end{equation*}

\begin{equation*}
R_\text{2}= (509.5\pm 0.5)\,\si{\ohm}
\end{equation*}


In Tabelle \ref{tab:messung1} finden sich die gemessenen Daten der Spannungsamplituden $U_C$ und die zugehörigen Zeiten $t$.
Der Verlauf der Spannungsamplitude ist in Abbildung \ref{fig:spannungsamplitude} abgebildet. Die zugehörige Einhüllende ist in Abbildung \ref{fig:einhuellende} dargestellt.
Da  Gleichung \eqref{eqn:blablup} ergibt sich die Einhüllende mittels
\begin{equation}
A=A_\text{0}exp{-2 \pi \mu t}
\end{equation}
%A_0 =  6.5992883063 ± 0.0353016378746
%mü =  851.764459304 ± 7.4567304364

%%%%%%%%%%%%%%%%%%%%%%%%%%%%%%%%%%%%%%%%%%%%%%%%%%%%%%%%%%%%%%%%%%%%%%%%%%%%%%%%%%%%%%%%%%5
\subsection{Bestimmung von $R_{\text{ap}}$}

Nach Gleichung \eqref{eqn:apigrenz} ergibt sich der Widerstand $R_{\text{ap}}$, bei dem der 
aperiodische Grenzfall eintritt zu
\begin{equation}
	R_{\text{ap}} = \pm \sqrt{\frac{4L}{C}} \, \text{,}
	\label{eqn:rap}
\end{equation}
wobei der positive Wert -- also die physikalisch relevante Größe -- betrachtet wird.
Mit den notierten Werten und Gauß'scher Fehlerfortpflanzung erhält man schließlich
\begin{equation*}
	R_{\text{ap}} = (4390 \pm 9) \, \si{\ohm} \text{.}
\end{equation*}

Der gemessene Wert ist $R_{\text{ap}} = \SI{3400}{\ohm}$.

%%%%%%%%%%%%%%%%%%%%%%%%%%%%%%%%%%%%%%%%%%%%%%%%%%%%%%%%%%%%%%%%%%%%%%%%%%%%%%%%%%%%%%%%%%%%%%
\subsection{Phasenverschiebung}

Die Frequenzabhängigkeit der Phasenverschiebung ist in Abbildung \ref{fig:phasenplot} dargestellt.

\begin{figure}
	\centering
	\includegraphics[width=0.7\textwidth]{build/taskd.pdf}
	\caption{Frequenzabhängigkeit der Phasenverschiebung zwischen Kondensator- und Erregerspannung}
	\label{fig:phasenplot}
\end{figure}
