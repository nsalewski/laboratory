\section{Auswertung}
\label{sec:Auswertung}
Die zu Beginn notierten Werte für die Komponenten des Schaltkreises lauten:
\begin{equation*}
  L=(10.11 \pm 0.03) \,\si{\milli\henry}
\end{equation*}

\begin{equation*}
C=(2.098 \pm 0.006) \cdot 10^{-9} \, \si{\farad}
\end{equation*}

\begin{equation*}
R_\text{1}= (48.1 \pm 0.1) \, \si{\ohm}
\end{equation*}

\begin{equation*}
R_\text{2}= (509.5\pm 0.5)\,\si{\ohm}
\end{equation*}


In Tabelle \ref{tab:messung1} finden sich die gemessenen Daten der Spannungsamplituden $U_C$ und die zugehörigen Zeiten $t$.
Der Verlauf der Spannungsamplitude ist in Abbildung \ref{fig:spannungsamplitude} abgebildet. Die zugehörige Einhüllende ist in Abbildung \ref{fig:einhuellende} dargestellt.
Nach Gleichung \eqref{eqn:blablup} ergibt sich die Einhüllende mittels
\begin{equation}
A=A_\text{0}exp{-2 \pi \mu t}
\end{equation}
