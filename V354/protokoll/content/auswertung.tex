\section{Auswertung}
\label{sec:Auswertung}
Die zu Beginn notierten Werte für die Komponenten des Schaltkreises lauten:
\begin{equation*}
	L=(10.11 \pm 0.03) \,\si{\milli\henry}
\end{equation*}

\begin{equation*}
	C=(2.098 \pm 0.006) \cdot 10^{-9} \, \si{\farad}
\end{equation*}

\begin{equation*}
	R_\text{1}= (48.1 \pm 0.1) \, \si{\ohm}
\end{equation*}

\begin{equation*}
	R_\text{2}= (509.5\pm 0.5)\,\si{\ohm}
\end{equation*}


In Tabelle \ref{tab:messung1} finden sich die gemessenen Daten der Spannungsamplituden $U_C$ und die zugehörigen Zeiten $t$.
Der Verlauf der Spannungsamplitude ist in Abbildung \ref{fig:spannungsamplitude} abgebildet. Die zugehörige Einhüllende ist in Abbildung \ref{fig:einhuellende} dargestellt.
Da sich $U_\text{0}$ nach dem Ohmschen Gesetz nur um einen Faktor von $I_\text{0}$ unterscheidet, ergibt sich nach Gleichung \eqref{eqn:schwingi} die Einhüllende mittels
\begin{equation}
	A=A_\text{0}exp{-2 \pi \mu t}
\end{equation}
Die Ausgleichsrechnung mittels Scipy und Python liefert die Werte:
\begin{equation*}
	A_0 =  (6.60 \pm 0.04) \,\si{\volt}
\end{equation*}
\begin{equation*}
	\mu =  (851.8 \pm 7.5) \, \frac{1}{\si{\second}}
\end{equation*}

$R_\text{eff}$ lässt sich nun mittels $\mu$ über Gleichung \eqref{eqn:defis} ermitteln. $T_\text{ex}$ ergibt sich nun nach Gleichung \eqref{eqn:abkling}.
Mittels Scipy und Python ergeben sich die Größen samt ihrer Fehler zu:
\begin{equation*}
  T_{\text{ex}}=(18.7 \pm 0.2) \cdot 10^{-5}\,\si{\second}
\end{equation*}

\begin{equation*}
  R_{\text{eff}}= (108.2 \pm 1.0) \,\si{\ohm}
\end{equation*}

Ein Vergleich zwischen dem berechneten $R_\text{eff}$ und dem in der Schaltung eingebauten Widerstand $R_\text{1}$ zeigt eine Differenz von $\delta R=60.1 \,\si{\ohm}$.
Diese lässt sich begründen über den bisher nicht beachteten Innenwiderstand des Funktionengenerators von $R=50\,\si{\ohm}$. Die verbleibende Differenz lässt sich begründen mit kleinen Messunsicherheiten und liegt annähernd im Toleranzbereich.
Für die weiteren Berechnungen wird der Innenwiderstand des Generators nun mitbetrachtet.
