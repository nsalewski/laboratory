\section{Diskussion}
\label{sec:Diskussion}
Beim Vergleich der gemessenen beziehungsweise aus den Plots abgelesenen Messgrößen zeigen sich nur geringe Abweichungen gegenüber den Theoriewerten.
Im Detail zeigt ein Vergleich zwischen der theoretisch aus den Kenngrößen des Schwingkreises berechneten Güte $q_\mathrm{Theorie}$ mit der experimentell bestimmten Güte $q_\mathrm{Experiment}$ zeigt sich eine Abweichung von $8\%$.

Die Differenz zwischen dem berechneten $R_\text{eff}$ und dem in der Schaltung eingebauten Widerstand $R_\text{1}$ zeigt eine Differenz von $\Delta R=60.1 \,\si{\ohm}$.
Diese lässt sich auf den bisher nicht berücksichtigten Innenwiderstand des Funktionengenerators von $R=50\,\si{\ohm}$ zurückführen. Die verbleibende Differenz lässt sich mit kleinen Messunsicherheiten begründen und liegt annähernd im Toleranzbereich.



Für bla blup usw.
Beim Vergleich des abgelesenen Wertes für $\omega_+ - \omega_-$ mit dem theoretisch berechneten Wert ergibt sich eine Abweichung von denkste, sind viel weniger als 500\%
Diese liegen alle absolut im Rahmen der Toleranz.
Allerdings fiel bereils zu Beginn der Messung auf, dass der Funktionengenerator keinen richtigen Nadelimpuls lieferte. Stattdessen lieferte der Generator einen Rechteckimpuls. Vergleiche dazu Abbildung blabiupskföoj
Wahrscheinlich wird dies allerdings wenig Einfluss auf die Messung gehabt haben.
