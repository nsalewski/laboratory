\section{Auswertung}
\label{sec:Auswertung}
\FloatBarrier
\subsection{Bestimmung der Geschwindigkeiten der verschiedenen Gänge}
Mit dem Aufbau aus Abschnitt \ref{sec:} lassen sich die Geschwindigkeiten des Wagens bei den
jeweiligen Gängen mit
\begin{equation}
	v = \frac{s}{n \cdot 10^{-3} \,\si{\second}}
\end{equation}
bestimmen, wobei $n$ der Anzahl der gemessenen Impulse entspricht.
Der Faktor $10^{-3}$ entspricht der Impulsfrequenz und ergibt sich aus dem Produkt der
Zeitbasis ($\frac{1}{\si{\micro\second}}$ und dem Untersetzungsfaktor ($10^2$), womit sich
die Zeit $t$ zu $t = n \cdot 10^{-3}$ ergibt.
Die Strecke $s$ ($s = \SI{20}{\centi\meter}$) ist die ausgemessene Distanz zwischen den beiden
Lichtschranken.
\begin{table}
	\centering
	\caption{Messwerte zur Bestimmung der Geschwindigkeit $v$ bei verschiedenen Gängen.}
	\label{tab:pace}
	\begin{tabular}{cccc}
		\toprule
		Gang       & $n$        & $t$ / $\si{\second}$ & $v$ / $\si{\meter\per\second}$ \\
		\midrule
		6 vor      & 39538\pm20 & 3.954\pm0.002        & 0.05058\pm0.00003              \\
		6 zurück  & 39826\pm22 & 3.983\pm0.002        & 0.05022\pm0.00003              \\
		12 vor     & 19693\pm29 & 1.970\pm0.003        & 0.1016\pm0.0002                \\
		12 zurück & 19952\pm23 & 1.995\pm0.002        & 0.1002\pm0.0001                \\
		18 vor     & 13147\pm12 & 1.315\pm0.001        & 0.1521\pm0.0001                \\
		18 zurück & 13285\pm9  & 1.329\pm0.001        & 0.1506\pm0.0001                \\
		24 vor     & 9865\pm8   & 0.987\pm0.001        & 0.2027\pm0.0002                \\
		24 zurück & 9985\pm10  & 0.999\pm0.001        & 0.2003\pm0.0002                \\
		30 vor     & 7908\pm4   & 0.7908\pm0.0004      & 0.2529\pm0.0001                \\
		30 zurück & 8050\pm40  & 0.805\pm0.004        & 0.248\pm0.001                  \\
		36 vor     & 6558\pm6   & 0.6558\pm0.0006      & 0.3050\pm0.0003                \\
		36 zurück & 6678\pm6   & 0.6678\pm0.0006      & 0.2995\pm0.0003                \\
		42 vor     & 5598\pm4   & 0.5598\pm0.0004      & 0.3572\pm0.0003                \\
		42 zurück & 5729\pm7   & 0.5729\pm0.0007      & 0.3491\pm0.0004                \\
		48 vor     & 4840\pm50  & 0.484\pm0.005        & 0.414\pm0.004                  \\
		48 zurück & 5013\pm5   & 0.5013\pm0.0005      & 0.3989\pm0.0004                \\
		54 vor     & 4362\pm3   & 0.4362\pm0.0003      & 0.4585\pm0.0004                \\
		54 zurück & 4452\pm4   & 0.4452\pm0.0004      & 0.4493\pm0.0004                \\
		60 vor     & 3910\pm6   & 0.3910\pm0.0006      & 0.5115\pm0.0008                \\
		60 zurück & 4041\pm17  & 0.404\pm0.002        & 0.495\pm0.002                  \\
		\bottomrule
	\end{tabular}
\end{table}
Die sich ergebenden Werte für die Geschwindigkeit des Wagens sind in Tabelle \ref{tab:pace}
aufgetragen. Die Mittelwerte der gemessenen Impulsanzahl $n$ wurden mit Formel
\eqref{eqn:mittelwert} bestimmt und der zugehörige Fehler mit Formel
\eqref{eqn:mittelwertfehler}. Weiterhin wird der Fehler der Geschwindigkeit $v$ mit
Gauß'scher Fehlerfortpflanzung (Formel \eqref{eqn:fehlerfortpflanzung} ermittelt.



%%%%%%%%%%%%%%%%%%%%%%%%%%%%%%%%%%%%%%%%%%%%%%%%%%%%%%%%%%%%
\FloatBarrier
\subsection{Ruhefrequenz}
\begin{table}
	\centering
	\caption{Messwerte zur Bestimmung der Ruhefrequenz $\nu_0$.}
	\label{tab:ruhefr}
	\begin{tabular}{cc}
		\toprule
		Messwert & $\nu_0$ / $\si{\hertz}$ \\
		\midrule
		1        & 20741.25                \\
		2        & 20741.25                \\
		3        & 20742.50                \\
		4        & 20741.25                \\
		5        & 20742.50                \\
		6        & 20742.50                \\
		\bottomrule
	\end{tabular}
\end{table}
%%%%%%%%%%%%%%%%%%%%%%%%%%%%%%%%%%%%%%%%%%%%%%%
%b
\FloatBarrier
\subsection{Auswertung der Messung der Wellenlänge}
In Tabelle \ref{tab:juice} finden sich die direkt gemessenen Wellenlängen. Diese ergeben sich aus der Differenz des Abstands zwischen Mikrophon und Lautsprecher für zwei verschiedene Entartungen der LISSAJOUS-Figuren zu Geraden.
Da sowohl die Entartungen zu steigenden, als auch zu fallenden Geraden betrachtet wurden, muss der gemessene Abstand jeweils noch verdoppelt werden, sodass sich die ganze Wellenlänge $\lambda$ ergibt.
Nach Formel \eqref{eqn:mittelwert} ergibt sich der Mittelwert der Wellenlänge zu.
\begin{equation}
	\label{eqn:wellenlänge}
	\lambda=\SI{16.80(6)}{\milli\metre}\text{.}
\end{equation}
Der Fehler des Mittelwerts ergibt sich mit Formel \eqref{eqn:mittelwertfehler}.\\
Zur Bestimmung der Schallgeschwindigkeit wird zudem die Frequenz $\nu_{0}$ benötigt.
Die bestimmten Frequenzen finden sich in Tabelle \ref{tab:ruhefrequenz}.
Nach Formel \eqref{eqn:mittelwert} mit dem Fehler nach Formel \eqref{eqn:mittelwertfehler} ergibt sich $\nu_0$ zu
\begin{equation}
	\label{eqn:rrrruhe}
	\nu_0=\SI{20600(60)}{\Hz}\text{.}
\end{equation}
Die Schallgeschwindigkeit $c$ ergibt sich damit zu
\begin{equation}
	\label{eqn:Schallgeschwindigkeit}
	c=\nu_0 \cdot\lambda=\SI{346(2)}{\meter\per\second}\text{.}
\end{equation}
Der Fehler wurde hierbei nach Gauß'scher Fehlerfortpflanzung \eqref{eqn:fehlerfortpflanzung}.

Die Größe $\frac{\nu_0}{c}=\frac{1}{\lambda}$ ergibt sich somit zu
\begin{equation}
	\label{eqn:wichtige_größe}
	\frac{\nu_0}{c}=\frac{1}{\lambda}=\SI{59.5(2)}{\per\meter}
\end{equation}


\begin{table}
	\centering
	\caption{Messwerte der Ruhefrequenz $\nu_{\mathrm{0}}$ zur Bestimmung der Schallgeschwindigkeit $c$.}
	\label{tab:ruhefrequenz}
	\begin{tabular}{cc}
		\toprule
		Datenpunkt & $\nu_0$ / $\si{\hertz}$ \\
		\midrule
		1          & 20742                   \\
		2          & 20160                   \\
		3          & 20740                   \\
		4          & 20741                   \\
		5          & 20685                   \\
		6          & 20715                   \\
		7          & 20718                   \\
		8          & 20520                   \\
		9          & 20633                   \\
		10         & 20373                   \\
		\bottomrule
	\end{tabular}

\end{table}

\begin{table}
	\centering
	\caption{Messung der Wellenlänge über die Entartung der LISSAJOUS-Figuren zur Bestimmung der Schallgeschwindigkeit $c$.}
	\label{tab:juice}
	\begin{tabular}{ccc}
		\toprule
		Entartung & Abstand $d$/$\si{\milli\meter}$ & Wellenlänge $\lambda$/$\si{\milli\meter}$ \\
		\midrule
		1         & 8.5                             & 17.0                                       \\
		2         & 8.3                             & 16.6                                       \\
		3         & 8.4                             & 16.8                                       \\
		4         & 8.4                             & 16.8                                       \\
		5         & 8.4                             & 16.8                                       \\
		\bottomrule
	\end{tabular}
\end{table}
\FloatBarrier
%%%%%%%%%%%%%%%%%%
\subsection{Nachweis des Doppler-Effekts über die direkte Messung}
Der Doppler-Effekt wird nun untersucht für eine bewegte Quelle und einen ruhenden Empfänger.
Dafür werden die zuvor verwendeten Geschwindigkeiten des Wagens verwendet, auf dem die Quelle montiert ist.
In Tabelle \ref{tab:direkt} sind dazu die bestimmten Wagengeschwindigkeiten gegen die Differenz $\Delta \nu$ zwischen Ruhefrequenz $\nu_\mathrm{0}$ und der jeweils gemessenen Frequenz $\nu_{\mathrm{Q}}$ aufgetragen.
Hierbei wird die Geschwindigkeit des Wagens für die Fahrt von der Quelle weg als negativ betrachtet.
\begin{table}
	\centering
	\caption{}
	\label{tab:direkt}
	\begin{tabular}{cc}
		Wagengeschwindigkeit $v$/$\si{\meter\per\second}$ & $\Delta \nu$/$\si{\Hz}$ \\
		-0.495                                            & -30.375                 \\
		-0.449                                            & -26.875                 \\
		-0.399                                            & -24.125                 \\
		-0.349                                            & -20.625                 \\
		-0.299                                            & -17.875                 \\
		-0.248                                            & -15.125                 \\
		-0.2                                              & -11.875                 \\
		-0.151                                            & -8.625                  \\
		-0.1                                              & -5.875                  \\
		-0.05                                             & -2.375                  \\
		0.051                                             & 3.125                   \\
		0.102                                             & 6.375                   \\
		0.152                                             & 9.375                   \\
		0.203                                             & 12.125                  \\
		0.253                                             & 15.625                  \\
		0.305                                             & 18.125                  \\
		0.357                                             & 21.125                  \\
		0.414                                             & 24.375                  \\
		0.459                                             & 26.875                  \\
		0.512                                             & 30.375                  \\
	\end{tabular}
\end{table}
\begin{figure}
	\includegraphics{Bilder/vquellebewegt.pdf}
	\caption{Direkte Messung der Frequenz}
	\label{fig:doppler}
\end{figure}
\FloatBarrier
\subsection{Nachweis des Doppler-Effekts mit der Schwebungsmethode}
\begin{table}
	\centering
	\caption{}
	\label{tab:direkt}
	\begin{tabular}{cc}
		\toprule
		Wagengeschwindigkeit $v$/$\si{\meter\per\second}$ & $\Delta \nu$/$\si{\Hz}$ \\
		\midrule
		-0.495                                            & -30.625                 \\
		-0.449                                            & -27.5                   \\
		-0.399                                            & -24.375                 \\
		-0.349                                            & -21.25                  \\
		-0.299                                            & -18.125                 \\
		-0.248                                            & -15.0                   \\
		-0.2                                              & -12.125                 \\
		-0.151                                            & -9.375                  \\
		-0.1                                              & -7.5                    \\
		-0.05                                             & -3.0125                 \\
		0.051                                             & 3.025                   \\
		0.102                                             & 6.375                   \\
		0.152                                             & 9.375                   \\
		0.203                                             & 12.5                    \\
		0.253                                             & 15.5                    \\
		0.305                                             & 18.625                  \\
		0.357                                             & 21.625                  \\
		0.414                                             & 24.375                  \\
		0.459                                             & 27.5                    \\
		0.512                                             & 30.625                  \\
		\bottomrule
	\end{tabular}
\end{table}
\begin{figure}
	\includegraphics{Bilder/schwebung.pdf}
	\caption{Schwebungsmethode}
	\label{fig:doppler_schwebung}
\end{figure}
