\section{Auswertung}
\label{sec:Auswertung}
\FloatBarrier
\subsection{Ruhefrequenz}
\begin{table}
	\centering
	\caption{Messwerte zur Bestimmung der Ruhefrequenz $\nu_0$.}
	\label{tab:ruhefr}
	\begin{tabular}{cc}
		\toprule
		Messwert & $\nu_0$ / $\si{\hertz}$ \\
		\midrule
		1        & 20741.25                \\
		2        & 20741.25                \\
		3        & 20742.50                \\
		4        & 20741.25                \\
		5        & 20742.50                \\
		6        & 20742.50                \\
		\bottomrule
	\end{tabular}
\end{table}
%%%%%%%%%%%%%%%%%%%%%%%%%%%%%%%%%%%%%%%%%%%%%%%
%b
\FloatBarrier
\subsection{Auswertung der Messung der Wellenlänge}
In Tabelle \ref{tab:juice} finden sich die direkt gemessenen Wellenlängen. Diese ergeben sich aus der Differenz des Abstands zwischen Mikrophon und Lautsprecher für zwei verschiedene Entartungen der LISSAJOUS-Figuren zu Geraden.
Da sowohl die Entartungen zu steigenden, als auch zu fallenden Geraden betrachtet wurden, muss der gemessene Abstand jeweils noch verdoppelt werden, sodass sich die ganze Wellenlänge $\lambda$ ergibt.
Nach Formel \eqref{eqn:mittelwert} ergibt sich der Mittelwert der Wellenlänge zu.
\begin{equation}
	\label{eqn:wellenlänge}
	\lambda=\SI{16.80(6)}{\milli\metre}\text{.}
\end{equation}
Der Fehler des Mittelwerts ergibt sich mit Formel \eqref{eqn:mittelwertfehler}.\\
Zur Bestimmung der Schallgeschwindigkeit wird zudem die Frequenz $\nu_{0}$ benötigt.
Die bestimmten Frequenzen finden sich in Tabelle \ref{tab:ruhefrequenz}.
Nach Formel \eqref{eqn:mittelwert} mit dem Fehler nach Formel \eqref{eqn:mittelwertfehler} ergibt sich $\nu_0$ zu
\begin{equation}
	\label{eqn:rrrruhe}
	\nu_0=\SI{20600(60)}{\Hz}\text{.}
\end{equation}
Die Schallgeschwindigkeit $c$ ergibt sich damit zu
\begin{equation}
	\label{eqn:Schallgeschwindigkeit}
	c=\nu_0 \cdot\lambda=\SI{346(2)}{\meter\per\second}\text{.}
\end{equation}
Der Fehler wurde hierbei nach Gauß'scher Fehlerfortpflanzung \eqref{eqn:fehlerfortpflanzung}.

Die Größe $\frac{\nu_0}{c}=\frac{1}{\lambda}$ ergibt sich somit zu
\begin{equation}
	\label{eqn:wichtige_größe}
	\frac{\nu_0}{c}=\frac{1}{\lambda}=\SI{59.5(2)}{\per\meter}
\end{equation}


\begin{table}
	\centering
	\caption{Messwerte der Ruhefrequenz $\nu_{\mathrm{0}}$ zur Bestimmung der Schallgeschwindigkeit $c$.}
	\label{tab:ruhefrequenz}
	\begin{tabular}{cc}
		\toprule
		Datenpunkt & $\nu_0$ / $\si{\hertz}$ \\
		\midrule
		1          & 20742                   \\
		2          & 20160                   \\
		3          & 20740                   \\
		4          & 20741                   \\
		5          & 20685                   \\
		6          & 20715                   \\
		7          & 20718                   \\
		8          & 20520                   \\
		9          & 20633                   \\
		10         & 20373                   \\
		\bottomrule
	\end{tabular}

\end{table}

\begin{table}
	\centering
	\caption{Messung der Wellenlänge über die Entartung der LISSAJOUS-Figuren zur Bestimmung der Schallgeschwindigkeit $c$.}
	\label{tab:juice}
	\begin{tabular}{ccc}
		\toprule
		Entartung & Abstand $d$/$\si{\milli\meter}$ & Wellenlänge $\lambda$/$\si{\milli\meter}$ \\
		\midrule
		1         & 8.5                             & 17.0                                       \\
		2         & 8.3                             & 16.6                                       \\
		3         & 8.4                             & 16.8                                       \\
		4         & 8.4                             & 16.8                                       \\
		5         & 8.4                             & 16.8                                       \\
		\bottomrule
	\end{tabular}
\end{table}
\FloatBarrier
%%%%%%%%%%%%%%%%%%
\subsection{Nachweis des Doppler-Effekts über die direkte Messung}

\begin{figure}
	\includegraphics{Bilder/vquellebewegt.pdf}
	\caption{Direkte Messung der Frequenz}
	\label{fig:doppler}
\end{figure}
\FloatBarrier
\subsection{Nachweis des Doppler-Effekts mit der Schwebungsmethode}
\begin{figure}
	\includegraphics{Bilder/schwebung.pdf}
	\caption{Schwebungsmethode}
	\label{fig:doppler_schwebung}
\end{figure}
