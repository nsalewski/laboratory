\section{Auswertung}
\label{sec:Auswertung}
\FloatBarrier
\subsection{Bestimmung der Geschwindigkeiten der verschiedenen Gänge}
Mit dem Aufbau aus Abschnitt \ref{sec:} lassen sich die Geschwindigkeiten des Wagens bei den 
jeweiligen Gängen mit
\begin{equation}
	v = \frac{s}{n \cdot 10^{-3} \,\si{\second}} 
\end{equation}
bestimmen, wobei $n$ der Anzahl der gemessenen Impulse entspricht.
Der Faktor $10^{-3}$ entspricht der Impulsfrequenz und ergibt sich aus dem Produkt der
Zeitbasis ($\frac{1}{\si{\micro\second}}$ und dem Untersetzungsfaktor ($10^2$), womit sich
die Zeit $t$ zu $t = n \cdot 10^{-3}$ ergibt.
Die Strecke $s$ ($s = \SI{20}{\centi\meter}$) ist die ausgemessene Distanz zwischen den beiden
Lichtschranken.
\begin{table}
	\centering
	\caption{Messwerte zur Bestimmung der Geschwindigkeit $v$ bei verschiedenen Gängen.}
	\label{tab:pace}
	\begin{tabular}{cccc}
		\toprule
		 Gang & $n$ & $t$ / $\si{\second}$ & $v$ / $\si{\meter\per\second}$ \\
		\midrule
		6 vor & 39538 \pm 20 & 39.538 \pm 0.020 & 0.005058 \pm 0.000003 \\
		6 zurück & 39826 \pm22 & 39.826 \pm 0.022 & 0.005022  \pm  0.000003 \\
		12 vor & 19693 \pm 29 & 19.693 \pm 0.029 & 0.010156 \pm 0.000015 \\
		12 zurück & 19952 \pm 23 & 19.952 \pm 0.023 & 0.010024 \pm 0.000012 \\
		18 vor & 13147 \pm 12 & 13.147 \pm 0.012 & 0.015213 \pm 0.000014 \\
		18 zurück & 13285 \pm 9 & 13.285 \pm 0.009 & 0.015055 \pm 0.000010 \\
		24 vor & 9865 \pm 8 & 9.865 \pm 0.008 & 0.020274 \pm 0.000015 \\
		24 zurück & 9985 \pm 10 & 9.985 \pm 0.010 & 0.020030 \pm 0.000020 \\
		30 vor & 7908 \pm 4 & 7.908 \pm 0.004 & 0.025292 \pm 0.000011 \\
		30 zurück & 8050 \pm 40 & 8.050 \pm 0.040 & 0.024830 \pm 0.000130 \\
		36 vor & 6558 \pm 6 & 6.558 \pm 0.006 & 0.030499 \pm 0.000027 \\
		36 zurück & 6678 \pm 6 & 6.678 \pm 0.006 & 0.029947 \pm 0.000029 \\
		42 vor & 5598 \pm 4 & 5.598 \pm 0.004 & 0.035724 \pm 0.000027 \\
		42 zurück & 5729 \pm 7 & 5.729 \pm 0.007 & 0.034910 \pm 0.000040 \\
		48 vor & 4840 \pm 50 & 4.840 \pm 0.050 & 0.041400 \pm 0.000400 \\
		48 zurück & 5013 \pm 5 & 5.013 \pm 0.005 & 0.039890 \pm 0.000040 \\
		54 vor & 4362.0 \pm 3 & 4.362 \pm 0.003 & 0.045851 \pm 0.000040 \\
		54 zurück & 4452 \pm 4 & 4.452 \pm 0.004 & 0.044930 \pm 0.000040 \\
		60 vor & 3910 \pm 6 & 3.910 \pm 0.006 & 0.051150 \pm 0.000080 \\
		60 zurück & 4041 \pm 17 & 4.041 \pm 0.017 & 0.049490 \pm 0.000210 \\
		\bottomrule
	\end{tabular}
\end{table}
Die sich ergebenden Werte für die Geschwindigkeit des Wagens sind in Tabelle \ref{tab:pace}
aufgetragen. Die Mittelwerte der gemessenen Impulsanzahl $n$ wurden mit Formel 
\eqref{eqn:mittelwert} bestimmt und der zugehörige Fehler mit Formel
\eqref{eqn:mittelwertfehler}. Weiterhin wird der Fehler der Geschwindigkeit $v$ mit 
Gauß'scher Fehlerfortpflanzung (Formel \eqref{eqn:fehlerfortpflanzung} ermittelt.



%%%%%%%%%%%%%%%%%%%%%%%%%%%%%%%%%%%%%%%%%%%%%%%%%%%%%%%%%%%%
\FloatBarrier
\subsection{Ruhefrequenz}
\begin{table}
	\centering
	\caption{Messwerte zur Bestimmung der Ruhefrequenz $\nu_0$.}
	\label{tab:ruhefr}
	\begin{tabular}{cc}
		\toprule
		Messwert & $\nu_0$ / $\si{\hertz}$ \\
		\midrule
		1        & 20741.25                \\
		2        & 20741.25                \\
		3        & 20742.50                \\
		4        & 20741.25                \\
		5        & 20742.50                \\
		6        & 20742.50                \\
		\bottomrule
	\end{tabular}
\end{table}
%%%%%%%%%%%%%%%%%%%%%%%%%%%%%%%%%%%%%%%%%%%%%%%
%b
\FloatBarrier
\subsection{Auswertung der Messung der Wellenlänge}
In Tabelle \ref{tab:juice} finden sich die direkt gemessenen Wellenlängen. Diese ergeben sich aus der Differenz des Abstands zwischen Mikrophon und Lautsprecher für zwei verschiedene Entartungen der LISSAJOUS-Figuren zu Geraden.
Da sowohl die Entartungen zu steigenden, als auch zu fallenden Geraden betrachtet wurden, muss der gemessene Abstand jeweils noch verdoppelt werden, sodass sich die ganze Wellenlänge $\lambda$ ergibt.
Nach Formel \eqref{eqn:mittelwert} ergibt sich der Mittelwert der Wellenlänge zu.
\begin{equation}
	\label{eqn:wellenlänge}
	\lambda=\SI{16.80(6)}{\milli\metre}\text{.}
\end{equation}
Der Fehler des Mittelwerts ergibt sich mit Formel \eqref{eqn:mittelwertfehler}.\\
Zur Bestimmung der Schallgeschwindigkeit wird zudem die Frequenz $\nu_{0}$ benötigt.
Die bestimmten Frequenzen finden sich in Tabelle \ref{tab:ruhefrequenz}.
Nach Formel \eqref{eqn:mittelwert} mit dem Fehler nach Formel \eqref{eqn:mittelwertfehler} ergibt sich $\nu_0$ zu
\begin{equation}
	\label{eqn:rrrruhe}
	\nu_0=\SI{20600(60)}{\Hz}\text{.}
\end{equation}
Die Schallgeschwindigkeit $c$ ergibt sich damit zu
\begin{equation}
	\label{eqn:Schallgeschwindigkeit}
	c=\nu_0 \cdot\lambda=\SI{346(2)}{\meter\per\second}\text{.}
\end{equation}
Der Fehler wurde hierbei nach Gauß'scher Fehlerfortpflanzung \eqref{eqn:fehlerfortpflanzung}.

Die Größe $\frac{\nu_0}{c}=\frac{1}{\lambda}$ ergibt sich somit zu
\begin{equation}
	\label{eqn:wichtige_größe}
	\frac{\nu_0}{c}=\frac{1}{\lambda}=\SI{59.5(2)}{\per\meter}
\end{equation}


\begin{table}
	\centering
	\caption{Messwerte der Ruhefrequenz $\nu_{\mathrm{0}}$ zur Bestimmung der Schallgeschwindigkeit $c$.}
	\label{tab:ruhefrequenz}
	\begin{tabular}{cc}
		\toprule
		Datenpunkt & $\nu_0$ / $\si{\hertz}$ \\
		\midrule
		1          & 20742                   \\
		2          & 20160                   \\
		3          & 20740                   \\
		4          & 20741                   \\
		5          & 20685                   \\
		6          & 20715                   \\
		7          & 20718                   \\
		8          & 20520                   \\
		9          & 20633                   \\
		10         & 20373                   \\
		\bottomrule
	\end{tabular}

\end{table}

\begin{table}
	\centering
	\caption{Messung der Wellenlänge über die Entartung der LISSAJOUS-Figuren zur Bestimmung der Schallgeschwindigkeit $c$.}
	\label{tab:juice}
	\begin{tabular}{ccc}
		\toprule
		Entartung & Abstand $d$/$\si{\milli\meter}$ & Wellenlänge $\lambda$/$\si{\milli\meter}$ \\
		\midrule
		1         & 8.5                             & 17.0                                       \\
		2         & 8.3                             & 16.6                                       \\
		3         & 8.4                             & 16.8                                       \\
		4         & 8.4                             & 16.8                                       \\
		5         & 8.4                             & 16.8                                       \\
		\bottomrule
	\end{tabular}
\end{table}
\FloatBarrier
%%%%%%%%%%%%%%%%%%
\subsection{Nachweis des Doppler-Effekts über die direkte Messung}

\begin{figure}
	\includegraphics{Bilder/vquellebewegt.pdf}
	\caption{Direkte Messung der Frequenz}
	\label{fig:doppler}
\end{figure}
\FloatBarrier
\subsection{Nachweis des Doppler-Effekts mit der Schwebungsmethode}
\begin{figure}
	\includegraphics{Bilder/schwebung.pdf}
	\caption{Schwebungsmethode}
	\label{fig:doppler_schwebung}
\end{figure}
