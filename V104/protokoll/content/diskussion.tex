\section{Diskussion}
\label{sec:Diskussion}
Mit einem Studentschen t-Testes soll die Übereinstimmung der Größe $\frac{1}{\lambda}$
überprüft werden. Diese wurde auf drei verschiedenen Wegen bestimmt: Durch die direkte
Wellenlängenmessung, durch die direkte Frequenzmessung und durch die Schwebungsmethode.

\begin{table}
	\centering
	\caption{Messergebnisse der Größe $\frac{1}{\lambda}$.}
	\label{tab:alam}
	\begin{tabular}{ccc}
		\toprule
		Methode      & $\frac{1}{\lambda}$ / $\frac{1}{\si{\meter}}$ & Anzahl der Einzelmessungen \\
		\midrule
		Wellenlänge & $\num{59.5(2)} \, \si{\meter}$                & 5                          \\
		Frequenz     & $\num{59.76(25)} \, \si{\meter}$              & 20                         \\
		Schwebung    & $\num{60.74(30)} \, \si{\meter}$              & 20                         \\
		\bottomrule
	\end{tabular}
\end{table}
Der Studentsche t-Test ermöglicht es die Kongruenz der Messergebnisse (aufgetragen in Tabelle
\ref{tab:alam}) zu diskutieren.

Der zu bestimmende Wert \cite{ttest} für die  Wahrscheinlichkeit eines systematischen Fehlers
ergibt sich zu
\begin{equation}
	\label{eqn:amina}
	t = \frac{x_1 - x_2}{\sqrt{\frac{1}{m} + \frac{1}{n}} S} \, \mathrm{,}
\end{equation}
mit
\begin{equation}
	\label{eqn:koyim}
	S^2 = \frac{(m-1)s_1^2 + (n-1)s_2^2}{m+n-2} \mathrm{.}
\end{equation}
Die Größen $x_1$ und $x_2$ entsprechen den Mittelwerten aus den entsprechenden Messreihen,
$s_1$ und $s_2$ sind die Fehler der Mittelwerte und die Größen $m$ und $n$ stellen die
Anzahl der Einzelmessungen jeder Messreihe dar, wobei $m+n-2$ der Anzahl der Freiheitsgrade
entspricht.
Aus den Formeln \eqref{eqn:amina} bzw. \eqref{eqn:koyim} und den Werten aus Tabelle
\ref{tab:alam} lassen sich die drei verschiedenen Methoden zur Bestimmung der Größe
$\frac{1}{\lambda}$ miteinander vergleichen.

Der Vergleich der Bestimmung der Wellenlängenmessung und der direkten Frequenzmessung liefert
den Wert
\begin{equation*}
	t_{\mathrm{wf}} = 2,148
\end{equation*}
und damit die Wahrscheinlichkeit von $95\%$, dass kein systematischer Fehler vorliegt \cite{tttest}.

Für den Vergleich von der Wellenlängenmessung und der Schwebungsmethode ergibt sich
\begin{equation*}
	t_{\mathrm{ws}} = 8,698 \mathrm{.}
\end{equation*}
Daher kann nach dem Studentschen t-Test ebenso wie bei dem Vergleich von der direkten Frequenzmessung und der Schwebungsmethode
mit
\begin{equation*}
	t_{\mathrm{fs}} = 11,223 \mathrm{,}
\end{equation*}
davon ausgegangen werden, dass zu über $99.99\%$ kein systematischer Fehler vorliegt.
Insgesamt lässt sich sich eine geringe Diskrepanz bei der Bestimmung der Größe $\frac{1}{\lambda}$ mit den drei verschiedenen Methoden feststellen.
Zudem zeigt ein Vergleich der in Abschnitt \ref{sec:wellenlaenge} bestimmten Schallgeschwindigkeit $c_{\mathrm{Experiment}}=\SI{346(2)}{\meter\per\second}$ mit dem Theoriewert $c_{\mathrm{Theorie}}=\SI{343}{\meter\per\second}$ \cite{schall} zeigt lediglich
eine Abweichung von weniger als $1\%$.
\\Daher scheint die Messung sehr gut mit der Theorie übereinzustimmen.
\\Trotzdem lässt sich eine mögliche Fehlerquelle in der Messung der Wellenlänge über die LISSAJOUS-Figuren feststellen.
\\Wie in Abbildung \ref{fig:Lisas} zu sehen, ließen sich die Entartungen der LISSAJOUS-Figuren zu Geraden nur sehr schlecht einstellen. Die durch das Mikrophon aufgezeichnete Schwingung wies deutlich anharmonische Schwingungen und große Abweichungen vom erwarteten Schwingungsverlauf auf.

\begin{figure}
	\centering
	\includegraphics[width=0.8\textwidth]{Bilder/lissajou.jpeg}
	\caption{Bestmöglichste Entartung der LISSAJOUS-Figuren zu Geraden auf dem Oszilloskop.}
	\label{fig:Lisas}
\end{figure}
