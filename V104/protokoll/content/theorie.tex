\section{Theorie}
\label{sec:Theorie}

Bewegen sich eine Schallquelle und ein Empfänger mit einer relativen Geschwindigkeit 
zueinander, so ändert sich für den Empfänger die Schallfrequenz \cite{demtröder}.

Die Ausbreitung von Schallwellen ist an ein Medium gebunden.
Daher muss unterschieden werden, ob sich der Empfänger oder die Schallquelle mit einer 
Geschwindigkeit $v$ relativ zum Medium fortbewegen.

\subsection{Bewegter Empfänger, ruhende Schallquelle}
Bewegt sich der Empfänger relativ zum Medium mit der Geschwindigkeit $v$, nimmt er die
Schallwellen mit der Frequenz
\begin{equation}
	\label{eqn:empfbewegt}
	\nu_{\mathrm{E}} = \nu_0 ( 1 \pm \frac{v}{c} )
\end{equation}
wahr, wobei $\nu_0$ der von der Quelle ausgesandten Ruhefrequenz und $c$ der
Ausbreitungsgeschwindigkeit der Welle entspricht.
Das Plus in Formel \eqref{eqn:empfbewegt} wird verwendet, wenn sich der Empfänger auf die 
Schallquelle zubewegt (positiv definierte Richtung) und das Minus entsprechend, 
wenn sich der Empfänger von der Schallquell entfernt (negativ definierte Richtung).
Die sich ergebende Frequenzdifferenz $\Delta \nu$ von Ruhefrequenz $\nu_0$ und der
Empfängerfrequenz $\nu_{\mathrm{E}}$ beträgt
\begin{equation*}
	\Delta \nu = \nu_0 \frac{v}{c} \mathrm{.}
\end{equation*}

\subsection{Bewegte Schallquelle, ruhender Empfänger}
Bewegt sich hingegen die Schallquelle mit einer Geschwindigkeit $v$ relativ zum Medium,
ändert sich die Schallfrequenz beim Empfänger $\nu_{\mathrm{Q}}$ zu 
\begin{equation}
	\label{eqn:quellbewegt}
	\nu_0 \frac{1}{1 \mp \frac{v}{c}} \mathrm{.}
\end{equation}
Das Minus in Formel \eqref{eqn:quellbewegt} entspricht hierbei der Bewegung der Schallquelle 
auf den Empfänger zu und das Plus dementsprechend der entgegengesetzten Richtung, also dem 
Fall, dass sich die Schallquelle vom Empfänger entfernt.
Ist $|v| \ll c$, gilt $\nu_{\mathrm{Q}} \approx \nu_{\mathrm{E}}$. Wenn also die 
Ausbreitungsgeschwindigkeit der Welle viel größer als die Geschwindigkeit von Schallquelle bzw. 
Empfänger ist, ist die Frequenzänderung unabhängig davon, ob sich der Empfänger oder die 
Quelle relativ zum Medium bewegt.
