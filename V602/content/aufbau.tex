\subsection{Versuchsaufbau}
\label{sec:Versuchsaufbau}
Der Versuchsaufbau ist in Abbildung \ref{fig:aufbau} dargestellt.
Dabei sind die wichtigsten Bestandteile die Kupfer-Röntgenröhre, ein Lithiumfluorid-Kristall
und das Geiger-Müller-Zählrohr.
Das Röntgengerät ist mit einem Rechner verbunden und kann mithilfe des Programms "Measure"
bedient werden. 
Hier lässt sich der Kristallwinkel und der Drehmodus festlegen.
Die Beschleunigungsspannung wird auf $U_{\mathrm{B}}=\SI{35}{\kilo\volt}$ eingestellt.
Es ist darauf zu achten, dass die Schlitzöffnung am Geiger-Müller-Zählrohr waagerecht 
ausgerichtet ist, damit nur ein Winkel gemessen wird und kein ganzes Spektrum.
Bei den Messungen für die Untersuchung der Absorptionsspektren können die jeweiligen Absorber
vor dem Geiger-Müller-Zählrohr angebracht werden.
