\section{Durchführung}
\label{sec:Durchführung}

\subsection{Versuchsaufbau}
\label{sec:Versuchsaufbau}
Der Versuchsaufbau besteht aus einem Ultraschallechoskop, an dessen Ausgänge zwei Ultraschallsonden mit \SI{2}{\mega\Hz} gekoppelt sind, und einem Rechner zur Datenaufnahme und -analyse.\\
An das Ultraschallechoskop sind zwei Ultraschallsonden angeschlossen, mithilfe derer sich sowohl eine Impuls-Echo-Messung, als auch eine Durchschallmessung realisieren lässt.
Am Rechner werden die gemessenen Daten mittels des Programms \textquote{Echoview} ausgewertet.\\
Hierbei ist \textquote{Echoview} in der Lage, vier verschiedene Diagramme darzustellen.
Im linken oberen Graphen wird der A-Scan dargestellt, also die Amplitude gegen die Zeit aufgetragen.
Der linke untere Graph stellt die gewählte Verstärkung dar. Die Verstärkung lässt sich am Ultraschallechoskop über die Drehknöpfe zur laufzeit-bzw. tiefenabhängigen Verstärkung (TGC; Time Gain Control) und ebenso über die Verstärkung des Outputs und der Empfindlichkeit der Sonden regulieren.\\
Zu Beachten ist, dass eine Verstärkung nur gewählt werden darf, wenn die auszuwertende Messreihe nicht zur Untersuchung der Amplitudenhöhe dient.
Die beiden rechten Graphiken sind das berechnete Spektrum der Messdaten (FFT), bzw. ihr
Cepstrum.
Erzeugte Graphiken und Messdaten können aus dem Programm heraus exportiert werden.
Als zu untersuchende Versuchsobjekte stehen Acrylzylinder verschiedener Länge, Acrylplatten unterschiedlicher Dicke sowie das Modell eines menschlichen Auges im Maßstab 3:1 zur Verfügung.


\FloatBarrier
\subsection{Versuchsbeschreibung}
\label{sec:Versuchsbeschreibung}

\subsubsection{Überprüfung der Bragg-Bedingung}
Zur Überprüfung der Bragg-Bedingung nach Formel \eqref{eqn:braggii} wird der feste Kristallwinkel
$\theta = \SI{14}{\degree}$ eingestellt. Außerdem soll das Geiger-Müller-Zählrohr den
Winkelbereich $\SI{26}{\degree} \le \alpha_{\mathrm{GM}} \le \SI{30}{\degree}$ mit einer
Schrittweite von $\Delta \alpha = \SI{0,1}{\degree}$ ablaufen. Als Integrationszeit pro
Winkel werden $\Delta t = \SI{2}{\second}$ verwendet.
Die Messung kann gestartet werden und das Programm zeichnet die Anzahl der gemessenen Impulse
in Abhängigkeit des vom Geiger-Müller-Zählrohr abgelaufenen Winkels auf.
\subsubsection{Das Emissionsspektrum der Kupfer-Röntgenröhre}
Bei der Bestimmung des Emissionsspektrums wird kein fester Kristallwinkel sondern der
2:1 Koppelmodus verwendet.
Weiterhin wird der Winkelbereich $\SI{4}{\degree} \le \alpha_{\mathrm{GM}} \le \SI{26}{\degree}$
mit einer Schrittweiter von $\Delta \alpha = \SI{0,2}{\degree}$ bei einer Integrationszeit
von $\Delta t = \SI{5}{\second}$ eingestellt.
\subsubsection{Das Absorptionsspektrum}
Das Absorptionsspektrum soll für fünf verschiedene Absorber bestimmt werden. Davon sind
vier Elemente mit Kernladungszahlen $30 \le Z \le 50$ zu wählen. Hier stehen Germanium,
Brom, Zink und Zirkonium zur Verfügung. Der weitere Absorber soll die Bedingung $Z \ge 70$
erfüllen. Es wird Gold verwendet.
Das Geiger-Müller-Zählrohr soll den Bereich $\pm \SI{2}{\degree}$ um den elementspezifischen,
zuvor bestimmten Glanzwinkel ablaufen.
Bei den Messungen wird eine Schrittweite von $\Delta \alpha = \SI{0,1}{\degree}$ und
eine Integrationszeit von $\Delta t = \SI{20}{\sec}$ eingestellt.
