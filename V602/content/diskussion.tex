\section{Diskussion}
\label{sec:Diskussion}

Beim Vergleich des experimentell bestimmten Grenzwinkels des Emissionsspektrum der Röntgenröhre und der daraus berechneten maximalen Energie $E_\mathrm{Ex.;max}=\SI{33962}{\electronvolt}$ zeigt sich eine hohe Übereinstimmung mit dem erwarteten Wert.
Die Cu-Röntgenröhre wurde mit einem Beschleunigungspotential von $U_\mathrm{B}=\SI{35}{\kilo\volt}$ betrieben. Es wurde daher eine maximale Energie der Röntgenstrahlung von $E_\mathrm{max}=\SI{35}{\kilo\electronvolt}$ erwartet.
Es zeigt sich eine Abweichung von $3\%$ für die maximale Energie.
Da allerdings das Emissionspektrum nur in Winkelschritten von $\Delta \theta=0.2\textdegree$ aufgenommen wurde, könnte der Grenzwinkel auch genau zwischen zwei Messpunkten liegen. Bei $\theta=5.0\textdegree$ liegt noch die gleiche Anzahl an pro Sekunde registrierten Impulsen wie für die vorherigen Messpunkte vor. Dies wird als Grundrauschen angenommen.
Bei dem als Grenzwinkel angenommenen Winkel von $\theta_\mathrm{Grenz}=5.2\textdegree$ liegt allerdings schon die vierfache Anzahl an registrierten Impulsen vor. Der tatsächliche Grenzwinkel könnte also vielmehr genau zwischen den Messpunkten liegen.

Die berechnete Auflösung der Messung von $\Delta E_{\alpha}=\SI{82.8}{\electronvolt}$
beziehungsweise $\Delta E_{\beta}=\SI{198.9}{\electronvolt}$ liegt zumindest über der Auflösung der Messung durch den Versuchsaufbau welche durch den Einstellwinkel limitiert wird.
Trotzdem scheint gerade im Falle von $\Delta E_\alpha$ ein sehr knapp bemessenes Fehlerintervall vorzuliegen.
Während der Messung besonders der Absorber kam es immer wieder zu kleinen Messstörungen. Würde eine der Messstörung zufällig an einem der kritischen, betrachteten Punkte auftreten, wäre das angegebene Vetrauensintervall hinfällig. Zudem könnten die jeweiligen Maxima des charakteristischen Spektrums genau verfehlt worden seien, sodass sich bei einer Folgemessung mit gleichem Versuchsaufbau zufällig andere Halbwertsbreiten ergeben könnten.
In Anbetracht dessen scheint die Messauflösung $\Delta E_{\beta}$ zumindest ein sinnvoller Wert, da sie einen Fehlerbereich von etwa zwei Messpunkten des $\theta$-Winkels umfasst und somit zumindest einen zufälligen Messfehler bei einem Einstellwinkel noch abdecken würde.
