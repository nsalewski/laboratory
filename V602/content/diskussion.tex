\section{Diskussion}
\label{sec:Diskussion}

Beim Vergleich des experimentell bestimmten Grenzwinkels des Emissionsspektrum der Röntgenröhre und der daraus berechneten maximalen Energie $E_\mathrm{Ex.;max}=\SI{33962}{\electronvolt}$ zeigt sich eine hohe Übereinstimmung mit dem erwarteten Wert.\\
Bei der Bestimmung des Emissionsspektrum der Cu-Röntgenröhre war noch ein Absorber vor dem Geiger-Müller-Zählrohr angebracht. Dies spielte bei der Bestimmung der zur Auswertung notwendigen Winkel kaum eine Rolle. Aufgrund des Absorbers kam es lediglich zu einem zusätzlichen Absorptionsspektrum, welches dem Emissionsspektrum überlagert war, allerdings vor allem in Bereichen, welche für die Betrachtung des Emissionsspektrums weniger interessant waren.\\
Die Cu-Röntgenröhre wurde mit einem Beschleunigungspotential von $U_\mathrm{B}=\SI{35}{\kilo\volt}$ betrieben. Es wurde daher eine maximale Energie der Röntgenstrahlung von $E_\mathrm{max}=\SI{35}{\kilo\electronvolt}$ erwartet.
Es zeigt sich eine Abweichung von $3\%$ für die maximale Energie.\\
Da allerdings das Emissionspektrum nur in Winkelschritten von $\Delta \theta=0.2\textdegree$ aufgenommen wurde, könnte der Grenzwinkel auch genau zwischen zwei Messpunkten liegen. Bei $\theta=5.0\textdegree$ liegt noch die gleiche Anzahl an pro Sekunde registrierten Impulsen wie für die vorherigen Messpunkte vor. Dies wird als Grundrauschen angenommen.
Bei dem als Grenzwinkel angenommenen Winkel von $\theta_\mathrm{Grenz}=5.2\textdegree$ liegt allerdings schon die vierfache Anzahl an registrierten Impulsen vor. Der tatsächliche Grenzwinkel könnte vielmehr genau zwischen den Messpunkten liegen.

Die berechnete Auflösung der Messung von $\Delta E_{\alpha}=\SI{82.8}{\electronvolt}$
beziehungsweise $\Delta E_{\beta}=\SI{198.9}{\electronvolt}$ liegt zumindest über der Auflösung der Messung welche durch den Einstellwinkel der Schrittweite limitiert wird.
Trotzdem scheint gerade im Falle von $\Delta E_\alpha$ ein sehr knapp bemessenes Fehlerintervall vorzuliegen.
Während der Messung, besonders der Absorptionsspektren, kam es immer wieder zu kleinen Messstörungen. Würde eine der Messstörung zufällig an einem der kritischen, betrachteten Punkte auftreten, wäre das angegebene Vetrauensintervall hinfällig. Zudem könnten die jeweiligen Maxima des charakteristischen Spektrums genau verfehlt worden sein, sodass sich bei einer Folgemessung mit gleichem Versuchsaufbau zufällig andere Halbwertsbreiten ergeben könnten.
In Anbetracht dessen scheint die Messauflösung $\Delta E_{\beta}$ zumindest ein sinnvoller Wert, da sie einen Fehlerbereich von etwa zwei Messpunkten des $\theta$-Winkels umfasst und somit zumindest einen zufälligen Messfehler bei einem Einstellwinkel noch abdecken würde.\\
Ein Vergleich der experimentell bestimmten Werte für die Abschirmkonstante $\sigma$ und die Energie $E_\mathrm{K}$ in Tabelle \ref{tab:theo} zeigt zumeist nur recht geringe Abweichungen gegenüber den Theoriewerten.
Im Falle größerer Fehler sind diese auf Ablesefehler bei der graphischen Bestimmung der entsprechenden Absorptionskanten zurückzuführen.
\begin{table}
	\caption{Vergleich der experimentell bestimmten Werte mit den Theoriewerten.}
	\label{tab:theo}
	\centering
\begin{tabular}{ccccccc}
\toprule
Absorber & $E_\mathrm{Lit}$/$\si{\kilo\electronvolt}$& $E_\mathrm{Ex.}$/$\si{\kilo\electronvolt}$&$\Delta E$ &$\sigma_\mathrm{K,Lit}$ &$\sigma_\mathrm{K,Ex.}$&$\Delta \sigma$ \\
\midrule
Zn; K-Kante & 9.65     & 8.91 &$8\%$& 3.57       &  4.62         &$   30\%$ \\
Ge; K-Kante & 11.10    & 10.97 & $1\%$&      3.68       & 3.86    &$   5\%$  \\
Br; K-Kante & 13.47    & 13.33 & $1\%$&      3.85       & 4.02    &$   4\%$  \\
Zr; K-Kante & 18.00    & 17.73 & $2\%$&      4.1        &4.38     &$   7\%$  \\
Au; L-II-Kante& 13.73  & 13.68 & $0.4\%$&      -      &  -  & - \\
Au; L-III-Kante& 11.92 &  11.74& $2\%$&      -     &  -  & - \\
\bottomrule
\end{tabular}
\end{table}

Bei der Bestimmung der Ryberg-Energie zeigt sich eine Abweichung von etwa $3\%$ ($R_{\mathrm{Theo},\infty}=\SI{13.6}{\electronvolt}$ und $R_{\mathrm{Ex.},\infty}=\SI{14(2)}{\electronvolt}$). Zudem liegt der Theoriewert im Fehlerintervall des experimentell bestimmten Werts.\\
Zu Beachten ist, dass eigentlich in der Formel \eqref{eqn:blubba} mit $z_\mathrm{eff}$ gerechnet wird. Zur Berechnung der Rydberg-Energie wurde allerdings die Kernladungszahl $Z$ verwendet, da die Abschirmkonstante $\sigma$ bereits über den Theoriewert der Rydberg-Energie bestimmt wurde.

Bei der Betrachtung der Messdaten zum Emissionsspektrum fällt zudem auf, dass die eingestellte Schrittweite $\Delta \alpha=\SI{0.4}{\degree}$ durch den Versuchsaufbau nicht immer korrekt realisiert wurde. Ein negativer Einfluss wird hierdurch allerdings nicht verursacht, da die Messauflösung durch diese Ungenauigkeit erhöht wurde.
