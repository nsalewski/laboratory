\section{Theorie}
\label{sec:Theorie}
Unter Röntgenstrahlen werden elektromagnetische Strahlen im Wellenlängenbereich von etwa $\SI{1}{\pico\meter}$ bis $\SI{10}{\nano\meter}$ verstanden.
Zur Erzeugung von Röntgenstrahlen werden Elektronen aus einer Glühkathode emittiert und innerhalb einer evakuierten Röhre zu einer Anode hin beschleunigt.
Beim Auftreffen auf die Anode werden die Elektronen abgebremst und geben ihre kinetische Energie zum einen als Wärmeenergie und zum anderen als elektromagnetische Strahlung ab.\\
Das entstehende sogenannte Bremsspektrum ist kontinuierlich, da die auftreffenden Elektronen sowohl nur einen Teil, als auch ihre gesamte kinetische Energie als elektromagnetische Strahlung, also als Photon emittieren können.
Die maximal übertragene Energie beträgt $E_\mathrm{max}=\symup{e}_0U_\mathrm{B}=h\nu$.
Für die minimale Wellenlänge der emittierten Photonen ergibt sich daher
\begin{equation}
  \label{eqn:welle}
  \lambda_\mathrm{min}=\frac{h\cdot c}{e_0U} \text{.}
\end{equation}
Wird an die Glühkathode der Röntgenröhre eine größere Heizspannung angelegt, ergibt sich keine Änderung des kontinuierlichen Bremsspektrums, lediglich die Intensität des Photonenstroms wird größer.\\
Zudem kann Röntgenstrahlung auch auf eine zweite Art beim Auftreffen von Elektronen an der Anode erzeugt werden\\
Ein freies Elektron stößt hierbei gegen ein gebundenes Elektron auf einer inneren Schale eines Atoms des Anodenmaterials.
Dieses wird herausgelöst und ein Elektron von einer äußeren Schale rückt in die entstehende Lücke nach. Es wird ein Photon emittiert, dessen Energie der Energiedifferenz $h\nu=E_m-E_n$ zwischen den beiden Schalen entspricht.
Im sogenannten charakteristischen Spektrum zeigen sich scharfe Linien mit materialspezifischer Charakteristik.\\
Diese Linien werden mit $K_\alpha,K_\beta,L_\alpha,...$ bezeichnet. Dem griechischen Buchstaben kann dabei entnommen werden, von welcher Schale das zurückfallende Elektron
stammt. Der Großbuchstabe bezeichnet die Schale auf der der Elektronenübergang endet.
Aufgrund von Wechselwirkungen zwischen den Elektronen wird die anziehende Wechselwirkung zwischen dem Atomkern und einen Elektron verringert.
Für  die Bindungsenergie $E_n$ eines Elektron auf der n-ten Schale gilt:
\begin{equation}
  \label{eqn:blubba}
  E_n=-R_\infty z_\mathrm{eff}^2 \cdot \frac{1}{n^2}
\end{equation}
Hierbei ist $R_\infty=\SI{13.6}{\electronvolt}$ die Rydbergenergie. Die Abschirmung wird berücksichtigt über die Verwendung der effektiven Kernladung $Z_\mathrm{eff}=z-\sigma_n$. Es ist $\sigma_n$ die Abschirmungskonstante.
Die Abschirmkonstante lässt sich schließlich berechnen über:
\begin{equation}
  \sigma_n=z-\sqrt{\frac{n^2 E_n}{R_\infty}} \text{.}
\end{equation}
Dies ist allerdings nur eine Näherung. Wird eine Näherung höherer Ordnung betrachtet, ergibt sich für die Abschirmkonstante:
\begin{equation}
  \label{eqn:schirm}
  \sigma_n=z-\sqrt{\frac{E_n}{R_\infty}-\frac{\alpha^2\cdot z^4}{4}} \text{.}
\end{equation}
Das charakteristische Spektrum hat aufgrund von unterschiedlichen Bahndrehimpulsen und Elektronenspin eine Feinstruktur, welche sich für jede charakteristische Linie des Spektrums in Abhängigkeit von der Anzahl der Elektronen auf der jeweiligen betrachteten Elektronenschale, in einzelne Linien aufspaltet.
Die Feinstruktur lässt sich allerdings im vorliegenden Experiment nicht auflösen.

Bei der Absorption von Röntgenstrahlen treten drei Effekte auf. Im vorliegenden Experiment können allerdings nur zwei dieser Effekte aufgrund der zu geringen Energie der verwendeten Röntgenstrahlen von unter $\SI{1}{\mega\electronvolt}$ nachgewiesen werden.
Nicht beobachtet werden kann die Paarbildung. Bei dieser wird aus einem energiereichen Photon ein Elektron-Positron-Paar erzeugt.
Bei den vorliegenden Energien sind dagegen der Photoeffekt und der Compton-Effekt die dominanten Prozesse.
Der Photoeffekt tritt auf, sobald ein Photon mindestens die Bindungsenergie eines Hüllenelektron aufweist.
Das Hüllenelektron wird dann aus seiner Bindung geschlagen. Die übrigbleibende Energie nach Überwindung der Bindungsenergie erhält das Elektron als kinetische Energie.
Beim Compton-Effekt findet ein als elastisch angenommener Stoß zwischen dem Photon der Röntgenstrahlung und einem als annähernd frei angenommenen Elektronen des Absorbermaterials statt.\\
Hierbei gibt das Photon einen Teil seiner Energie an das Elektron ab und wird an diesem gestreut.
Das Elektron kann sich nun mit der übertragenen Energie fortbewegen.
Da allerdings nur lose gebundene Elektronen, zum Beispiel die Valenzelektronen, für den Compton-Effekt in Frage kommen, ist der Photoeffekt der dominante der beobachteten Prozesse.\\
Der Absorptionskoeffizient nimmt mit steigender Photonenenergie ab, steigt jedoch sprunghaft an, wenn das Energieniveau einer der Elektronenschalen des Atoms erreicht wird.
Im Absorptionsspektrum bilden sich daher scharfe Absorptionskanten aus. Diese werden als $K-;L-;...$ Absorptionskanten bezeichnet, je nachdem von welcher Elektronenschale das ausgelöste Elektron stammt.
Es bildet sich erneut eine Feinstruktur aus. Wird diese berücksichtigt, muss die Bindungsenergie über die Sommerfeldsche Feinstrukturformel berechnet werden:
\begin{equation}
  E_{n,j}=-R_\infty  \left(z_\mathrm{eff,1}^2 \cdot \frac{1}{n^2}+\alpha^2 z_\mathrm{eff,2}^4 \cdot \frac{1}{n^3}\left(\frac{1}{j+\frac{1}{2}}-\frac{3}{4n}\right)\right)\text{.}
\end{equation}
Es ist $\alpha \approx \frac{1}{137}$ die Sommerfeldsche Feinstrukturkonstante, $n$ die Hauptquantenzahl und $j$ der Gesamtdrehimpuls des betrachteten Elektrons.
Die Bestimmung der Abschirmkonstante $\sigma_\mathrm{L}$ aus der L-Kante ist unter Beachtung der Feinstruktur kompliziert, da hierfür die Abschirmzahlen jedes beteiligten Elektron betrachtet werden müssen.\\
Unter Verwendung der Energiedifferenz zwischen zwischen zwei L-Kanten $\Delta E_\mathrm{L}=E_\mathrm{L_\mathrm{II}}-E_\mathrm{L_\mathrm{III}}$ vereinfacht sich die Bestimmung der Abschirmkonstante zu
\begin{equation}
	\label{eqn:goldi}
  \sigma_{\mathrm{L}}=Z-\left(\frac{4}{\alpha}\sqrt{\frac{\Delta E_\mathrm{L}}{R_\infty}}-\frac{5\Delta E_\mathrm{L}}{R_\infty}\right)^{\frac{1}{2}}\cdot \left(1+\frac{19}{32}\alpha^2\frac{\Delta E_\mathrm{L}}{R_\infty}\right)^{\frac{1}{2}} \text{.}
\end{equation}

Die Energie $E$ der Röntgenstrahlung lässt sich über die Wellenlänge $\lambda$ der Röntgenstrahlung über die Bragg'sche Reflexion bestimmen.
Die Röntgenstrahlung fällt hierzu auf ein dreidimensionales Gitter mit einem bekannten Abstand $d$ der Netzebenen (im vorliegenden Experiment ein LiF-Kristall). Die Photonen der einfallenden Röntgenstrahlen werden an jedem Atom des Gitters gebeugt, es findet somit eine Interferenz der an verschiedenen Netzebenen gebeugten Photonen statt.
Beträgt der Gangunterschied zwischen zwei an verschiedenen Netzebenen reflektierten Strahlen ein n-Vielfaches der Wellenlänge des einfallenden Röntgenstrahls, so kommt es zu konstruktiver Interferenz.\\
Um die Intensität der Strahlung zu messen, wird ein Geiger-Müller-Zählrohr verwendet.
Für den sogenannten Glanzwinkel $\theta$ unter dem der Strahl einfällt, kommt es zur konstruktiven Interferenz und es ergibt sich die Bragg-Bedingung, über die sich die Wellenlänge des einfallenden Strahls bestimmen lässt:
\begin{equation}
	\label{eqn:braggii}
  2d \sin{\theta}=n\lambda \text{.}
\end{equation}
