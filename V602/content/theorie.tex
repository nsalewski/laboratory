\section{Theorie}
\label{sec:Theorie}
Unter Röntgenstrahlen werden elektromagnetische Strahlen im Wellenlängenbereich von etwa $\SI{1}{\pico\meter}$ bis $\SI{10}{\nano\meter}$ verstanden.
Zur Erzeugung von Röntgenstrahlen werden Elektronen aus einer Glühkathode emmitiert und innerhalb einer evakuierten Röhre zu einer Anode hin beschleunigt.
Beim Auftreffen auf die Anode kann Röntgenstrahlung über zwei verschiedene Vorgänge entstehen.
Zum Einen werden die auftreffenden Elektronen abgebremst. Sie geben dabei einen Teil ihrer Energie als elektromagnetische Strahlung ab, es werden als Photonen (Röntgenquanten) emittiert.
Das entstehende sogenannte Bremsspektrum ist kontinuierlich, da die auftreffenden Elektronen sowohl nur einen Teil, als auch ihre gesamte kinetische Energie an die Photonen abgeben können. Die verbleibende, nicht als Photon emittierte Energie, erwärmt die Anode.
Die maximal übertragene Energie beträgt $E_\mathrm{max}=\symup{e}_0U_\mathrm{B}=h\nu$.
Für die minimale Wellenlänge der emittierten Photonen ergibt sich
\begin{equation}
  \lambda_\mathrm{min}=\frac{h\cdot c}{e_0U} \text{.}
\end{equation}
Wird an die Glühkathode der Röntgenröhre eine größere Heizspannung angelegt, ergibt sich also keine Änderung des kontinuierlichen Bremsspektrums, lediglich die Intensität des Photonenstroms wird größer.
Im Gegensatz zum Bremsspektrum, dessen prinzipielle Gestalt nur durch die Beschleunigungsspannung beeinflusst wird, ist das charakteristische Spektrum materialspezifisch.
Ein freies Elektron stößt hierbei gegen ein gebundenes Elektron auf einer inneren Schale eines Atoms des Anodenmaterials.
Dieses wird herausgelöst und ein Elektron von einer äußeren Schale rückt in die entstehende Lücke nach. Es wird ein Photon emittiert, dessen Energie der Energiedifferenz $h\nu=E_m-E_n$ zwischen den beiden Schalen entspricht.
Im charakteristischen Spektrum zeigen sich scharfe Linien mit materialspezifischer Charakteristik.
Diese Linien werden mit $K_\alpha,K_\beta,L_\alpha$ bezeichnet. Dem griechischen Buchstaben kann dabei entnommen werden, von welcher Schale das zurückfallende Elektron
stammt. Der Großbuchstabe bezeichnet die Schale auf der der Elektronenübergang endet.
Aufgrund von Wechselwirkungen zwischen den Elektronen wird die anziehende Wechselwirkung zwischen dem Atomkern und einen Elektron verringert.
Für  die Bindungsenergie $E_n$ eines Elektron auf der n-ten Schale gilt:
\begin{equation}
  E_n=-R_\infty z_\mathrm{eff}^2 \cdot \frac{1}{n^2}
\end{equation}
Hierbei ist $R_\infty=\SI{13.6}{\electronvolt}$ die Rydbergenergie. Die Abschirmung wird berücksichtigt über die Verwendung der effektiven Kernladung $Z_\mathrm{eff}=z-\sigma_n$. Es ist $\sigma_n$ die Abschirmungskonstante.
Die Abschirmkonstante lässt sich schließlich berechnen über:
\begin{equation}
  \sigma_n=z-\sqrt{\frac{n^2 E_n}{R_\infty}}
\end{equation}
Das charakteristische Spektrum hat aufgrund von unterschiedlichen Bahndrehimpulsen und Elektronenspin eine Feinstruktur, welche sich für jede charakteristische Linie des Spektrums in Abhängigkeit von der Anzahl der Elektronen auf der jeweiligen betrachteten Elektronenschale, in einzelne Linien aufspaltet.
Die Feinstruktur lässt sich allerdings im vorliegenden Experiment nicht auflösen.

Bei der Absorption von Röntgenstrahlen treten drei Effekte auf. Im vorliegenden Experiment können allerdings nur zwei dieser Effekte aufgrund der zu geringen Energie der verwendeten Röntgenstrahlen nachgewiesen werden.
Nicht beobachtet werden kann die Paarbildung. Bei dieser wird aus einem energiereichen Photon ein Elektron-Positron-Paar erzeugt. Dieses Phänomen tritt allerdings bei den vorliegenden Energien der Photonen von unter $\SI{1}{\mega\electronvolt}$ kaum auf.
Bei niedrigen Energien von blaaablup tritt bereits der Photoeffekt auf. Das auftreffende Photon wird vom Atom absorbiert und schlägt dabei ein Elektron aus einer kernnahen Schale heraus.

Der Absorptionskoeffizient nimmt mit steigender Photonenenergie ab, steigt jedoch sprunghaft an, wenn ein Energieniveau einer der Elektronenschalen des Atoms erreicht wird.
Im Absorptionsspektrum bilden sich scharfe Absorptionskanten aus. Diese werden als $K-;L-;...$ Absorptionskanten bezeichnet, jenachdem von welcher Elektronenschale das ausgelöste Elektron stammte.


Die Energie $E$ der Röntgenstrahlung lässt sich über die Wellenlänge $\lambda$ der Röntgenstrahlung über die Bragg'sche Reflexion bestimmen.
Die Röntgenstrahlung fällt hierzu auf ein dreidimensionales Gitter mit einem bekannten Abstand $d$ der Netzebenen (im vorliegenden Experiment ein LiF-Kristall). Die Photonen der einfallenden Röntgenstrahlen werden an jedem Atom des Gitters gebeugt, es findet somit eine Interferenz der an verschiedenen Netzebenen gebeugten Photonen statt.
Beträgt der Gangunterschied zwischen zwei an verschiedenen Netzebenen reflektierten Strahlen ein n-Vielfaches der Wellenlänge des einfallenden Strahls, so kommt es zu konstruktiver Interferenz.
Für den sogenannten Glanzwinkel $\theta$ unter dem der Strahl einfällt, kommt es genau zur konstruktiven Interferenz und  es ergibt sich die Bragg-Bedingung, über die sich die Wellenlänge des einfallenden Strahls bestimmen lässt:
\begin{equation}
  2d \sin{\theta}=n\lambda \text{.}
\end{equation}



\cite{Anleitung}
