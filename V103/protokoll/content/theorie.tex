\section{Theorie}
\label{sec:Theorie}

In der Werkstofftechnik stellt der Elastizitätsmodul $E$ einen wichtigen Materialkennwert dar.
Der Elastizitätsmodul ist ein Faktor, welcher die Gestaltsdeformation eines Körpers unter Wirkung einer Normalspannung oder eines Drucks $\sigma$ beschreibt.
Allgemein werden Kräfte, welche an der Oberfläche eines Körpers angreifen, als Spannung bezeichnet. Als Schub-oder Tangentialspannung wird hierbei die oberflächenparallele Komponente, als \textbf{Normalspannung} $\sigma$ die zur Oberfläche senkrechte Komponente der Spannung, bezeichnet.
Ist die durch die Normalspannung verursachte relative Änderung einer linearen Körperdimension (zum Beispiel einer Länge $L$) hinreichend klein,
besteht ein linearer Zusammenhang zwischen der angreifenden Spannung $\sigma$ und der relativen Änderung $\frac{\Delta L}{L}$ mit dem Elastizitätsmodul $E$ als Proportionalitätsfaktor.
\begin{equation}
	\label{eqn:hook}
	\sigma=E \cdot \frac{\Delta L}{L}
\end{equation}
Dieser Zusammenhang wird als \textbf{Hooksches Gesetz} bezeichnet.
Falls die Längenänderung $\Delta L$ direkt genau bestimmt werden kann, könnte prinzipiell nach Formel \eqref{eqn:hook} der Elastizitätsmodul bestimmt werden.
Dafür wären allerdings hochpräzise Messgeräte notwendig. Daher wird in diesem Versuch der Elastizitätsmodul über die Biegung eines Probestabes infolge einer angreifenden Kraft untersucht.
Die Durchbiegung $D(x)$ ist bei ansonsten unveränderten Eigenschaften des Probestabes relativ groß gegenüber $\Delta L$ und lässt daher auch bei weniger präzisen Messgeräten eine Bestimmung des Elastizitätsmodul $E$ zu.

Greift am Stabende eine Kraft $F$ an, so verursacht diese auf jeden Querschnitt $Q$ des Probestabes ein Drehmoment $M_{\mathrm{F}}$, sodass der Stab aus seiner Ruhelage ausgelenkt wird. Das Drehmoment $M_{\mathrm{F}}$ ergibt sich hierbei aus dem Produkt zwischen dem Hebelarm für den betrachteten Querschnitt $Q$ des Stabes und angreifender Kraft $F$
Aufgrund dieser Durchbiegung der Probe treten in dieser Normalspannungen auf, die der Deformation entgegen wirken.
In den oberen Schichten des Stabes treten Zugspannungen auf, in den unteren Schichten entsprechend Druckspannungen.
Mittig des Stabes (in Abbildung \ref{fig:Durchbiegung} durch eine gestrichelte Linie dargestellt) liegt die sogenannte \textbf{neutrale Faser}. In dieser treten keine Spannungen auf.
\begin{figure}
	\centering
	\includegraphics[width=0.7\textwidth]{Bilder/durchbiegungstab.png}
	\caption{Durchbiegung des einseitig eingespannten Stabes bei angreifender Kraft $F$ am anderen Stabende. \cite{Anleitung}}
	\label{fig:Durchbiegung}
\end{figure}
Da die Zug-und Druckspannung an jedem Querschnitt $Q$ des Probestabes entgegengesetzt gleich sind, erzeugen sie ein Drehmoment $M_{\mathrm{\sigma}}$ auf $Q$.
Die Durchbiegung $D(x)$ erreicht nun einen stationären Zustand, wenn $M_{\mathrm{F}}$ und $M_{\mathrm{\sigma}}$ übereinstimmen.
Da $M_{\mathrm{\sigma}}$ das Integral vom Produkt aus $y$ und $\sigma(y)$ über den Querschnitt $Q$ des Stabes ist, ergibt sich

\begin{equation}
	\label{eqn:momentengleichung}
	M_{\mathrm{F}}=M_{\mathrm{\sigma}} \rightarrow F(L-x)=\int_{\symup{Q}}^{} y \cdot \sigma(y) \symup{dq}
\end{equation}
$dq$ ist hierbei das Flächenelement mit Abstand $y$ von der neutralen Faser.
Die Normalspannung $\sigma(y)$ lässt sich mittel des Hookschen Gesetz \eqref{eqn:hook} bestimmen.
Unter Verwendung (differential-) geometrischer Überlegungen lässt sich Gleichung \eqref{eqn:momentengleichung} schreiben als
\begin{equation}
	\label{eqn:integralmomentengleichung}
	F(L-x)=E \frac{\symup{d}^2}{\symup{d}x^2} \int_{\symup{Q}}^{} y^2 \symup{dq}=I \cdot E \frac{\symup{d}^2}{\symup{d}x^2}
\end{equation}
Analog zum Massenträgheitsmoment $\theta$ wird $I$ als \textbf{Flächenträgheitsmoment} definiert.
Die Integration von Gleichung  \eqref{eqn:integralmomentengleichung} liefert somit sofort einen Ausdruck für die Auslenkung $D$ in Abhängigkeit zu $x$

\begin{equation}
	\label{eqn:d_x_einseitig}
	D(x)=\frac{F}{2EI}\left(Lx^2-\frac{x^3}{3}\right) \text{.}
\end{equation}
Hierbei ist $L$ die Länge des Stabes vom eingespannten Ende bis zum Punkt, an dem die Kraft $F$ angreift und $x$ ist der Abstand zwischen dem Einspannpunkt und dem Messpunkt (vgl. Abbildung \ref{fig:Durchbiegung}).

Greift die Kraft $F$ mittig des Stabes an, wirkt nur noch die halbe Kraft an der Querschnittsfläche $Q$. Für die beiden Stabhälften wirken nun unterschiedliche Drehmomente $M_{\mathrm{F}}$ und für die Durchbiegung ergeben sich die beiden Gleichungen
\begin{equation}
	\label{eqn:d_x_beidseitig_eins}
	D(x)=\frac{F}{48EI}\left(3L^2x-4x^3\right) \mathrm{ für } 0\leq x\leq\frac{L}{2}
\end{equation}
\begin{equation}
	\label{eqn:d_x_beidseitig_zwei}
	D(x)=\frac{F}{48EI}\left(4x^3-12 Lx^2+9 L^2x-L^3\right) \mathrm{ für } \frac{L}{2}\leq x \leq L
\end{equation}
