\section{Diskussion}
\label{sec:Diskussion}
Da es sehr viele verschiedene Kupfer-Zink-Legierungen mit unterschiedlicher Zusammensetzung gibt, ist es nicht möglich, einen eindeutigen Literaturwert für den Messingstab zu finden, da dessen genaue Legierung unbekannt ist.
Es wird allerdings vermutet, dass der verwendete Messingstab aufgrund der errechneten Dichte $\rho$ einen relativ hohen Zinkanteil hat.
Daher wird als Literaturwert für den Elastizitätsmodul $E$ auch ein Wert für Messing mit hohem Zinkanteil herangezogen.

Allgemein lässt sich sagen, dass die bestimmten Elastizitätsmoduln sehr nah an den Literaturwert liegen.
Während des Experiments wurde allerdings festgestellt, dass die sehr empfindlichen Messuhren zur Messung der Auslenkung eine mögliche Fehlerquelle darstellen.
Stöße an den Tisch oder auch eine leichte Verdrehung des Laufrads des Taststifts führen so leicht zu Messfehlern.

Ein Vergleich des bestimmten Elastizitätsmodul des runden Stabs (Messing) $E_{\mathrm{Messing}}= (74.9 \pm 0.4)\,\si{\giga\pascal}$ mit dem Literaturwert nach \cite{Hans}
$E_{\mathrm{Theorie}}=78\, \si{\giga\pascal}$ zeigt, dass der bestimmte E-Modul knapp unterhalb des Theoriewerts liegt. Es zeigt sich eine Abweichung von $4\%$.
In Tabelle \ref{tab:messergebnisse} sind die berechneten Elastizitätsmoduln für den Aluminiumstab, sowie ihre Abweichung zum Literaturwert $E_{\mathrm{Aluminium}}= 69\, \si{\giga\pascal}$ nach \cite{Hans} eingetragen.

\begin{table}
	\centering
	\caption{Vergleich der E-Moduln für den Aluminiumstab mit dem Literaturwert}
	\label{tab:messergebnisse}
	\begin{tabular}{ccc}
		\toprule
		Messung                              & Messergebnis /$\si{\giga\pascal}$ & Abweichung zur Literatur \\
		\midrule
		einseitig eingespannt                & $(70.42 \pm 0.22)$                & $2.1\%$                \\
		rechte Seite; beidseitig eingespannt & $(67.4 \pm 0.03)$                 & $2.3\%$                \\
		linke Seite; beidseitig eingespannt  & $(71.3 \pm 0.4)$                  & $3.3\%$                \\
		\bottomrule
	\end{tabular}
\end{table}

Generell lässt sich sagen, dass die Messwerte eine geringe Diskrepanz 
gegenüber der bestimmten Regressionsgeraden aufweisen.
Da die Bestimmung des Elastizitätsmoduls über die lineare 
Ausgleichsrechnung nur eine geringe Abweichung gegenüber dem Literaturwert 
aufweist, sind verbleibende Fehler auf geringe Messunsicherheiten und 
statistische Fehler zurückzuführen.
