\section{Auswertung}
\label{sec:Auswertung}

%%%%%%%%%%%%%%%%%%%%%%%%%%%%%%%%%%%%%%%%%%%%%%%%%%%%%%%%%%%%%%%%%%%%%%%%%%%%%%%%%%%%
\FloatBarrier
\subsection{Durchbiegung des Stabes bei beidseitiger Einspannung}

Die Messwerte zur beidseitigen Einspannung sind in Tabelle \ref{tab:beidi} aufgetragen.

\begin{table}
	\centering
	\caption{Messergebnisse für die Durchbiegung eines Stabes bei beidseitiger Einspannung.}
	\label{tab:beidi}
	\begin{tabular}{cccc}
		\toprule
		$x$ / $\si{\centi\meter}$ & $D_0(x)$ / $\si{\milli\meter}$ & $D_{\mathrm{M}}(x)$ / $\si{\milli\meter}$ & $D(x)$ / $\si{\milli\meter}$ \\
		\midrule
		0 & 1.00 & 1.09 & 0.09 \\
		1 & 0.99 & 1.23 & 0.24 \\
		2 & 0.98 & 1.38 & 0.40 \\
		3 & 0.96 & 1.52 & 0.56 \\
		4 & 0.97 & 1.68 & 0.71 \\
		5 & 0.96 & 1.84 & 0.88 \\
		6 & 0.96 & 1.97 & 1.01 \\
		7 & 0.95 & 2.12 & 1.17 \\
		8 & 0.97 & 2.26 & 1.29 \\
		9 & 0.95 & 2.40 & 1.45 \\
		10 & 0.95 & 2.53 & 1.58 \\
		11 & 0.93 & 2.64 & 1.71 \\
		12 & 0.93 & 2.77 & 1.84 \\
		13 & 0.93 & 2.90 & 1.97 \\
		14 & 0.91 & 3.00 & 2.09 \\
		15 & 0.90 & 3.09 & 2.19 \\
		16 & 0.89 & 3.19 & 2.30 \\
		17 & 0.88 & 3.28 & 2.40 \\
		18 & 0.86 & 3.35 & 2.49 \\
		19 & 0.84 & 3.42 & 2.58 \\
		20 & 0.83 & 3.48 & 2.65 \\
		21 & 0.81 & 3.53 & 2.72 \\
		22 & 0.79 & 3.58 & 2.79 \\
		23 & 0.78 & 3.60 & 2.82 \\
		24 & 0.70 & 3.64 & 2.94 \\
		25 & 0.70 & 3.65 & 2.95 \\
		26 & 0.67 & 3.65 & 2.98 \\
		27 & 0.65 & -- & -- \\
		28 & 1.48 & 4.41 & 2.93 \\
		29 & 1.46 & 4.38 & 2.92 \\
		30 & 1.44 & 4.34 & 2.90 \\
		31 & 1.41 & 4.29 & 2.88 \\
		32 & 1.40 & 4.22 & 2.82 \\
		33 & 1.38 & 4.15 & 2.77 \\
		34 & 1.35 & 4.06 & 2.71 \\
		35 & 1.33 & 3.96 & 2.63 \\
		36 & 1.31 & 3.85 & 2.54 \\
		37 & 1.28 & 3.73 & 2.45 \\
		38 & 1.26 & 3.61 & 2.35 \\
		39 & 1.23 & 3.49 & 2.26 \\
		40 & 1.25 & 3.37 & 2.12 \\
		41 & 1.18 & 3.25 & 2.07 \\
		42 & 1.17 & 3.10 & 1.93 \\
		43 & 1.15 & 2.95 & 1.80 \\
		44 & 1.14 & 2.80 & 1.66 \\
		45 & 1.12 & 2.64 & 1.52 \\
		46 & 1.11 & 2.49 & 1.38 \\
		47 & 1.09 & 2.34 & 1.25 \\
		48 & 1.08 & 2.19 & 1.11 \\
		49 & 1.09 & 2.02 & 0.93 \\
		50 & 1.06 & 1.85 & 0.79 \\
		51 & 1.05 & 1.70 & 0.65 \\
		52 & 1.05 & 1.54 & 0.49 \\
		\bottomrule
	\end{tabular}
\end{table}

\begin{figure}
	\centering
	\includegraphics{Bilder/c.pdf}
	\caption{Durchbiegung der Stäbe.}
	\label{fig:Stabus}
\end{figure}

Die Durchbiegung $D(x)$ an den Orten $0 \leq x \leq \frac{L}{2}$
also $\SI{0}{\centi\meter} \leq x \leq \SI{26}{\centi\meter}$ 
ist in Abbildung \ref{fig:Stabus} gemäß Formel \eqref{eqn:d_x_beidseitig_eins} gegen das 
Polynom $3L^2x - 4x^3$ aufgetragen.

Die Regressionsgerade der Form
\begin{equation*}
	D(x) = a \cdot x \mathrm{,}
\end{equation*}
mit $a = \frac{F}{48EI}$ gemäß Formel \eqref{eqn:d_x_beidseitig_eins}, wurde mit Scipy 
\cite{scipy} berechnet.
Der Parameter $a$ ergibt sich zu
\begin{equation*}
	a = 19,61 \pm 0,08 \mathrm{.}
\end{equation*}
Damit erhält man den Elastizitätsmodul $E$ zu
\begin{equation*}
	E = (5,903 \pm 0,025) \, \si{\pascal} \mathrm{.}
\end{equation*}
%%%%%%%%%%%%%%%%%%%%%%%%%%%%%%%%%%%%%%%%%%%%%%%%%%%%%%%%%%%%%%%%%%%%%%%%%%%%%%%%%%%%%%%%%%%
\begin{figure}
	\centering
	\includegraphics{Bilder/c2.pdf}
	\caption{Durchbiegung der Stäbe amk.}
	\label{fig:StabusMaximus}
\end{figure}

Die Regeressionsgerade für die zweite Hälfte des beidseitig eingespannten Stabes  mit den
zugehörigen Messwerten ist in Abbildung \ref{fig:StabusMaximus} dargestellt.
Für die Steigung der $a$ der Regressionsgerade ergibt sich
\begin{equation*}
	a = 20,79 \pm 0,30  \mathrm{.}
\end{equation*}
Damit erhält man für den Elastizitätsmodul $E$ den Wert 
\begin{equation*}
	E = (5,57 \pm 0,08) \, \si{\pascal} \mathrm{.}
\end{equation*}

