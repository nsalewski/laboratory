\section{Theorie}
\label{sec:Theorie}

Die Theorie wird in zwei Abschnitte gegliedert: die Phänomene der Interferenz und Kohärenz.


\subsection{Interferenz}
\label{sec:interferenz}

\subsubsection{Der Begriff der Interferenz}
\label{sec:interferenzdef}
Die Interferenz beschreibt das Phänomen, das auftritt, wenn Wellen -- hier Licht als 
elektromagnetische Welle -- überlagert werden. 
Die Intensitäten der Wellen werden nicht nur addiert, sondern zusätzlich mit einem
Interferenzterm multipliziert. Daher können sich Wellen verstärken, was \textbf{konstruktive 
Interferenz} genannt wird und gegenseitig auslöschen (\textbf{destruktive Interferenz}).

Die Lichtausbreitung im Vakuum lässt sich durch ebene elektromagnetische Wellen
\begin{equation}
	\label{eqn:ebenewelle}
	\vec{E}(x,t) = \vec{E}_0 \cos(k x - \omega t - \delta) 
\end{equation}
beschreiben, wobei $\vec{E}$ der elektrischen Feldstärke, $x$ dem Ort, $t$ der Zeit, $k=\frac{2\pi}{\lambda}$ der Wellenzahl, $\lambda$
der Wellenlänge, $\omega$ der Kreisfrequenz und $\delta$ dem Phasenunterschied bezüglich eines 
festen Bezugspunktes entspricht.
Die Beschreibbarkeit der Ausbreitung des Lichts durch elektromagnetische Wellen impliziert die 
Gültigkeit der Maxwell'schen Gleichungen. Diese sind lineare Differentialgleichungen. Daher 
gilt das Prinzip der linearen Superposition, welches besagt, dass ein elektrisches Feld
$\vec{E}$, das aus mehreren einzelnen elektrischen Feldern $\vec{E}_{\mathrm{i}}$
zusammengesetzt wird, der Summe dieser, also $\vec{E} = \sum \vec{E}_{\mathrm{i}}$, entspricht.

Weiterhin ist die Feldstärke von Licht, aufgrund der hohen Frequenz, von
Messgeräten nicht messbar. Diesbezüglich wird die Intensität von Licht, die Lichtleistung pro 
Fläche, betrachtet.
Aus den Maxwell'schen Gleichungen ergibt sich diese zu
\begin{equation}
	\label{eqn:intensity}
	I = C \, |\vec{E}|^2 \mathrm{, } \, C = \mathrm{const.}
\end{equation}
Mit den Gleichungen \eqref{eqn:ebenewelle} und \eqref{eqn:intensity} und der linearen 
Superposition ergibt sich für die Gesamtintensität $I_{\mathrm{GES}}$ von zwei Lichtwellen mit 
der gleichen Amplitude $\vec{E}_0$, die an einem festen Ort $x$ einfallen,
\begin{equation}
	I_{\mathrm{GES}} = \frac{C}{t_2 - t_1} \int_{t_1}^{t_2} |\vec{E_1} + \vec{E_2}|^2(x,t) \,  \symup{d}t \mathrm{,}
\end{equation}
wobei das Beobachtungsintervall $t_2 - t_1$ groß gegen die Periodendauer $T=\frac{2\pi}{\omega}$
sein soll.
Einsetzen und Ausmultiplizieren liefert schließlich
\begin{equation}
	I_{\mathrm{GES}} = 2C \, \vec{E_0}^2 (1+\cos(\delta_2 - \delta_1)) \mathrm{,}
\end{equation}
den Interferenzterm. Die Intensitäten weichen bei einem Phasenunterschied von
$\delta_2 - \delta_1 = n \cdot 2\pi$ um $2C \vec{E_0}^2$ vom Mittelwert ab und verschwinden
bei einem Phasenunterschied von $(2n+1) \cdot \pi$, mit $n \in \mathbb{N}$.


\subsubsection{Diskussion über die Voraussetzungen zur Messung von Interferenzerscheinungen}
\label{sec:messunginterferenz}

Werden Lichtwellen aus verschiedenen Quellen überlagert, sind keine Interferenzeffekte zu 
beobachten. Dies hat die Ursache, dass bei der Emission von Lichtwellen Elektronen 
Emissionszentren darstellen: Die Elektronen werden durch hinzugefügte Energie in einen 
angeregten Zustand versetzt und emittieren bei der Rückkehr in den Grundzustand Energie in Form 
eines Wellenzuges endlicher Länge, also einer Wellengruppe. Des Weiteren treten diese Emissionen 
statistisch verteilt in der Elektronenhülle des Atoms bzw. Moleküls auf und daher sind die Phasen
$\delta_1$ bzw. $\delta_2$ statistische Funktionen der Zeit. 

Daraus folgt, dass die Mittelung über die Zeit
\begin{equation}
	\frac{1}{t_2-t_1} \int_{t_1}^{t_2} C \, \cos(\delta_2(t)-\delta_1(t)) \, \symup{d}t
\end{equation}
verschwindet, da das Beobachtungsintervall $t_2-t_1$ groß gegen die Periodendauer $T$ ist und 
der Phasenunterschied $\delta_2-\delta_1$ beliebige Werte annimmt.

Licht aus verschiedenen Quellen ist also nicht interferenzfähig, es ist \textbf{inkohärent}.
Interferenzeffekte lassen sich also nur erzeugen, wenn das Licht aus der selben Quelle stammt, 
sogenanntes \textbf{kohährentes} Licht.


\subsection{Kohärenz}
\label{sec:kohärenz} 










\cite{Anleitung}
