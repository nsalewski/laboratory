\section{Theorie}
\label{sec:Theorie}

Die Theorie wird in zwei Abschnitte gegliedert: Zum Einen in den Begriff der Interferenz und 
die Voraussetzungen für Interferenzeffekte und zum Anderen in die Interferenzeffekte mit 
kohärentem Licht.


\subsection{Interferenz}
\label{sec:interferenz}

\subsubsection{Der Begriff der Interferenz}
\label{sec:interferenzdef}
Die Interferenz beschreibt das Phänomen, das auftritt, wenn Wellen -- hier Licht als 
elektromagnetische Welle -- überlagert werden. 
Die Intensitäten der Wellen werden nicht nur addiert, sondern zusätzlich mit einem
Interferenzterm multipliziert. Daher können sich Wellen verstärken, was \textbf{konstruktive 
Interferenz} genannt wird und gegenseitig auslöschen (\textbf{destruktive Interferenz}).

Die Lichtausbreitung im Vakuum lässt sich durch ebene elektromagnetische Wellen
\begin{equation}
	\label{eqn:ebenewelle}
	\vec{E}(x,t) = \vec{E}_0 \cos(k x - \omega t - \delta) 
\end{equation}
beschreiben, wobei $\vec{E}$ der elektrischen Feldstärke, $x$ dem Ort, $t$ der Zeit, $k=\frac{2\pi}{\lambda}$ der Wellenzahl, $\lambda$
der Wellenlänge, $\omega$ der Kreisfrequenz und $\delta$ dem Phasenunterschied bezüglich eines 
festen Bezugspunktes entspricht.
Die Beschreibbarkeit der Ausbreitung des Lichts durch elektromagnetische Wellen impliziert die 
Gültigkeit der Maxwell'schen Gleichungen. Diese sind lineare Differentialgleichungen. Daher 
gilt das Prinzip der linearen Superposition, welches besagt, dass ein elektrisches Feld
$\vec{E}$, das aus mehreren einzelnen elektrischen Feldern $\vec{E}_{\mathrm{i}}$
zusammengesetzt wird, der Summe dieser, also $\vec{E} = \sum \vec{E}_{\mathrm{i}}$, entspricht.

Weiterhin ist die Feldstärke von Licht, aufgrund der hohen Frequenz, von
Messgeräten nicht messbar. Diesbezüglich wird die Intensität von Licht, die Lichtleistung pro 
Fläche, betrachtet.
Aus den Maxwell'schen Gleichungen ergibt sich diese zu
\begin{equation}
	\label{eqn:intensity}
	I = C \cdot |\vec{E}|^2 \mathrm{, } \, C = \mathrm{const.}
\end{equation}
Mit den Gleichungen \eqref{eqn:ebenewelle} und \eqref{eqn:intensity} und der linearen 
Superposition ergibt sich für die Gesamtintensität $I_{\mathrm{GES}}$ von zwei Lichtwellen mit 
der gleichen Amplitude $\vec{E}_0$, die an einem festen Ort $x$ einfallen,
\begin{equation}
	I_{\mathrm{GES}} = \frac{C}{t_2 - t_1} \int_{t_1}^{t_2} |\vec{E_1} + \vec{E_2}|^2(x,t) \,  \symup{d}t \mathrm{,}
\end{equation}
wobei das Beobachtungsintervall $t_2 - t_1$ groß gegen die Periodendauer $T=\frac{2\pi}{\omega}$
sein soll.
Einsetzen und Ausmultiplizieren liefert schließlich
\begin{equation}
	I_{\mathrm{GES}} = 2C \, \vec{E_0}^2 (1+\cos(\delta_2 - \delta_1)) \mathrm{,}
\end{equation}
den Interferenzterm. Die Intensitäten weichen bei einem Phasenunterschied von
$\delta_2 - \delta_1 = n \cdot 2\pi$ um $2C \vec{E_0}^2$ vom Mittelwert ab und verschwinden
bei einem Phasenunterschied von $(2n+1) \cdot \pi$, mit $n \in \mathbb{N}$.


\subsubsection{Diskussion über die Voraussetzungen zur Messung von Interferenzerscheinungen}
\label{sec:messunginterferenz}

Werden Lichtwellen aus verschiedenen Quellen überlagert, sind keine Interferenzeffekte zu 
beobachten. Dies hat die Ursache, dass bei der Emission von Lichtwellen Elektronen 
Emissionszentren darstellen: Die Elektronen werden durch hinzugefügte Energie in einen 
angeregten Zustand versetzt und emittieren bei der Rückkehr in den Grundzustand Energie in Form 
eines Wellenzuges endlicher Länge, also einer Wellengruppe. Des Weiteren treten diese Emissionen 
statistisch verteilt in der Elektronenhülle des Atoms bzw. Moleküls auf und daher sind die Phasen
$\delta_1$ bzw. $\delta_2$ statistische Funktionen der Zeit. 

Daraus folgt, dass die Mittelung über die Zeit
\begin{equation}
	\frac{1}{t_2-t_1} \int_{t_1}^{t_2} C \, \cos(\delta_2(t)-\delta_1(t)) \, \symup{d}t
\end{equation}
verschwindet, da das Beobachtungsintervall $t_2-t_1$ groß gegen die Periodendauer $T$ ist und 
der Phasenunterschied $\delta_2-\delta_1$ beliebige Werte annimmt.

Licht aus verschiedenen Quellen ist also nicht interferenzfähig, es ist \textbf{inkohärent}.
Interferenzeffekte lassen sich also nur erzeugen, wenn das Licht aus der selben Quelle stammt, 
sogenanntes \textbf{kohährentes} Licht.


\subsection{Interferenzeffekte mit kohärentem Licht}
\label{sec:kohärenz} 

Um Interferenzeffekte mit kohärentem Licht erzeugen zu können, muss die Quelle Licht mit 
festem $k$, $\omega$ und $\delta$ (vergleiche Gleichung \eqref{eqn:ebenewelle}) emittieren.
Weiterhin muss der Strahlengang des Lichts getrennt werden, damit sich die beiden Teilwellen 
an einem Punkt P überlagern können.
Der Unterschied der Weglängen der beiden Teilwellen wird \textbf{Wegunterschied $\Delta$} 
genannt.
Hierbei ist zu beachten, dass der Wegunterschied $\Delta$ nicht zu groß gegen die Länge eines 
Wellenzugs sein darf. Dies liegt daran, dass ein Emissionsakt eine endliche Dauer $\tau$ und der 
emittierte Wellenzug somit eine endliche Länge hat. Ist der Wegunterschied $\Delta$ zu groß
gegen die Länge dieses Wellenzuges, treffen Teilwellen aus unterschiedlichen Wellenzügen 
zeitgleich am Punkt P auf und können aufgrund der inkonstanten Phasenbeziehung zueinander nicht 
interferieren.
Der Wegunterschied $\Delta$, ab dem keine Interferenzeffekte auftreten, heißt \textbf{
	Kohärenzlänge $\ell$}. Sie ergibt sich aus der Anzahl $N$ der im Interferenzbild 
auftretenden Intensitätsmaxima und der Wellenlänge $\lambda$ zu 
\begin{equation}
	\ell = N \lambda \, \mathrm{.}
\end{equation}
Außerdem ergibt sich aus dem \textbf{Fourier'schem Theorem}, dass ein Wellenzug endlicher 
Länge nicht monochromatisch ist und ein Frequenz- und Wellenlängenspektrum hat. Dieses 
Spektrum sorgt für ein unklares Interferenzbild, weil eine Frequenz aus dem Spektrum in dem 
Punkt bei einem gewissen Wegunterschied $\Delta$ ein Maximum haben kann, während eine andere
Frequenz aus dem Spektrum destruktiv interferiert. Daher muss entweder das Spektrum oder der
Wegunterschied sehr klein sein, damit sich Maxima und Minima nicht in einem Punkt überschneiden.

Das Frequenzspektrum eines Wellenzuges ergibt sich durch eine Fourier-Transformation der 
Feldstärke zu 
\begin{equation}
	g(\omega) = 2E_0 \frac{\sin((\omega-\omega_0)\frac{\tau}{2}}{\omega-\omega_0}
\end{equation}
und damit die Intensität als Betragsquadrat zu
\begin{equation}
	\label{eqn:waveintensity}
	G(\omega) = |g(\omega)|^2 = 4E_0^2 \frac{\sin^2((\omega-\omega_0)\frac{\tau}{2}}{(\omega \omega_0)^2} \mathrm{.}
\end{equation}
Die Intensitätsverteilung \eqref{eqn:waveintensity} hat ein Maximum bei $\omega=\omega_0$ und 
als Verteilungsfunktion die Breite $\Delta \omega = \frac{2\pi}{\tau}$.
Mit dem Zusammenhang 
\begin{equation}
	\lambda_0 := \frac{2\pi \symup{c}}{\omega_0} \mathrm{,}
\end{equation}
der Breite $\Delta \omega = \frac{2\pi}{\tau}$ der Verteilungsfunktion $G(\omega)$ und 
Differentiation ergibt sich
\begin{equation}
	\Delta \lambda = \frac{\lambda_0^2}{\symup{c}\tau}
\end{equation}
und wegen der Dauer eines Wellenzuges $\tau = \frac{\ell}{\symup{c}}$ 
(auch \textbf{Kohärenzzeit}) ein Zusammenhang zwischen der Kohärenzlänge $\ell$, der Dauer
eines Wellenzuges $\tau$ und der Breite $\Delta \lambda$ der Wellenlängen- und Frequenzverteilung zu 
\begin{equation}
	\Delta \lambda = \frac{\lambda_0^2}{\ell} \mathrm{.}
\end{equation}

Des Weiteren zu beachten ist, dass in der Realität verwendete Lichtquellen nicht punktförmig 
sind, sondern eine endliche Ausdehnung haben. Strahlen, die von einer ausgedehnten Lichtquelle 
(siehe Abbildung \ref{fig:...}) aus durch eine Linse auf einen Beobachtungspunkt P fokussiert 
werden, können paarweise interferieren, wenn sie aus dem gleichen Punkt emittiert wurden. 
Durch den Winkel $\psi$ in Abbildung \ref{fig:...} tritt eine zusätzliche Phasenverschiebung 
$\Delta \phi$ zwischen den Strahlen 3 und 4 auf, sodass das Interferenzbild im Punkt P 
zerstört werden kann, wenn die Phasenverschiebung $\Delta \phi$ durch den Winkel $psi$ nicht 
viel kleiner als $\pi$ ist.
Somit ergibt sich mit 
\begin{equation}
	\Delta \phi = \frac{2\pi}{\lambda} a \sin(\psi) \mathrm{,}
\end{equation}
die \textbf{Kohärenzbedingung für ausgedehnte Lichtquellen} zu 
\begin{equation}
	a \sin(\psi) \ll \frac{lambda}{2} \mathrm{.}
\end{equation}












\cite{Anleitung}
