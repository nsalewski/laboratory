\section{Auswertung}
\label{sec:Auswertung}
\subsection{Bestimmung der Wellenlänge des Lasers}
Zur Bestimmung der Wellenlänge des Lasers werden Datenpaare aus den gezählten Interferenzmaxima am Photoelement $z$ und der Verschiebestrecke des Spiegels $\Delta d$ benötigt.
Die Verschiebestrecke ergibt sich über die abgelesene Verschiebung an der Mikrometerschraube multipliziert mit dem Faktor $\frac{1}{HÜ}$.
Dabei ist $HÜ=5.046$ die Hebelübersetzung.

Ebenso wie die Datenpaare befinden sich die aus ihnen nach Formel \eqref{eqn:sf} berechneten Wellenlängen $\lambda $ in Tabelle %\tabref{tab:spieglein}.
\begin{table}
  \caption{Datenpaare zur Berechnung der Wellenlänge des Lasers.}
  \label{tab:spieglein}
  \centering
\begin{tabular}{ccc}
  \toprule
$\Delta d$/$10^{-4}\si{\meter}$ & Anzahl $z$ der Intensitätsmaxima & Wellenlänge $\lambda$/$\si{\nano\meter}$ \\
\midrule
4.16 & 1028.0 & 809.67 \\
4.18 & 1006.0 & 831.32 \\
4.08 & 999.0 & 817.31 \\
4.24 & 979.0 & 866.39 \\
4.1 & 984.0 & 833.79 \\
4.26 & 991.0 & 859.9 \\
4.2 & 1023.0 & 821.38 \\
4.06 & 1025.0 & 792.71 \\
\bottomrule
\end{tabular}
\end{table}


Die Wellenlängen werden mittels python/numpy \cite{numpy} gemittelt und es ergibt sich:
\begin{equation}
\lambda=  \SI{829(23)}{\nano\meter}
\end{equation}
