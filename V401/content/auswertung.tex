\section{Auswertung}
\label{sec:Auswertung}
\subsection{Bestimmung der Wellenlänge des Lasers}
Zur Bestimmung der Wellenlänge des Lasers werden Datenpaare aus den gezählten Interferenzmaxima $z$ am Photoelement und der Verschiebestrecke des Spiegels $\Delta d$ benötigt.
Die Verschiebestrecke ergibt sich über die abgelesene Verschiebung an der Mikrometerschraube multipliziert mit dem Faktor $\frac{1}{H}$.
Dabei ist $H=5.046$ die Hebelübersetzung.

Ebenso wie die Datenpaare befinden sich die aus ihnen nach Formel \eqref{eqn:lambda} berechneten Wellenlängen $\lambda $ in Tabelle \ref{tab:spieglein}.
\begin{table}
  \caption{Datenpaare zur Berechnung der Wellenlänge des Lasers.}
  \label{tab:spieglein}
  \centering
\begin{tabular}{ccc}
  \toprule
$\Delta d$/$10^{-4}\si{\meter}$ & Anzahl $z$ der Intensitätsmaxima & Wellenlänge $\lambda$/$\si{\nano\meter}$ \\
\midrule
4.16 & 1028.0 & 809.67 \\
4.18 & 1006.0 & 831.32 \\
4.08 & 999.0 & 817.31 \\
4.24 & 979.0 & 866.39 \\
4.1 & 984.0 & 833.79 \\
4.26 & 991.0 & 859.9 \\
4.2 & 1023.0 & 821.38 \\
4.06 & 1025.0 & 792.71 \\
\bottomrule
\end{tabular}
\end{table}


Die Wellenlängen werden mittels python/numpy \cite{numpy} gemittelt und es ergibt sich:
\begin{equation}
\lambda=  \SI{829(23)}{\nano\meter} \text{.}
\end{equation}

\FloatBarrier

\subsection{Bestimmung des Brechungsindex von Luft}

Für die Bestimmung der Brechungsindizes werden folgende Werte verwendet:

\begin{align}
	T_0 &= \SI{273,15}{\kelvin} \\
	p_0 &= \SI{1,0132}{\bar}  \\
	T &= \SI{293,15}{\kelvin} \\
	b &= \SI{50}{\milli\meter} \\
	\lambda &= \SI{635}{\nano\meter}
\end{align}



\begin{table}
	\caption{Messwerte für die Berechnung des Brechungsindex von Luft.}
	\label{tab:luftbrech}
	\centering
	\begin{tabular}{ccc}
	\toprule
	$p$ / $\si{\bar}$ & Anzahl $z$ der Intensitätsmaxima & Brechungsindex \\
	\midrule
		0.75 & 25 & 1.000230 \\
		0.80 & 25 & 1.000216 \\
		0.80 & 32 & 1.000276 \\
		0.80 & 34 & 1.000293 \\
		0.80 & 32 & 1.000276 \\
		0.80 & 34 & 1.000293 \\
		0.80 & 33 & 1.000285 \\
		0.80 & 33 & 1.000285 \\
		0.80 & 34 & 1.000293 \\
	\bottomrule
	\end{tabular}
\end{table}

In Tabelle \ref{tab:luftbrech} sind die Messwerte zur Bestimmung des Brechungsindex von Luft
und der jeweils bestimmte Brechungsindex aufgetragen.
Der Brechungsindex wird mit Formel \eqref{eqn:n} berechnet, wobei sich $\Delta n$ aus Formel
\eqref{eqn:deltan} ergibt.

Es ergibt sich der Brechungsindex für Luft mit dem Fehler des Mittelwerts von
\begin{gather}
	n_{\mathrm{LUFT}} = (1,000272 \pm 0.000027) \mathrm{,}\\
  \Delta n_{\mathrm{LUFT}}=n_{\mathrm{LUFT}}-n_{\mathrm{Vakuum}}=0.000272\pm0.000027
\end{gather}

\subsection{Bestimmung des Brechungsindex von \texorpdfstring{$CO_2$}{math} }

Analog zur Bestimmung des Brechungsindex von Luft verläuft die Berechnung des
Brechungsindex von CO$_2$.

\begin{table}
	\caption{Messwerte für die Berechnung des Brechungsindex von \texorpdfstring{CO$_2$}{math}.}
	\label{tab:co2brech}
	\centering
	\begin{tabular}{ccc}
	\toprule
	$p$ / $\si{\bar}$ & Anzahl $z$ der Intensitätsmaxima & Brechungsindex \\
	\midrule
		0.80 & 54 & 1.000466 \\
		0.75 & 42 & 1.000387 \\
		0.85 & 59 & 1.000479 \\
		0.80 & 56 & 1.000483 \\
		0.85 & 58 & 1.000471 \\
		0.80 & 54 & 1.000466 \\
		0.82 & 58 & 1.000488 \\
		0.80 & 53 & 1.000457 \\
		0.80 & 56 & 1.000483 \\
		0.80 & 52 & 1.000449 \\
	\bottomrule
	\end{tabular}
\end{table}

Die Messwerte mit zugehörigem Brechungsindex sind in Tabelle \ref{tab:co2brech} aufgetragen.
Es ergibt sich der Brechungsindex zu
\begin{gather}
	n_{\mathrm{CO}_2} = 1.000463 \pm 0.000028 \mathrm{.}\\
  \Delta n_{\mathrm{CO}_2}=n_{\mathrm{CO}_2}-n_{\mathrm{Vakuum}}=0.000463\pm0.000028
\end{gather}
