\section{Durchführung}
\label{sec:Durchführung}

\subsection{Versuchsaufbau}
\label{sec:Versuchsaufbau}
Der Versuchsaufbau besteht aus einem Ultraschallechoskop, an dessen Ausgänge zwei Ultraschallsonden mit \SI{2}{\mega\Hz} gekoppelt sind, und einem Rechner zur Datenaufnahme und -analyse.\\
An das Ultraschallechoskop sind zwei Ultraschallsonden angeschlossen, mithilfe derer sich sowohl eine Impuls-Echo-Messung, als auch eine Durchschallmessung realisieren lässt.
Am Rechner werden die gemessenen Daten mittels des Programms \textquote{Echoview} ausgewertet.\\
Hierbei ist \textquote{Echoview} in der Lage, vier verschiedene Diagramme darzustellen.
Im linken oberen Graphen wird der A-Scan dargestellt, also die Amplitude gegen die Zeit aufgetragen.
Der linke untere Graph stellt die gewählte Verstärkung dar. Die Verstärkung lässt sich am Ultraschallechoskop über die Drehknöpfe zur laufzeit-bzw. tiefenabhängigen Verstärkung (TGC; Time Gain Control) und ebenso über die Verstärkung des Outputs und der Empfindlichkeit der Sonden regulieren.\\
Zu Beachten ist, dass eine Verstärkung nur gewählt werden darf, wenn die auszuwertende Messreihe nicht zur Untersuchung der Amplitudenhöhe dient.
Die beiden rechten Graphiken sind das berechnete Spektrum der Messdaten (FFT), bzw. ihr
Cepstrum.
Erzeugte Graphiken und Messdaten können aus dem Programm heraus exportiert werden.
Als zu untersuchende Versuchsobjekte stehen Acrylzylinder verschiedener Länge, Acrylplatten unterschiedlicher Dicke sowie das Modell eines menschlichen Auges im Maßstab 3:1 zur Verfügung.


\subsection{Versuchsbeschreibung}
\label{sec:Versuchsbeschreibung}
Zu Beginn des Versuchs muss das Michelson-Interferometer für die Messung justiert werden.
Dazu wird der Laser eingeschaltet und an die Position des Photoelements (vgl. Abbildung \ref{fig:aufbau}) eine Mattscheibe eingebracht.\\
Der justierbare Spiegel wird so ausgerichtet, dass die beiden hellsten Intensitätsmaxima der beiden ankommenden Strahlen möglichst genau zur Deckung gebracht werden. Das Photoelement wird entsprechend so ausgerichtet, dass das Intensitätsmaxima beider Strahlen genau auf den Eintrittsspalt des Photoelements liegt.
\\Für die Messung der Wellenlänge des Lasers wird der verschiebbare Spiegel genutzt.
Dazu wird der Synchronmotor eingeschaltet und eine Verschieberichtung ausgewählt.
Zu beachten ist hierbei, dass der Motor nicht zu schnell bewegt wird, da sonst das Photoelement nicht alle ankommenden konstruktiven Interferenzen, also Intensitätsmaxima der ankommenden Strahlen, als getrennte Intensitätsmaxima eindeutig zählen kann und somit das falsche Ergebnis liefert.
\\Bei Bedarf muss ein kleinerer Gang ausgewählt werden. Im vorliegenden Experiment wurde die Messung im kleinsten Gang durchgeführt.
Nachdem etwa 1000 konstruktive Interferenzen der Teilstrahlen am Photoelement registriert wurden, wird die exakte Anzahl an registrierten Interferenzen notiert sowie die Verschiebestrecke des Spiegels über die Verschiebung der Mikrometerschraube bestimmt. Dazu wird die Position zu Beginn und zum Ende der Verschiebung an der Mikrometerschraube abgelesen und mittels der Hebelübersetzung die Verschiebestrecke $\Delta d$ berechnet.
\\Die Messung wird etwa sieben- bis zehnmal durchgeführt.

Zur Bestimmung des Brechungsindex von Luft wird der verschiebbare Spiegel nicht mehr bewegt.\\
Die Messzelle wird mittels der Vakuumpumpe auf den Druck $p$ evakuiert, welcher notiert wird. Beim langsamen Wiedereinlassen der Luft werden erneut die konstruktiven Interferenzen am Photoelement gezählt und sobald wieder der Normaldruck $p_{\mathrm{0}}$ in der Messzelle herrscht, wird deren Anzahl $z$ notiert. \\Die Messung wird sieben-bis zehnmal wiederholt.

Für die Messung des Brechungsindex von $\mathrm{CO_2}$ wird die Messzelle evakuiert und der Druck $p$ notiert.\\
Es wird langsam $\mathrm{CO_2}$ in die Messzelle strömen gelassen und erneut die Anzahl an beobachteten Intensitätsmaxima $z$ notiert sobald in der Messzelle wieder der Normaldruck $p_0$ vorliegt.
\\Da sich noch Luftrückstände in der Messzelle befinden könnten, wird die Messung ebenso etwa sieben-bis zehnmal wiederholt.
