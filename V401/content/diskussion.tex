\section{Diskussion}
\label{sec:Diskussion}
Allgemein lässt sich sagen, dass der Versuchsaufbau sehr empfindlich gegenüber Erschütterungen war. Bereits Unruhe am Nebentisch führte zur Zählung von Impulsen durch das Zählwerk, ohne dass eine Variation der optischen Weglänge vollzogen wurde.
\\Auffällig ist, dass die Bestimmung des Brechungsindex für beide Gase mit einer sehr hohen Genauigkeit gelang. Die Abweichungen zum Theoriewert liegen deutlich unter der Auflösung des Experiments und jeweils auch im Bereich des Messunsicherheit (vergleiche dazu Tabelle \ref{tab:lalelu}).
\\Im Gegensatz dazu ist die Abweichung in der Wellenlängenmessung mit über $30\%$ sehr groß.
\\Eine mögliche Erklärung wäre, dass die Intensitätsmaxima zu schnell an der Öffnung des Photoelements vorbeizogen, sodass einige Maxima verschluckt wurden. Allerdings wurde für die Messreihe bereits der kleinste Gang des Spiegelantriebs verwendet.
\\Bereits zuvor zeigte sich bei Messungen in höheren Gängen, welche nicht im Rahmen der Messreihe der Auswertung stattfanden, dass für die angegebene Wellenlänge des Lasers jeweils etwa ein Viertel zuwenig Intensitätsmaxima am Photoelement festgestellt wurden. \\Auch ein mehrfaches Nachjustieren des Photoelements brachte keine Verbesserung.
\\Da die Messung der Intensitätsmaxima für die Berechnung der Brechungsindizes allerdings sehr genau war, ist eine generelle Fehlfunktion des Photoelements und der Zählapparatur aber auszuschließen. \\
Ein Problem mit dem verschiebbaren Spiegel oder ein Ablesefehler an der Mikrometerschraube wäre auch denkbar.\\
Ein Ablesefehler an der Mikrometerschraube ist aber recht unwahrscheinlich, da auch für verschieden große Verschiebestrecken in verschiedenen Messbereichen immer etwa die gleiche Abweichung zum Theoriewert festgestellt wurde.\\
Am wahrscheinlichsten ist also, dass am Photoelement dauerhaft ein Teil der Interferenzen nicht erfasst wurden und daher ein ungenaues Ergebnis zustande kam.
Als Theoriewert für den verbauten Laser für Tabelle \ref{tab:lalelu} wurde der am Laser angegebene Wert benutzt.
\begin{table}
	\caption{Vergleich der experimentell bestimmten Größen mit den Theoriewerten sowie die zugehörigen relativen Fehlern.}
	\label{tab:lalelu}
	\centering
	\begin{tabular}{cccc}
		\toprule
		&Experimenteller Wert&Theoriewert&relativer Fehler\\
		\midrule
Wellenlänge $\lambda$ des Lasers&\SI{829(23)}{\nano\meter}&\SI{635}{\nano\meter}&30.6\%\\
Brechungsindex $n_{\mathrm{LUFT}}$&1.000272 \pm 0.000027& 1.0002765 \cite{co2}&$4.5\cdot 10^{-6}$\%\\
Brechungsindex $n_{\mathrm{CO}_2}$ &1.000463 \pm 0.000028&1.0004476 \cite{co2}&$1.5\cdot 10^{-5}$\%\\
  \bottomrule
	\end{tabular}
\end{table}
