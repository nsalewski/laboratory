\section{Diskussion}
\label{sec:Diskussion}
Aus der Bestimmung der Strömungsgeschwindigkeit lässt sich erkennen, dass die Geschwindigkeit 
bei kleineren Rohrdurchmessern größer ist, was auch zu erwarten ist.
Für die Konstante $2\frac{\nu_0}{c}$ gemäß Formel \eqref{eqn:verschiebung} ergibt sich eine 
Abweichung von $\SI{34.68}{\percent}$.
Bei der Betrachtung der Eindringtiefe wurden nicht die in der Anleitung angegebenen Verhältnisse zwischen der Eindringtiefe $x_{sec}$ in Microsekunden und der Eindringtiefe $x_{m}$ in Millimetern verwendet, da sich bei Verwendung dieser Verhältnisse für den Innendurchmesser des Rohres, welcher aus der Darstellung der Abhängigkeit der Streuintensität $I_{\mathrm{S}}$ von der Eindringtiefe bestimmt wurde, eine deutliche Abweichung vom erwarteten Wert zeigte.
Abgelesen aus dem Plot wurde der Innendurchmesser so zunächst bestimmt zu $d=\SI{8.5}{\milli\meter}$, dies entspricht einer Abweichung von $15\%$ zu den Kenndaten der Messapparatur.
Die Verhältnisse wurden daher exakter berechnet. Mit den so berechneten Verhältnissen ergibt sich der Innendurchmesser im Rahmen von Ableseungenauigkeiten genau zum erwarteten Wert des Versuchsaufbau.
Bei der Betrachtung des Strömungsprofil zeigt sich nahezu exakt der erwartete parabelförmige Verlauf. Es zeigt sich zudem, dass die in der Versuchsbeschreibung angegebene Näherung zwischen der Eindringtiefe $x_\mathrm{mm}$ und der Eindringtiefe $x_\mathrm{sec}$ nur in grober Näherung richtig ist. Bei der im vorliegenden Versuch verwendeten besseren Näherung zeigt sich, dass theoretisch noch mindestens ein Messpunkt bei geringerer Eindringtiefe hätte aufgenommen werden müssen, um das gesamte Strömungsprofil innerhalb des Strömungsrohr analysieren zu können.
Bei der Betrachtung des Strömungsprofil zeigt sich zudem deutlich, wie ungenau die Messdaten aufgenommen werden konnten. Für den Bereich außerhalb der Dopplerflüssigkeit wird eine konstante Strömungsgeschwindigkeit angenommen. In der Messung mit Pumpleistung $P=\SI{45.2}{\percent}$ zeigen sich allerdings Schwankungen, welche auf Ablesefehler zurückzuführen sind.
Für die Streuintensität $I_\mathrm{S}$ zeigt sich ebenfalls bei einer Pumpleistung $P=\SI{45.2}{\percent}$ einige kleinere Abweichungen vom erwarteten Verlauf. Auch hier werden Ablesefehler als Grund vermutet, da sich in der Messung für eine Pumpleistung von $P=\SI{70}{\percent}$ ziemlich genau der erwartete Verlauf zeigt.
