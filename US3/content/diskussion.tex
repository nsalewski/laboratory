\section{Diskussion}
\label{sec:Diskussion}
Bei der Betrachtung der Eindringtiefe wurden nicht die in der Anleitung angegebenen Verhältnisse zwischen der Eindringtiefe $x_{sec}$ in Microsekunden und der Eindringtiefe $x_{m}$ in Millimetern verwendet, da sich bei Verwendung dieser Verhältnisse für den Innendurchmesser des Rohres, welcher aus der Darstellung der Abhängigkeit der Streuintensität $I_{\mathrm{S}}$ von der Eindringtiefe bestimmt wurde, eine deutliche Abweichung vom erwarteten Wert zeigte.
Abgelesen aus dem Plot wurde der Innendurchmesser so zunächst bestimmt zu $d=\SI{8.5}{\milli\meter}$, dies entspricht einer Abweichung von $15\%$ zu den Kenndaten der Messapparatur.
Die Verhältnisse wurden daher exakter berechnet. Mit den so berechneten Verhältnissen ergibt sich der Innendurchmesser im Rahmen von Ableseungenauigkeiten genau zum erwarteten Wert des Versuchsaufbau.  
