\subsection{Versuchsaufbau}
\label{sec:Versuchsaufbau}
Der Versuchsaufbau besteht zum Einen aus einer Strömungsröhre mit drei verschiedenen 
Innendurchmessern ($\SI{7}{\milli\meter}$, $\SI{10}{\milli\meter}$ und $\SI{16}{\milli\meter}$).
Weiterhin ist die Strömungsröhre mit einem Gemisch aus Wasser, Glycerin und Glaskugeln gefüllt.
Den drei Teilrohren ist jeweils ein Doppler-Prisma (vgl. Abbildung
\ref{fig:dopplerprisma}) zugeordnet, dessen drei Flächen den Doppler-Winkeln $\SI{15}{\degree}$,
$\SI{30}{\degree}$ und $\SI{60}{\degree}$ gegenüber der strömenden Flüssigkeit entsprechen. 
Des Weiteren wird ein Ultraschall Doppler-Generator, eine Ultraschallsonde (Frequenz 
$\SI{2}{\mega\hertz}$) und ein Rechner mit dem Programm FlowView verwendet, um die Messungen
aufzunehmen. Außerdem wird eine Zentrifugalpumpe gebraucht, mit der die
Strömungsgeschwindigkeit
varriert wird. 
