\section{Auswertung}
\label{sec:Auswertung}

%%%%%%%%%%%%%%%%%%%%%%%%%%%%%%%%%%%%%%%%%%%%%%%%%%%%%%%%%%%%%%%%%%%%%%%%%%%%%%%%%%%%%%%%%%%%%%
% Teil b)
\FloatBarrier
\subsection{Dispersionskurven nach Dispersionsrelation}

\subsubsection{Dispersionskurve der LC-Kette}
Die Dispersionskurve ergibt sich nach Gleichung \eqref{eqn:dispersion} mit 
\begin{equation}
	\label{eqn:dispisi}
	\omega = \sqrt{\frac{2}{LC}(1-\cos\theta)} \, \text{,}
\end{equation}
da negative Frequenzen nicht betrachtet werden.
Die Phasenverschiebung pro Kettenglied $\theta$ ergibt sich durch 
\begin{equation}
	\theta = \frac{\phi}{n} \, \text{,}
\end{equation}
wobei $\phi$ der gemessenen Phasenverschiebung zwischen Eingangsspannung $U_{\text{ein}}$
und Ausgangsspannung $U_{\text{aus}}$ entspricht.

Die sich ergebenden Werte sind in Tabelle \ref{tab:dispersionlc} aufgetragen.

\begin{table}
	\caption{Messdaten zur Untersuchung der Dispersionskurve einer LC-Kette}
	\label{tab:dispersionlc}
	\centering
	\begin{tabular}{cccc}
	\toprule
	$\phi$/$\si{\radian}$ & $f$/$\si{\kilo\Hz}$ & $\omega$/$\si{\kilo\Hz}$ & $\theta$/$\si{\radian}$ \\
	\midrule
		$\pi$ & 7.2 & 45.2 & 0.22 \\
		2$\pi$ & 14.3 & 89.8 & 0.45 \\
		3$\pi$ & 21.4 & 134.5 & 0.67 \\
		4$\pi$ & 28.3 & 177.8 & 0.90 \\
		5$\pi$ & 34.7 & 218.0 & 1.12 \\
		6$\pi$ & 41.0 & 257.6 & 1.35 \\
		7$\pi$ & 46.4 & 291.5 & 1.57 \\
		8$\pi$ & 51.5 & 323.6 & 1.80 \\
		9$\pi$ & 56.0 & 351.9 & 2.02 \\
		10$\pi$ & 58.7 & 368.8 & 2.24 \\
		11$\pi$ & 61.1 & 383.9 & 2.47 \\
		12$\pi$ & 63.2 & 397.1 & 2.69 \\
		13$\pi$ & 64.4 & 404.6 & 2.92 \\
	\bottomrule
	\end{tabular}
\end{table}

Die theoretische Dispersionskurve nach Gleichung \eqref{eqn:dispisi} mit den zugehörigen
Messwerten ist in Abbildung \ref{fig:dispilc} dargestellt.

\begin{figure}
	\centering
	\includegraphics{Bilder/b1.pdf}
	\caption{Dispersionskurve der LC-Kette: Kreisfrequenz $\omega$ gegen die Phasenverschiebung pro Kettenglied $\theta$ aufgetragen}
	\label{fig:dispilc}
\end{figure}

\FloatBarrier
\subsubsection{Dispersionskurve der LC$_1$C$_2$-Kette}
Die Dispersionskurve für die LC$_1$C$_2$-Kette ergibt sich nach Formel \eqref{eqn:dispersion2}.

Analog wie bei der LC-Kette erhält man die Messdaten, die in Tabelle \ref{tab:dispersioncc}
aufgetragen sind.

\begin{table}
	\caption{Messdaten zur Untersuchung der Dispersionskurve einer LC$_1$C$_2$-Kette}
	\label{tab:dispersioncc}
	\centering
	\begin{tabular}{cccc}
	\toprule
	$\phi$/$\si{\radian}$ & $f$/$\si{\kilo\Hz}$ & $\omega$/$\si{\kilo\Hz}$ & $\theta$/$\si{\radian}$ \\
	\midrule
		$\pi$ & 8.65 & 54.3 & 0.22 \\
		2$\pi$ & 17.05 & 107.1 & 0.45 \\
		3$\pi$ & 25.10 & 157.7 & 0.67 \\
		4$\pi$ & 32.45 & 203.9 & 0.90 \\
		5$\pi$ & 38.90 & 244.4 & 1.12 \\
		6$\pi$ & 43.80 & 275.2 & 1.35 \\
		7$\pi$ & 73.80 & 463.7 & 1.57 \\
		8$\pi$ & 75.10 & 471.9 & 1.80 \\
	\bottomrule
	\end{tabular}
\end{table}

Die theoretische Dispersionskurve nach Gleichung \eqref{eqn:dispersion2} mit den zugehörigen Messwerten ist in Abbildung \ref{fig:dispicc} dargestellt. 

\begin{figure}
	\centering
	\includegraphics{Bilder/b2.pdf}
	\caption{Dispersionskurve der LC$_1$C$_2$-Kette: Kreisfrequenz $\omega$ gegen die Phasenverschiebung pro Kettenglied $\theta$ aufgetragen}
	\label{fig:dispicc}
\end{figure}



%%%%%%%%%%%%%%%%%%%%%%%%%%%%%%%%%%%%%%%%%%%%%%%%%%%%%%%%%%%%%%%%%%%%%%%%%%%%%%%%%%%%%%%%%%%%%%
%c
\FloatBarrier
\subsection{Phasengeschwindigkeit in einer $LC$-Kette}

Aus den bestimmten Eigenfrequenzen $f$ der $LC$-Kette und den zugehörigen Phasenverschiebungen $\theta$ lassen sich nach Formel \eqref{eqn:v-phase} die jeweiligen Phasengeschwindigkeiten berechnen.
Die Theoriekurve ergibt sich ebenso nach Formel \eqref{eqn:v-phase}.
$\theta$, also die Phasenverschiebung pro Kettenglied, ergibt sich aus der Phasenverschiebung $\phi$ über die gesamte $LC$-Kette dividiert durch die Anzahl der Kettenglieder $n=14$.
In Tabelle \ref{tab:c} finden sich die bestimmten Nummern der Entartungen $n$ der Lissajou-Figuren, welche jeweils der Phasenverschiebung über der ganzen $LC$-Kette mit $\phi=n\pi$ entsprechen mit den zugehörigen Eigenfrequenzen $f$.
Zudem werden die Eigenfrequenzen umgerechnet in die Kreisfrequenz $\omega$ und die Phasenverschiebung pro Kettenglied $\theta$ in die Tabelle eingetragen.
In der fünften Spalte ist die Phasengeschwindigkeit nach $v_{\mathrm{Ph}}=\frac{\omega}{\theta}$ eingetragen.
In Abbildung \ref{fig:plotc} sind die gefundenden Messwerte der Phasengeschwindigkeit im Vergleich zur Theoriekurve aufgetragen.
\begin{table}
  \caption{Phasengeschwindigkeiten zu den jeweiligen Eigenfrequenzen der $LC$-Kette}
\label{tab:c}
\centering
\begin{tabular}{ccccc}
\toprule
Phasenverschiebung $\phi$/$\pi\si{\radian}$ & $f$/$\si{\kilo\Hz}$ & $\omega$/$\si{\kilo\Hz}$ & $\theta$/$\si{\radian}$ & $v_{\mathrm{Ph}}$/\si{\kilo\metre\per\second} \\
\midrule
1.0 & 7.2 & 45.2 & 0.22 & 201.6 \\
2.0 & 14.3 & 89.8 & 0.45 & 200.2 \\
3.0 & 21.4 & 134.5 & 0.67 & 199.73 \\
4.0 & 28.3 & 177.8 & 0.9 & 198.1 \\
5.0 & 34.7 & 218.0 & 1.12 & 194.32 \\
6.0 & 41.0 & 257.6 & 1.35 & 191.33 \\
7.0 & 46.4 & 291.5 & 1.57 & 185.6 \\
8.0 & 51.5 & 323.6 & 1.8 & 180.25 \\
9.0 & 56.0 & 351.9 & 2.02 & 174.22 \\
10.0 & 58.7 & 368.8 & 2.24 & 164.36 \\
11.0 & 61.1 & 383.9 & 2.47 & 155.53 \\
12.0 & 63.2 & 397.1 & 2.69 & 147.47 \\
13.0 & 64.4 & 404.6 & 2.92 & 138.71 \\
\bottomrule
\end{tabular}
\end{table}
\begin{figure}
  \centering
 \includegraphics{Bilder/c.pdf}
  \caption{Phasengeschwindigkeit der $LC$-Kette an jedem Kettenglied aufgetragen gegen die Kreisfrequenz}
  \label{fig:plotc}
\end{figure}
%%%%%%%%%%%%%%%%%%%%%%%%%%
%d
\subsection{Spannungsverlauf der $LC$-Kette bei offenem Ende}
Die gemessenen Spannungsamplituden an jedem Kettenglied bei der 1. und 2. Eigenschwingung der $LC$-Kette finden sich in Tabelle \ref{tab:ei}.

\begin{table}
   \centering
\caption{Wichtige Tabelle}
\label{tab:ei}
\begin{tabular}{ccc}
  \toprule
Nummer des Kettenglieds & $U$/$\si{\volt}$ bei 1.Eigenschwingung & $U$/$\si{\volt}$ bei 2.Eigenschwingung \\
\midrule
0 & 0.93 & 1.11 \\
1 & 1.17 & 0.81 \\
2 & 1.2 & 0.3 \\
3 & 1.14 & 0.07 \\
4 & 0.84 & 0.35 \\
5 & 0.63 & 0.6 \\
6 & 0.3 & 0.69 \\
7 & 0.09 & 0.78 \\
8 & 0.225 & 0.6 \\
9 & 0.65 & 0.3 \\
10 & 0.84 & 0.15 \\
11 & 0.75 & 0.66 \\
12 & 0.78 & 1.05 \\
13 & 0.9 & 1.05 \\
\bottomrule
\end{tabular}
\end{table}

Die Grundschwingung, also die erste Eigenschwingung der $LC$-Kette, betrug $w_{\mathrm{0}}=7.5 \,\si{\kilo\Hz}$, die erste Oberschwingung, also die 2. Eigenschwingung betrug $w_{\mathrm{1}}=14.7 \,\si{\kilo\Hz}$.
Nach Formel  \eqref{eqn:abschluss} wird bei der beidseitig offenen $LC$-Kette, also $R=\infty$ an den Enden der LC-Kette ein Maximum der Spannungsamplitude angenommen, sowie kein Phasensprung am Kettenende erwartet.
In Abbildung \ref{fig:plotdeins} sind die jeweilig gemessenen Spannungsamplituden bei den ersten beiden Eigenschwingungen gegen die Kettenglieder aufgetragen.
Der erwartete Spannungsverlauf einer stehenden Cosinus-Welle wird durch die Messdaten annähernd erreicht. Die stehende Welle weist bei der Grundschwingung einen Spannungsknoten etwa in der Mitte der $LC$-Kette auf, während an den Enden Spannungsbäuche auftreten.
Daher findet auch wie erwartet kein Phasensprung am Kettenende statt.
 Der Spannungsverlauf bei der ersten Eigenschwingung bildet sich eine stehende Welle mit halber Wellenlänge.
In Abbildung \ref{fig:plotd} erkennt man ebenso den erwarteten Verlauf der Spannungsamplituden für die zweite Eigenschwingung.
Erneut zeigen sich an Anfang und Ende der $LC$-Kette Spanungsbäuche, es findet also wieder wie erwartet kein Phasensprung am Kettenende statt.
Etwa in der Mitte der $LC$-Kette zeigt sich ein dritter Wellenbauch und es ergeben sich somit 2 Schwingungsknoten und 3 Schwingungsbäuche.
Die erste Oberschwingung weist also genau eine Wellenlänge auf.
\begin{figure}
  \centering
 \includegraphics{Bilder/d.pdf}
  \caption{Spannungsverlauf der offenen $LC$-Kette für $f=7.5 \,\si{\kilo\Hz}$(Grundschwingung)}
  \label{fig:plotdeins}
\end{figure}

\begin{figure}
  \centering
 \includegraphics{Bilder/d2.pdf}
  \caption{Spannungsverlauf der offenen $LC$-Kette für $f=14.7 \,\si{\kilo\Hz}$(erste Oberschwingung)}
  \label{fig:plotd}
\end{figure}

\subsection{Spannungsverlauf der mit dem Wellenwiderstand $Z$ abgeschlossenen $LC$-Kette}
Für eine geschlossene $LC$-Kette mit dem Wellenwiderstand $Z$ als Abschlusswiderstand wird keine Reflektion am Kettenende erwartet und damit dürften sich auch keine stehenden Wellen bilden.
Der Wellenwiderstand ergibt sich nach Formel \eqref{eqn:wellenwiderstand} etwa zu $Z=246 \,\si{\ohm}$.
In Tabelle \ref{tab:lame} finden sich die gemessenen Spannungsamplituden für die erste Eigenschwingung der $LC$-Kette an den jeweiligen Kettengliedern.
In Abbildung \ref{fig:really?!} sind die Spannungsamplituden gegen die Orte auf der $LC$-Kette -also die Nummern der jeweiligen Masche der Kette- aufgetragen.
\begin{table}
  \centering
\caption{Spannungsamplituden and den einzelnen Kettengliedern für die erste Eigenfrequenz der mit dem Wellenwiderstand abgeschlossenen $LC$-Kette}
\label{tab:lame}
\begin{tabular}{cc}
\toprule
Nummer des Kettenglieds & $U$/$\si{\volt}$ bei 1.Eigenschwingung und Abschlusswiderstand $R=Z$ \\
\midrule
0 & 0.025 \\
1 & 0.025 \\
2 & 0.025 \\
3 & 0.025 \\
4 & 0.025 \\
5 & 0.025 \\
6 & 0.025 \\
7 & 0.025 \\
8 & 0.025 \\
9 & 0.025 \\
10 & 0.025 \\
11 & 0.025 \\
12 & 0.025 \\
13 & 0.025 \\
\bottomrule
\end{tabular}
\end{table}

Es zeigt sich der erwartete Verlauf. Es lässt sich eine konstante Spannungsamplitude an jedem Kettenglied feststellen.
Somit hat sich keine stehende Welle ausgeprägt und es findet, wie erwartet, für den Wellenwiderstand als Abschlusswiderstand keine Reflektion am Kettenende statt.


\begin{figure}
  \centering
 \includegraphics{Bilder/d3.pdf}
  \caption{Spannungsverlauf einer mit dem Wellenwiderstand $Z$ abgeschlossenen $LC$-Kette bei der ersten Eigenfrequenz ($7.5 \,\si{\kilo\Hz}$)}
  \label{fig:really?!}
\end{figure}
