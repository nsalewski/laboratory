\section{Diskussion}
\label{sec:Diskussion}
Insgesamt lässt sich sagen, dass die erhaltenen Ergebnissen etwa der Theorie folgen.
\FloatBarrier
%%%%%%%%%%%%%%%%%%%%%%%%%%%%%%%%%%%%%%%%%%%%%%%%%%%%%%%%%%%%%%%%%%%%%%%%%%%%%%%%%%%%%%%%%%%%%%%
Bei der Ermittlung der Grenzfrequenzen von LC und LC$_1$C$_2$-Kette treten geringe
Diskrepanzen zu den theoretisch erwarteten Werten auf.
Die experimentell bestimmte Grenzfrequenz $f_{\text{exp}}=(64,115 \pm 1,546) \, \si{\kilo\hertz}$
zeigt nur eine relative Abweichung vom theoretisch berechneten Wert ($f_{\text{theo}}=\SI{64,311}{\kilo\hertz}$) von $0,3\%$.
Die Grenzfrequenzen der LC$_1$C$_2$-Kette sind in Tabelle \ref{tab:halleluja} aufgetragen.
\begin{table}
	\caption{Experimentell beziehungsweise theoretisch bestimmte Grenzfrequenzen der LC$_1$C$_2$-Kette mit zugehörigen relativen Fehlern.}
	\label{tab:halleluja}
	\centering
	\begin{tabular}{ccc}
		\toprule
		$f_{\text{n,theo}}$/$\si{\kilo\hertz}$ & $f_{\text{n,exp}}$/$\si{\kilo\hertz}$ & $\Delta f_n$ / \% \\
		\midrule
		45,475                                 & 52,175                                & 14,7              \\
		66,511                                 & 66,801                                & 0,4               \\
		80,571                                 & 79,883                                & 0,9               \\
		\bottomrule
	\end{tabular}
\end{table}

Bei beiden Ketten liegen die Fehler -- bis auf bei der 1. Grenzfrequenz der LC$_1$C$_2$-Kette --
im jeweils bestimmten Fehlerintervall und sind daher auf
zufällige Fehler, sowie auf Ablesefehler und Fehler beim Bestimmen der Skalierung des
Millimeterpapiers zurückzuführen.
Der im Vergleich große Fehler von $14,7 \%$ bei der 1. Grenzfrequenz lässt sich durch die
schlechte Ablesbarkeit dieser Grenzfrequenz auf der Durchlasskurve begründen.
\FloatBarrier
%%%%%%%%%%%%%%%%%%%%%%%%%%%%%%%%%%%%%%%%%%%%%%%%%%%%%%%%%%%%%%%%%%%%%%%%%%%%%%%%%%%%%%%%%%%%%%%

Zu der Dispersionskurve der LC-Kette lässt sich anmerken, dass nur geringe Abweichungen zwischen
den Messwerten und der Theoriekurve zu erkennen sind.

Bei der Dispersionskurve der LC$_1$C$_2$-Kette hingegen, sind zwar im niedrigen Frequenzbereich
geringe Diskrepanzen zwischen Theoriekurve und Messwerten festzustellen, aber bei höheren
Frequenzen -- also den Messwerten, die dem oberen Ast zuzuordnen sind -- sind größere Abweichung
zu erkennen.

Diese Abweichungen können darauf zurückgeführt werden, dass der regelbare Wellenwiderstand $Z$.
der einer Funktion der Frequenz entspricht, nicht bei steigender Frequenz hochgeregelt,
sondern konstant gehalten wurde.
%%%%%%%%%%%%%%%%%%%%%%%%%%%%%%%%%%%%%%%%%%%%%%%%%%%%%%%%%%%%%%%%%%%%%%%%%%%%%%%%%%%%%%%%%%%%%%

Bei der Bestimmung der Phasengeschwindigkeit an jedem Kettenglied liegen die gemessenen Werte tendenziell leicht über der Theoriekurve, folgen aber trotzdem ihrem prinzipiellen Verlauf.
Abweichungen lassen sich  zurückführen auf Schwierigkeiten beim Bestimmen der Eigenfrequenzen des Systems, da sich teilweise nur geringe Abweichungen in den Lissajou-Figuren zeigten.
Dies galt besonders für große Frequenzen.

Für die stehenden Wellen auf der $LC$-Kette bei den ersten beiden Eigenschwingungen zeigt sich tendenziell der erwartete Verlauf.
Für die Grundschwingung ist allerdings am linken Ende der Kette das Maximum nicht an der nullten Masche, sondern an der ersten Masche und für das rechte Ende der Kette liegt das Maximum niedriger als am linken Ende und es zeigt sich eine unerwartet geringe Spannungsamplitude an der 11. und 12. Masche.\\
Eventuell werden diese Störungen durch die mehrfach reflektierte Welle, auch am linken Kettenende wird erneut eine vollständige Reflektion ohne Phasensprung erwartet, verursacht.
Zudem sollte die minimale Spannungsamplitude theoretisch nicht an der siebten Masche auftreten, sondern zwischen der sechsten und siebten Masche.

Näher an dem erwarteten Spannungsverlauf eines Kosinus liegt der Verlauf der Spannungsamplituden für die erste Oberschwingung.
Der mittlere Spannungsbauch liegt allerdings auch hier nicht mittig auf der $LC$-Kette und am rechten Ende zeigt sich an der vorletzen Masche die identische Spannungsamplitude wie an der letzten Masche.
Erneut könnten weitere reflektierte Wellen der Grund für die Unregelmäßigkeiten sein.
%Tabelle mit den Grenzfrequenzen
