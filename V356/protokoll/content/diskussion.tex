\section{Diskussion}
\label{sec:Diskussion}
Insgesamt lässt sich sagen, dass die erhaltenen Ergebnissen etwa der Theorie folgen.
Bei der Bestimmung der Phasengeschwindigkeit an jedem Kettenglied liegen die gemessenen Werte tendenziell leicht über der Theoriekurve, folgen aber trotzdem ihrem prinzipiellen Verlauf.
Abweichungen lassen sich  zurückführen auf Schwierigkeiten beim Bestimmen der Eigenfrequenzen des Systems, da sich teilweise nur geringe Abweichungen in den Lissajou-Figuren zeigten.
Dies galt besonders für große Frequenzen.

Für die stehenden Wellen auf der $LC$-Kette bei den ersten beiden Eigenschwingungen zeigt sich tendenziell der erwartete Verlauf.
Für die Grundschwingung ist allerdings am linken Ende der Kette das Maximum nicht an der 0.Masche, sondern an der ersten Masche und für das rechte Ende der Kette liegt das Maximum niedriger als am linken Ende und es zeigt sich eine unerwartet geringe Spannungsamplitude an der 11. und 12. Masche.\\
Eventuell werden diese Störungen durch die mehrfach reflektierte Welle, auch am linken Kettenende wird erneut eine vollständige Reflektion ohne Phasensprung erwartet, verursacht.
Zudem sollte die minimale Spannungsamplitude theoretisch nicht an der 7. Masche auftreten, sondern zwischen der 6. und 7. Masche.

Näher an dem erwarteten Spannungsverlauf eines Kosinus liegt der Verlauf der Spannungsamplituden für die erste Oberschwingung.
Der mittlere Spannungsbauch liegt allerdings auch hier nicht mittig auf der $LC$-Kette und am rechten Ende zeigt sich an der vorletzen Masche die identische Spannungsamplitude wie an der letzten Masche.
Erneut könnten weitere reflektierte Wellen der Grund für die Unregelmäßigkeiten sein.
%Tabelle mit den Grenzfrequenzen
