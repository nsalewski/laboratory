\section{Durchführung}
\label{sec:Durchführung}

\subsection{Bestimmung einer Durchlasskurve}
Die Durchlasskurven einer LC- beziehungsweise LC$_1$C$_2$-Kette können
mit dem Aufbau in Abbildung \ref{fig:durchlasskurve} bestimmt werden.

Der Wellenwiderstand $Z$ am Eingang und Ausgang der Kette wird durch einen
regelbaren Widerstand realisiert und mittels
\begin{equation}
	Z = \sqrt{\frac{L}{C}}
\end{equation}
beziehungsweise
\begin{equation}
	Z = \sqrt{\frac{2L}{C_1+C_2}}
\end{equation}
auf den genäherten Wellenwiderstand $Z$ für kleine Frequenzen festgelegt (vergleiche Formel \eqref{eqn:wellenwiderstand}).
Auf einem X-Y-Schreiber lässt sich die Ausgangsspannung $U_{\text{aus}}$ in Abhängigkeit von der Frequenz $\omega$ der Eingangsspannung $U_{\text{ein}}$ aufzeichnen.
Die Eingangsspannung liefert ein wobbelbarer NF-Generator.
Außerdem kann an diesem NF-Generator eine Range eingestellt werden, in der die Frequenzen um eine festgelegte Frequenz variieren.
Dies ermöglicht eine Aufzeichnung um die Grenzfrequenz -- beziehungsweise 
bei der LC$_1$C$_2$-Kette um die Grenzfrequenzen.

Des Weiteren müssen an dem X-Y-Schreiber die Position sowie die Sensitivität der Achsen eingestellt werden.
Hierbei muss beachtet werden, dass der Schreiber nicht den Rand überschreitet und der betrachtete Messbereich geeignet groß dargestellt wird.

Nachdem die Kalibrierung abgeschlossen und die Grenzfrequenzen zu erkennen sind, kann der Aufzeichnungsvorgang am X-Y-Schreiber gestartet werden.
Bei diesem müssen einige Referenzpunkte für die Frequenz bestimmt werden, 
da die Achsen des verwendeten Millimeterpapiers nicht skaliert sind.

Bei der Aufzeichnung der Durchlasskurve für die LC$_1$C$_2$-Kette wird jeder zweite Kondensator der LC-Kette ausgetauscht.


\subsection{Dispersionsrelation}

Hierbei soll die Frequenzabhängigkeit der Phasenverschiebung $\phi$ 
zwischen Eingangsspannung $U_{\text{ein}}$ und Ausgangsspannung $U_{\text{aus}}$ gemäß der Dispersionsrelation nach Formel \eqref{eqn:dispersion}
-- beziehungsweise Formel \eqref{eqn:dispersion2} untersucht werden.
Zum Nachweis der Dispersionsrelation lässt sich der Aufbau aus Abbildung 
\ref{fig:} verwenden.

Dafür wird die Eingangsspannung sowie die Ausgangsspannung auf den X- beziehungsweise Y-Eingang 
des Oszilloskops gegeben und das Oszilloskop im Zwei-Kanal-Modus betrieben.
Dann werden für die LC- und die LC$_1$C$_2$-Kette die Frequenzen ermittelt, bei denen auf 
dem Oszilloskop eine Phasenverschiebung von $\pi$ zu erkennen ist.
Dies wird realisiert, indem am NF-Generator die Frequenzen $\omega$ der Eingangsspannung 
hochgeregelt werden bis auf dem Oszilloskop eine Phasenverschiebung von $\pi$ also eine 
Gerade zu erkennen ist.

Zu beachten ist hier, dass bei der LC$_1$C$_2$-Kette -- wie in Abschnitt \ref{sec:} erläutert 
-- der optische Ast auftritt.









\subsection{Nachweis von stehenden Wellen}
