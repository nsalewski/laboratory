\section{Diskussion}
\label{sec:Diskussion}
Allgemein lässt sich die sowohl die Schwingungsdauer als auch die Schwebungsdauer der gekoppelten Pendel nur recht ungenau bestimmen, da sämtliche Zeiten manuell gestoppt wurden.\\
Ein zusätzlicher Fehler ergibt sich eventuell über die teils klemmenden Knöpfe der Stoppuhr.
Eine genauere Messung ließe sich beispielsweise realisieren, indem die Zeitnahme mittels Lichtschranken und einer entsprechenden Steuerungsschaltung automatisiert wird.
In Tabelle \ref{tab:discuss} und \ref{tab:discuss2} sind die gemessenen und nach der Theorie
berechneten Werte eingetragen.

\begin{table}
	\caption{Vergleich zwischen Theorie und Experiment für die Pendellänge $l=\SI{0.45}{\meter}$ .}
	\label{tab:discuss}
	\begin{tabular}{cccc}
		\toprule
		Messung                    & Experiment               & Theorie                  & prozentuale Abweichung \\
		\midrule
		$T_{\mathrm{+}}$           & $\SI{1.325(2)}{\second}$ & $\SI{1.346(4)}{\second}$ & $1.6\%$                \\
		$\omega_{\mathrm{+}}$      & $\SI{4.742(9)}{\Hz}$     & $\SI{4.67(2)}{\Hz}$      & $1.5\%$                \\
		$T_{\mathrm{-}}$           & $\SI{1.13(1)}{\second}$  & $\SI{1.13(1)}{\second}$  & $0.0\% $               \\
		$\omega_{\mathrm{-}}$      & $\SI{5.58(5)}{\Hz}$      & $\SI{5.58(5)}{\Hz}$      & $0.0\%$                \\
		$T_{\mathrm{S,einfach}}$   & $\SI{6.4(1)}{\second}$   & $\SI{6.9(4)}{\second}$   & $7.2\%$                \\
		$T_{\mathrm{S,fünffach}}$ & $\SI{6.96(3)}{\second}$  & $\SI{6.9(4)}{\second}$   & $0.9\%$                \\
		$\omega_{\mathrm{S,1}}$    & $\SI{0.84(5)}{\Hz}$      & $\SI{0.91(6)}{\Hz}$      & $7.7\%$                \\
		$\omega_{\mathrm{S,2}}$    & $\SI{0.903(4)}{\Hz}$     & $\SI{0.91(6)}{\Hz}$      & $0.7\%$                \\
		\bottomrule
	\end{tabular}
\end{table}

\begin{table}
	\caption{Vergleich zwischen Theorie und Experiment für die Pendellänge 72 \,\si{\centi\meter}}
	\label{tab:discuss2}
	\begin{tabular}{cccc}
		\toprule
		Messung                    & Experiment               & Theorie                  & prozentuale Abweichung \\
		\midrule
		$T_{\mathrm{+}}$           & $\SI{1.609(4)}{\second}$ & $\SI{1.702(4)}{\second}$ & $5.5\%$                \\
		$\omega_{\mathrm{+}}$      & $\SI{3.904(10)}{\Hz}$    & $\SI{3.692(8)}{\Hz}$     & $5.7\%$                \\
		$T_{\mathrm{-}}$           & $\SI{1.410(6)}{\second}$ & $\SI{1.492(8)}{\second}$ & $5.5\% $               \\
		$\omega_{\mathrm{-}}$      & $\SI{4.455(18)}{\hertz}$ & $\SI{4.212(22)}{\Hz}$    & $5.8\%$                \\
		$T_{\mathrm{S,einfach}}$   & $\SI{11.82(8)}{\second}$ & $\SI{12.1(5)}{\second}$  & $2.3\%$                \\
		$T_{\mathrm{S,fünffach}}$ & $\SI{12.32(5)}{\second}$ & $\SI{12.1(5)}{\second}$  & $1.8\%$                \\
		$\omega_{\mathrm{S,1}}$    & $\SI{0.551(20)}{\Hz}$    & $\SI{0.521(20)}{\Hz}$    & $5.8\%$                \\
		$\omega_{\mathrm{S,2}}$    & $\SI{0.510(21)}{\Hz}$    & $\SI{0.521(20)}{\Hz}$    & $2.1\%$                \\

		\bottomrule
	\end{tabular}
\end{table}

Die größte Abweichung in der Zeitmessung zeigt sich bei der einfach gemessenen Schwebungsdauer. Die Zeit war hier recht kurz, sodass sich ein Fehler verursacht durch verspätete Reaktion beim Starten oder Stoppen der Uhr besonders auswirkt.
Sämtliche weitere Abweichungen liegen deutlich unter der Auflösung der Messung, Fehler durch verzögertes Betätigen der Stoppuhr werden auf bis zu $\SI{0.2}{\second}$ geschätzt (nach \cite{CS-Pro}).
Beim Vergleich der $\omega_{\mathrm{S,1}}$ berechnet nach Formel \eqref{eqn:twtest} und dem $\omega_{\mathrm{S,2}}$ berechnet über die Schwingungsdauer fällt auf, dass die Schwingungsfrequenz berechnet über die Schwingungsdauer deutlich näher an den Theoriewerten liegt.
Der größere Fehler bei dem  $\omega_{\mathrm{S,1}}$ entsteht, da sich dieses über die Differenz der gleich -und gegensinnigen Schwingungsfrequenz bestimmt und sich kleine Fehler somit hier größer auswirken.
