\section{Auswertung}
\label{sec:Auswertung}
Alle im folgenden berechneten Mittelwerte werden mit
\begin{equation}
	\label{eqn:mittelwert}
	\overline x=\frac{1}{N}\sum \limits_{i=1}^{N} x_i
\end{equation}
bestimmt.
Der zugehörige Fehler des Mittelwerts bestimmt sich über
\begin{equation}
	\label{eqn:mittelwertfehler}
	\Delta \overline x= \frac{1}{\sqrt{N}} \sqrt{\frac{1}{N-1} \sum \limits_{i=1}^{N} (x_i- \overline x)^2}\text{.}
\end{equation}
Wenn fehlerbehaftete Größen in einer späteren Formel weiter verwendet werden, so wird der sich fortpflanzende Fehler
mit Hilfe der Gauß’schen Fehlerfortpflanzung berechnet:
\begin{equation}
	\label{eqn:fehlerfortpflanzung}
	\Delta f = \sqrt{ \sum \limits_{i = 1}^{N} (\frac{\partial f}{\partial x_i})^2 \cdot (\Delta x_i)^2}\text{.}
\end{equation}
\FloatBarrier
Für die Berechnung der Theoriewerte werden folgende fehlerbehaftete Größen verwendet:
\begin{gather*}
	l=\SI{0.450(3)}{\centi\meter} \text{,}
	g=\SI{9.81190(4)}{\meter\per\square\second} \text{.}
\end{gather*}
Hierbei ist $l$ die Pendellänge samt Ablesefehler und die Erdanziehung $g$ wird nach \cite{G} für Dortmund angegeben.
\subsection{Messung 1: Pendellänge 0.45}
Um eine kleinere Messunsicherheit zu erhalten, wird jeweils für 5 Schwingungen gemessen. Daher wird der gemessene Wert zu Beginn durch fünf dividiert, um die Dauer für eine Schwingung zu erhalten.
\\In Tabelle \ref{tab:alinksrechts} sind die Periodendauern für die beiden Pendel ohne Kopplungsfeder aufgetragen.
\\Nach Formel \eqref{eqn:mittelwert} ergibt sich der Mittelwert der Periodendauer mit dem Fehler des Mittelwerts nach Formel \eqref{eqn:mittelwertfehler} zu
\begin{gather*}
	T_{\mathrm{mittel, rechts}}=\SI{1.300(5)}{\second}\\
	T_{\mathrm{mittel, links}}=\SI{1.305(6)}{\second}
\end{gather*}
\begin{table}
	\centering
	\caption{Periodendauer der ungekoppelten Pendel.}
	\label{tab:alinksrechts}
	\begin{tabular}{cc}
		\toprule
		$T_{\mathrm{links}}$/ $\si{\second}$ & $T_{\mathrm{rechts}}$/ $\si{\second}$ \\
		\midrule
		1.298                                & 1.282                                 \\
		1.286                                & 1.294                                 \\
		1.300                                & 1.310                                 \\
		1.320                                & 1.312                                 \\
		1.286                                & 1.304                                 \\
		1.286                                & 1.292                                 \\
		1.320                                & 1.282                                 \\
		1.334                                & 1.322                                 \\
		1.326                                & 1.278                                 \\
		1.292                                & 1.320                                 \\
		\bottomrule
	\end{tabular}
\end{table}


\begin{table}
	\centering
	\caption{Periodendauer für gleich-und gegensinnige Schwingung.}
	\label{tab:gleichgegen}
	\begin{tabular}{cc}
		\toprule
		$T_{\mathrm{gleichsinnig}}$/ $\si{\second}$ & $T_{\mathrm{gegensinnig}}$/ $\si{\second}$ \\
		\midrule
		1.336                                       & 1.146                                      \\
		1.312                                       & 1.106                                      \\
		1.328                                       & 1.066                                      \\
		1.316                                       & 1.144                                      \\
		1.334                                       & 1.126                                      \\
		1.328                                       & 1.186                                      \\
		1.328                                       & 1.132                                      \\
		1.320                                       & 1.086                                      \\
		1.322                                       & 1.120                                      \\
		1.326                                       & 1.146                                      \\
		\bottomrule
	\end{tabular}
\end{table}
Ebenso wird die Periodendauer für die gleich-und gegensinnige Schwingung nach Formel \eqref{eqn:mittelwert} mit dem Fehler des Mittelwerts nach Formel \eqref{eqn:mittelwertfehler} berechnet zu
\begin{gather*}
	T_{\mathrm{mittel, gleichsinnig}}=T_{\mathrm{+}}=\SI{1.325(2)}{\second} \text{,}\\
	T_{\mathrm{mittel, gegensinnig}}=T_{\mathrm{-}} =\SI{1.13(1)}{\second} \text{.}
\end{gather*}
Aus den Periodendauern lässt sich die experimentell bestimmte Schwingungsfrequenz jeweils über $\omega=\frac{2\pi}{T}$ bestimmen.
\\Es ergibt sich
\begin{gather*}
	\omega_{\mathrm{+}}= \SI{4.742(9)}{\Hz} \text{,}\\
	\omega_{\mathrm{-}}= \SI{5.58(5)}{\Hz} \text{.}
\end{gather*}
Der Theoriewert für $\omega_{\mathrm{+}}$ ergibt sich nach Formel \eqref{eqn:wgleich} zu
\begin{align*}
	\omega_{\mathrm{+}}= \SI{4.67(2)}{\Hz} \text{.}
\end{align*}
Die theoretischen Werte für $T_{\mathrm{+}}$ und $T_{\mathrm{-}}$ ergeben sich nach Formel \eqref{eqn:tgleich} beziehungsweise \eqref{eqn:tgegen}.
Die theoretische Periodendauer der gleichsinnigen Schwingung ergibt sich direkt zu
\begin{gather*}
	T_{\mathrm{+}}=\SI{1.346(4)}{\second} \text{,}
\end{gather*}
Die Fehler ergeben sich hierbei über die Gaußsche Fehlerfortpflanzung nach Formel \eqref{eqn:fehlerfortpflanzung}.
Da die Kopplungskonstante $K$ der Feder nicht bekannt ist, wird der Theoriewert $T_{\mathrm{-}}$ über den Theoriewert für $\omega_{\mathrm{-}}$ bestimmt.
Nach Formel \eqref{eqn:kappa} lässt sich $\omega_{\mathrm{-}}$ wie folgt berechnen:
\begin{equation}
	\label{eqn:omegaminus}
	\omega_{\mathrm{-}}=\sqrt{\frac{\omega_{\mathrm{+}}^2 \cdot(1+\kappa)}{1-\kappa}} \text{.}
\end{equation}
Für den Kopplungsgrad $\kappa$ ergibt sich mit den experimentell bestimmten Schwingungsdauern für die gleich-und gegensinnige Schwingung nach Formel \eqref{eqn:kappa}
\begin{align*}
	\kappa=\SI{0.161(9)}{} \text{.}
\end{align*}
Damit ergibt sich für
\begin{align*}
	\omega_{\mathrm{-}}=\SI{5.58(5)}{\Hz} \text{.}
\end{align*}
Schließlich lässt sich der Theoriewert für $T_{\mathrm{-}}$ über
\begin{align*}
	T_{\mathrm{-}}=\frac{2\pi}{\omega_{\mathrm{-}}}=\SI{1.13(1)}{\second}
\end{align*}
bestimmen.

In Tabelle \ref{tab:schwebung} finden sich die gemessenen Schwebungsdauern der Messung der gekoppelten Schwingungen.\\
In der linken Spalte sind hierbei die Messwerte für die erste Schwebungsdauer und in der rechten Spalte die bereits auf eine Schwebung umgerechnete Dauer der Messung über fünf Schwebungsdauern aufgetragen.
\begin{table}
	\centering
	\caption{Schwebungsdauer gemessen über eine bzw. fünf Perioden.}
	\label{tab:schwebung}
	\begin{tabular}{cc}
		\toprule
		$T_{\mathrm{Schwebung einfach}}$/ $\si{\second}$ & $T_{\mathrm{Schwebung}}$/ $\si{\second}$ \\
		\midrule
		5.64                                             & 6.814                                    \\
		6.44                                             & 6.984                                    \\
		6.72                                             & 6.846                                    \\
		6.32                                             & 6.872                                    \\
		6.36                                             & 7.098                                    \\
		6.90                                             & 6.900                                    \\
		6.87                                             & 7.106                                    \\
		6.03                                             & 6.900                                    \\
		6.36                                             & 7.080                                    \\
		6.44                                             & 6.972                                    \\
		\bottomrule
	\end{tabular}
\end{table}
\\Der Mittelwert über die einfache Messung ergibt sich zu
\begin{align*}
	T_{\mathrm{S}}=\SI{6.4(1)}{\second} \text{.}
\end{align*}
\\Der Mittelwert samt Fehler wird ebenso wie der Mittelwert der fünffachen Messung erneut nach Formel \eqref{eqn:mittelwert} und Formel \eqref{eqn:mittelwertfehler} bestimmt.
\\Es ergibt sich
\begin{align*}
	T_{\mathrm{S}}=\SI{6.96(3)}{\second}\text{.}
\end{align*}
Der Theoriewert der Schwebungsdauer berechnet sich nach Formel \eqref{eqn:twschwebe} mit den bereits bestimmten Theoriewerten für $T_{\mathrm{+}}$ und $T_{\mathrm{-}}$ zu
\begin{align*}
	T_{\mathrm{S}}=\SI{6.9(4)}{\second}\text{.}
\end{align*}
Sowohl der Theoriewert, als auch der experimentell bestimmte Wert für die Schwebungsfrequenz $\omega_{\mathrm{S}}$ ergibt sich nach Formel \eqref{eqn:twschwebe} mit den bereits bestimmten Werten für $\omega_{\mathrm{+}}$ und $\omega_{\mathrm{-}}$.
\begin{align*}
	\omega_{\mathrm{S,Experiment}}=\SI{0.84(5)}{\Hz}\text{,} \\
	\omega_{\mathrm{S,Theorie}}=\SI{0.91(6)}{\Hz}\text{.}
\end{align*}






\FloatBarrier
\subsection{Messung 2: Pendellänge 0.73}

\begin{table}
	\centering
	\caption{Messwerte für die Periodendauer der ungekoppelten Pendel bei einer Pendellänge von $l=\SI{0.73}{\meter}$.}
	\label{tab:rivaldo}
	\begin{tabular}{cc}
		\toprule
		$T_{\mathrm{links}}$ / $ \si{\second}$ & $T_{\mathrm{rechts}}$ / $\si{\second}$\\
		\midrule
		1.602 & 1.606 \\
		1.59 & 1.612 \\
		1.626 & 1.626 \\
		1.602 & 1.584 \\
		1.592 & 1.606 \\
		1.618 & 1.626 \\
		1.606 & 1.618 \\
		1.592 & 1.602 \\
		1.606 & 1.626 \\
		1.614 & 1.626 \\
		\bottomrule
	\end{tabular}
\end{table}

Es ergeben sich die Werte 
\begin{equation*}
	T_{\mathrm{mittel,links}} = \SI{1.605(4)}{\second}
\end{equation*}
und
\begin{equation*}
	T_{\mathrm{mittel,rechts}} = \SI{1.613(4)}{\second} \mathrm{.}
\end{equation*}

%%%%%%%%%%%%%%%%%%%

\begin{table}
	\centering
	\caption{Messwerte für die Periodendauer der gekoppelten Pendel bei gleich- und
	gegenphasiger Schwinung bei einer Pendellänge von $l=\SI{0.73}{\meter}$.}
	\label{tab:beckham}
	\begin{tabular}{cc}
		\toprule
		$T_{\mathrm{gleichphasig}}$ / $\si{\second}$ & $T_{\mathrm{gegenphasig}}$ 
		/ $\si{\second}$ \\
		\midrule
		1.590 & 1.374 \\
		1.600 & 1.408 \\
		1.594 & 1.418 \\
		1.620 & 1.418 \\
		1.608 & 1.432 \\
		1.626 & 1.426 \\
		1.608 & 1.426 \\
		1.606 & 1.406 \\
		1.624 & 1.406 \\
		1.618 & 1.390 \\
		\bottomrule
	\end{tabular}
\end{table}

Es ergeben sich die Werte 
\begin{equation*}
	T_{\mathrm{mittel,gleichphasig}} = T_+ = \SI{1.609(4)}{\second}
\end{equation*}
und
\begin{equation*}
	T_{\mathrm{mittel,gegenphasig}} = T_- = \SI{1.410(6)}{\second} \mathrm{.}
\end{equation*}

Weiterhin werden die Schwingungsfrequenz mit $\omega = \frac{2\pi}{T}$ zu
\begin{equation*}
	\omega_+ = \SI{3.904(10)}{\hertz}
\end{equation*}
und
\begin{equation*}
	\omega_- = \SI{4.455(18)}{\hertz} \mathrm{.}
\end{equation*}
Der Theoriewert für $\omega_+$ ergibt sich nach Formel \eqref{eqn:wgleich} zu
\begin{equation*}
	\omega_+ = \SI{3.692(8)}{\hertz} \mathrm{.}
\end{equation*}


\begin{table}
	\centering
	\caption{Messwerte für die Schwebungsdauer bei einer Pendellänge von $l=\SI{0.73}{\meter}$.}
	\label{tab:kalou}
	\begin{tabular}{cc}
		\toprule
		$T_{\mathrm{Schwebung,einfach}}$ / $\si{\second}$ & $T_{\mathrm{Schwebung}}$ / $\si{\second}$ \\
		\midrule
		12.46 & 12.192 \\
		12.06 & 12.700 \\
		11.86 & 12.426 \\
		11.58 & 12.276 \\
		11.72 & 12.258 \\
		11.68 & 12.180 \\
		11.83 & 12.402 \\
		11.66 & 12.374 \\
		11.73 & 12.192 \\
		11.65 & 12.230 \\
		\bottomrule
	\end{tabular}
\end{table}
