\section{Durchführung}
\FloatBarrier
\label{sec:Durchführung}
Zuerst werden die Massen der beiden Pendel auf die gleiche Höhe eingestellt, welche mit 
einem Maßband gemessen wird. Die Masse der Gewichte beträgt nach \cite{Anleitung}
$m = \SI{1}{\kilo\gram}$. 

Es werden zunächst die Periodendauern der Pendel ohne Kopplung gemessen.
Hierzu werden die Pendel ausgelenkt und eine Stoppuhr gestartet.
Nach fünf Perioden wird die Zeit gestoppt und notiert.
Für die beiden Pendel sollen jeweils zehn Zeiten gemessen werden.

Für die Messung der Schwingungsdauer $T_+$ der gleichsinnigen Schwingung werden die beiden 
Pendel durch eine Feder gekoppelt. Die beiden Pendel werden unter dem gleichen 
Winkel -- wie in Abbildung \ref{fig:gleich} -- ausgelenkt. Analog 
wie bei der Bestimmung der ungekoppelten Schwingungen werden zehn Zeiten für fünf
Perioden der gleichsinnigen Schwingung notiert.

Weiterhin werden die Pendel für die Bestimmung der Schwingungsdauer $T_-$ der gegensinnigen 
Schwingung -- wie in Abbildung \ref{fig:gegen} -- gegengleich ausgelenkt und wieder zehn 
Messwerte für fünf Periodendauern ermittelt.

Zuletzt wird die Messung für die Schwebung durchgeführt, indem beide Pendel unter 
unterschiedlichen Winkeln ausgelenkt werden (vgl. Abbildung \ref{fig:schwebe}).
Hier wird zuerst die Zeit für eine Schwebungsdauer -- also einen Stillstand der 
Pendelbewegung -- bestimmt und weiterhin die Zeit für fünf Perioden -- also 
fünf Stillstände.
