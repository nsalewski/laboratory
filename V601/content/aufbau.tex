\subsection{Versuchsaufbau}
\label{sec:Versuchsaufbau}
Der vorliegende Versuchsaufbau besteht aus einem mit Hg-Dampf gefüllten Glasrohr.
Dieses befindet sich in einer beheizbaren Blechkammer. Die Blechkammer kann mittels eines Heizgenerators geheizt werden und ihre Innentemperatur kann mittels eines elektrischen Thermometers gemessen werden.
In dem Glasrohr befindet sich unten ein Heizdraht mit niedriger Elektronenaustrittsarbeit. Dieser wird mittels eines Kontaktspannungsgerät mit konstanter Heizleistung beheizt. Mittig der Glasröhre befindet sich eine Beschleunigungsamplitude und am oberen Ende der Glasröhre eine Auffängerelektrode.
Die beiden Elektroden werden über gesteuerte Gleichspannungsquellen versorgt. Diese sind so konfiguriert, dass sich ihre Ausgangspannung proportional zur Zeit ändern kann.
Der Auffängerstrom kann mittels eines Pikoamperemeters gemessen werden.
Das Pikoamperemeter ist als Gleichstromverstärker realisiert und sein Ausschlag ist proportional zur anliegenden Eingangstrom. Das Ausgangssignal des Pikoamperemeters wird daher verwendet, um den Auffängerstrom $I_\mathrm{A}$ auf dem XY-Schreiber aufzugeben.
Mithilfe eines XY-Schreibers können die gemessenen Ströme und Spannungen und ihre Abhängigkeiten zueinander graphisch dargestellt werden.
