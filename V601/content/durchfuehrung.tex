\section{Durchführung}
\label{sec:Durchführung}

\subsection{Versuchsaufbau}
\label{sec:Versuchsaufbau}
Der Versuchsaufbau besteht aus einem Ultraschallechoskop, an dessen Ausgänge zwei Ultraschallsonden mit \SI{2}{\mega\Hz} gekoppelt sind, und einem Rechner zur Datenaufnahme und -analyse.\\
An das Ultraschallechoskop sind zwei Ultraschallsonden angeschlossen, mithilfe derer sich sowohl eine Impuls-Echo-Messung, als auch eine Durchschallmessung realisieren lässt.
Am Rechner werden die gemessenen Daten mittels des Programms \textquote{Echoview} ausgewertet.\\
Hierbei ist \textquote{Echoview} in der Lage, vier verschiedene Diagramme darzustellen.
Im linken oberen Graphen wird der A-Scan dargestellt, also die Amplitude gegen die Zeit aufgetragen.
Der linke untere Graph stellt die gewählte Verstärkung dar. Die Verstärkung lässt sich am Ultraschallechoskop über die Drehknöpfe zur laufzeit-bzw. tiefenabhängigen Verstärkung (TGC; Time Gain Control) und ebenso über die Verstärkung des Outputs und der Empfindlichkeit der Sonden regulieren.\\
Zu Beachten ist, dass eine Verstärkung nur gewählt werden darf, wenn die auszuwertende Messreihe nicht zur Untersuchung der Amplitudenhöhe dient.
Die beiden rechten Graphiken sind das berechnete Spektrum der Messdaten (FFT), bzw. ihr
Cepstrum.
Erzeugte Graphiken und Messdaten können aus dem Programm heraus exportiert werden.
Als zu untersuchende Versuchsobjekte stehen Acrylzylinder verschiedener Länge, Acrylplatten unterschiedlicher Dicke sowie das Modell eines menschlichen Auges im Maßstab 3:1 zur Verfügung.


\subsection{Versuchsbeschreibung}
\label{sec:Versuchsbeschreibung}
Vor Beginn jeder Messung muss der XY-Schreiber geeignet kalibriert werden.
Das verwendete Millimeterpapier lässt sich hierbei elektrostatisch auf dem XY-Schreiber fixieren.
Bevor ein Signal auf die Eingänge des XY-Schreibers gegeben wird, muss der Nullpunkt der Messung festgelegt werden. Dazu wird mit den beiden "Zero"-Knöpfen der Messskalen der Schreibkopf mit etwas Abstand zum Rand in der linken, unteren Ecke des Millimeterpapiers platziert.
Zur Justierung der Empfindlichkeit der beiden Eingänge wird der Versuch ohne Aufzeichnung einer Messkurve durchgeführt. Dafür werden die zu messenden Signale auf die jeweiligen Eingänge des XY-Schreibers gegeben.
Die Empfindlichkeit beider Eingänge wird so geregelt, dass die jeweiligen maximalen und minimalen Werte mit etwas Abstand zum Rand liegen.
Für jede Messung können nun Messkurven aufgezeichnet werden. Hierfür wird die Schutzhülle vom Schreibkopf des XY-Schreibers entfernt, und der Schreibkopf vorsichtig auf das Millimeterpapier gesetzt.\\
Nach dem Abschluss jeder Messreihe ist es notwendig, die x-Achse zu skalieren. Hierfür wird das auf die Y-Achse aufgegebene Signal entfernt und die Messung wird bezüglich des Signals auf dem X-Eingang wiederholt. Der Schreibkopf wird hierfür nicht auf das Millimeterpapier gesetzt.
In regelmäßigen Abständen der gemessenen Spannung am Voltmeter werden Zwischenwerte auf der X-Achse markiert und die zugehörigen Spannungen notiert.

Zur Messung der integralen Energieverteilung der beschleunigten Elektronen wird der Auffängerstrom $I_\mathrm{A}$ in Abhängigkeit der Bremsspannnung $U_\mathrm{A}$ bei konstanter Beschleunigungsspannung $U_\mathrm{B}=\SI{11}{\volt}$ gemessen. Der Auffängerstrom wird hierfür auf den Y-Eingang des XY-Schreibers aufgegeben und die Bremsspannung auf den X-Eingang. Die Bremsspannung wird über die anliegende gesteuerte Gleichspannungsquelle hierbei von $\SI{0}{\volt}$ bis $\SI{11}{\volt}$ variiert.
Die Messung wird einmal bei Zimmertemperatur und einmal bei etwa $150^\circ C$ durchgeführt.
Während der Messung soll die Temperatur im Versuchsaufbau möglich konstant gehalten werden. Dafür wird die Temperatur permanent am elektronischen Thermometer abgelesen und der Heizgenerator entsprechend nachgeregelt.

Zur Aufnahme der Franck-Hertz-Kurve wird die Temperatur im Versuchsaufbau auf den maximal möglichen Wert des vorliegenden Versuchaufbaus von etwa $180^\circ C$ erhöht. Die Bremsspannung wird hierfür permanent auf $\SI{1}{\volt}$ gestellt und die Beschleunigungsspannung von $\SI{0}{\volt}$ bis $\SI{60}{\volt}$ über die gesteuerte Gleichspannungsquelle variiert.
Die variable Beschleunigungsspannung wird auf den X-Eingang und mittels des Pikoamperemeters eine Spannung proportional zum Auffängerstrom $I_\mathrm{A}$ auf den Y-Eingang des XY-Schreibers aufgegeben.

Zur Bestimmung der Ionisatiosnspannung von Quecksilber wird die Innentemperatur des Aufbaus auf etwa $100^\circ C$ geregelt, der Auffängerstrom wird hierbei auf den Y-Eingang gegeben.
An den X-Eingang wird die Beschleunigungsspannung $U_\mathrm{B}$ aufgegeben. Die Bremsspannung ist hierbei konstant $U_\mathrm{A}=\SI{-30}{\volt}$.
