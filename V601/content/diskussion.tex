\section{Diskussion}
\label{sec:Diskussion}
Bei der ersten Messung zur differentiellen Energieverteilung war prinzipiell aufgrund der großenn freien Weglänge $\bar{w}$ erwartet worden, dass sich ein recht scharfes Maximum der differentiellen Energieverteilung bei $U_\mathrm{A}=U_\mathrm{B}-K$ ausbildet. Stattdessen zeigt sich bereits zu Beginn der Messung ein großes Maximum und erneut eines ungefähr bei der erwarteten Gegenspannung. Dies deutet darauf hin, dass bereits zu Beginn der Messung die Energie vieler Elektronen nicht ausreicht, um gegen die Gegenspannung der Auffängerelektrode anzulaufen. Dies würde prinzipiell bedeuten, dass die Beschleunigungsspannung effektiv nicht dem eingestellten Wert von $U_\mathrm{B}=\SI{11}{\volt}$ entsprach, oder eventuell auch schwankte.
Selbiges ließe sich für die Bremsspannung $U_\mathrm{A}$ vermuten. Allerdings würde sich dann wiederum das zweite Maximum bei $U_\mathrm{A}=\SI{7.5}{\volt}$ schwerlich erklären lassen.
Allgemein ließe sich eine Unschärfe im Maximum der differentiellen Energieverteilung mit der Fermi-Dirac-Statistik erklären. Danach hätten die ausgelösten Elektronen aber eine größere, als durch die Beschleunigungsspannung $U_\mathrm{B}$ induzierte Energie. Wodurch der starke Abfall auch bei niedrigen Bremsspannungen verursacht wird, lässt sich also nicht eindeutitg klären.

Bei der zweiten Messung zur differentiellen Energieverteilung (vgl. Blatt 2 im Anhang), zeigen sich ähnliche Ungenauigkeiten in der Messung. Erwartet worden wäre etwa eine konstante differentielle Energieverteilung bis $U_\mathrm{A}=\frac{\Delta E}{\symup{e}_0}-K$, und für größere Bremsspannungen wären keine auftreffenden Elektronen an der Auffangelektrode mehr erwartet worden.
Tendenziell war aufgrund der höheren Stoßwahrscheinlichkeit der Elektronen mit den Quecksilberatomen aufgrund der kürzeren freien Weglänge $\bar{w}$
eine stetig geringer werdende Anzahl an auftreffenden Elektronen an der Auffängerelektrode erwartet worden
Die kürzere freie Weglänge wird hierbei verursacht durch den höheren Dampfdruck, welcher wiederum verursacht wird durch die höhere Temperatur.
Da keine Raumrichtung ausgezeichnet ist, wurde eine geringere Geschwindigkeit in Richtung der Auffängerelektrode und da die Bremsspannung kontinuierlich erhöht wurde, ein stetig sinkender Auffängerstrom erwartet.
Es zeigt sich recht grob der erwartete Stufenverlauf.
Mit dem abgelesenen Kontaktpotential $K_1=\SI{3.5}{\volt}$ aus der ersten Messung zur differentiellen Energieverteilung und dem $U_1=\frac{\Delta E}{\symup{e}_0=\SI{5.3(1)}{\volt}}$ aus der Messreihe der Franck-Hertz-Kurve müsste sich die Stufe bei $U_\mathrm{A}=\SI{1.8(1)}{\volt}$ ausbilden.
Dies passt relativ gut mit der Stufe in Abbildung \ref{fig:aufgabeA2} bei etwa $U_\mathrm{A}=\SI{1.9}{\volt}$ zusammen.
Auffallend ist dennoch, dass auch diese Messung deutlich vom erwarteten Verlauf der Graphen abweicht.

Das mittels der Franck-Hertz-Kurve bestimmte Kontaktpotential weicht deutlich vom mittels der differentiellen Energieverteilung bestimmten Kontaktpotential ab.
Ein Vergleich beider Kontaktpotentiale ist aber prinzipiell auch schwierig, da sie bei verschiedenen Temperaturen bestimmt wurden.
Die Ionisationsspannung wird mit beiden bestimmten Kontaktpotentialen
\begin{gather*}
K_{\mathrm{diff}}=\SI{3.5}{\volt}\text{,}\\
K_\mathrm{F.-H.}=\SI{0.95(10)}{\volt}\text{,}
\end{gather*}
berechnet.
Es ergibt sich:
\begin{gather*}
U_\mathrm{ion,diff}=\SI{22.5}{\volt}\text{,}\\
U_\mathrm{ion,F.-H.}=\SI{25.05}{\volt}\text{,}
\end{gather*}
Beide Werte weichen hierbei deutlich vom Literaturwert $U_\mathrm{ion,Lit.}=\SI{10.4375}{\volt}$ \cite{ion} ab, und zwar um $141\%$ beziehungsweise um $116\%$.
Die Ungenauigkeit liegt begründet in der schwierigen Bestimmung der Tangenten mit der größten Steigung in der Messung der Ionisationsspannung (vgl. Anhang Blatt 4).
Ein Vergleich der experimentell bestimmten ersten Anregungsenergie des Hg-Atoms $\Delta E_{1\mathrm{,ex.}}=\SI{5.3(1)}{\electronvolt}$zeigt nur eine geringe Abweichung zum Literaturwert $\Delta E_{1 \mathrm{,theo}}=\SI{4.9}{\electronvolt}$\cite{gott} von etwa $8\%$.
