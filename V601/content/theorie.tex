\section{Theorie}
\label{sec:Theorie}

Der Franck-Hertz-Versuch belegt die Aussage von Niels Bohrs Atommodell, dass Atome diskrete 
Energieniveaus besitzen. 
Dies bedeutet, dass Atome -- in diesem Fall Quecksilber -- durch diskrete Anregungsenergien 
in einen höherenergetischen Zustand versetzt werden können und bei der Rückkehr in den 
Anfangszustand einen Lichtquanten mit einer Energie emittieren, die der Energiedifferenz zwischen
Anfangszustand und angeregtem Zustand, also
\begin{equation}
	\symup{h} \nu = E_1 - E_0
\end{equation}
entspricht. Hierbei entspricht $\symup{h}$ dem Planck'schen Wirkungsquantum und $\nu$ der
Frequenz des emittierten Lichtquants.
Die Anregung der Atome kann durch zwei unterschiedliche Methoden realisiert werden.
Zum Einen durch Wechselwirkung elektromagnetischer Strahlung mit den Atomen.
Zum Anderen mit den hier betrachteten Stoßprozessen von Elektronen mit den Atomen, den 
sogenannten \textbf{Elektronenstoßexperimenten}.
Um dieses Phänomen näher zu behandeln wird zunächst der prinzipielle Aufbau des 
Franck-Hertz-Versuchs erläutert.
\subsection{Prinzipieller Aufbau des Franck-Hertz-Versuchs}
%Bild Aufbau
Der prinzipielle Aufbau des Franck-Hertz-Versuchs ist in Abbildung \ref{fig:franckhertztheory}
dargestellt.
In einem evakuierten Gefäß befinden sich ein Glühdraht, eine Beschleunigungselektrode, eine
Auffängerelektrode und die Quecksilberatome. 
Die Dichte der Quecksilberatome ist gemäß der Dampfdruck-Kurve temperaturabhängig.
Der Glühdraht bestehend aus einem Material mit niedriger Austrittsarbeit wird durch eine 
Gleichspannung erhitzt und es treten gemäß dem glühelektrischen Effekt Elektronen 
aus dem Draht aus.
Da zwischen dem Glühdraht und der Beschleunigungselektrode eine positive Spannung anliegt,
erhalten die Elektronen -- vorrausgesetzt sie besitzen unmittelbar nach dem Austritt aus dem
Draht keine kinetische Energie -- bis zur Beschleunigungselektrode die kinetische Energie 
\begin{equation}
	E_{\mathrm{kin}} = \frac{\symup{m}_0 v_{\mathrm{vor}}^2}{2} = \symup{e}_0 U_{\mathrm{B}} \mathrm{,}
\end{equation}
wobei $\symup{m}_0$ der Elektronenmasse und $\symup{e}_0$ der Elementarladung entspricht.

\cite{Anleitung}
