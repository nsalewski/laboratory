\section{Auswertung}
\label{sec:Auswertung}
Es wurden Bauteile mit folgenden Größen verwendet:
\begin{equation}
	\label{eqn:bauteile}
	\begin{aligned}
		M_{\mathrm{Kugel}}                       & = & (512.2 \pm 0.2) \,\si{\gram}\text{,}                       \\
		D_{\mathrm{Kugel}} = 2R_{\mathrm{Kugel}} & = & (50.760\pm 0.004)\,\si{\milli\meter} \text{,}              \\
		\Theta_{\mathrm{Halterung}}              & = & 22.5 \cdot 10^{-7} \, \si{\kilo\gram\square\metre}\text{,} \\
		L_{\mathrm{Draht}}                       & = & 585 \,\si{\milli\meter} \text{.}
	\end{aligned}
\end{equation}
\FloatBarrier

\subsection{Bestimmung des Schubmoduls $G$}
Zur Bestimmung des Schubmoduls $G$ nach Gleichung \eqref{eqn:schubischu} ist zu beachten,
dass das gesamte Trägheitsmoment $\Theta_{\mathrm{Gesamt}}$ des Drehschwingers sich zusammensetzt aus
$\Theta_{\mathrm{Halterung}}$ und $\Theta_{\mathrm{Kugel}}$.
Aus Gleichung \eqref{eqn:toughtimes} folgt
\begin{equation*}
	\label{eqn:dattD}
	D=\left(\frac{2\pi}{T}\right)^2 \cdot \Theta_{\mathrm{Gesamt}} \mathrm{.}
\end{equation*}
Mit Gleichung \eqref{eqn:richti} ergibt sich für $G$ der Ausdruck
\begin{equation*}
	\label{eqn:richtigerG}
	G=\left(\frac{2\pi}{T}\right)^2 \cdot \Theta_{\mathrm{Gesamt}}\cdot \frac{2L}{\pi\cdot R^4} \mathrm{.}
\end{equation*}
\subsubsection{Bestimmung des Trägheitsmoment}
Zur Bestimmung des Schubmoduls $G$ wird zunächst $\Theta_{\mathrm{Kugel}}$ nach Formel \eqref{eqn:trägee} bestimmt.
Mit den Kenngrößen des Drehschwingers (vgl. \refeq{eqn:bauteile}) und dem Fehler nach Gaußscher Fehlerfortpflanzung mit Formel \eqref{eqn:fehlerfortpflanzung} ergibt sich
\begin{equation*}
	\Theta_{\mathrm{Kugel}}=(1.3197\pm 0.0006) \cdot 10^{-4} \,\si{\kilo\gram \square\metre} \text{.}
\end{equation*}
Damit ergibt sich $\Theta_{\mathrm{Gesamt}}$ zu:
\begin{equation*}
	\label{eqn:trägemasse}
	\Theta_{\mathrm{Gesamt}}= \Theta_{\mathrm{Kugel}}+\Theta_{\mathrm{Halterung}}= (1.3422\pm 0.0006)
	\cdot 10^{-4} \,\si{\kilo\gram \square\metre} \text{.}
\end{equation*}
\subsubsection{Bestimmung des Drahtradius}

\begin{table}
	\centering

	\caption{Durchmesser und Radius an 6 verschiedenen Messpunkten des Drahtes.}
	\label{tab:niceslabel}
	\begin{tabular}{ccc}
		\toprule
		Messpunkt & $D_{\mathrm{Draht}}$ / $\si{\micro\meter}$ & $R_{\mathrm{Draht}}$ / $\si{\micro\meter}$ \\
		\midrule
		1         & 180                                        & 90.0                                       \\
		2         & 177                                        & 88.5                                       \\
		3         & 171                                        & 85.5                                       \\
		4         & 170                                        & 85.0                                       \\
		5         & 168                                        & 84.0                                       \\
		6         & 166                                        & 83.0                                       \\
		\bottomrule
	\end{tabular}
\end{table}
Zudem wird zur Bestimmung des Schubmoduls $G$ der Radius des Drahtes benötigt.
In Tabelle \ref{tab:niceslabel} sind die gemessenen Durchmesser des Drahtes an sechs verschiedenen Stellen sowie der Radius $R_{\mathrm{Draht}}=0.5 \cdot D_{\mathrm{Draht}}$ eingetragen.
Nach Formel \eqref{eqn:mittelwert} ergibt sich der Mittelwert sowie nach Formel \eqref{eqn:mittelwertfehler} der Fehler des Mittelwerts des Drahtradius zu:
\begin{equation*}
	R_{\mathrm{Draht}}= (86.0 \pm 1.1)\,\si{\micro\meter} \text{.}
\end{equation*}
\FloatBarrier

\subsubsection{Bestimmung der Schwingungsdauer ohne äußeres Magnetfeld}
\begin{table}
	\centering
	\caption{Schwingungsdauer $T$ ohne äußeres Magnetfeld}
	\label{tab:schwingung_ohne}
	\begin{tabular}{cc}

		\toprule
		Messpunkt & Schwingungsdauer $T$/$\si{\second}$ \\
		\midrule
		1         & 20.032                              \\
		2         & 20.039                              \\
		3         & 20.044                              \\
		4         & 20.033                              \\
		5         & 20.032                              \\
		6         & 20.033                              \\
		7         & 20.034                              \\
		8         & 20.037                              \\
		9         & 20.03                               \\
		10        & 20.048                              \\
		\bottomrule
	\end{tabular}
\end{table}
In Tabelle \ref{tab:schwingung_ohne} sind die Schwingungsdauern $T$ für 10 Perioden für den Drehschwinger ohne äußeres Magnetfeld dargestellt.
Der Mittelwert der Schwingungsdauer gibt sich nach Formel \eqref{eqn:mittelwert} samt dem Fehler des Mittelwerts nach Formel \eqref{eqn:mittelwertfehler} zu:
\begin{equation*}
	T= (20.0362 \pm0.0018) \,\si{\second} \text{.}
\end{equation*}


Somit ergibt sich der Schubmodul $G$ mit $\Theta_{\mathrm{Gesamt}}$, $R_{\mathrm{Draht}}$ und $T$ zu:
\begin{equation*}
	G= (90 \pm 5) \,\si{\giga\pascal} \text{.}
\end{equation*}
\FloatBarrier

\subsection{Berechnung der weiteren elastischen Materialkonstanten}
Da die Messung zur Bestimmung des Elastizitätsmodul $E$ entfiel, wird dieses angegeben mit:
\begin{equation*}
	E= (210.0 \pm 0.5 )\,\si{\giga\pascal} \text{.}
\end{equation*}
Nach Formel \eqref{eqn:zusammenhang} ergeben sich nun der Kompressionsmodul $Q$ und die Querkontraktionszahl $\mu$ zu:
\begin{align*}
	\mu  = \frac{E}{2G}-1       & = 0.168\pm 0.003 \mathrm{,}                     \\
	Q    = \frac{E}{3(1-2 \mu)} & = (105.6\pm 1.1) \,\si{\giga\pascal} \mathrm{.}
\end{align*}
Die Fehler wurden hierfür mit der Gaußschen Fehlerfortpflanzung nach Gleichung \eqref{eqn:fehlerfortpflanzung} bestimmt.
\subsection{Bestimmung des magnetischen Moments $m$ des Helmholtzspulenpaars}
Durch Umformen von Gleichung \eqref{eqn:periodemagnet} ergibt sich
\begin{equation}
	\label{eqn:linearB}
	B=\frac{1}{m}\cdot \left(\left(\frac{2\pi}{T_\mathrm{m}}\right)^2 \cdot \Theta_{\mathrm{Gesamt}}-D\right) \text{.}
\end{equation}
beziehungsweise
\begin{equation*}
	B=a\cdot\frac{1}{T_{\mathrm{m}}^2}+b \text{  nach Gleichung \eqref{eqn:ausgleichsgrade}.}
\end{equation*}
Es lässt sich somit ein linearer Zusammenhang zwischen der magnetischen Flussdichte $B$ und $\left(\frac{1}{T_{\mathrm{m}}}\right)^2$, also dem reziproken Quadrat der Schwingungsdauer mit den Regressionsparametern $a=\frac{1}{m}\cdot 4\pi^2 \cdot \Theta_{\mathrm{Gesamt}}$ sowie $b=-\frac{D}{m}$ nach Gleichung \eqref{eqn:ausgleichsgrade} feststellen.
Aus diesem linearen Zusammenhang wird nun das magnetische Moment $m$ bestimmt.
Das B-Feld eines Helmholtzspulenpaars ergibt sich nach \cite{LordHelmchen} zu
\begin{equation}
	\label{eqn:MisterMagneto}
	B_{\mathrm{H}}= \frac{8}{\sqrt{125}}\cdot \mu_{\mathrm{0}} \frac{I\cdot N}{R} \text{.}
\end{equation}
In Tabelle \ref{tab:vielvieldaten} befinden sich die fünf Periodendauern $T_{\mathrm{i}}$ zu jeder Stromstärke der Messung mit eingeschalteten Helmholtz-Spulen.
Zudem wurde für jede Stromstärke die mittlere Schwingungsdauer sowie der zugehörige Fehler nach Formel \eqref{eqn:mittelwert} bzw. \eqref{eqn:mittelwertfehler} bestimmt.
Außerdem wird jeweils die magnetische Flussdichte $B$ angegeben. Der zugehörige Fehler wurden über die Gaußsche Fehlerfortpflanzung nach Formel \eqref{eqn:fehlerfortpflanzung} bestimmt.

\begin{table}
	\centering
	\caption{Messdaten zur Bestimmung des magnetischen Moments}
	\label{tab:vielvieldaten}
	\begin{tabular}{ccccccccc}
		\toprule
		$I$ / $\si{\ampere}$ & $T_{\mathrm{1}}$ /$\si{\second}$ & $T_{\mathrm{2}}$ /$\si{\second}$ & $T_{\mathrm{3}}$ /$\si{\second}$ & $T_{\mathrm{4}}$ /$\si{\second}$ & $T_{\mathrm{5}} $/$\si{\second}$ & $T_{\mathrm{mittel}}$/ $\si{\second}$ & $B$/$\si{\milli\tesla}$ \\
		\midrule
		0.1                  & 14.154                           & 14.137                           & 14.107                           & 14.085                           & 14.052                           & (14.107 \pm 0.018)                    & 0.4496                  \\
		0.2                  & 12.143                           & 12.106                           & 12.080                           & 12.045                           & 12.013                           & (12.077 \pm 0.023)                    & 0.8992                  \\
		0.3                  & 10.215                           & 10.186                           & 10.156                           & 10.131                           & 10.105                           & (10.159 \pm 0.019)                    & 1.3488                  \\
		0.4                  & 8.893                            & 8.882                            & 8.856                            & 8.845                            & 8.821                            & (8.859\pm 0.013)                      & 1.7984                  \\
		0.5                  & 8.022                            & 7.997                            & 7.998                            & 7.972                            & 7.972                            & (7.992\pm 0.009)                      & 2.2480                  \\
		0.6                  & 7.359                            & 7.344                            & 7.327                            & 7.333                            & 7.341                            & (7.341\pm 0.005)                      & 2.6980                  \\
		0.7                  & 6.839                            & 6.840                            & 6.838                            & 6.834                            & 6.829                            & (6.836\pm0.002)                       & 3.1471                  \\
		0.8                  & 6.389                            & 6.406                            & 6.396                            & 6.403                            & 6.409                            & (6.401 \pm0.004)                      & 3.5967                  \\
		0.9                  & 6.159                            & 6.067                            & 6.030                            & 6.009                            & 6.030                            & (6.059 \pm0.027)                      & 4.0463                  \\
		1.0                  & 5.861                            & 5.820                            & 5.698                            & 5.619                            & 5.656                            & (5.730\pm0.050)                       & 4.4959                  \\
		\bottomrule
	\end{tabular}
\end{table}
Eine lineare Regression mit python/matplotlib \cite{matplotlib} für Gleichung \eqref{eqn:linearB} nach Formel \eqref{eqn:ausgleichsgrade} liefert:
\begin{align*}
	m & = & (0.0339 \pm 0.0004) \,\si{\ampere\square\meter} \text{,}                              \\
	D & = & (7.3 \pm 1.3) \cdot 10^{-6} \,\si{\kilo\gram\square\meter\per\square\second} \text{.}
\end{align*}
Die zugehörigen Messdaten samt der Regressionsgraden findet sich in Abbildung \ref{fig:magnetischesMoment}.
\begin{figure}
	\centering
	\includegraphics{Bilder/b.pdf}
	\caption{Plot.}
	\label{fig:magnetischesMoment}
\end{figure}
Das magnetische Moment ergibt sich also zu $m=(0.0339 \pm0.0004) \,\si{\ampere\square\meter}$.




%%%%%%%%%%%%%%%%%%%%%%%%%%%%%%%%%%%%%%%%%%%%%%%%%%%%%%%%%%%%%%%%%%%%%%%%%%%%%%%%%%%%%%%%%%%
\FloatBarrier
\subsection{Bestimmung der Horizontalkomponente des Erdmagnetfeldes}

Die Messwerte der Periodendauer $T$ der Drehschwingung zur Bestimmung der Horizontalkomponente
des Erdmagnetfeldes sind in Tabelle \ref{tab:magnetusmaximus} aufgetragen.
\begin{table}
	\caption{Messwerte der Periodendauer $T$ zur Bestimmung der Horizontalkomponente
	des Magnetfeldes.}
	\label{tab:magnetusmaximus}
	\centering
	\begin{tabular}{cc}
		\toprule
		Nummer & $T$/$\si{\second}$ \\
		\midrule
		1      & 19.957             \\
		2      & 19.948             \\
		3      & 19.949             \\
		4      & 19.940             \\
		5      & 19.944             \\
		6      & 19.944             \\
		7      & 19.936             \\
		8      & 19.933             \\
		9      & 19.937             \\
		10     & 19.930             \\
		\bottomrule
	\end{tabular}
\end{table}
Damit ergibt sich die Periodendauer $T$ als Mittelwert (bestimmt mit Formel
\eqref{eqn:mittelwert} dieser Messwerte mit zugehörigem Fehler (bestimmt mit Formel
\eqref{eqn:mittelwertfehler}) zu
\begin{equation*}
	T = (19,942 \pm 0,003) \, \si{\second} \mathrm{.}
\end{equation*}

Die Horizontalkomponente des Erdmagnetfeldes ergibt sich durch Formel \eqref{eqn:periodemagnet}
und der Richtgröße eines Zylinders $D$ (Formel \eqref{eqn:richti}) zu
\begin{equation}
	B = 4\pi^2\frac{\Theta_{\mathrm{Gesamt}}}{m T^2} - \frac{\pi G R^4}{2Lm} \mathrm{.}
\end{equation}
Damit erhält man mit Gaußscher Fehlerfortpflanzung (Formel \eqref{eqn:fehlerfortpflanzung}) die
Horizontalkomponente des Erdmagnetfeldes
\begin{equation*}
	B = (3 \pm 29) \, \si{\micro\tesla} \mathrm{.}
\end{equation*}
