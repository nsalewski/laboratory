\subsection{Versuchsaufbau}
\label{sec:Versuchsaufbau}
Der Versuchsaufbau besteht aus einem Ultraschallechoskop, an dessen Ausgänge zwei Ultraschallsonden mit \SI{2}{\mega\Hz} gekoppelt sind, und einem Rechner zur Datenaufnahme und -analyse.\\
An das Ultraschallechoskop sind zwei Ultraschallsonden angeschlossen, mithilfe derer sich sowohl eine Impuls-Echo-Messung, als auch eine Durchschallmessung realisieren lässt.
Am Rechner werden die gemessenen Daten mittels des Programms \textquote{Echoview} ausgewertet.\\
Hierbei ist \textquote{Echoview} in der Lage, vier verschiedene Diagramme darzustellen.
Im linken oberen Graphen wird der A-Scan dargestellt, also die Amplitude gegen die Zeit aufgetragen.
Der linke untere Graph stellt die gewählte Verstärkung dar. Die Verstärkung lässt sich am Ultraschallechoskop über die Drehknöpfe zur laufzeit-bzw. tiefenabhängigen Verstärkung (TGC; Time Gain Control) und ebenso über die Verstärkung des Outputs und der Empfindlichkeit der Sonden regulieren.\\
Zu Beachten ist, dass eine Verstärkung nur gewählt werden darf, wenn die auszuwertende Messreihe nicht zur Untersuchung der Amplitudenhöhe dient.
Die beiden rechten Graphiken sind das berechnete Spektrum der Messdaten (FFT), bzw. ihr
Cepstrum.
Erzeugte Graphiken und Messdaten können aus dem Programm heraus exportiert werden.
Als zu untersuchende Versuchsobjekte stehen Acrylzylinder verschiedener Länge, Acrylplatten unterschiedlicher Dicke sowie das Modell eines menschlichen Auges im Maßstab 3:1 zur Verfügung.

\section{Durchführung}
\label{sec:Durchführung}
\subsection{Bestimmung der Periodendauer zur Berechnung der elastischen Konstanten}
Zur Messung der Periodendauer zur Berechnung des Schubmoduls $G$ wird der Magnet in der Kugelmasse vertikal ausgerichtet. Dadurch steht der Magnet senkrecht auf der horizontalen Komponente des Erdmagnetfelds und wird durch diese somit nicht beeinflusst.
Über eine Vierteldrehung am Justierrad (vgl. \ref{fig:pendel}) wird das System zum Schwingen angeregt.
Falls das System deutliche Pendelbewegungen ausführt, können diese über ein Bremskissen unter der Kugelmasse gedämpft werden, sodass das System möglicht nur noch Drehschwingungen ausführt.
Die Schwingungsdauer $T$ wird zehn mal gemessen.
\subsection{Magnetisches Moment eines Permanentmagneten}
Zur Bestimmung des magnetischen Moments des Helmholtz-Spulenpaars wird der Magnet in der Kugelmasse parallel zu den Feldlinien des Spulenpaars ausgerichtet.
Der Drehschwinger wird erneut über das Justierrad um einen kleinen Auslenkungswinkel ausgelenkt (vgl. Formel \eqref{eqn:jkadsfl}. Diese gilt nur für kleine Auslenkungswinkel).
Die Stromstärke für das Helmholtzspulenpaar wird von $A=0.1\,\si{\ampere}$ in $0.1\,\si{\ampere}$-Schritten bis $1\,\si{\ampere}$ hochgeregelt.
Für jede Stromstärke wird die Periodendauer $T$ fünfmal gemessen.
\subsection{Messung der Horizontalkomponente des Erdmagnetfelds}
Zur Bestimmung der Horizontalkomponente des Erdmagnetfelds wird der Magnet in Nord-Süd-Richtung, also parallel zu den Feldlinien des Erdmagnetfelds, ausgerichtet.
Das System wird in Drehschwingungen versetzt und die Periodendauer $T$ wird anschließend zehn Mal gemessen.
