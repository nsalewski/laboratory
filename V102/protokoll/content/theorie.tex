\section{Theorie}
\label{sec:Theorie}

\subsection{Begriffe aus der Elastizitätstheorie}
\label{sec:begriffe}

Kräfte, die an Festkörpern angreifen, lassen sich in Volumenkräfte und Oberflächenkräfte 
unterteilen. Letztere werden in diesem Versuch genauer betrachtet. 
Volumenkräfte wirken auf jedes Volumenelement des Festkörpers und können Translationsbewegungen 
und Rotationsbewegungen hervorrufen.
Im Gegensatz dazu können \textbf{Oberflächenkräfte} -- Kräfte, die an Oberflächenelementen
angreifen -- Gestalts- und Volumenänderungen bewirken. Sie können neben der Oberfläche auch an 
jeder Querschnittsfläche des Körpers nachgewiesen werden.

Es wird der Begriff der \textbf{Spannung} eingeführt, welche als Oberflächenkraft pro Fläche
definiert ist. Des Weiteren unterscheidet man zwischen der \textbf{Normalspannung $\sigma$}
(auch \textbf{Druck $P$}) und der \textbf{Tangentialspannung $\tau$} (auch
\textbf{Schubspannung}), welche dem orthogonalen bzw. dem tangentialen Anteil der Spannung
entsprechen. 

Wirkt eine Spannung auf einen Festkörper, treten Verformungen -- in Form von Gestalts- und 
Volumenänderungen -- auf. Diese Verformungen entsprechen auf molekularer Ebene einem veränderten 
Gleichgewichtszustand der Atom- bzw. Molekülstruktur, der durch die elektrostatischen 
Kräfte zwischen den Atomen bzw. Molekülen eintritt. Wenn die wirkende Spannung abgenommen wird
und der Körper in seinen ursprünglichen Zustand -- mit gleicher Form und Größe -- zurückkehrt,
spricht man von einer \textbf{elastischen Deformation}.

Weiterhin ergibt sich ein proportionaler Zusammenhang zwischen der Deformation -- in Form 
einer relativen Längen- oder Volumenänderung $\frac{\Delta L}{L}$ bzw.
$\frac{\Delta V}{V}$ -- und der Spannung, wenn der Betrag der wirkenden Oberflächenkraft nicht 
zu groß ist, was in Abbildung \ref{fig:Hooki} durch den Hookeschen Bereich illustriert ist.
% Hier später Kraftverlauf zwischen benachbarten Atomen einfügen \Picture
Liegt die Kraft in diesem Bereich, wird ein proportionaler Zusammenhang angenommen, der sich zu
\begin{align}
	\label{eqn:elastimodi}
	&\sigma = E \, \frac{\Delta L}{L} \,\,\,\, \mathrm{oder}  &P = Q \, \frac{\Delta V}{V}
\end{align}
ergibt.

Allgemein werden 36 Konstanten (jeweils 6 Konstanten zur Charakterisierung der Spannung und
Deformation) in Form einer Matrix benötigt, um den proportionalen Zusammenhang zwischen
Spannung und Deformation zu beschreiben.
Zusätzlich ist das Schema der Elastizitätskonstantenmatrix symmetrisch \cite{Ergebnis der 
exakten Naturwissenschaften vierter band , googlen 21 elastsoooss}, sodass im allgemeinen Fall 
21 Elastizitätskonstanten benötigt werden.
Im sogenannten kubischen Kristallsystem verbleiben aufgrund weiterer Symmetrien 3 notwendige 
Elastizitätskonstanten und bei \textbf{isotropen Körpern} -- Körper mit richtungsunabhängigen 
Elastizitätskonstanten -- wird diese Anzahl auf nur 2 reduziert.

Diese 2 Konstanten werden als \textbf{Schubmodul $G$} (auch \textbf{Torsionsmodul}) und 
\textbf{Kompressionsmodul $Q$} bezeichnet.
Letztere ist eine Kenngröße für die Volumenelastizität. Der Schubmodul charakterisiert die 
Gestaltselastizität.

Weiterhin definiert man in der Praxis zwei weitere Konstanten: der
\textbf{Elastizitätsmodul $E$} und die \textbf{Poissonsche Querkontraktionszahl $\mu$}.

Der Elastizitätsmodul wurde in Gleichung \eqref{eqn:elastimodi} definiert und ist der 
Proportionalitätsfaktor zwischen dem Druck $P$ und der relativen Volumenänderung
$\frac{\Delta V}{V}$.
Bewirkt eine Oberflächenspannung $\vec{\sigma}$ an einem Festkörper eine Längenänderung, 
entspricht die Poissonsche Querkontraktionszahl $\mu$ dem Proportionalitätsfaktor zwischen 
der Längenänderung parallel zur Spannungsrichtung $\frac{\Delta L}{L}$ und der Längenänderung
orthogonal zur Spannungsrichtung $\frac{\Delta B}{B}$. Die Poissonsche Querkontraktionszahl
wird in Abbildung \ref{fig:querkontrakti} durch das Beispiel eines gedehnten Stabes erklärt und
als
\begin{equation*}
	\mu := - \frac{\Delta B}{B} \cdot \frac{L}{\Delta L}
\end{equation*}
%Hier Bild für QUerkontraktionszahl einfügen...............
definiert.

Da bei isotropen Körpern nur zwei Konstanten benötigt werden, müssen die zwei zusätzlichen
Konstanten durch zwei Gleichungen miteinander verknüpft sein.
Diese Zusammenhänge ergeben sich zu 
\begin{align}
	E &= 2G (\mu + 1) \\
	E &= 3Q (1 - 2\mu) \mathrm{.}
\end{align}
\FloatBarrier

\subsection{Der Schubmodul $G$}

Der Schubmodul $G$ ist eine Kenngröße für die Gestaltselastizität eines Körpers und entspricht 
dem Proportionalitätsfaktor zwischen einer Tangentialspannung $\tau$ und einem Scherungswinkel 
$\alpha$:
\begin{equation}
	\label{eqn:schubimoduli}
	\tau = G \alpha \mathrm{.}
\end{equation}
Die dem Schubmodul $G$ zugehörige Deformation wird also als \textbf{Scherung} bezeichnet.
Diese Deformation ist durch eine an einem Körper angreifende Tangentialspannung bedingt, die 
den Körper -- wie in Abbildung \ref{fig:würfeli} illustriert -- um einen Scherungswinkel $\alpha$
neigt.
%Hier Bild von Würfel
Da der Scherungswinkel $\alpha$ in Abbildung \ref{fig:würfeli} allerdings nur schwierig zu messen
ist, wird im Folgenden die Torsion eines Zylinders betrachtet.

\subsubsection{Das Drehmoment eines Zylinders}
%Hier Bild Torsion
Der Zylinder ist an einem Ende fest eingespannt und wird am anderen Ende durch zwei
gegenüberliegende, in entgegengesetzte Richtung wirkende Kräfte gedrillt (siehe Abbildung 
\ref{fig:torsioni}).
Durch diese Kräfte wirkt ein Drehmoment $M$ auf den Zylinder und erzeugt eine Drehung des losen
Endes des Zylinders gegenüber dem fest eingespannten Ende des Zylinder. Diese Drehung wird 
durch den Torsionswinkel $\varphi$ festgelegt.
Da das Drehmoment über den Durchmesser des Zylinders variiert, wird der Zylinder in Hohlzylinder 
mit dem Radius $r$ und Dicke $\symup{d} r$ zerlegt, von denen sich die einzelnen Drehmomente 
über den gesamten Zylinderradius integrieren lassen, sodass man das Gesamtdrehmoment $M$ 
erhält.

Des Weiteren legt der Torsionswinkel $\varphi$ den Scherungswinkel $\alpha$ zwischen den 
Mantelschichten fest, wie in Abbildung \ref{fig:verdrilltorsion} dargestellt.
%Hier Bild von Zylinder
Es ergibt sich der Zusammenhang zwischen Torsionswinkel $\varphi$ und Scherungswinkel $\alpha$ 
zu 
\begin{equation}
	\label{eqn:variphi}
	\alpha = \frac{r \varphi}{L} \mathrm{,}
\end{equation}
mit der Länge $L$ und dem Radius $r$ der Hohlzylinder.

Mit Gleichung \eqref{eqn:schubimoduli} soll nun ein Zusammenhang zwischen dem Drehmoment $M$ 
und dem Torsionswinkel $\varphi$ gefunden werden. 
Das infinitiesimale Drehmoment $\symup{d} M$, welches auf einen Punkt mit der Entfernung $r$ 
zur Zylinderachse wirkt, ergibt sich durch  
\begin{equation*}
	\symup{d} M = r \, \symup{d}K \mathrm{,}
\end{equation*}
wobei $\symup{d} K$ der Tangentialkraft entspricht.
Weiterhin ist die Tangentialspannung $\tau$ aus Gleichung \eqref{eqn:elastimodi} als 
Tangentialkraft pro Flächeneinheit 
\begin{equation*}
	\tau = \frac{\symup{d} K}{\symup{d} F}
\end{equation*}
definiert.
Mit diesen beiden Definitionen und Gleichungen \eqref{eqn:schubimoduli} und \eqref{eqn:variphi}
ergibt sich das Drehmoment eines Hohlzylinders $\symup{d} M$ zu 
\begin{equation}
	\symup{d} M = 2\pi \frac{G}{L} \varphi r^3 \symup{d}r \mathrm{.}
\end{equation}
Integration über den Zylinderradius $R$ liefert das Drehmoment
\begin{equation}
	\label{eqn:drehmoment}
	M = \int_0^R \symup{d} M = \frac{\pi}{2} G \frac{R^4}{L} \varphi \mathrm{.}
\end{equation}






\cite{Anleitung}
