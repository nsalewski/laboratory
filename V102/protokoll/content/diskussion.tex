\section{Diskussion}
\label{sec:Diskussion}
Über den bereits angegebenen Elastizitätsmodul $E$ wurde das Material des Drahtes bestimmt.
Es handelt sich hierbei wahrscheinlich um Stahl. Dessen Elastizitätsmodul beträgt nach \cite{stahlharteJungs} $E=210 \,\si{\giga\pascal}$.
Der Schubmodul von Stahl beträgt nach \cite{stahlharteJungs} $G=81 \,\si{\giga\pascal}$. Ein Vergleich mit unserem Experiment $G_{\mathrm{Experiment}}=(90 \pm 5) \,\si{\giga\pascal}$ .
Somit weicht unser experimentell bestimmtes Ergebnis um etwa $11\%$ vom Theoriewert ab.

Die Horizontalkomponente des Erdmagnetfelds (in Dortmund) hat nach \cite{Potsdam} den Wert 
$B_{\mathrm{Literatur}}=19.3 \,\si{\micro\tesla}$.
Es liegt also eine Abweichung von circa $543\%$ vom experimentell bestimmten Wert
zum Literaturwert vor. Da der Literaturwert im berechneten Fehlerintervall ($B_{\mathrm{Experiment}}=(3\pm29)\,\si{\micro\tesla}$) liegt, ist die große Abweichung auf die Messungenauigkeiten 
der sechs fehlerbehafteten zu bestimmenden Größen zurückzuführen. 

Weiterhin wurde bei der Messung bereits festgestellt, dass die Messapparatur sehr empfindlich auf Luftzug reagiert. Es reichte bereits,
wenn jemand sich in der Nähe der Apparatur bewegte (zum Beispiel hinter dem Experiment herging), um deutliche Änderungen in der Schwingungsdauer festzustellen.
