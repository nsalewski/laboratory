\section{Diskussion}
\label{sec:Diskussion}
Über den bereits angegebene Elastizitätsmodul $E$ wurde das Material des Drahtes bestimmt.
Es handelt sich hierbei wahrscheinlich um Stahl. Dessen Elastizitätsmodul beträgt nach \cite{stahlharteJungs} $E=210 \,\si{\giga\pascal}$.
Der Schubmodul von Stahl beträgt nach \cite{stahlharteJungs} $G=81 \,\si{\giga\pascal}$. Ein Vergleich mit unserem Experiment $G_{\mathrm{Experiment}}=(90 \pm 5) \,\si{\giga\pascal}$ .
Somit weicht unser experimentell bestimmtes Ergebnis um etwa $11\%$ vom Theoriewert ab.
Ein Vergleich der horizontalen Komponente des Erdmagnetfelds (für Dortmund) nach \cite{Potsdam} ergibt $B_{\mathrm{Theorie}}=19.33 \,\si{\micro\tesla}$.
Beim Vergleich mit dem experimentellen Ergebnis $B_{\mathrm{Experiment}}=(3\pm29)\,\si{\micro\tesla}$ zeigt sich eine Abweichung von $533\%$. Da allerdings der Theoriewert trotzdem im sehr großen Fehlerbereich des experimentellen
Ergebnisses liegt, scheint der Versuchsaufbau nicht geeignet zu sein, um die horizontale Komponente des Erdmagnetfelds zu bestimmen.

Bei der Messung wurde bereits festgestellt, dass die Messapparatur sehr empfindlich auf Luftzug reagiert. Es reichte bereits,
wenn jemand sich in der Nähe der Apparatur bewegte (zum Beispiel hinter dem Experiment herging), um deutliche Änderungen in der Schwingungsdauer festzustellen.
