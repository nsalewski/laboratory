\section{Durchführung}
\label{sec:Durchführung}

\subsection{Versuchsaufbau}
\label{sec:Versuchsaufbau}
Der Versuchsaufbau besteht aus einem Ultraschallechoskop, an dessen Ausgänge zwei Ultraschallsonden mit \SI{2}{\mega\Hz} gekoppelt sind, und einem Rechner zur Datenaufnahme und -analyse.\\
An das Ultraschallechoskop sind zwei Ultraschallsonden angeschlossen, mithilfe derer sich sowohl eine Impuls-Echo-Messung, als auch eine Durchschallmessung realisieren lässt.
Am Rechner werden die gemessenen Daten mittels des Programms \textquote{Echoview} ausgewertet.\\
Hierbei ist \textquote{Echoview} in der Lage, vier verschiedene Diagramme darzustellen.
Im linken oberen Graphen wird der A-Scan dargestellt, also die Amplitude gegen die Zeit aufgetragen.
Der linke untere Graph stellt die gewählte Verstärkung dar. Die Verstärkung lässt sich am Ultraschallechoskop über die Drehknöpfe zur laufzeit-bzw. tiefenabhängigen Verstärkung (TGC; Time Gain Control) und ebenso über die Verstärkung des Outputs und der Empfindlichkeit der Sonden regulieren.\\
Zu Beachten ist, dass eine Verstärkung nur gewählt werden darf, wenn die auszuwertende Messreihe nicht zur Untersuchung der Amplitudenhöhe dient.
Die beiden rechten Graphiken sind das berechnete Spektrum der Messdaten (FFT), bzw. ihr
Cepstrum.
Erzeugte Graphiken und Messdaten können aus dem Programm heraus exportiert werden.
Als zu untersuchende Versuchsobjekte stehen Acrylzylinder verschiedener Länge, Acrylplatten unterschiedlicher Dicke sowie das Modell eines menschlichen Auges im Maßstab 3:1 zur Verfügung.


\subsection{Versuchsbeschreibung}
\label{sec:Versuchsbeschreibung}
Zur Aufnahme der \textbf{Charakteristik des Zählrohrs} wird eine $\beta$-Quelle vor das Endfenster des Zählrohrs gestellt und die Zählrate über ein Intervall von einer Minute in Abhängigkeit zur Betriebsspannung $U$ gemessen.\\
Die Betriebsspannung wird hierbei in Schritten von $\SI{10}{\volt}$ heraufgeregelt.
Da wahrscheinlich ein Defekt im Versuchsaufbau vorliegt, läuft die Messung statt über das Intervall $\SI{350}{\volt}$ bis $\SI{700}{\volt}$ über das Intervall $\SI{450}{\volt}$ bis $\SI{900}{\volt}$.\\
Zudem wird die Stromstärke $I$ des Zählrohrstrom mittels eines empfindlichen Strommessgeräts zur Untersuchung der pro Teilchen vom Zählrohr freigesetzten Ladungsmenge erfasst.

In einem nur qualitativen Experiment werden werden die \textbf{Nachentladungen} am Oszilloskopschirm sichtbar gemacht.
Dazu wird die $\beta$-Quelle so weit vom Eintrittsfenster des Zählrohrs entfernt, dass lediglich ein Impuls auf dem Bildschirm samt den zugehörigen Nachentladungen zu sehen ist.\\
Es wird das Oszilloskopbild bei geringer Betriebsspannung nahezu ohne auftretende Nachentladungen mit $U=\SI{450}{\volt}$, mit dem Oszilloskopbild mit deutlich sichtbaren Nachentladungen bei einer Betriebsspannung von $U=\SI{700}{\volt}$, verglichen.

Die Untersuchung der \textbf{Totzeit} erfolgt über zwei Methoden.\\
Zum Einen wird sie aus dem Oszilloskopbild bestimmt.
Hierzu wird aus dem Oszilloskopbild die Zeitdifferenz zwischen zwei Impulsen maximaler Höhe bestimmt. Hierzu muss die Strahlungsintensität hoch sein. Die Quelle wird daher nahe vor dem Eintrittsfenster des Zählrohrs platziert.

Zudem wird die Totzeit über die \textbf{Zwei-Quellen-Methode} bestimmt.
Hierzu wird zunächst eine Messung der Impulsrate mit nur einer Quelle über ein Zeitintervall von $\SI{1}{\minute}$ durchgeführt.
Anschließend wird eine zweite Quelle hinzugefügt und die Impulsrate beider Quellen gemeinsam bestimmt. Schließlich wird die zweite Quelle ebenfalls über das gleiche Zeitintervall alleine vermessen.
Es ist darauf zu achten, dass beim Hinzufügen, beziehungsweise Entfernen der Quellen die Position der jeweils andere Quelle bezüglich des Zählrohrs möglichst nicht geändert wird.
