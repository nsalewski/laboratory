\section{Diskussion}
\label{sec:Diskussion}
Zwischen den beiden bestimmten Totzeiten zeigen sich sehr große Abweichungen.
Es ist daher davon auszugehen, dass die Totzeit nicht mit beiden Methoden im vorliegenden Fall zuverlässig bestimmt werden kann. Es ist davon auszugehen, dass die über die Zwei-Quellen-Methode bestimmte Totzeit näher an der tatsächlichen Totzeit des Geiger-Müller-Zählrohr liegt, da bei der Bestimmung am Oszilloskopbild große Ablesefehler möglich sind.
Mit Gewissheit lässt sich aber keine Aussage üüber die tatsächliche Totzeit des Geiger-Müller-Zählrohrs treffen.
