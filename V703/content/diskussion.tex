\section{Diskussion}
\label{sec:Diskussion}
Die Güte des Geiger-Müller-Zählrohrs ist definiert über die Breite des Plateaubereichs und der hier vorliegenden Steigung.
Im vorliegenden Versuch ergibt sich die Steigung pro $\SI{100}{\volt}$ zu
\begin{equation*}
  P_{\mathrm{S}}=\SI{1.00(8)E-02}{\percent} \text{.}
\end{equation*}
Die Breite des Plateaubereichs lässt sich nicht eindeutig angeben, da sich zum Ende des Plateaubereichs kein weiterer Anstieg der gemessenen Impulse zeigt.
Seine Breite wird daher angenommen zu
\begin{equation*}
  \Delta U_\mathrm{P}=\SI{390}{\volt}\text{.}
\end{equation*}
Zwischen den beiden bestimmten Totzeiten zeigen sich sehr große Abweichungen.
Es ist daher davon auszugehen, dass die Totzeit nicht mit beiden Methoden im vorliegenden Fall zuverlässig bestimmt werden kann. Es ist davon auszugehen, dass die über die Zwei-Quellen-Methode bestimmte Totzeit näher an der tatsächlichen Totzeit des Geiger-Müller-Zählrohr liegt, da bei der Bestimmung am Oszilloskopbild große Ablesefehler möglich sind.
Mit Gewissheit lässt sich aber keine Aussage über die tatsächliche Totzeit des Geiger-Müller-Zählrohrs treffen.
Die bestimmte Totzeit über die Zwei-Quellen-Methode scheint allerdings auch zu gering.
Für eine höhere Genauigkeit in der Bestimmung der Totzeit müsste die Strahlungsintensität geringer sein, die Strahlungsquellen also weiter weg vom Eintrittsfenster des Geiger-Müller-Zählrohrs platziert werden.


Bei der Bestimmung der pro Teilchen freigesetzten Ladungsmenge fällt auf, dass diese mit steigendem anliegendem Beschleunigungspotential größer wird.
Dies lässt sich durch das größer werdende elektrische Feld im Zählrohr erklären.
Ein Vergleich mit Abbildung \ref{fig:voltage_ions} zeigt zudem, dass die richtige Größenordnung gemessen wurde. Ebenso wie in der Abbildung ist die pro Teilchen freigesetzte Ladungsmenge für den Geiger-Müller-Bereich in der Größenordnung $\SI{10E10}{\electronvolt}$.
