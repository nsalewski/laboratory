\section{Diskussion}
\label{sec:Diskussion}
Generell lässt sich sagen, dass die einzelnen Messwerte in jedem Plot nur gering von der Ausgleichsgraden
beziehungsweise von der Theoriekurve abweichen. Allerdings viel bereits während der Messung auf, dass Voltmeter und Amperemeter durch Stöße an den Tisch
stark schwankten und zum Teil nicht mehr bis zum zuvor gemessenen Wert stiegen.
Auffällig ist zudem, dass bei einer Änderung des jeweiligen Messbereichs ein anderer Wert als zuvor gemessen wurde.
Deshalb wurde eine Änderung des Messbereichs weitmöglichst vermieden.
Dennoch sieht man eine Änderung des Messbereichs zum Teil auch in den gemessenen Werten. So wurde beispielsweise in der Messung
der Klemmenspannung der Monozelle ohne Gegenspannung (vergleiche mit Plot \ref{fig:plot_a}) nach der sechsten Messung der Messbereich des Voltmeters geändert.
Ebenso bei der Messung der Monozelle mit Gegenspannung (siehe auch Plot \ref{fig:monozellebelastet}). Dort wurde für die letzten beiden Messpunkte
ebenfalls der Messbereich des Voltmeters geändert.
Diese Ungenauigkeit der Messung bei der Änderung des Messbereichs lässt sich erklären mit einem dadurch geänderten Innenwiderstand des Messgeräts.
