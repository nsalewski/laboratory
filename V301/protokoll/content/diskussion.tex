\section{Diskussion}
\label{sec:Diskussion}
Generell lässt sich aussagen, dass die einzelnen Messwerte in den jeweiligen Abbildungen eine geringe Diskrepanz gegenüber der Ausgleichsgeraden - beziehungsweise der Theoriekurve in Abbildung \ref{fig:aufe} - aufweisen. Allerdings fiel bereits während der Messung auf, dass das Voltmeter und das Amperemeter durch Stöße an den Tisch stark schwankten und zum Teil daraufhin nicht den zuvor gemessenen Wert anzeigten.
Auffällig ist zudem, dass die Messgeräte bei einer Änderung der Skalierung einen andereren Wert als zuvor angezeigt haben. Daher wurde eine Änderung des Messbereichs weitmöglichst vermieden.
Dennoch sieht man eine Änderung des Messbereichs zum Teil auch in den gemessenen Werten. So wurde beispielsweise in der Messung der Klemmenspannung der Monozelle ohne Gegenspannung (vergleiche mit Abbildung \ref{fig:plot_a}) nach der sechsten Messung der Messbereich des Voltmeters geändert.
Ebenso bei der Messung der Monozelle mit Gegenspannung (siehe auch Abbildung \ref{fig:plot_monozellebelastet}). Dort wurde für die letzten beiden Messpunkte ebenfalls der Messbereich des Voltmeters geändert.
Diese Ungenauigkeit der Messung bei der Änderung des Messbereichs lässt sich erklären mit einem dadurch geänderten Innenwiderstand des Messgeräts.

Die Abweichungen der Messpunkte gegenüber den Theoriekurven lassen sich weitesgehend - wie schon am Ende der Auswertung angesprochen - auf zufällige Fehler bei der Messung zurückführen.
Die systematischen Fehler - wie beispielsweise bei der direkten Messung der Leerlaufspannung $U_0$, bedingt durch unberücksichtigte Innenwiderstände der Messgeräte oder dem Vernachlässigen von Leitungswiderständen - fallen vernachlässigbar gering aus, sodass sie gegenüber den zufälligen Fehlern vernachlässigbar sind.
Mögliche zufällige Fehler stellen Ablesefehler oder auch Schwankungen der Netzspannung dar.

