\section{Durchführung}
\label{sec:Durchführung}

Die einzelnen Brückenschaltungen werden in den Unterpunkten a) - e) erläutert.

\subsection{Wheatstonesche Brücke}
\label{sec:durcha}
Zunächst wird die Wheatstonesche Brücke wie in Abbildung \ref{fig:wheatstonebrücke} aufgebaut.

Es sollen zwei unbekannte Widerstände gemessen werden, die jeweils dem $R_x$ in der Abbildung entsprechen.
Als Nullindikator dient ein digitales Oszilloskop.
Ein Tiefpassfilter am Oszilloskop ermöglicht es, die hochfrequenten Störspannungen zu dämpfen.

Nachdem der Aufbau abgeschlossen ist, können die Spannungsquelle $U_{\text{S}}$ sowie das Oszilloskop eingeschaltet werden.
Es wird ein Wechselstrom verwendet.
Nun muss der Widerstand $R_3$ solange variiert werden, bis auf dem Oszilloskop nur noch die hochfrequenten Störspannungen zu erkennen sind.
Wenn dies der Fall ist, kann die Skala des Oszilloskops empfindlicher eingestellt werden und
die Variation des Widerstandes $R_3$ kann wiederholt werden.
Dieser Vorgang wird solange ausgeführt, bis die höchste Empfindlichkeit des Oszilloskops erreicht ist.
Dann wird die Abgleichbedingung \eqref{eqn:abgleichbedingung} als erfüllt angesehen und der Messdurchgang beendet.
Das Potentiometer, das für das Verhältnis $\frac{R_3}{R_4}$ verwendet wird, hat einen Gesamtwiderstand von $R_{\text{GES}} = R_3 + R_4 = \SI{1000}{\ohm}$.
Damit erhält man $R_4$, wenn $R_3$ ermittelt wurde, durch $R_4 = \SI{1000}{\ohm} - R_3$.

Für jeden unbekannten Widerstand $R_x$ werden drei Messdurchgänge mit drei verschiedenen Referenzwiderständen $R_2$ durchgeführt.

\subsection{Kapazitätsmessbrücke}
\label{sec:durchb}
Der Versuchsaufbau ist in Abbildung \ref{fig:kapazitätsmessbrücke} dargestellt.

Zunächst soll ein R-C-Kombinationsglied gemessen werden, also ein Kondensator (Kapazität $C_x$) mit nicht vernachlässigbarem ohmschen Widerstand $R_x$.
Um die Abgleichbedingung zu erfüllen (Brückenspannung Null) wird ein alternierendes Messen durchgeführt. %evtl besser: wird alternierend gemessen
Das heißt, dass das Stellglied $R_2$ und das Potentiometer ($\frac{R_3}{R_4}$) abwechselnd reguliert werden, sodass die Brückenspannung minimal wird.
Dieser Vorgang wird solange wiederholt bis das Oszilloskop bei größter Empfindlichkeit nur noch die hochfrequenten Störspannungen anzeigt.
Danach wird die Messung mit zwei weiteren Referenzkapazitäten $C_2$ wiederholt.

Des Weiteren sollen die Kapazitäten zweier Kondensatoren ohne das Stellglied $R_2$ bestimmt werden.
Hierfür wird einfach das Stellglied aus dem Versuchsaufbau entfernt.
Die Durchführung funktioniert nun wieder analog zu \ref{sec:durcha}.
Auch hier werden wieder drei Messungen mit drei verschiedenen Referenzkapazitäten durchgeführt.

\subsection{Induktivitätsmessbrücke}
\label{sec:durchc}

In Abbildung \ref{fig:induktivitätsmessbrücke} ist der Aufbau der Induktivitätsmessbrücke dargestellt.
Es soll die Induktivität $L_x$ und der Verlustwiderstand $R_x$ einer unbekannten Spule gemessen werden.
Verwendet wird hierfür ein L-R-Kombinationsglied.
Die Durchführung ist analog zur Messung des R-C-Kombinationsglieds \ref{sec:durchb}.
Die Werte für $R_2$, $R_3$ und $R_4$, die die Abgleichbedingungen erfüllen, werden also wieder alternierend gemessen.
Variiert wird hier für drei Messungen jeweils die Referenzinduktivität $L_2$.

\subsection{Maxwell-Brücke}

Der Versuchsaufbau der Maxwell-Brücke ist in Abbildung \ref{fig:maxwellbrücke} dargestellt.
Mit der Maxwell-Brücke soll das L-R-Kombinationsglied aus \ref{sec:durchc} ein weiteres Mal gemessen werden.
Hierfür wird der Referenzwiderstand $R_2$ bei drei Messungen jeweils variiert.
Die Referenzkapazität $C_4$ wird hingegen nicht verändert.
Die beiden Stellglieder $R_3$ und $R_4$ werden wie in den vorherigen Messungen alternierend gemessen, bis die auf dem Oszilloskop angezeigte Spannung bei der höchsten Empfindlichkeit nur den hochfrequenten Störspannungen entspricht.
Daraus erhält man mit den Abgleichbedingungen die unbekannte Kapazität und den unbekannten Verlustwiderstand.

\subsection{Wien-Robinson-Brücke}
Der Aufbau der Wien-Robinson-Brücke ist in Abbildung \ref{fig:wienrobinsonbrücke} dargestellt.
Die Abgleichbedingung ist bei der Wien-Robinson-Brücke abhängig von der Spannungsfrequenz.
Daher ist die Frequenz der Wechselspannung bei dieser Brückenschaltung nicht konstant wie
bei den vorherigen Messungen, sondern wird variiert zwischen $\omega = \SI{20}{\Hz}$ und $\omega = \SI{30}{\kHz}$.
Zunächst wird nach dem Start der Messung die Spannungsfrequenz $\omega_{\text{min}}$ bestimmt, bei welcher die Brückenspannung $U_{\text{Br}}$ ein Minimum erreicht.
Danach wird die eigentliche Messung gestartet.
Hierbei werden in geeigneten Abständen um die Spannungsfrequenz $\omega_{\text{min}}$ die Messwerte für $U_{\text{S}}$ und $U_{\text{Br}}$ abgelesen.
Da die Brückenspannung am Oszilloskop abgelesen wird, ist zu beachten, dass die Amplituden vollständig auf dem Oszilloskop zu sehen sind.
\newpage
