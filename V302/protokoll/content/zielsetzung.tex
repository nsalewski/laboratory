\section{Zielsetzung}
\label{sec:Zielsetzung}
Im vorliegenden Versuch sollen mittels Brückenschaltungen verschiedene physikalische Größen, wie Ohmsche Widerstände, Induktivitäten und Kapazitäten
anhand verschiedener geeigneter Brückenschaltungen bestimmt werden.
Die unbekannten Bauteile werden über die Abgleichbedingung der jeweiligen Brückenschaltung über bereits bekannte Schaltungselemente bestimmt.
Zudem soll der Verlust in Kondensatoren und Spulen betrachtet und technisch mittels eines Ohm'schen Widerstands abgeglichen werden.
Schließlich soll die frequenzabhängige Brückenspannung einer Wien-Robinson-Brücke betrachtet werden und mittels der Bestimmung des Klirrfaktors soll die Güte
eines Sinusspannungsgenerators bestimmt werden.
