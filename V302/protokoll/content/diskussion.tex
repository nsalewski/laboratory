\section{Diskussion}
\label{sec:Diskussion}
Auffällig ist, dass bei der Messung der Induktivität mit der Maxwell-Brücke die jeweilige Abweichung vom Mittelwert geringer ist als jeweiligen Abweichungen bei der Induktiviätsmesssbrücke, was nach der Theorie auch so zu erwarten war.
Die größeren Fehler des Mittelwerts lassen sich durch die hohen angenommenen Fehler der einzelnen Messelemente begründen ($R_3$ und $R_4$).
Beim Vergleich der Theoriekurve des Quotienten $\frac{U_{Br}}{U_s}$ mit den gemessenen Werten zeigt sich eine sehr hohe Übereinstimmung beider Kurven. Lediglich der erste Messpunkt
bei $\omega=\SI{20}{\Hz}$ und die Messpunkte ab etwa $\omega=\SI{12000}{\Hz}$ weichen deutlich von der Theoriekurve ab. Dies könnte allerdings auch auf ein zweites Minimum hindeuten, da die Messwerte erneuut abfallen. Die Theoriekurve würde weitere Minima nicht berücksichtigen, da sich ihr Minimum durch das feste $\omega_0$ bestimmt.
Erfreulich ist, dass die Messpunkte um $\omega_0$ sehr gut auf der Theoriekurve liegen.
Da zuvor bei niedrigen Frequenzen gemessen wurde ($\omega=\SI{1000}{\Hz})$, kann angenommen werden, dass die Brückenspannung hier relativ gut zu Null abgeglichen werden kann. Gemessen wurde das Verschwinden der Brückenspannung
$U_{Br}$ etwa bei $\omega_0=\SI{240}{Hz}$. Nach der Theorie ergibt sich $\omega_0=\SI{241.69}{Hz}$. Dies ist eine Abweichung von etwa $0.7\%$, welche angesichts der Messunsicherheiten vernachlässigbar ist.
Bereits ein leichter Schlag auf den Tisch verursachte Schwankungen im Multimeter bei der Messung der Wien-Robinsonbrücke, sodass nicht ausgeschlossen werden kann, dass eine Beeinflussung auch bereits schon in den vorangegangenen Messungen durch Stöße an den Tisch stattfand.
