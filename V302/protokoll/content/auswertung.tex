\section{Auswertung}
\label{sec:Auswertung}

\subsection{a) Wheatstonesche Brücke}
Mit der Wheatstoneschen Brücke (Abb: \ref{fig:wheatstonebrücke}) sollen zwei unbekannte Widerstände bestimmt werden.
Für den unbekannten Widerstand $R_x$ gilt nach Formel \eqref{eqn:widerstand}
\begin{equation}
	R_x = R_2 \frac{R_3}{R_4}
\end{equation}
Der erste bestimmte unbekannte Widerstand war der Widerstand 13.
	\begin{table}
		\centering
		\caption{Messdaten für Wert 13}
		\label{tab:wheat1}
	\begin{tabular}{cccc}
		\toprule
		$R_2$ / $\Omega$ & $R_3$ / $\Omega$ & $R_4$ / $Omega$ & $R_x$ / $\Omega$ \\
		\midrule
		332 & 491,8 & 508,2 & 321,3 \pm 1 \\
		664 & 325,6 & 674,4 & 320,6 \pm 1 \\
		1000 & 242,9 & 757,1 & 320,8 \pm1 \\
		\bottomrule
	\end{tabular}
	\end{table}
