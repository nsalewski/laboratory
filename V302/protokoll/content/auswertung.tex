\section{Auswertung}
\label{sec:Auswertung}

\subsection{a) Wheatstonesche Brücke}
Mit der Wheatstoneschen Brücke (Abb: \ref{fig:wheatstonebrücke}) sollen zwei unbekannte Widerstände bestimmt werden.
Für den unbekannten Widerstand $R_x$ gilt nach Formel \eqref{eqn:widerstand}
\begin{equation}
	R_x = R_2 \frac{R_3}{R_4}
\end{equation}
Der erste bestimmte unbekannte Widerstand war der Widerstand 13.
	\begin{table}
		\centering
		\caption{Messdaten für Wert 13}
		\label{tab:wheat1}
	\begin{tabular}{cccc}
		\toprule
		$R_2$ / $\Omega$ & $R_3$ / $\Omega$ & $R_4$ / $\Omega$ & $R_x$ / $\Omega$ \\
		\midrule
		332 & 491,8 & 508,2 & 321,3 \pm 1,7 \\
		664 & 325,6 & 674,4 & 320,6 \pm 1,7 \\
		1000 & 242,9 & 757,1 & 320,8 \pm 1,7 \\
		\bottomrule
	\end{tabular}
	\end{table}

Die Messwerte mit dem berechneten Widerstand 13 sind in Tabelle \ref{tab:wheat1} dargestellt.
Für den Referenzwiderstand $R_2$ soll ein Fehler von $0,2\%$ und für das Verhältnis $\frac{R_3}{R_4}$ ein Fehler von $0,5\%$ angenommen werden (\cite{Anleitung}).
Daraus ergeben sich die Fehler für $R_x$ mit der Gaußschen Fehlerfortpfplanzung.

Damit ergibt sich für den zu bestimmenden Widerstand $R_{13} = (320,9 \pm 1,7??) \Omega$.
\newpage
Der zweite bestimmte Widerstand war der Widerstand 12.

	\begin{table}
		\centering
		\caption{Messdaten für Wert 12}
		\label{tab:wheat2}
	\begin{tabular}{cccc}
		\toprule
		$R_2$ / $\Omega$ & $R_3$ / $\Omega$ & $R_4$ / $\Omega$ & $R_x$ / $\Omega$ \\
		\midrule
		332 & 542,1 & 457,9 & 393,0 \pm 2,1 \\
		664 & 371,8 & 628,2 & 393,0 \pm 2,1 \\
		1000 & 282.0 & 718,0 & 392,8 \pm 2,1 \\
		\bottomrule
	\end{tabular}
	\end{table}
Die Messwerte und der ermittelte Widerstand $R_{12}$ sind in Tabelle \ref{tab:wheat2} dargestellt.

Für den zu bestimmenden Widerstand ergibt sich $R_{12} = (392,9 \pm 2,1??) \Omega$.

\subsection{b) Kapazitätsmessbrücke}
Der Aufbau einer Kapazitätsmessbrücke ist in Abbildung ??? dargestellt.
Zunächst sollen die Widerstände einer RC-Kombination bestimmt werden. 
Für den ohmschen Widerstand $R_x$ gilt nach Formel ??? $R_x = R_2 \frac{R_3}{R_4}$
Für den kapazitiven Widerstand $C_x$ gilt nach Formel ??? $C_x = C_2 \frac{R_4}{R_3}$
	\begin{table}
		\centering
		\caption{Messdaten für das RC-Kombinations-Glied 8}
		\label{tab:kapakombi}
	\begin{tabular}{cccccc}
		\toprule
		$C_2$ / $nF$ & $R_2$ / $\Omega$ & $R_3$ / $\Omega$ & $R_4$ / $\Omega$ & $R_x$ / $\Omega$ & $C_x$ / $nF$ \\
		\midrule
		597 & 225,2 & 717,0 & 283,0 & 570,6 \pm 3,1 & 235,6 \pm 1,3 \\
		750 & 279,9 & 671,8 & 328,2 & 572,9 \pm 3,0 & 366,4 \pm 2,0 \\
		994 & 167,2 & 773,4 & 226,6 & 570,7 \pm 3,3 & 291,2 \pm 1,6 \\
		\bottomrule
	\end{tabular}
	\end{table}
Die Werte für das RC-Kombinations-Glied sind in Tabelle \ref{tab:kapakombi} dargestellt.
Der Fehler von $R_2$ wurde auf $\Delta R_2 = 0,5 \Omega$ festgelegt.

Damit erhält man für die zu bestimmenden Größen: $R_x = (571,4 \pm 3,1??) \Omega$ und $C_x = (297,7 \pm 1,6??) nF$.



Nun sollen die Kapazitäten zweier Kondensatoren bestimmt werden.
Hierfür wurden Die Kondensatoren 1 und 3 verwendet.
Es wurden jeweils der Kondensator $C_2$ variiert und der ohmsche Widerstand $R_2$ wurde auf Null gesetzt.

	\begin{table}
		\centering
		\caption{Messdaten für den Kondensator 1}
		\label{tab:kapa1}
	\begin{tabular}{ccccc}
		\toprule
		$C_2$ / $nF$ & $R_2$ / $\Omega$ & $R_3$ / $\Omega$ & $R_4$ / $\Omega$ & $C_x$ / $nF$ \\
		\midrule
		597 & 0 & 476,9 & 523,1 & 654,8 \pm 3,5 \\
		750 & 0 & 529,8 & 470,2 & 665,6 \pm 3,6 \\
		994 & 0 & 602,7 & 397,3 & 655,2 \pm 3,5 \\
		\bottomrule
	\end{tabular}
	\end{table}

Die Werte für den Kondensator 1 sind in Tabelle \ref{tab:kapa1} dargestellt.


	\begin{table}
		\centering
		\caption{Messdaten für den Kondensator 3}
		\label{tab:kapa3}
	\begin{tabular}{ccccc}
		\toprule
		$C_2$ / $nF$ & $R_2$ / $\Omega$ & $R_3$ / $\Omega$ & $R_4$ / $\Omega$ & $C_x$ / $nF$ \\
		\midrule
		597 & 0 & 589,1 & 410,9 & 416,4 \pm 2,2 \\
		750 & 0 & 639,5 & 360,5 & 422,8 \pm 2,3 \\
		994 & 0 & 704,9 & 295,1 & 416,1 \pm 2,2 \\
		\bottomrule
	\end{tabular}
	\end{table}
Die Werte für den Kondensator 3 sind in Tabelle \ref{tab:kapa3} dargestellt.


\subsection{c) Induktivitätsmessbrücke}

\blindtext

Die Induktivitätsmessbrücke ist in Abbildung ??? dargestellt.
Die unbekannte Induktivität $L_x$ ($L_{17}$) wurde mittels Formel ??? bestimmt, der 
Verlustwiderstand $R_x$ ($R_{17}$) mit Formel ???.

	\begin{table}
		\centering
		\caption{Messdaten für das L-R-Kombinationsglied 17}
		\label{tab:induk}
	\begin{tabular}{cccccc}
		\toprule
		$L_2$ / $mH$ & $R_2$ / $\Omega$ & $R_3$ / $\Omega$ & $R_4$ / $\Omega$ & $R_x$ / $\Omega$ & $L_x$ / $mH$ \\
		\midrule
		20,1 & 38,8 & 680,1 & 319,9 & 82,5 \pm 2,5 & 42,7 \pm 0,2 \\
		27,5 & 60,5 & 609,2 & 390,8 & 94,3 \pm 2,5 & 42,9 \pm 0,2 \\
		14,6 & 33,3 & 749,2 & 250,8 & 99,5 \pm 3,0 & 43,6 \pm 0,2 \\
		\bottomrule
	\end{tabular}
	\end{table}

Die Werte für das L-R-Kombinationsglied sind in Tabelle \ref{tab:induk} dargestellt.
Der Fehler von $R_x$ und $L_x$ durch die fehlerbehafteten Größen $L_2$ (0,2 \%), $R_2$ (3 \%) und $\frac{R_3}{R_4}$ (0,5 \%) wurden mittels Gaußscher Fehlerfortpfplanzung bestimmt.


\subsection{d) Maxwell-Brücke}

	\begin{table}
		\centering
		\caption{Messdaten für das L-R-Kombinationsglied 17}
		\label{tab:maxwk}
	\begin{tabular}{cccccc}
		\toprule
		$R_2$ / $\Omega$ & $C_4$ / $nF$ & $R_3$ / $\Omega$ & $R_4$ / $\Omega$ & $R_x$ / $\Omega$ & $L_x$ / $mH$ \\
		\midrule
		332 & 597 & 215,2 & 754,1 & 94,7 \pm 4,0 & 42,7 \pm 1,3 \\
		664 & 597 & 108,0 & 753,2 & 95,2 \pm 4,0 & 42,8 \pm 1,3 \\
		1000 & 597 & 72,0 & 753,2 & 95,6 \pm 4,1 & 43,0 \pm 1,3 \\
		\bottomrule
	\end{tabular}
	\end{table}





