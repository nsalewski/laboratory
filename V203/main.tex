\input{header.tex}

\subject{V203}
\title{Verdampfungswärme und Dampfdruck-Kurve}
\date{
	Durchführung: 2016-10-18
	\hspace{3em}
	Abgabe: 2016-10-25
}

\begin{document}

\maketitle
\thispagestyle{empty}
\tableofcontents
\newpage

\section{Zielsetzung}

\section{Theorie}

\begin{itemize}
		\item Zustandsdiagramm:
			\begin{itemize}
			\item Druck $p$ gegen Temperatur $T$ aufgetragen
			\item Aggregatzustände (fest, flüssig, gasförmig) können als drei Bezirke abgegrenzt werden
			\item Bezirke werden durch Kurven getrennt.
				Die Menge aller Tupel, die nicht auf einer Kurve liegen besitzen 2 Freiheitsgrade. Die Tupel auf der Kurve einen und das Tupel am Tripelpunkt (TP) besitzt keinen Freiheitsgrad
			\item auf Kurven koexistieren zwei Aggregatzustände
			\item Kurve, die AZ gasförmig und flüssig trennt heißt Dampfdruck-Kurve
			\item diese wird durch Verdampfungswärme $L$ charakterisiert
			\begin{itemize}
					\item charakteristische Größe für jeden Stoff
					\item nicht konstant, hängt von Temperatur ab
					\item geht gegen $0$, wenn sich Temperatur dem kritischen Punkt (KP) nähert.
					\item es existiert Temperatur-Bereich in dem L nahezu konstant
			\end{itemize}
			\end{itemize}
		\item Mikroskopische Vorgänge bei der Verdampfung: 
			\begin{itemize}

			\item Moleküle an Wasseroberfläche besitzen Geschwindigkeitsverteilung gemäß der Maxwell-Boltzmann Verteilung.
			\item Entsprechender Anteil mit ausreichender kinetischer Energie verdampft.
			\item Hierbei wird Arbeit gegen Van-der-Waals-Kräfte(Molekularkräfte) verrichtet. 
			\item Nötige Energie aus Wärmevorrat der Substanz oder von außen zugeführt.
			\item Energie um $1$ mol Flüssigkeit in Dampf gleicher Temperatur umzuwandeln heißt molare Verdampfungswärme $L$.
			\item Bei Kondensation wird Wärmeenergie mit dem Betrag von $L$ wieder an die Umgebung abgegeben.
			\item Moleküle in Dampfphase erzeugen Druck durch Stöße an Wand und Flüssigkeitsoberfläche.
			\item Moleküle, die auf Oberfläche treffen werden teilweise wieder eingefangen
			\item Sind äußere Bedingungen gleichbleibend stellt sich thermodynamisches Gleichgewicht ein. Druck bleibt gleich im zeitlichen Mittel: Stoffmenge die verdampft entspricht Stoffmenge die kondensiert.
			\item dieser Druck heißt Sättigungsdampfdruck
			\item steigt $T$, verschiebt sich Geschwindigkeitsverteilung, mehr Moleküle haben kinetische Energie um aus Obeerfläche austreten können, höherer Druck
			\item Druck hängt nicht vom Volumen ab, verändert sich Volumen, verdampft entsprechend mehr Flüssigkeit bis Sättigungsdampfdruck erreicht ist.
			\item Also lässt sich das Verhältnis vom gesättigten Dampf nicht durch allgemeine Gasgleichung $p V = R T$ beschreiben.
			\end{itemize}
		\item Ableitung einer Differentialgleichung für die Dampfdruck-Kurve
			\begin{itemize}
				\item betrachte reversiblen Kreisprozess
				\item isobares, isothermes Verdampfen und anschließendes isobares, isothermes Kondensieren bis zum Ausgangszustand
				\item daraus erhält man durch Anwendung des 1. und 2. Hauptsatzes der Thermodynamik die Clausius-Clapeyronsche Gleichung 
				\begin{equation}
					\frac{L \symup{d}T}{T} = \symup{d}p (V_D - V_F)
				\end{equation}
				\item Lösen der DGL schwierig bei komplexen $T$-Abhängigkeiten von $V_D$ und $V_F$
				\item Daher werden Vereinfachungen verwendet, die gelten wenn $T$ weit unter KP liegt
					\begin{enumerate}
						\item $V_F$ vernachlässigbar gegenüber $V_D$
						\item $V_D$ gehorcht idealer Gasgleichung
						\item $L$ unabhängig von Druck und Temperatur
					\end{enumerate}
				\item daraus erhält man vereinfachte gleichung
					\begin{equation}
						\frac{R}{p} \symup{d}p = \frac{L}{T^2} \symup{d}T
					\end{equation}
				\item Integration liefert schließlich $T$-Abhängigkeit von $p$
					\begin{equation}
						p = p_0 \exp{-\frac{L}{R} \frac{1}{T}}
					\end{equation}
			\end{itemize}
		
	
\end{itemize}

\section Ausgleichspolynom


\end{document}
