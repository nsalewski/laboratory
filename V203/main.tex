\input{header.tex}

\subject{V203}
\title{Verdampfungswärme und Dampfdruck-Kurve}
\date{
	Durchführung: 2016-10-18
	\hspace{3em}
	Abgabe: 2016-10-25
}

\begin{document}

\maketitle
\thispagestyle{empty}
\tableofcontents
\newpage

\section{Zielsetzung}

\section{Theorie}

\begin{itemize}
		\item Zustandsdiagramm:
			\begin{itemize}
			\item Druck $p$ gegen Temperatur $T$ aufgetragen
			\item Aggregatzustände (fest, flüssig, gasförmig) können als drei Bezirke abgegrenzt werden
			\item Bezirke werden durch Kurven getrennt.
				Die Menge aller Tupel, die nicht auf einer Kurve liegen besitzen 2 Freiheitsgrade. Die Tupel auf der Kurve einen und das Tupel am Tripelpunkt (TP) besitzt keinen Freiheitsgrad
			\item auf Kurven koexistieren zwei Aggregatzustände
			\item Kurve, die AZ gasförmig und flüssig trennt heißt Dampfdruck-Kurve
			\item diese wird durch Verdampfungswärme $L$ charakterisiert
			\begin{itemize}
					\item charakteristische Größe für jeden Stoff
					\item nicht konstant, hängt von Temperatur ab
					\item geht gegen $0$, wenn sich Temperatur dem kritischen Punkt (KP) nähert.
					\item es existiert Temperatur-Bereich in dem L nahezu konstant
			\end{itemize}
			\end{itemize}
		\item Mikroskopische Vorgänge bei der Verdampfung
\end{itemize}




\end{document}
