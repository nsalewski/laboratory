\subsection{Versuchsaufbau}
\label{sec:Versuchsaufbau}
Der Versuchsaufbau besteht aus einem Ultraschallechoskop, an dessen Ausgänge zwei Ultraschallsonden mit \SI{2}{\mega\Hz} gekoppelt sind, und einem Rechner zur Datenaufnahme und -analyse.\\
An das Ultraschallechoskop sind zwei Ultraschallsonden angeschlossen, mithilfe derer sich sowohl eine Impuls-Echo-Messung, als auch eine Durchschallmessung realisieren lässt.
Am Rechner werden die gemessenen Daten mittels des Programms \textquote{Echoview} ausgewertet.\\
Hierbei ist \textquote{Echoview} in der Lage, vier verschiedene Diagramme darzustellen.
Im linken oberen Graphen wird der A-Scan dargestellt, also die Amplitude gegen die Zeit aufgetragen.
Der linke untere Graph stellt die gewählte Verstärkung dar. Die Verstärkung lässt sich am Ultraschallechoskop über die Drehknöpfe zur laufzeit-bzw. tiefenabhängigen Verstärkung (TGC; Time Gain Control) und ebenso über die Verstärkung des Outputs und der Empfindlichkeit der Sonden regulieren.
Zu Beachten ist, dass eine Verstärkung nur gewählt werden darf, wenn die auszuwertende Messreihe nicht zur Untersuchung der Amplitudenhöhe dient.\\
Die beiden rechten Graphiken sind das berechnete Spektrum der Messdaten (FFT), bzw. ihr
Cepstrum.
Erzeugte Graphiken und Messdaten können aus dem Programm heraus exportiert werden.
Als zu untersuchende Versuchsobjekte stehen Acrylzylinder verschiedener Länge, Acrylplatten unterschiedlicher Dicke sowie das Modell eines menschlichen Auges im Maßstab 3:1 zur Verfügung.
