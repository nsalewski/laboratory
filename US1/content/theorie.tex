\section{Theorie}
\label{sec:Theorie}

\subsection{Grundlagen}
Ultraschallwellen sind Schallwellen im Frequenzbereich von $\SI{20}{\kilo\hertz}$ bis
$\SI{1}{\giga\hertz}$ mit denen die Struktur von Materialien untersucht werden kann.
Schallwellen sind longitudinale Wellen der Form
\begin{equation}
	p(x,t) = p_0 + v_0 Z \cos(\omega t - kx) \mathrm{,}
\end{equation}
wobei $Z$ als akustische Impedanz oder Schallkennwiderstand bezeichnet wird.
Schallwellen können -- wie elektromagnetische Wellen -- reflektiert und gebrochen werden.
Da sie sich aber aufgrund von Druckschwankungen im Raum fortbewegen, ist die
Schallgeschwindigkeit $c$ materialabhängig. Bei Flüssigkeiten zum Beispiel hängt die
Schallgeschwindigkeit von der Kompressibilität $\kappa$ und der Dichte $\rho$ ab;
\begin{equation}
	c_{\mathrm{Fl}} = \sqrt{\frac{1}{\kappa\rho}} \, \mathrm{.}
\end{equation}
Die Schallgeschwindigkeit in Festkörpern ergibt sich durch
\begin{equation}
	c_{\mathrm{Fe}} = \sqrt{\frac{E}{\rho}} \, \mathrm{,}
\end{equation}
mit dem Elastizitätsmodul $E$.
Damit lässt sich die akustische Impedanz
\begin{equation}
	Z=c \rho
\end{equation}
bestimmen.
Des Weiteren wird Schall von dem Material absorbiert, sodass die Schallamplitude exponentiell
mit dem Ort $x$ abfällt;
\begin{equation}
	\label{eqn:dampf}
	I(x) = I_0 \, \symup{e}^{-\alpha x} \mathrm{.}
\end{equation}
Die Absorptionsstärke wird vom materialabhängigen Absorptionskoeffzienten $\alpha$ festgelegt.
Luft beispielsweise absorbiert Schall sehr stark, sodass in der Praxis ein Kontaktmittel an der
Grenzfläche von dem untersuchten Material aufgetragen wird.
Wenn Schall auf eine Grenzfläche trifft, wird ein Teil reflektiert -- der Anteil wird durch den
Reflexionskoeffizienten
\begin{equation}
	R =(\frac{Z_1-Z_2}{Z_1+Z_2})^2
\end{equation}
bestimmt -- und ein Teil transmittiert (Anteil ergibt sich aus $T=1-R$).

Ultraschall kann beispielsweise mit dem piezo-elektrischen Effekt erzeugt werden.
Durch ein elektrisches Wechselfeld kann ein piezo-elektrischer Kristall -- z.B. Quarze -- zu
Schwingungen angeregt werden. Wird der Kristall mit
Schwingungen, die der Eigenfrequenz des Kristalls entsprechen, angeregt, werden
Ultraschallwellen mit höherem Schalldruck ausgesandt. Piezokristalle können auch durch
Wechselwirkung mit Schallwellen zu Schwingungen angeregt werden.
Im Folgenden werden die grundlegenden Messverfahren mit Ultraschall erläutert.
\subsection{Messverfahren mit Ultraschall}
Messungen mit Ultraschallwellen werden durch Laufzeitmessungen realisiert.
Es lässt sich zwischen zwei elementaren Messverfahren unterscheiden:
Das Durchschallungs-Verfahren bringt keine Aussage über die Lage einer Fehlstelle in einem
Material. Mit einem Ultraschallsender wird ein Impuls am Anfang des Probestücks ausgesendet
und am Ende des Probestücks wird die abgeschwächte Intensität des Impulses gemessen (vgl.
Abbildung \ref{fig:durchschall}).

Das Impuls-Echo-Verfahren hingegen ermöglicht eine Aussage über die Fehlstelle des Probestücks.
Hierbei wird der ausgesandte Impuls nach einer Reflexion am Ende des Probestücks am Anfang des
Probestücks gemessen (vgl. Abbildung \ref{fig:echo}).
Liegt eine Fehlstelle im Probestück vor, wird der Schallimpuls an dieser reflektiert und die
verkürzte Laufzeit $t$ legt die Lage der Fehlstelle gemäß
\begin{equation}
	\label{eqn:laufzeit}
	s = \frac{1}{2} c t
\end{equation}
fest.



\cite{Anleitung}
