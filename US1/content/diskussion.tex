\section{Diskussion}
\label{sec:Diskussion}
Die mit der Cepstrumanalyse bestimmten Dicken der Platten sind mit zugehörigen gemessenen Dicken
und Fehlern in Tabelle \ref{tab:cepstru} zu finden.
Es lässt sich anmerken, dass die Fehler im Rahmen der Messungenauigkeit liegen.
\begin{table}
	\centering
	\caption{Messergebnisse für die Dicke der Platten mit zugehörigen gemessenen Dicken und relativen Fehlern.}
	\label{tab:cepstru}
	\begin{tabular}{cccc}
	\toprule
		Platte & $d_{\mathrm{exp}}$ / $\si{\milli\meter}$ & $d_{\mathrm{theo}}$ / $\si{\milli\meter}$ & Fehler \\
	\midrule
		1 & \SI{6,1(2)}{\milli\meter} & \SI{6,1}{\milli\meter} & \SI{0}{\percent} \\
		2 & \SI{10,2(2)}{\milli\meter} & \SI{10,0}{\milli\meter} & \SI{2}{\percent} \\
	\bottomrule
	\end{tabular}
\end{table}

In Tabelle \ref{tab:augentheorie} sind die die bestimmten Abstände im Auge mit den zugehörigen
Literaturwerten und relativen Fehlern aufgetragen.
\begin{table}
	\centering
	\caption{Messergebnisse für die Abstände im Auge mit zugehörigen Literaturwerten \cite{auge} und relativen Fehlern.}
	\label{tab:augentheorie}
	\begin{tabular}{cccc}
	\toprule
		Abstand & $d_{\mathrm{exp}}$ / $\si{\milli\meter}$ & $d_{\mathrm{theo}}$ / $\si{\milli\meter}$ & Fehler \\
	\midrule
		Iris -- Anfang Linse & \SI{1,85}{\milli\meter} & \SI{3,6}{\milli\meter} & \SI{48,6}{\percent} \\
		Anfang Linse -- Ende Linse & \SI{3,09}{\milli\meter} & \SI{3,6}{\milli\meter} & \SI{14,2}{\percent} \\
		Ende Linse -- Retina & \SI{10,74}{\milli\meter} & \SI{15,18}{\milli\meter} & \SI{29,2}{\percent} \\
	\bottomrule
	\end{tabular}
\end{table}
Es ist auffällig, dass die Bereiche mit der Glaskörperflüssigkeit einen deutlich höhren 
Fehler aufweisen, was mit der verwendeten -- möglicherweise falschen --
Schallgeschwindigkeit $c_{\mathrm{GK}}$ 
zusammenhängen kann. 

