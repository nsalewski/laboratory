\section{Diskussion}
\label{sec:Diskussion}
\begin{table}
	\centering
	\caption{Messergebnisse für die bestimmten Schallgeschwindigkeiten im Vergleich zum Literaturwert.}
	\label{tab:schall}
	\begin{tabular}{cccc}
		\toprule
		Messverfahren& $c_\mathrm{Ex.}$ / $\si{\meter\per\second}$ & $c_\mathrm{Theo.}$ / $\si{\meter\per\second}$ & Fehler \\
		\midrule
		Impuls-Echo-Verfahren&\num{2738(11)}&\num{2750}&\SI{0.4}{\percent}\\
		Durchschallverfahren&\num{2770(40)}&\num{2750}&\SI{0.7}{\percent}\\
		\bottomrule
	\end{tabular}
\end{table}
In Tabelle \ref{tab:schall} sind die bestimmten Schallgeschwindigkeiten in Acryl im Vergleich zum Literaturwert nach \cite{schall} aufgetragen.
Es zeigen sich nur sehr geringe Abweichungen. Zudem weichen die beiden über die verschiedenen Messverfahren bestimmten Schallgeschwindigkeiten lediglich um $1\%$ voneinander ab, beziehungsweise liegen sogar im gegenseitigen Fehlerbereich.
Dies deutet auf eine in sich konsistente Messung der Schallgeschwindigkeit hin.
Die mit der Cepstrumanalyse bestimmten Dicken der Platten sind mit zugehörigen gemessenen Dicken
und Fehlern in Tabelle \ref{tab:cepstru} zu finden.
Es lässt sich anmerken, dass die Fehler im Rahmen der Messungenauigkeit liegen.
\begin{table}
	\centering
	\caption{Messergebnisse für die Dicke der Platten mit zugehörigen gemessenen Dicken und relativen Fehlern.}
	\label{tab:cepstru}
	\begin{tabular}{cccc}
	\toprule
		Platte & $d_{\mathrm{exp}}$ / $\si{\milli\meter}$ & $d_{\mathrm{theo}}$ / $\si{\milli\meter}$ & Fehler \\
	\midrule
		1 & \num{6,1(2)}& \num{6,1}& \SI{0}{\percent} \\
		2 & \num{10,2(2)}& \num{10,0}& \SI{2}{\percent} \\
	\bottomrule
	\end{tabular}
\end{table}
\begin{table}
	\centering
	\caption{Messergebnisse für die Abstände im Auge mit zugehörigen Literaturwerten \cite{auge} und relativen Fehlern.}
	\label{tab:augentheorie}
	\begin{tabular}{cccc}
		\toprule
		Abstand & $d_{\mathrm{exp}}$ / $\si{\milli\meter}$ & $d_{\mathrm{theo}}$ / $\si{\milli\meter}$ & Fehler \\
		\midrule
		Iris -- Anfang Linse & \num{1,85} & \num{3,6}& \SI{48,6}{\percent} \\
		Anfang Linse -- Ende Linse & \num{3,09} & \num{3,6}& \SI{14,2}{\percent} \\
		Ende Linse -- Retina & \num{10,74} & \num{15,18} & \SI{29,2}{\percent} \\
		\bottomrule
	\end{tabular}
\end{table}
In Tabelle \ref{tab:augentheorie} sind die die bestimmten Abstände im Auge mit den zugehörigen
Literaturwerten und relativen Fehlern aufgetragen.
Es ist auffällig, dass die Bereiche mit der Glaskörperflüssigkeit einen deutlich höhren
Fehler aufweisen, was mit der verwendeten -- möglicherweise falschen --
Schallgeschwindigkeit $c_{\mathrm{GK}}$
zusammenhängen kann.
\FloatBarrier
\newpage
