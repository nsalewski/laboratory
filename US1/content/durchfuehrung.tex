\section{Durchführung}
\label{sec:Durchführung}

\subsection{Versuchsaufbau}
\label{sec:Versuchsaufbau}
Der Versuchsaufbau besteht aus einem Ultraschallechoskop, an dessen Ausgänge zwei Ultraschallsonden mit \SI{2}{\mega\Hz} gekoppelt sind, und einem Rechner zur Datenaufnahme und -analyse.\\
An das Ultraschallechoskop sind zwei Ultraschallsonden angeschlossen, mithilfe derer sich sowohl eine Impuls-Echo-Messung, als auch eine Durchschallmessung realisieren lässt.
Am Rechner werden die gemessenen Daten mittels des Programms \textquote{Echoview} ausgewertet.\\
Hierbei ist \textquote{Echoview} in der Lage, vier verschiedene Diagramme darzustellen.
Im linken oberen Graphen wird der A-Scan dargestellt, also die Amplitude gegen die Zeit aufgetragen.
Der linke untere Graph stellt die gewählte Verstärkung dar. Die Verstärkung lässt sich am Ultraschallechoskop über die Drehknöpfe zur laufzeit-bzw. tiefenabhängigen Verstärkung (TGC; Time Gain Control) und ebenso über die Verstärkung des Outputs und der Empfindlichkeit der Sonden regulieren.\\
Zu Beachten ist, dass eine Verstärkung nur gewählt werden darf, wenn die auszuwertende Messreihe nicht zur Untersuchung der Amplitudenhöhe dient.
Die beiden rechten Graphiken sind das berechnete Spektrum der Messdaten (FFT), bzw. ihr
Cepstrum.
Erzeugte Graphiken und Messdaten können aus dem Programm heraus exportiert werden.
Als zu untersuchende Versuchsobjekte stehen Acrylzylinder verschiedener Länge, Acrylplatten unterschiedlicher Dicke sowie das Modell eines menschlichen Auges im Maßstab 3:1 zur Verfügung.


\subsection{Versuchsbeschreibung}
\label{sec:Versuchsbeschreibung}
Zu Beginn der Messung werden die verwendeten Acrylzylinder ebenso wie die Acrylplatten mit einer Schiebelehre vermessen.
Über einen Drehschalter lässt sich das Messverfahren am Ultraschallechoskop einstellen.
Für das Impuls-Echo-Verfahren ist dieser Drehschalter auf die Position 1/1, bzw. 2/2 zu drehen, jenachdem mit welcher Sonde die Messung durchgeführt wird.\\
Für das Durchschallungsverfahren wird der Modus 1/2 bzw. 2/1 gewählt.
Eine Sonde ist hierbei jeweils der Sender und die andere Sonde dient als Empfänger.
Für die Messung der Laufzeit mit dem Impuls-Echo Verfahren wird eine Sonde über bidestilliertes Wasser an einen Acrylzylinder gekoppelt, welcher auf ein Papiertuch zu stellen ist, um Kratzer am Versuchsobjekt zu vermeiden.\\
In das Messprogramm ist zur Berechnung der Eindringtiefe die Schallgeschwindigkeit $c$ nach den Literaturdaten, bzw. die durch eine Laufzeitmessung nach Formel \eqref{eqn:laufzeit} ermittelte Schallgeschwindigkeit einzutragen.
Um Impulse eindeutiger identifizieren zu können, kann für Laufzeitmessungen die Amplitude über die Einstellung am Ultraschallechoskop verstärkt werden.\\
Die Messung mit dem Impuls-Echo-Verfahren zur Bestimmung der Schallgeschwindigkeit soll für sieben verschiedene Zylinderlängen erfolgen. Hierfür können auch mehrere Zylinder gestapelt werden
Mittels des Impuls-Echo-Verfahrens soll zudem die Dämpfungskonstante des Zylindermaterials bestimmt werden. Zu Beachten ist, dass hierfür keinesfalls eine Verstärkung verwendet werden darf. Außerdem dürfen keine Zylinder gestapelt werden, da sich an der Übergangsfläche sowohl eine reflektierte, als auch eine transmittierte Schallwelle bildet, somit ist eine Bestimmung des Amplitudenverhältnis nicht sinnvoll.
Es sollen möglichst viele Zylinderlängen vermessen werden. Für große Zylinderlängen kann allerdings die Amplitude der reflektierten Schallwelle so gering ausfallen, dass sie nicht mehr zu bestimmen ist.
\\Zudem soll die Schallgeschwindigkeit für möglichst viele Zylinder mit dem Durchschallverfahren bestimmt werden. Hierfür wird beidseitig an den Acrylzylinder, welcher in eine Halterung waagerecht gelegt wird, jeweils eine Sonde mit Kopplungsgel gekoppelt.
Zur Analyse eines Spektrums/Cepstrums wird ein etwa \SI{40}{\milli\meter} langer Acrylzylinder auf zwei Acrylplatten gesetzt sowie eine Sonde an den Zylinder gekoppelt.
Alle Grenzflächen werden durch bidestilliertes Wasser gekoppelt. Die Verstärkung ist so zu wählen, dass 3 Mehrfachreflexionen zu erkennen sind.
Im A-Scan werden die beiden Cursor so neben die Mehrfachreflexionen gesetzt, dass ein Spektrum/Cepstrum mittels der FFT-Funktion erzeugt werden kann.\\
Zur Untersuchung der Innenabstände des Augenmodells wird eine Messung mit dem Impuls-Echo-Verfahren durchgeführt. Hierzu wird eine Sonde mit Koppelgel vorsichtig auf die Hornhaut des Augenmodells gesetzt und vorsichtig so ausgerichtet, dass im Messprogramm fünf Peaks zu erkennen sind (Reflexion durch die Hornhaut, Iris, Vorder- und Rückseite der Linse und der Retina).
