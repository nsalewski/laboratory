\section{Durchführung}
\label{sec:Durchführung}

\subsection{Versuchsaufbau}
\label{sec:Versuchsaufbau}
Der Versuchsaufbau besteht aus einem Ultraschallechoskop, an dessen Ausgänge zwei Ultraschallsonden mit \SI{2}{\mega\Hz} gekoppelt sind, und einem Rechner zur Datenaufnahme und -analyse.\\
An das Ultraschallechoskop sind zwei Ultraschallsonden angeschlossen, mithilfe derer sich sowohl eine Impuls-Echo-Messung, als auch eine Durchschallmessung realisieren lässt.
Am Rechner werden die gemessenen Daten mittels des Programms \textquote{Echoview} ausgewertet.\\
Hierbei ist \textquote{Echoview} in der Lage, vier verschiedene Diagramme darzustellen.
Im linken oberen Graphen wird der A-Scan dargestellt, also die Amplitude gegen die Zeit aufgetragen.
Der linke untere Graph stellt die gewählte Verstärkung dar. Die Verstärkung lässt sich am Ultraschallechoskop über die Drehknöpfe zur laufzeit-bzw. tiefenabhängigen Verstärkung (TGC; Time Gain Control) und ebenso über die Verstärkung des Outputs und der Empfindlichkeit der Sonden regulieren.\\
Zu Beachten ist, dass eine Verstärkung nur gewählt werden darf, wenn die auszuwertende Messreihe nicht zur Untersuchung der Amplitudenhöhe dient.
Die beiden rechten Graphiken sind das berechnete Spektrum der Messdaten (FFT), bzw. ihr
Cepstrum.
Erzeugte Graphiken und Messdaten können aus dem Programm heraus exportiert werden.
Als zu untersuchende Versuchsobjekte stehen Acrylzylinder verschiedener Länge, Acrylplatten unterschiedlicher Dicke sowie das Modell eines menschlichen Auges im Maßstab 3:1 zur Verfügung.


\subsection{Versuchsbeschreibung}
\subsubsection{für die Ablenkung im elektrischen Feld}
In der ersten Messung soll die Proportionalität zwischen der Leuchtfleckverschiebung $D$ und der 
Ablenkspannung $U_{\mathrm{d}}$ überprüft werden. Diese Messung wird für drei verschiedene 
Beschleunigungsspannungen $U_{\mathrm{B}}$ durchgeführt.
Bevor die Spannungsversorgung der Kathodenstrahlröhre angeschaltet wird, muss der Schalter 
circa eine Minute als Anheizzeit in der Stellung "STANDBY" stehen. Wenn der Leuchtfleck
zu unscharf wird, kann dieser durch Regulierung der Fokussierungs- und 
Wehnelt-Spannung fokussiert werden.
Das Koordinatennetz, auf dem die Verschiebung des Leuchtflecks abgelesen wird, hat neun 
äquidistante (Abstand: $1 inch$) Linien. Der Leuchtfleck soll bei Variation der 
Ablenkspannung $U_{\mathrm{d}}$ auf die neun Linien positioniert werden. Ist dies der Fall,
wird jeweils die Verschiebung als $n*1inch$ und die zugehörige Ablenkspannung 
$U_{\mathrm{d}}$ abgelesen. Es ergeben sich also neun Messtupel.

Für die Bestimmung einer unbekannten Frequenz einer Sinusspannung wird der 
Kathodenstrahl-Oszillosgraph verwendet. Es wird Sägezahnfrequenz solange variiert bis 
stehende Wellen zu erkennen sind. Dann wird die zugehörige Frequenz notiert. 
Es sollen vier Frequenzen für die Verhältnisse $n = \frac{1}{2}, 1, 2$ und $3$ bestimmt werden.
Damit der Leuchtschirm nicht beschädigt wird, sollte der Elektronenstrahl nicht zu lange auf 
die gleiche Stelle abgelenkt werden.

\subsubsection{für die Ablenkung im magnetischen Feld}
Die Leuchtfleckverschiebung $D$ in Abhängigkeit von der Flussdichte $B$ wird für die 
Beschleunigungsspannungen $U_{\mathrm{B}}=\SI{250}{\volt}$ und $U_{\mathrm{B}}=\SI{450}{\volt}$
gemessen. Damit nur das Magnetfeld des Spulenpaares auf die Kathodenstrahlöhre wirkt
wird die Kathodenstrahlöhre in Richtung der Horizontalkomponente des Erdmagnetfeldes verschoben.
Diese Richtung kann mit dem Deklinatorium-Inklinatorium bestimmt werden, indem dieser einen 
Winkel von $0°$ anzeigt. Der Strom soll wieder abgelesen werden, wenn sich der 
Leuchtfleck auf den äquidistanten Linien befindet.

Für die Bestimmung der Horizontalkomponente des Erdmagnetfeldes $B_{\mathrm{hor}}$ wird eine 
möglichst niedrige Beschleunigungsspannung verwendet. 
Der Spulenstrom wird ausgeschaltet und die Kathodenstrahlröhre wieder in die Richtung der 
Horizontalkomponente des Erdmagnetfeldes gedreht. Die Position des abgelenkten Leuchtflecks 
auf dem Schirm wird markiert. Daraufhin wird die Kathodenstrahlröhre um $90°$ gedreht und 
der Spulenstrom wieder hochgeregelt bis das Erdmagnetfeld wieder auf die vorherige Ablenkung
kompensiert wird. Der kompensierende Spulenstrom wird notiert.
