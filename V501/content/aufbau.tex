\subsection{Versuchsaufbau}
\label{sec:Versuchsaufbau}
\subsubsection{Aufbau zur Ablenkung durch ein elektrisches Feld}
Der Aufbau zur Untersuchung der Proportionalität zwischen der Leuchtfleckverschiebung $D$ und 
der Ablenkspannung $U_{\mathrm{d}}$ ist in Abbildung \ref{fig:verschiebungefeld} dargestellt.
Dieser besteht aus der Kathodenstrahlröhre und einem Voltmeter zur Spannungsmessung. Für den 
Schirm am Ende der Kathodenstrahlröhre soll das Y-System eingestellt werden, sodass die 
Verschiebung nur in y-Richtung gemessen wird. Es ist darauf zu achten, dass eine Platte 
der Ablenksysteme geerdet ist.

Ein Kathodenstrahl-Oszillograph kann mit einem Aufbau wie in Abbildung \ref{fig:oszillo}
realisiert werden. Hierbei wird im Vergleich zum vorherigen Aufbau nur ein Sinusgenerator für 
die Spannungsversorgung der in y-Richtung ablenkenden Platten und ein Sägezahngenerator 
für die Spannungsversorgung der in x-Richtung ablenkenden Platten hinzugefügt. 
Da bei dem Messvorgang die Sägezahnfrequenz bestimmt werden soll, wird außerdem ein 
Frequenzzähler verwendet.

\subsubsection{Aufbau zur Ablenkung durch ein magnetiches Feld}
Für die Messung der Leuchtfleckverschiebung durch ein magnetisches Feld wird der Aufbau wie 
in \ref{fig:verschiebungefeld} verwendet. Zusätzlich wird ein Magnetfeld in Form eines 
Helmholtz-Spulenpaares senkrecht zur Kathodenstrahlröhre erzeugt. Der Spulenstrom kann manuell
reguliert werden.
Zur Bestimmung der Richtung des Erdmagnetfeldes wird ein Deklinatorium-Inklinatorium 
verwendet.
