\section{Diskussion}
\label{sec:Diskussion}
Ein Vergleich der experimentell bestimmten Apparaturkonstanten $K_\mathrm{Ex.}=\SI{53(1)}{\centi\meter}$ mit der aus den Apparaturdaten bestimmten Apparaturkonstanten $K=\SI{35.75}{\centi\meter}$ zeigt eine Abweichung von etwa $48\%$. Eine mögliche Erklärung für die recht große Abweichung ist, dass die Elektronen auf ihrem Weg von der Beschleunigungselektrode zu den Ablenkplatten gebremst werden und ihre Geschwindigkeit beim Durchlaufen des elektrischen Feldes geringer ist als nach dem Passieren der Beschleunigungselektrode. Damit würden die Elektroden eine längere Zeit $\Delta t$ zum Passieren der Ablenkplatten benötigen und daher eine größere Ablenkung erfahren.\\
Die Elektronen könnten beispielsweise gebremst werden, indem sie mit Luftmolekülen wechselwirken. Dies würde bedeuten, dass das Vakuum in der Kathodenstrahlröhre eventuell nicht so gut ist wie angenommen und daher nicht so gut als ideales Vakuum genähert werden kann.
Selbiger Grund lässt sich für die großen Abweichungen zwischen den theoretischen und den experimentell bestimmten Werten für die spezifische Elektronenladung $\frac{\symup{e}_0}{\symup{m}_0}$ anbringen. Wird angenommen, dass auch hier das Vakuum in der Kathodenstrahlröhre nicht optimal ist, folgt, dass die Elektronen auch hier länger brauchen, um das magnetische Feld zu passieren. Auf sie hat also effektiv bei Eintritt in das magnetische Feld eine geringere Beschleunigungsspannung gewirkt, da sie auf dem Weg zwischen Beschleunigungselektrode und Ablenkplatten ebenfalls durch Luftmoleküle gebremst wurden.
\\Für die Totalintensität des Erdmagnetfelds ergibt sich im Experiment ein Wert von $B_\mathrm{Ex.}=\SI{6.16e-05}{\tesla}$. Ein Vergleich mit dem Literaturwert $B_\mathrm{Theo.}=\SI{4.9046e-05}{\tesla}$ nach \cite{bob} zeigt eine Abweichung von etwa $25\%$. Mögliche Fehler lassen sich mit Störfeldern in der Umgebung erklären. So wurde auch das Deklinatorium-Inklinatorium durch Störfelder, beispielsweise durch den Versuchsaufbau verursacht, bereits beeinflusst und ist ein weiterer möglicher Fehler in der Messung.
