\section{Diskussion}
\label{sec:Diskussion}
Ein Vergleich der experimentell bestimmten Apparaturkonstanten $K_\mathrm{Experiment}=\SI{53(1)}{\centi\meter}$ mit der aus den Apparaturdaten bestimmten Apparaturkonstanten $K=\SI{35.75}{\centi\meter}$ zeigt eine Abweichung von etwa $48\%$. Eine mögliche Erklärung für die recht große Abweichung ist, dass die Elektronen auf ihrem Weg von der Beschleunigungselektrode zu den Ablenkplatten gebremst werden und ihre Geschwindigkeit beim Durchlaufen des elektrischen Feldes geringer ist als nach dem Passieren der Beschleunigungselektrode. Damit würden die Elektroden eine längere Zeit $\Delta t$ zum Passieren der Ablenkplatten benötigen und daher eine größere Ablenkung erfahren.
Die Elektronen könnten beispielsweise gebremst werden, indem sie mit Luftmolekülen wechselwirken. Dies würde bedeuten, dass das Vakuum in der Kathodenstrahlröhre eventuell nicht so gut ist wie angenommen und daher nicht so gut als ideales Vakuum genähert werden kann.
