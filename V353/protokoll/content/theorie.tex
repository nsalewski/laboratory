\section{Theorie}
\label{sec:Theorie}
Relaxation bezeichnet in der Physik den Übergang eines Systems hin zu einem stabilen Endzustand, nachdem es aus diesem entfernt wurde.
Die Änderungsgeschwindigkeit $\dot{A}$ einer physikalischen Größe $A$ ist dabei meist proportional zur Abweichung zwischen $A$ zu einem Zeitpunkt $t$ und seinem Endzustand $A(\infty)$.
Die Integration dieses Zusammenhangs vom Zeitpunkt $0$ bis zu einem Zeitpunkt $t$ liefert:
\begin{equation}
  \label{eqn:relaxation}
  A(t)=A(\infty) (  A(0)-A(\infty) ) \cdot e^{ct}\, ;\,\, mit\,\, c<0 .
\end{equation}

Auf-und Entladevorgänge von Kondensatoren sind Beispiele für Relaxationsvorgänge und sollen im Folgenden detaillierter betrachtet werden.
Bei einem geladenen Kondensator mit der Kapazität $C$ befindet sich die Ladung $Q$ auf den Platten.
Zwischen den Kondensatorplatten liegt dabei die Spannung $U_\text{C}$ vor. Mit
\begin{equation*}
  U_\text{C}=\frac{Q}{C}
\end{equation*}
und dem Ohmschen Gesetz lässt sich $\dot{Q}=I$ zu
\begin{equation}
\dot{Q}=-\frac{1}{RC}Q(t)
\end{equation}
umformen.
Da der Kondensator sich nach unendlich langer Zeit entladen wird, folgt unter Verwendung von Formel \eqref{eqn:relaxation} für den zeitlichen Verlauf der Entladung

\begin{equation}
  \label{eqn:aufladung}
  Q(t)=Q(0)(e^{-\frac{t}{RC}})
\end{equation}
Für die Aufladung des Kondensators durch eine Spannungsquelle mit $U_0$ über einen Widerstand erhält man mit ähnlichen Überlegungen unter Beachtung der Anfangsbedingungen $Q(0)=0$ und $Q(\infty)=CU_\text{0}$

\begin{equation}
  \label{eqn:entladung}
  Q(t)=CU_\text{0}(  1-e^{-\frac{t}{RC}} ) .
\end{equation}
$RC$ wird hierbei als Zeitkonstante des Systems bezeichnet.\\
\\Relaxationsvorgänge sind auch bei periodisch angeregten Systemen zu beobachten.
Liegt an einem RC-Kreis eine Wechselspannung $U(t)=U_\text{0}\cos(\omega t)$ an, und $\omega \ll \frac{1}{RC}$ gilt, ist $U_\text{C} \approx U(t)$.
Für große $\omega$ ist dies nicht mehr erfüllt, zudem nimmt die Amplitude von $U_\text{C}$ mit größer werdenden $\omega$ ab.
\begin{figure}
  \centering
  \includegraphics[width=0.7\textwidth]{bilder/periodisch.png}
  \caption{Schaltung zur Diskussion von Relaxationsvorgängen unter Einfluss von periodischer Anregung \cite{Anleitung}}
  \label{fig:periodisch}
\end{figure}
Unter Verwendung des Ansatz
\begin{equation*}
U_\text{C}=A(\omega)\cos(\omega t +\phi(\omega)) ,
\end{equation*}
erhält man mit den Kirchhoffschen Regeln und einigen Umformungen
\begin{equation}
\phi(\omega)=\arctan(-\omega RC)
\end{equation}
Weitere Umformungen liefern
\begin{equation}
A(\omega)=\frac{U_\text{0}}{\sqrt{1+\omega ^2R^2C^2}} .
\end{equation}
Für sehr kleine, beziehungsweise sehr große Frequenzen erhält man die folgenden Grenzfälle:

\begin{table*}
\centering
  \label{tab:tab1}
  \begin{tabular}{cc}

    $\omega \ll \frac{1}{RC} $&  $ \omega \gg \frac{1}{RC}$ \\
    \midrule
    keine Phasenverschiebung & $\phi \to \frac{\pi}{2}$\\
    $A(\omega) \to U_\text{0} $ & $A(\omega) \to 0$\\
  \end{tabular}
\end{table*}

Somit ist sofort ersichtlich, dass das RC-Glied als Tiefpass verwendet werden kann. Große Frequenzen werden weitestgehend herausgefiltert und kleine Frequenzen passieren das RC-Glied nahezu unverändert. \\
\\Für genügend große Frequenzen  $\omega$ ergibt sich aus \ref{fig:periodisch} wiederum mit den Kirchhoffschen Regeln, dass $U_\text{C}$ vernachlässigbar klein ist, also $U(t) \approx U_\text{R}(t)$ gilt.
Mit Umformungen erhält man
\begin{equation}
  U_\text{C}(t)=\frac{1}{RC} \int_{0}^{t} U(t') dt'
\end{equation}
Der RC-Kreis kann also verwendet werden,  um eine angelegte Spannung $U(t)$ zu integrieren.
