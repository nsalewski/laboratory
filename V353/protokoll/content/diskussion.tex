\section{Diskussion}
\label{sec:Diskussion}
Bereits bei der Verwendung der ersten Methode zur Bestimmung der Zeitkonstanten $RC$ (vgl. Abschnitt \ref{sec:RC}) fiel %diesmal hab ichs richtig geschrieben :P
auf, dass der Generator keine reine Rechteckspannung lieferte, sondern Schwankungen und Abweichungen von der idealen Rechteckspannung zeigte.

\begin{table}
	\centering
	\caption{Alle bestimmten Zeitkonstanten $RC$ samt ihrer prozentualen Fehler}
	\label{tab:diskussion2}
	\begin{tabular}{cccc}
		\toprule
		$RC$/$10^{-3} \si{\second}$  bestimmt über... & ...Entladekurve & ...$A(\omega)$ & ...Phasenverschiebung \\
		\midrule
		Berechneter Wert \pm Fehler                    & (0.48 \pm 0.03) & (6.2 \pm 0.5)  & (3,67 \pm 0,45)       \\
		Prozentualer Fehler                            & 6.25\%          & 8.1\%          & 13.1\%                \\
		\bottomrule
	\end{tabular}
\end{table}
Die bestimmten Zeitkonstanten $RC$ (vgl. \ref{tab:diskussion2}) weichen deutlich voneinander ab. Dies deutet, ebenso wie die starke Abweichung der Messwerte von der Regressionskurve
in Abbildung \ref{fig:plotb} für kleine Frequenzen ($\omega \ll 35\,\si{\Hz}$) darauf hin, dass die Messung fehlerbehaftet war. Beispielsweise könnte ein Wackelkontakt an einer der Steckverbindungen vorgelegen haben.
Des Weiteren war das Maximum von $A(\omega)$ bei etwa $35 \,\si{\Hz}$ nicht zu erwarten. Für kleine Frequenzen scheint der Funktionengenerator nicht die gleiche Spannungsamplitude liefern zu können, wie für große Frequenzen.
Wie in Tabelle \ref{tab:diskussion} zu sehen, scheint hier ein Defekt am Funktionengenerator vorzuliegen. Erwartet worden wäre, dass die gelieferte Spannungsamplitude frequenzunabhängig ist, also nahezu konstant bleibt.

\begin{table}
	\centering
	\caption{Starke Frequenzabhängigkeit der Generatorspannung für niedrige Frequenzen}
	\label{tab:diskussion}
	\begin{tabular}{ccc}
		\toprule
		$\omega$/$\si{\Hz}$ & $U_\text{C}$/ $\si{\volt}$ & $U(t)$/ $\si{\volt}$ \\
		\midrule
		4.24                & 2.80                       & 56.0                 \\
		10                  & 4.40                       & 88.0                 \\
		15                  & 4.88                       & 97.6                 \\
		20                  & 5.04                       & 102.0                \\
		35                  & 5.20                       & 107.0                \\
		\bottomrule
	\end{tabular}
\end{table}
Diese Frequenzabhängigkeit der Generatorspannung für kleine Frequenzen ist sehr wahrscheinlich der Grund, warum sowohl in der Bestimmung der Zeitkonstanten $RC$ über die Frequenzabhängigkeit der Amplitude in
Abbildung \ref{fig:plotb} als auch im Polarplot die Tupel $\{A(\omega_{\text{i}}), \phi(\omega_{\text{i}}) \}$ (siehe Abbildung \ref{fig:polari}) so stark von der Theoriekurve abweichen.

Außerdem zu erwähnen ist, dass die gemessenen Werte im hochfrequenteren Bereich eine geringe
Diskrepanz zur Theoriekurve zeigen. Diese Abweichungen lassen sich auf zufällige Fehler zurückführen.
Die Integratorfunktion des RC-Gliedes konnte außerdem - bei erforderlichen hohen Frequenzen -
nachgewiesen werden.

Weiterhin müssen systematische Fehler - wie zum Beispiel die Vernachlässigung der Innenwiderstände von Generator und Oszilloskop - berücksichtigt werden.
