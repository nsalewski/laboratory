\section{Fehlerrechnung}
\label{sec:fehlerrechnung}
Alle berechneten Mittelwerte werden mit folgender Formel bestimmt:
\begin{equation}
  \label{eqn:mittelwert}
  \overline x=\frac{1}{N}\sum \limits_{i=1}^{N} x_i .
\end{equation}
Der zugehörige Fehler des Mittelwerts bestimmt sich mit:

\begin{equation}
  \label{eqn:mittelwertfehler}
  \Delta \overline x= \frac{1}{\sqrt{N}} \sqrt{\frac{1}{N-1} \sum \limits_{i=1}^{N} (x_i- \overline x)^2}.
\end {equation}
Wenn fehlerbehaftete Größen in einer späteren Formel weiter verwendet werden, so wird der sich fortpflanzende Fehler
mit Hilfe der Gauß’schen Fehlerfortpflanzung berechnet:

\begin{equation}
  \label{eqn:fehlerfortpflanzung}
  \Delta f = \sqrt{ \sum \limits_{i = 1}^{N} (\frac{\partial f}{\partial x_i})^2 \cdot (\Delta x_i)^2}.
\end{equation}
Haben Messgeräte baubedingte Unsicherheiten, so errechnen sich die Fehler des Mittelwerts nach der Regel zur Fehlerfortpflanzung von Gerätefehlern wie folgt:
 \begin{equation}
   \frac{\Delta z}{z}=\sqrt{(\frac{\Delta x}{x})^2+(\frac{\Delta y}{y})^2}
\end{equation}

Die Regression von Polynomen und Ausgleichsgrade, sowie die Bestimmung der zugehörigen Fehler werden mit
IPython 5.1.0 mittels Scipy 0.18.1 durchgeführt.
Parameter eventueller Ausgleichsgeraden
\begin{equation}
\label{eqn:ausgleichsgrade}
y=a \cdot x +b .
\end{equation}
werden bestimmt über

\begin{equation}
\label{eqn:ausgleichsgrade_a}
a= \frac{ \overline{xy}- \overline{x} \overline{y}}{\overline{x^2}-\overline{x}^2} .
\end{equation}

\begin{equation}
\label{eqn:ausgleichsgrade_b}
b= \frac{ \overline{x^2} \overline{y}- \overline{x} \cdot \overline{xy}}{\overline{x^2}-\overline{x}^2} .
\end{equation}
