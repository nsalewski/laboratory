\section{Auswertung}
\label{sec:Auswertung}
\subsection{Bestimmung der Zeitkonstanten RC}
Zur Bestimmung der Zeitkonstanten RC werden aus der Entladekurve in Abbildung \ref{fig:plotrc} die Werte für $U_\text{C}$ abgelesen.
$U_\text{0}$ wurde bestimmt zu $U_\text{0}=\SI{7.28}{\ohm}$.
Die abgelesenen Daten finden sich in Tabelle \ref{tab:taba}.
Durch Umformung wird Formel \eqref{eqn:aufladung} zu
\begin{equation}
	\label{eqn:ausgleich1}
	\ln\left(\frac{U_\text{C}}{U_\text{0}}\right)=-\frac{1}{RC}t .
\end{equation}
Zur Bestimmung der Zeitkonstanten $RC$ wird nun $\ln\left(\frac{U_\text{C}}{U_\text{0}}\right)$ gegen die Zeit $t$ aufgetragen.
Aus der Steigung der Regressionsgeraden nach Gleichung \eqref{eqn:ausgleich1} ergibt sich nun die berechnete Zeitkonstante zu

\begin{equation*}
	\centering
	RC = (0.48 \pm 0.03) \cdot 10^{-3} \si{\second} .
\end{equation*}
\begin{table}
	\caption{Messdaten zur Bestimmung der Zeitkonstanten $RC$}
	\label{tab:taba}
	\centering
	\begin{tabular}{ccc}
		\toprule
		$U_\text{C}$/$\si{\volt}$ & $\ln{(\frac{U_\text{C}}{U_\text{0}})}$ & $t$ /$10^{-3}\si{\second}$ \\
		\midrule
		6.04                      & -0.186726850262                        & 0.0                        \\
		4.04                      & -0.588886170236                        & 0.25                       \\
		2.76                      & -0.96990018248                         & 0.5                        \\
		1.8                       & -1.39734419731                         & 0.75                       \\
		1.08                      & -1.90816982107                         & 1.0                        \\
		0.56                      & -2.56494935746                         & 1.25                       \\
		0.24                      & -3.41224721785                         & 1.5                        \\
		\bottomrule
	\end{tabular}
\end{table}
Die Messdaten sowie die berechnete lineare Regression befinden sich in Abbildung \ref{fig:plota}.
\begin{figure}
	\centering
	\includegraphics[angle=90]{bilder/F0000TEK.JPG}
	\caption{Aufgabenteil a: Entladekurve des Kondensators zur Bestimmung der Zeitkonstanten $RC$}
	\label{fig:plotrc}
\end{figure}

\begin{figure}
	\centering
	\includegraphics{build/test.pdf}
	\caption{Lineare Regression zur Bestimmung der Zeitkonstanten $RC$}
	\label{fig:plota}
\end{figure}

%%%%%%%%%%%%%%%%%%%%%%%%%%%%%%%%%%%%%%%%%%%%%%%%%%%%%%%%%%%%%%%%%%%%%%%%%%%%%%%%%%%%%%%%%%%

Die Zeitkonstante $RC$ soll nun erneut bestimmt werden, indem die Frequenzabhängigkeit der Amplitude $A(\omega)$ der Spannung $U_\text{C}$ am Kondensator untersucht wird.
Dazu wird $A(\omega)$ in Abhängigkeit zur Frequenz $\omega$ aufgetragen. Die verwendeten Daten befinden sich in Tabelle \ref{tab:tab2}.
Eine Regression mit Gleichung \eqref{eqn:amplitude} liefert für die Zeitkonstante :
\begin{equation*}
	\centering
	RC=(10 \pm 3)\cdot 10^{-3} \si{\second}
\end{equation*}

\begin{table}
	\caption{Messdaten zur Bestimmung der Zeitkonstanten $RC$ über die Frequenzabhängigkeit der Amplitude $A(\omega)$}
	\label{tab:tab2}
	\centering
	\begin{tabular}{cc}
		\toprule
		$A(\omega)$/ $10^{-3} \si{\volt}$ & $\omega$/$\si{\Hz}$ \\
		\midrule
		2800.0                            & 4.24                \\
		4400.0                            & 10.0                \\
		4880.0                            & 15.0                \\
		5040.0                            & 20.0                \\
		5200.0                            & 35.0                \\
		5120.0                            & 60.0                \\
		4880.0                            & 90.0                \\
		4320.0                            & 150.0               \\
		3600.0                            & 230.0               \\
		2880.0                            & 320.0               \\
		2080.0                            & 500.0               \\
		1120.0                            & 1000.0              \\
		680.0                             & 2000.0              \\
		230.0                             & 5000.0              \\
		120.0                             & 10000.0             \\
		\bottomrule
	\end{tabular}
\end{table}

Die Messdaten und die berechnete Regression sind in Abbildung \refeq{fig:plotb} abgebildet.
\begin{figure}
	\centering
	\includegraphics{build/amplitude.pdf}
	\caption{Regression zur Bestimmung der Zeitkonstanten $RC$ über die Frequenzabhängigkeit der Amplitude}
	\label{fig:plotb}
\end{figure}



Nun wird die Zeitkonstante ein weiteres Mal mithilfe der Phasenverschiebung zwischen Generator- und Kondensatorspannung bestimmt.
Die Phasenverschiebung $\phi$ ergibt sich zu
\begin{equation*}
	\phi = \frac{a}{b} \, 2\pi \text{,}
\end{equation*}
wobei $a$ dem gemessenen zeitlichen Abstand der Nulldurchgänge der Spannungskurven entspricht und $b$ als Periodendauer durch $b = \frac{1}{\omega}$ bestimmt wird.

\begin{table}
	\caption{Messdaten zur Bestimmung der Zeitkonstanten $RC$ über die Phasenverschiebung $\phi$}
	\label{tab:phasen}
	\centering
	\begin{tabular}{cccc}
		\toprule
		$\omega$ / $\si{\Hz}$ & $a$ / $\si{\second}$ & $b$ / $\si{\second}$ & $\phi$ / $\si{\radian}$ \\
		\midrule
		4.2                   & 0.00000              & 0.23585              & 0.00000                 \\
		10.0                  & 0.00020              & 0.10000              & 0.01257                 \\
		15.0                  & 0.00076              & 0.06667              & 0.07163                 \\
		20.0                  & 0.00080              & 0.05000              & 0.10053                 \\
		35.0                  & 0.00072              & 0.02857              & 0.15834                 \\
		60.0                  & 0.00068              & 0.01667              & 0.25635                 \\
		90.0                  & 0.00067              & 0.01111              & 0.37888                 \\
		150.0                 & 0.00063              & 0.00667              & 0.59376                 \\
		230.0                 & 0.00052              & 0.00435              & 0.75147                 \\
		320.0                 & 0.00044              & 0.00313              & 0.88467                 \\
		500.0                 & 0.00033              & 0.00200              & 1.03673                 \\
		1000.0                & 0.00018              & 0.00100              & 1.13097                 \\
		2000.0                & 0.00009              & 0.00050              & 1.13097                 \\
		5000.0                & 0.00005              & 0.00020              & 1.57080                 \\
		10000.0               & 0.00002              & 0.00010              & 1.25664                 \\
		\bottomrule
	\end{tabular}
\end{table}

Die sich ergebenden Werte für die Frequenz $\omega$ und die Phasenverschiebung $\phi$ sind in Tabelle \ref{tab:phasen} aufgetragen.

\begin{figure}
	\centering
	\includegraphics{build/phase.pdf}
	\caption{Regression zur Bestimmung der Zeitkonstanten $RC$ über die Phasenverschiebung von Generator- und Kondensatorspannung}
	\label{fig:phasi}
\end{figure}



Nach Formel \eqref{eqn:phase} wird eine Ausgleichskurve bestimmt (siehe Abbildung \ref{fig:phasi}
).

Für die Zeitkonstante $RC$ ergibt sich damit der Wert $RC = (3,67 \pm 0,45) \cdot 10^{-3} \, \si{\second}$.

\begin{figure}
	\centering
	\includegraphics{polaar.pdf}
	\caption{Polardarstellung von Messtupeln $\{A(\omega_{\text{i}}), \phi(\omega_{\text{i}}) \}$ und erwarteter Theoriekurve}
	\label{fig:polari}
\end{figure}

Mit der bestimmten Zeitkonstante $RC$ lassen sich in einem Polarkoordinatensystem die gemessenen Tupel $\{A(\omega_{\text{i}}), \phi(\omega_{\text{i}}) \}$ sowie eine Theoriekurve gemäß Formel ??? also
\begin{equation}
	\frac{A({\omega})}{U_0} = - \frac{\sin(\phi(\omega))}{\omega \, RC} \text{,}
\end{equation}
auftragen (siehe Abbildung \ref{fig:polari}).

\begin{table}
	\caption{Messdaten von Phasenverschiebung $\phi(\omega)$ und Relativamplitude $\frac{A(\omega)}{U_0}$}
	\centering
	\label{tab:polars}
	\begin{tabular}{cc}
		\toprule
		$\phi$ / $\si{\radian}$ & $\frac{A}{U_0}$ \\
		\midrule
		0.00000                 & 0.38462         \\
		0.01257                 & 0.60440         \\
		0.07163                 & 0.67033         \\
		0.10053                 & 0.69231         \\
		0.15834                 & 0.71429         \\
		0.25635                 & 0.70330         \\
		0.37888                 & 0.67033         \\
		0.59376                 & 0.59341         \\
		0.75147                 & 0.49451         \\
		0.88467                 & 0.39560         \\
		1.03673                 & 0.28571         \\
		1.13097                 & 0.15385         \\
		1.13097                 & 0.09341         \\
		1.57080                 & 0.03159         \\
		1.25664                 & 0.01648         \\
		\bottomrule
	\end{tabular}
\end{table}

Die benötigten Messwerte sind in Tabelle \ref{tab:polars} augetragen.
\subsection{d) Integration}


\begin{figure}
	\caption{Aufgabenteil d: Rechteckspannung}
	\centering
	\begin{subfigure}{0.48\textwidth}
		\centering
		\includegraphics[width=0.98\textwidth]{build/integration1.pdf}
		\label{fig:intrechteck}
	\end{subfigure}
	\begin{subfigure}{0.48\textwidth}
		\centering
		\includegraphics[width=0.98\textwidth]{bilder/ALL0001/F0001TEK.JPG}
		\label{fig:rechteck}
	\end{subfigure}
\end{figure}
In Abbildung \ref{fig:rechteck} ist die Rechteckspannung sowie die über das RC-Glied integrierte Rechteckspannung dargestellt.
Die integrierte Rechteckspannung ist wie zu erwarten eine Dreiecksspannung.
\begin{figure}
	\caption{Aufgabenteil d: Sinusspannung}
	\centering
	\begin{subfigure}{0.48\textwidth}
		\centering
		\includegraphics[width=0.98\textwidth]{build/integration2.pdf}
		\label{fig:intsinus}
	\end{subfigure}
	\begin{subfigure}{0.48\textwidth}
		\centering
		\includegraphics[width=0.98\textwidth]{bilder/ALL0002/F0002TEK.JPG}
		\label{fig:sinus}
	\end{subfigure}
\end{figure}

Der Spannungsverlauf der Sinusspannung ist in Abbildung \ref{fig:sinus} dargestellt. Wie zuvor ist die Kosinusspannung - integrierte Sinusspannung - außerdem in diesem Plot aufgezeichnet.

\begin{figure}
	\caption{Aufgabenteil d: Dreiecksspannung}
	\centering
	\begin{subfigure}{0.48\textwidth}
		\centering
		\includegraphics[width=0.98\textwidth]{build/integration3.pdf}
		\label{fig:intdreieck}
	\end{subfigure}
	\begin{subfigure}{0.48\textwidth}
		\centering
		\includegraphics[width=0.98\textwidth]{bilder/ALL0003/F0003TEK.JPG}
		\label{fig:dreieck}
	\end{subfigure}

\end{figure}

In Abbildung \ref{fig:dreieck} ist Dreieck und Parabelspannung zu sehen
